%%%%%%%%%%%%%%%%%%%%%%%%%%%%%%%%%%%%%%%%%
% Lachaise Assignment
% LaTeX Template
% Version 1.0 (26/6/2018)
%
% This template originates from:
% http://www.LaTeXTemplates.com
%
% Authors:
% Marion Lachaise & François Févotte
% Vel (vel@LaTeXTemplates.com)
%
% License:
% CC BY-NC-SA 3.0 (http://creativecommons.org/licenses/by-nc-sa/3.0/)
% 
%%%%%%%%%%%%%%%%%%%%%%%%%%%%%%%%%%%%%%%%%

%----------------------------------------------------------------------------------------
%	PACKAGES AND OTHER DOCUMENT CONFIGURATIONS
%----------------------------------------------------------------------------------------

\documentclass{article}

%%%%%%%%%%%%%%%%%%%%%%%%%%%%%%%%%%%%%%%%%
% Lachaise Assignment
% Structure Specification File
% Version 1.0 (26/6/2018)
%
% This template originates from:
% http://www.LaTeXTemplates.com
%
% Authors:
% Marion Lachaise & François Févotte
% Vel (vel@LaTeXTemplates.com)
%
% License:
% CC BY-NC-SA 3.0 (http://creativecommons.org/licenses/by-nc-sa/3.0/)
% 
%%%%%%%%%%%%%%%%%%%%%%%%%%%%%%%%%%%%%%%%%

%----------------------------------------------------------------------------------------
%	PACKAGES AND OTHER DOCUMENT CONFIGURATIONS
%----------------------------------------------------------------------------------------

\usepackage{amsmath,amsfonts,amssymb, tikz-cd} % Math packages

\usepackage{enumerate} % Custom item numbers for enumerations


\usepackage[framemethod=tikz]{mdframed} % Allows defining custom boxed/framed environments

\usepackage{listings} % File listings, with syntax highlighting
\lstset{
	basicstyle=\ttfamily, % Typeset listings in monospace font
}

%----------------------------------------------------------------------------------------
%	DOCUMENT MARGINS
%----------------------------------------------------------------------------------------

\usepackage{geometry} % Required for adjusting page dimensions and margins

\geometry{
	paper=letterpaper, % Paper size, change to letterpaper for US letter size
	top=2.5cm, % Top margin
	bottom=3cm, % Bottom margin
	left=2.5cm, % Left margin
	right=2.5cm, % Right margin
	headheight=14pt, % Header height
	footskip=1.5cm, % Space from the bottom margin to the baseline of the footer
	headsep=1.2cm, % Space from the top margin to the baseline of the header
	%showframe, % Uncomment to show how the type block is set on the page
}

%----------------------------------------------------------------------------------------
%	FONTS
%----------------------------------------------------------------------------------------

\usepackage[utf8]{inputenc} % Required for inputting international characters
\usepackage[T1]{fontenc} % Output font encoding for international characters


%----------------------------------------------------------------------------------------
%	COMMAND LINE ENVIRONMENT
%----------------------------------------------------------------------------------------

% Usage:
% \begin{commandline}
	%	\begin{verbatim}
		%		$ ls
		%		
		%		Applications	Desktop	...
		%	\end{verbatim}
	% \end{commandline}

\mdfdefinestyle{commandline}{
	leftmargin=10pt,
	rightmargin=10pt,
	innerleftmargin=15pt,
	middlelinecolor=black!50!white,
	middlelinewidth=2pt,
	frametitlerule=false,
	backgroundcolor=black!5!white,
	frametitle={Command Line},
	frametitlefont={\normalfont\sffamily\color{white}\hspace{-1em}},
	frametitlebackgroundcolor=black!50!white,
	nobreak,
}

% Define a custom environment for command-line snapshots
\newenvironment{commandline}{
	\medskip
	\begin{mdframed}[style=commandline]
	}{
	\end{mdframed}
	\medskip
}

%----------------------------------------------------------------------------------------
%	FILE CONTENTS ENVIRONMENT
%----------------------------------------------------------------------------------------

% Usage:
% \begin{file}[optional filename, defaults to "File"]
	%	File contents, for example, with a listings environment
	% \end{file}

\mdfdefinestyle{file}{
	innertopmargin=1.6\baselineskip,
	innerbottommargin=0.8\baselineskip,
	topline=false, bottomline=false,
	leftline=false, rightline=false,
	leftmargin=2cm,
	rightmargin=2cm,
	singleextra={%
		\draw[fill=black!10!white](P)++(0,-1.2em)rectangle(P-|O);
		\node[anchor=north west]
		at(P-|O){\ttfamily\mdfilename};
		%
		\def\l{3em}
		\draw(O-|P)++(-\l,0)--++(\l,\l)--(P)--(P-|O)--(O)--cycle;
		\draw(O-|P)++(-\l,0)--++(0,\l)--++(\l,0);
	},
	nobreak,
}

% Define a custom environment for file contents
\newenvironment{file}[1][File]{ % Set the default filename to "File"
	\medskip
	\newcommand{\mdfilename}{#1}
	\begin{mdframed}[style=file]
	}{
	\end{mdframed}
	\medskip
}

%----------------------------------------------------------------------------------------
%	NUMBERED QUESTIONS ENVIRONMENT
%----------------------------------------------------------------------------------------

% Usage:
% \begin{question}[optional title]
	%	Question contents
	% \end{question}

\mdfdefinestyle{question}{
	innertopmargin=1.2\baselineskip,
	innerbottommargin=0.8\baselineskip,
	roundcorner=5pt,
	nobreak,
	singleextra={%
		\draw(P-|O)node[xshift=1em,anchor=west,fill=white,draw,rounded corners=5pt]{%
			Question \theQuestion\questionTitle};
	},
}

\newcounter{Question} % Stores the current question number that gets iterated with each new question

% Define a custom environment for numbered questions
\newenvironment{question}[1][\unskip]{
	\bigskip
	\stepcounter{Question}
	\newcommand{\questionTitle}{~#1}
	\begin{mdframed}[style=question]
	}{
	\end{mdframed}
	\medskip
}

%----------------------------------------------------------------------------------------
%	WARNING TEXT ENVIRONMENT
%----------------------------------------------------------------------------------------

% Usage:
% \begin{warn}[optional title, defaults to "Warning:"]
	%	Contents
	% \end{warn}

\mdfdefinestyle{warning}{
	topline=false, bottomline=false,
	leftline=false, rightline=false,
	nobreak,
	singleextra={%
		\draw(P-|O)++(-0.5em,0)node(tmp1){};
		\draw(P-|O)++(0.5em,0)node(tmp2){};
		\fill[black,rotate around={45:(P-|O)}](tmp1)rectangle(tmp2);
		\node at(P-|O){\color{white}\scriptsize\bf !};
		\draw[very thick](P-|O)++(0,-1em)--(O);%--(O-|P);
	}
}

% Define a custom environment for warning text
\newenvironment{warn}[1][Warning:]{ % Set the default warning to "Warning:"
	\medskip
	\begin{mdframed}[style=warning]
		\noindent{\textbf{#1}}
	}{
	\end{mdframed}
}

%----------------------------------------------------------------------------------------
%	INFORMATION ENVIRONMENT
%----------------------------------------------------------------------------------------

% Usage:
% \begin{info}[optional title, defaults to "Info:"]
	% 	contents
	% 	\end{info}

\mdfdefinestyle{info}{%
	topline=false, bottomline=false,
	leftline=false, rightline=false,
	nobreak,
	singleextra={%
		\fill[black](P-|O)circle[radius=0.4em];
		\node at(P-|O){\color{white}\scriptsize\bf i};
		\draw[very thick](P-|O)++(0,-0.8em)--(O);%--(O-|P);
	}
}

% Define a custom environment for information
\newenvironment{info}[1][Info:]{ % Set the default title to "Info:"
	\medskip
	\begin{mdframed}[style=info]
		\noindent{\textbf{#1}}
	}{
	\end{mdframed}
}
 % Include the file specifying the document structure and custom commands

%----------------------------------------------------------------------------------------
%	ASSIGNMENT INFORMATION
%----------------------------------------------------------------------------------------

\title{Homework 4 Measure Theor} % Title of the assignment

\author{Gandhar Kulkarni (mmat2304)} % Author name and email address

\date{} % University, school and/or department name(s) and a date

%----------------------------------------------------------------------------------------

\begin{document}

\maketitle % Print the title
\section{} %Problem 1
Let $(X, \mathcal{S}, \mu)$ be a measure space, and let $f:X \rightarrow \mathbb{R}$ be a measurable function. Suppose $$\mu(\{x \in X: |f(x)|\geq \varepsilon\})=0, $$ for all $\varepsilon>0.$ Prove that $f=0$ a.e.

\emph{Solution:} For all $n \in  \mathbb{N},$ we can say that $\mu(\{x \in X: |f(x)|\geq 1/n\})=0.$ Define $\{x \in X: |f(x)|\geq 1/n\}$ as $N_n.$
See that $N_n \subseteq N_{n+1}.$ Consider $\mu(\cap_{n=1}^{\infty}N_n)=\mu(\lim_{n \to \infty}N_n)=\lim_{n \to \infty} \mu(N_n)=0.$ See that $\lim_{n \to \infty}N_n=\{x \in X: |f(x)| >0\}.$
This is precisely the definition of $f=0$ a.e., proving the statement. 
\section{} %Problem 2
Let $f:\mathbb{R} \rightarrow \mathbb{R}$ be a function such that the set $$\{x \in \mathbb{R}: a \leq f(x) \leq b\}, $$
is measurable for any $a<b.$ Prove that $f$ is measurable.

\emph{Solution:} Consider the collection $\{[a-1/n,b+n]\}_{n \in \mathbb{N}}.$ Thus union of this collection is $(a,\infty).$ Let $C_n:= f^{-1}([a-1/n,b+n]).$
See that $C_n$ is measurable for all $n.$ Since measurable sets are closed under countable union, we have $\cup_{n=1}^{\infty}C_n=f^{-1}([a-1/n,b+n])=f^{-1}((a,\infty))$ is 
also measurable. Our choice of $a$ was arbitrary, which implies that $f$ is a measurable function. 
\section{} %Problem 3
Suppose that $f:[a,b]\rightarrow \mathbb{R}$ is a function such that 
$$\{x \in [a,b]: f(x)=c \}, $$ is measurable for each $c \in \mathbb{R}.$ Is $f$ necessarily measurable?

\emph{Solution:} We know that $f^{-1}(\{c\})$ is measurable for all $c \in \mathbb{R}.$ See that the function $f:[0,1]\rightarrow \mathbb{R}$
where $f(x)=x$ if $x \in V,$ where $V$ is the Vitali set, and $x+1$ otherwise. See that this function is actually one-one, so the pre-image has only one point, which is measurable.
However, $f^{-1}([0,1]\cap f([0,1]))$ is $V,$ a non-measurable set. Thus $f$ needn't be measurable as a function even though the fibre of each point is measurable as a set.
\section{} %Problem 4
\section{} %Problem 5
\section{} %Problem 6
Prove that an increasing function $f: [a,b] \rightarrow \mathbb{R}$ is measurable.

\emph{Solution:} We know from the previous homework that monotone functions are measurable, since we can explicitly find $f^{-1}((a,\infty)),$ where $f$ is 
a monotone function. Since $f:[a,b] \rightarrow \mathbb{R}$ in this case is given to be increasing, it is also monotone. Therefore it must also be measurable.

$e^{i \pi}$
\section{} %Problem 7
\section{} %Problem 8
If $f: \mathbb{R} \rightarrow \mathbb{R}$ is continuous and $g: \mathbb{R} \rightarrow \mathbb{R}$ is Lebesgue measurable,
then $g \circ f$ is Lebesgue measurable. True or false?

\emph{Solution:} Let $f: [0,2] \rightarrow [0,1]$ be the inverse of the function $K: [0,1]\rightarrow [0,2],$ which is given by $K(x)=\Lambda(x)+x,$ where $\Lambda$ is the Cantor function.
It is strictly increasing, hence one-one. It is also surjective onto $[0,2]$ as it takes every value in that range due to the continuity of $\Lambda$ and $x.$ Thus $K$ is biijective, and the function 
$f;=K^{-1}$ is also bijective and continuous, thus a measurable function.
\section{} %Problem 9
 
%----------------------------------------------------------------------------------------
%	INTRODUCTION
%----------------------------------------------------------------------------------------


\end{document}
