%%%%%%%%%%%%%%%%%%%%%%%%%%%%%%%%%%%%%%%%%
% Lachaise Assignment
% LaTeX Template
% Version 1.0 (26/6/2018)
%
% This template originates from:
% http://www.LaTeXTemplates.com
%
% Authors:
% Marion Lachaise & François Févotte
% Vel (vel@LaTeXTemplates.com)
%
% License:
% CC BY-NC-SA 3.0 (http://creativecommons.org/licenses/by-nc-sa/3.0/)
% 
%%%%%%%%%%%%%%%%%%%%%%%%%%%%%%%%%%%%%%%%%

%----------------------------------------------------------------------------------------
%	PACKAGES AND OTHER DOCUMENT CONFIGURATIONS
%----------------------------------------------------------------------------------------

\documentclass{article}

\input{structure.tex} % Include the file specifying the document structure and custom commands

%----------------------------------------------------------------------------------------
%	ASSIGNMENT INFORMATION
%----------------------------------------------------------------------------------------

\title{} % Title of the assignment

\author{Gandhar Kulkarni (mmat2304)} % Author name and email address

\date{} % University, school and/or department name(s) and a date

%----------------------------------------------------------------------------------------

\begin{document}

\maketitle % Print the title

%----------------------------------------------------------------------------------------
%	INTRODUCTION
%----------------------------------------------------------------------------------------

\section{} %Problem 1 
\begin{enumerate}
	\item
	\item This statement is false. For sake of contradiction, let $(X,||\cdot||)$ be a normed linear space such that the induced metric is the discrete 
	metric. Then for $x,y \in X, x \neq y$ we must have $||x-y|| =1.$ Note that $2x\neq 2y,$ so we must have $||2x-2y||=2||x-y||=2,$ but by the discrete 
	metric the answer should still be $1!$ Thus there can be no such norm.
\end{enumerate}
\section{} %Problem 2 
We wish to show that the function $||\cdot|| $ on $X$ satisfies the triangle inequality iff the closed unit ball is convex.
Assume that the function $||\cdot||$ is indeed a norm. Then let $x,y \in D,$ the closed unit ball. Then $||x||,||y|| \leq 1.$ Now we have for $\alpha \in 
[0,1]$ $z=\alpha x+(1-\alpha)y.$ See that $$||z|| =||\alpha x+(1-\alpha)y|| \leq \alpha ||x|| + (1-\alpha) ||y|| \leq \alpha \cdot 1 + (1-\alpha) \cdot 1 
\leq 1,$$ thus we have $z \in D.$

Take two elements $x,y \in X$ both non-zero, since if either were zero the inequality would be trivial. Then $$||x+y|| = (||x||+||y||)\cdot 
\left|\left|\alpha \frac{x}{||x||}+ (1-\alpha)\frac{y}{||y||}\right|\right|,$$ where $\alpha=\frac{||x||}{||x||+||y||}.$ Note that 
$\frac{x}{||x||}=\frac{y}{||y||}=1,$ thus we can use the convexity condition to see that $\frac{||x+y||}{||x||+||y||} \leq 1,$ which is the triangle 
inequality.
\section{} %Problem 3 
\begin{enumerate}
	\item Pick a $f \in C([a,b]).$ Then $|f|\leq M=\sup\{|f(x)|:x \in [a,b]\}.$ Now see that $$\int_a^b|f(t)|^pdt \leq (b-a)M^p \geq 0.$$ Thus $||f||_p \geq 
	0.$
	For $f=0,$ we have $M=0,$ so $\int_a^b|0|^pdt=0.$ If $\int_a^b|f(t)|^pdt =0,$ then see that $0 \leq (b-a)M^p \geq 0.$ Thus we must have $$(b-a)M^p=0 
	\implies M=0 \implies f=0.$$ To see the next axiom, for $\alpha \in \mathbb{K},$ we have $\int_a^b|\alpha f(t)|^pdt =\int_a^b|\alpha|^p|f(t)|^pdt.$
	Then $$||\alpha f||_p= \left(|\alpha|^p\int_a^b|f(t)|^pdt\right)^{\frac{1}{p}} =|\alpha| ||f||_p.$$
	Now let $f,g \in C[a,b].$ 

\end{enumerate}
\section{} %Problem 4 
\begin{enumerate}
	\item $||f_1-F||_{\infty}$ is to be found, where $F$ is the subspace of constant functions. Unfolding the term, we get 
	$$||f_1-F||_{\infty}= \inf_{c \in \mathbb{R} }\{ \sup_{t \in [0,1]} |t-c|  \},$$ which for $c \in [-1,0] \cup [1,2] $ is $1,$ while in $(0,1)$ it 
	decreases to $\frac{1}{2}$ then goes back up to $1.$ Thus, we must have $||f_1-F||_{\infty}=\frac{1}{2}.$
	\item We want to now see the distance between $f_2=t^2$ and $G,$ the space of all polynomials with degree at most $1.$ Then for some polynomial $-ax-b 
	\in G,$ we want to see $\sup_{t \in [0,1]}\{|t^2+ax+b|\}$ 
\end{enumerate}
\section{} %Problem 5 
\section{} %Problem 6 
Assume that $X$ is a Banach space. Let $(x_n)$ be an absolutely convergent sequence in $X,$ that is, $||x_n||\rightarrow \alpha,$ as $n \to \infty.$ Thus 
for all $\varepsilon > 0,$ $\exists N \in \mathbb{N}$ such that for all $n \geq N,$ $$| ||x_n|| - \alpha| < \varepsilon.$$
Now we have $ $
\section{} %Problem 7 
\section{} %Problem 8 
\section{} %Problem 9 
\section{} %Problem 10 
\section{} %Problem 11
\section{} %Problem 12 

\end{document}
