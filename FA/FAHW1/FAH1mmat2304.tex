%%%%%%%%%%%%%%%%%%%%%%%%%%%%%%%%%%%%%%%%%
% Lachaise Assignment
% LaTeX Template
% Version 1.0 (26/6/2018)
%
% This template originates from:
% http://www.LaTeXTemplates.com
%
% Authors:
% Marion Lachaise & François Févotte
% Vel (vel@LaTeXTemplates.com)
%
% License:
% CC BY-NC-SA 3.0 (http://creativecommons.org/licenses/by-nc-sa/3.0/)
% 
%%%%%%%%%%%%%%%%%%%%%%%%%%%%%%%%%%%%%%%%%

%----------------------------------------------------------------------------------------
%	PACKAGES AND OTHER DOCUMENT CONFIGURATIONS
%----------------------------------------------------------------------------------------

\documentclass{article}

%%%%%%%%%%%%%%%%%%%%%%%%%%%%%%%%%%%%%%%%%
% Lachaise Assignment
% Structure Specification File
% Version 1.0 (26/6/2018)
%
% This template originates from:
% http://www.LaTeXTemplates.com
%
% Authors:
% Marion Lachaise & François Févotte
% Vel (vel@LaTeXTemplates.com)
%
% License:
% CC BY-NC-SA 3.0 (http://creativecommons.org/licenses/by-nc-sa/3.0/)
% 
%%%%%%%%%%%%%%%%%%%%%%%%%%%%%%%%%%%%%%%%%

%----------------------------------------------------------------------------------------
%	PACKAGES AND OTHER DOCUMENT CONFIGURATIONS
%----------------------------------------------------------------------------------------

\usepackage{amsmath,amsfonts,amssymb, tikz-cd} % Math packages

\usepackage{enumerate} % Custom item numbers for enumerations


\usepackage[framemethod=tikz]{mdframed} % Allows defining custom boxed/framed environments

\usepackage{listings} % File listings, with syntax highlighting
\lstset{
	basicstyle=\ttfamily, % Typeset listings in monospace font
}

%----------------------------------------------------------------------------------------
%	DOCUMENT MARGINS
%----------------------------------------------------------------------------------------

\usepackage{geometry} % Required for adjusting page dimensions and margins

\geometry{
	paper=letterpaper, % Paper size, change to letterpaper for US letter size
	top=2.5cm, % Top margin
	bottom=3cm, % Bottom margin
	left=2.5cm, % Left margin
	right=2.5cm, % Right margin
	headheight=14pt, % Header height
	footskip=1.5cm, % Space from the bottom margin to the baseline of the footer
	headsep=1.2cm, % Space from the top margin to the baseline of the header
	%showframe, % Uncomment to show how the type block is set on the page
}

%----------------------------------------------------------------------------------------
%	FONTS
%----------------------------------------------------------------------------------------

\usepackage[utf8]{inputenc} % Required for inputting international characters
\usepackage[T1]{fontenc} % Output font encoding for international characters


%----------------------------------------------------------------------------------------
%	COMMAND LINE ENVIRONMENT
%----------------------------------------------------------------------------------------

% Usage:
% \begin{commandline}
	%	\begin{verbatim}
		%		$ ls
		%		
		%		Applications	Desktop	...
		%	\end{verbatim}
	% \end{commandline}

\mdfdefinestyle{commandline}{
	leftmargin=10pt,
	rightmargin=10pt,
	innerleftmargin=15pt,
	middlelinecolor=black!50!white,
	middlelinewidth=2pt,
	frametitlerule=false,
	backgroundcolor=black!5!white,
	frametitle={Command Line},
	frametitlefont={\normalfont\sffamily\color{white}\hspace{-1em}},
	frametitlebackgroundcolor=black!50!white,
	nobreak,
}

% Define a custom environment for command-line snapshots
\newenvironment{commandline}{
	\medskip
	\begin{mdframed}[style=commandline]
	}{
	\end{mdframed}
	\medskip
}

%----------------------------------------------------------------------------------------
%	FILE CONTENTS ENVIRONMENT
%----------------------------------------------------------------------------------------

% Usage:
% \begin{file}[optional filename, defaults to "File"]
	%	File contents, for example, with a listings environment
	% \end{file}

\mdfdefinestyle{file}{
	innertopmargin=1.6\baselineskip,
	innerbottommargin=0.8\baselineskip,
	topline=false, bottomline=false,
	leftline=false, rightline=false,
	leftmargin=2cm,
	rightmargin=2cm,
	singleextra={%
		\draw[fill=black!10!white](P)++(0,-1.2em)rectangle(P-|O);
		\node[anchor=north west]
		at(P-|O){\ttfamily\mdfilename};
		%
		\def\l{3em}
		\draw(O-|P)++(-\l,0)--++(\l,\l)--(P)--(P-|O)--(O)--cycle;
		\draw(O-|P)++(-\l,0)--++(0,\l)--++(\l,0);
	},
	nobreak,
}

% Define a custom environment for file contents
\newenvironment{file}[1][File]{ % Set the default filename to "File"
	\medskip
	\newcommand{\mdfilename}{#1}
	\begin{mdframed}[style=file]
	}{
	\end{mdframed}
	\medskip
}

%----------------------------------------------------------------------------------------
%	NUMBERED QUESTIONS ENVIRONMENT
%----------------------------------------------------------------------------------------

% Usage:
% \begin{question}[optional title]
	%	Question contents
	% \end{question}

\mdfdefinestyle{question}{
	innertopmargin=1.2\baselineskip,
	innerbottommargin=0.8\baselineskip,
	roundcorner=5pt,
	nobreak,
	singleextra={%
		\draw(P-|O)node[xshift=1em,anchor=west,fill=white,draw,rounded corners=5pt]{%
			Question \theQuestion\questionTitle};
	},
}

\newcounter{Question} % Stores the current question number that gets iterated with each new question

% Define a custom environment for numbered questions
\newenvironment{question}[1][\unskip]{
	\bigskip
	\stepcounter{Question}
	\newcommand{\questionTitle}{~#1}
	\begin{mdframed}[style=question]
	}{
	\end{mdframed}
	\medskip
}

%----------------------------------------------------------------------------------------
%	WARNING TEXT ENVIRONMENT
%----------------------------------------------------------------------------------------

% Usage:
% \begin{warn}[optional title, defaults to "Warning:"]
	%	Contents
	% \end{warn}

\mdfdefinestyle{warning}{
	topline=false, bottomline=false,
	leftline=false, rightline=false,
	nobreak,
	singleextra={%
		\draw(P-|O)++(-0.5em,0)node(tmp1){};
		\draw(P-|O)++(0.5em,0)node(tmp2){};
		\fill[black,rotate around={45:(P-|O)}](tmp1)rectangle(tmp2);
		\node at(P-|O){\color{white}\scriptsize\bf !};
		\draw[very thick](P-|O)++(0,-1em)--(O);%--(O-|P);
	}
}

% Define a custom environment for warning text
\newenvironment{warn}[1][Warning:]{ % Set the default warning to "Warning:"
	\medskip
	\begin{mdframed}[style=warning]
		\noindent{\textbf{#1}}
	}{
	\end{mdframed}
}

%----------------------------------------------------------------------------------------
%	INFORMATION ENVIRONMENT
%----------------------------------------------------------------------------------------

% Usage:
% \begin{info}[optional title, defaults to "Info:"]
	% 	contents
	% 	\end{info}

\mdfdefinestyle{info}{%
	topline=false, bottomline=false,
	leftline=false, rightline=false,
	nobreak,
	singleextra={%
		\fill[black](P-|O)circle[radius=0.4em];
		\node at(P-|O){\color{white}\scriptsize\bf i};
		\draw[very thick](P-|O)++(0,-0.8em)--(O);%--(O-|P);
	}
}

% Define a custom environment for information
\newenvironment{info}[1][Info:]{ % Set the default title to "Info:"
	\medskip
	\begin{mdframed}[style=info]
		\noindent{\textbf{#1}}
	}{
	\end{mdframed}
}
 % Include the file specifying the document structure and custom commands

%----------------------------------------------------------------------------------------
%	ASSIGNMENT INFORMATION
%----------------------------------------------------------------------------------------

\title{} % Title of the assignment

\author{Gandhar Kulkarni (mmat2304)} % Author name and email address

\date{} % University, school and/or department name(s) and a date

%----------------------------------------------------------------------------------------

\begin{document}

\maketitle % Print the title

%----------------------------------------------------------------------------------------
%	INTRODUCTION
%----------------------------------------------------------------------------------------

\section{} %Problem 1 
\begin{enumerate}
	\item
	\item This statement is false. For sake of contradiction, let $(X,||\cdot||)$ be a normed linear space such that the induced metric is the discrete 
	metric. Then for $x,y \in X, x \neq y$ we must have $||x-y|| =1.$ Note that $2x\neq 2y,$ so we must have $||2x-2y||=2||x-y||=2,$ but by the discrete 
	metric the answer should still be $1!$ Thus there can be no such norm.
\end{enumerate}
\section{} %Problem 2 
We wish to show that the function $||\cdot|| $ on $X$ satisfies the triangle inequality iff the closed unit ball is convex.
Assume that the function $||\cdot||$ is indeed a norm. Then let $x,y \in D,$ the closed unit ball. Then $||x||,||y|| \leq 1.$ Now we have for $\alpha \in 
[0,1]$ $z=\alpha x+(1-\alpha)y.$ See that $$||z|| =||\alpha x+(1-\alpha)y|| \leq \alpha ||x|| + (1-\alpha) ||y|| \leq \alpha \cdot 1 + (1-\alpha) \cdot 1 
\leq 1,$$ thus we have $z \in D.$

Take two elements $x,y \in X$ both non-zero, since if either were zero the inequality would be trivial. Then $$||x+y|| = (||x||+||y||)\cdot 
\left|\left|\alpha \frac{x}{||x||}+ (1-\alpha)\frac{y}{||y||}\right|\right|,$$ where $\alpha=\frac{||x||}{||x||+||y||}.$ Note that 
$\frac{x}{||x||}=\frac{y}{||y||}=1,$ thus we can use the convexity condition to see that $\frac{||x+y||}{||x||+||y||} \leq 1,$ which is the triangle 
inequality.
\section{} %Problem 3 
\begin{enumerate}
	\item Pick a $f \in C([a,b]).$ Then $|f|\leq M=\sup\{|f(x)|:x \in [a,b]\}.$ Now see that $$\int_a^b|f(t)|^pdt \leq (b-a)M^p \geq 0.$$ Thus $||f||_p \geq 
	0.$
	For $f=0,$ we have $M=0,$ so $\int_a^b|0|^pdt=0.$ If $\int_a^b|f(t)|^pdt =0,$ then see that $0 \leq (b-a)M^p \geq 0.$ Thus we must have $$(b-a)M^p=0 
	\implies M=0 \implies f=0.$$ To see the next axiom, for $\alpha \in \mathbb{K},$ we have $\int_a^b|\alpha f(t)|^pdt =\int_a^b|\alpha|^p|f(t)|^pdt.$
	Then $$||\alpha f||_p= \left(|\alpha|^p\int_a^b|f(t)|^pdt\right)^{\frac{1}{p}} =|\alpha| ||f||_p.$$
	Now let $f,g \in C[a,b].$ Then we have to prove Minkowski's inequality to show the triangle inequality. 
	\begin{align*}
		||f+g||^p_p &= \int_{a}^b |f+g|^p dx\\
		&= \int_{a}^b |f+g|\cdot|f+g|^{p-1} dx\\
		&\leq \int_{a}^b |f||f+g|^{p-1} dx + \int_{a}^b |g||f+g|^{p-1} dx\\
		&\leq \left(\int_{a}^b |f|^{p} dx+\int_{a}^b |g|^{p} dx\right)\left(\int_{a}^b |f+g|^{(p-1)\frac{p}{p-1}} dx \right)^{1-\frac{1}{p}} 
		(\text{H\"{o}lder's inequality})\\
		&= (||f||_p+||g||_p)\frac{||f+g||_p^p}{||f+g||_p},
	\end{align*}
	which yields the required result.
	\item 

\end{enumerate}
\section{} %Problem 4 
\begin{enumerate}
	\item $||f_1-F||_{\infty}$ is to be found, where $F$ is the subspace of constant functions. Unfolding the term, we get 
	$$||f_1-F||_{\infty}= \inf_{c \in \mathbb{R} }\{ \sup_{t \in [0,1]} |t-c|  \},$$ which for $c \in [-1,0] \cup [1,2] $ is $1,$ while in $(0,1)$ it 
	decreases to $\frac{1}{2}$ then goes back up to $1.$ Thus, we must have $||f_1-F||_{\infty}=\frac{1}{2}.$
	\item We want to now see the distance between $f_2=t^2$ and $G,$ the space of all polynomials with degree at most $1.$ Then for some polynomial $-ax-b 
	\in G,$ we want to see $\sup_{t \in [0,1]}\{|t^2+ax+b|\}$ 
\end{enumerate}
\section{} %Problem 5 
Let $Y$ and $X/Y$ be Banach spaces. Then take $(x_n)$ to be a Cauchy sequence in $X.$ That is, for all $\varepsilon > 0$ there exists $N \in \mathbb{N}$ 
such that for $m,n \geq N$ we have $$||x_m-x_n||< \varepsilon.$$
By the canonical projection to $X/Y$ we can see that the sequence $(x_n+Y)$ is also Cauchy, since we have $$||x_m-x_n +Y|| \leq ||x_m-x_n+0|| < \varepsilon. 
$$ Since $X/Y$ is Banach, we have $(x_n+Y) \to (x_0+Y).$ Now let $y_n:= $ 

Let $X$ and $X/Y$ be Banach spaces. Then take $(y_n),$ a Cauchy sequence in $Y.$ Thus for all $\varepsilon > 0,$ there exists $N \in \mathbb{N}$ such that 
for $m,n \geq N,$ we have $||y_m-y_n||< \varepsilon.$ As a Cauchy sequence in $X,$ this must converge to some element $y_0 \in X.$ We now need to show that 
$y_0 \in Y.$ But since $Y$ is closed and $y_0$ is a limit point, we must have $y_0 \in Y.$

Let $X$ and $Y$ be Banach spaces. 
\section{} %Problem 6 
Assume that $X$ is a Banach space. Let $(x_n)$ be an absolutely convergent sequence in $X,$ that is, $||x_n||\rightarrow \alpha,$ as $n \to \infty.$ Thus 
for all $\varepsilon > 0,$ $\exists N \in \mathbb{N}$ such that for all $n \geq N,$ $$| ||x_n|| - \alpha| < \varepsilon.$$
Now we have $ $
\section{} %Problem 7 
Let $\ell^p$ be the space of all $p-$power summable sequences. Then $||(x_n)||=(\sum_{i=1}^{\infty}|x_i|^p)^\frac{1}{p}.$ Now we want to show that the space 
$K=\{(x_n) \in \ell^p: x_i=0 \forall i>n, n \in \mathbb{N}\}$ is dense in $\ell^p.$ Take any $(x_n)^{(0)}_{n \in \mathbb{N}} \in \ell^p.$ There is a 
sequence of elements in $\ell^p$ $(x_n)^{(m)} \subseteq K$ such that $x_n^{(m)}=x_n$ if $n \leq m,$ and $0$ otherwise. See that 
$||(x_n)^{(m)}-(x_n)^{(0)}||_p=\left(\sum_{i=m+1}^{\infty}|x_i|^p\right)^{\frac{1}{p}}.$ Since $||(x_n)^{(m)}||_p \to ||(x_n)^{(0)}||_p$ as $m \to \infty,$ 
we must have for a choice of $\varepsilon >0$ there being $N \in \mathbb{N}$ such that for $m \geq N,$ $|||(x_n)^{(m)}||_p - ||(x_n)^{(0)}||_p|< 
\varepsilon. $ Multiplying on both sides by $\frac{||(x_n)^{(m)}||^p_p - ||(x_n)^{(0)}||^p_p}{||(x_n)^{(m)}||_p - ||(x_n)^{(0)}||_p},$ and after a change in 
the value of $\varepsilon,$ we get $$||(x_n)^{(m)}||^p_p - ||(x_n)^{(0)}||^p_p< \varepsilon.$$ Thus we have shown that $K$ is dense in $\ell^p.$ For a fixed 
$m,$ and a fixed $n,$ look at $x_n^{(m)}.$ This is a real number or a complex number, which can be approximated by a sequence of rationals (or elements of 
$\mathbb{Q}(i)$) such that $x_n^{(m)}$ is its limit. Now we have a sequence $(t_k),$ where $t_k \to x_n^{(m)}$ as $k \to \infty.$ 

Now let us see that $K'=\{(x_n) \in K: x_n \in \mathbb{Q} \text{ or} \mathbb{Q}(i)\} \subset K$ is countable, and as we saw above must be dense in $\ell^p.$ 
Thus it is separable.  
\section{} %Problem 8 
\section{} %Problem 9 
\section{} %Problem 10 
Let $P([0,1])$ be the space of all real polynomials defined on $[0,1]$ be a real vector space. Let the norm of a polynomial $f(x)=a_0+a_1x+\dots+a_nx^n \in 
P([0,1])$ be given thus: $||f|| = |a_0|+|a_1|+\dots+|a_n|.$ Now see that the operator $I:P([0,1]) \rightarrow P([0,1])$ such that $x^t \mapsto 
\frac{x^{t+1}}{t+1},$ we then see that $$||I|| = \sup_{0 \neq x \in P([0,1])} \frac{||If||}{||f||} = \frac{a_0x+\dots 
\frac{a_n}{n+1}a_{n+1}}{a_0+\dots+a_nx^n }=\frac{|a_0|+\left|\frac{a_1}{2}\right|+\dots + \left|\frac{a_n}{n+1}\right|}{|a_0|+\dots+ |a_n|} \leq 1.$$ Also 
see that this supremum is indeed attained since $I(a_0)=a_0x,$ and in this case $\frac{||I(a_0)||}{||a_0||}=1.$ Thus we have $||I|| =1.$ We wish to find the 
inverse of this operator, see that the differential operator $D$ such that $x^t \mapsto tx^{t-1},$ is the required inverse. However, see that for 
$f(x)=x^n,$ we have $Df=I^{-1}f=nx^{n-1}.$ Now we have $$||D|| = \sup_{0 \neq x \in P([0,1])} \frac{||Df||}{||f||} \geq 
\frac{||Dx^n||}{||x^n||}=\frac{n}{1}.$$ Thus we have that our operator $D$ is unbounded, since for any chosen $N \in \mathbb{N}$ we can choose $x^{N+1},$ 
such that $ \frac{||Dx^{N+1}||}{||x^{N+1}||}$ is larger. 
\section{} %Problem 11
\section{} %Problem 12 

\end{document}
