%%%%%%%%%%%%%%%%%%%%%%%%%%%%%%%%%%%%%%%%%
% Lachaise Assignment
% LaTeX Template
% Version 1.0 (26/6/2018)
%
% This template originates from:
% http://www.LaTeXTemplates.com
%
% Authors:
% Marion Lachaise & François Févotte
% Vel (vel@LaTeXTemplates.com)
%
% License:
% CC BY-NC-SA 3.0 (http://creativecommons.org/licenses/by-nc-sa/3.0/)
% 
%%%%%%%%%%%%%%%%%%%%%%%%%%%%%%%%%%%%%%%%%

%----------------------------------------------------------------------------------------
%	PACKAGES AND OTHER DOCUMENT CONFIGURATIONS
%----------------------------------------------------------------------------------------

\documentclass{article}

%%%%%%%%%%%%%%%%%%%%%%%%%%%%%%%%%%%%%%%%%
% Lachaise Assignment
% Structure Specification File
% Version 1.0 (26/6/2018)
%
% This template originates from:
% http://www.LaTeXTemplates.com
%
% Authors:
% Marion Lachaise & François Févotte
% Vel (vel@LaTeXTemplates.com)
%
% License:
% CC BY-NC-SA 3.0 (http://creativecommons.org/licenses/by-nc-sa/3.0/)
% 
%%%%%%%%%%%%%%%%%%%%%%%%%%%%%%%%%%%%%%%%%

%----------------------------------------------------------------------------------------
%	PACKAGES AND OTHER DOCUMENT CONFIGURATIONS
%----------------------------------------------------------------------------------------

\usepackage{amsmath,amsfonts,amssymb, tikz-cd} % Math packages

\usepackage{enumerate} % Custom item numbers for enumerations


\usepackage[framemethod=tikz]{mdframed} % Allows defining custom boxed/framed environments

\usepackage{listings} % File listings, with syntax highlighting
\lstset{
	basicstyle=\ttfamily, % Typeset listings in monospace font
}

%----------------------------------------------------------------------------------------
%	DOCUMENT MARGINS
%----------------------------------------------------------------------------------------

\usepackage{geometry} % Required for adjusting page dimensions and margins

\geometry{
	paper=letterpaper, % Paper size, change to letterpaper for US letter size
	top=2.5cm, % Top margin
	bottom=3cm, % Bottom margin
	left=2.5cm, % Left margin
	right=2.5cm, % Right margin
	headheight=14pt, % Header height
	footskip=1.5cm, % Space from the bottom margin to the baseline of the footer
	headsep=1.2cm, % Space from the top margin to the baseline of the header
	%showframe, % Uncomment to show how the type block is set on the page
}

%----------------------------------------------------------------------------------------
%	FONTS
%----------------------------------------------------------------------------------------

\usepackage[utf8]{inputenc} % Required for inputting international characters
\usepackage[T1]{fontenc} % Output font encoding for international characters


%----------------------------------------------------------------------------------------
%	COMMAND LINE ENVIRONMENT
%----------------------------------------------------------------------------------------

% Usage:
% \begin{commandline}
	%	\begin{verbatim}
		%		$ ls
		%		
		%		Applications	Desktop	...
		%	\end{verbatim}
	% \end{commandline}

\mdfdefinestyle{commandline}{
	leftmargin=10pt,
	rightmargin=10pt,
	innerleftmargin=15pt,
	middlelinecolor=black!50!white,
	middlelinewidth=2pt,
	frametitlerule=false,
	backgroundcolor=black!5!white,
	frametitle={Command Line},
	frametitlefont={\normalfont\sffamily\color{white}\hspace{-1em}},
	frametitlebackgroundcolor=black!50!white,
	nobreak,
}

% Define a custom environment for command-line snapshots
\newenvironment{commandline}{
	\medskip
	\begin{mdframed}[style=commandline]
	}{
	\end{mdframed}
	\medskip
}

%----------------------------------------------------------------------------------------
%	FILE CONTENTS ENVIRONMENT
%----------------------------------------------------------------------------------------

% Usage:
% \begin{file}[optional filename, defaults to "File"]
	%	File contents, for example, with a listings environment
	% \end{file}

\mdfdefinestyle{file}{
	innertopmargin=1.6\baselineskip,
	innerbottommargin=0.8\baselineskip,
	topline=false, bottomline=false,
	leftline=false, rightline=false,
	leftmargin=2cm,
	rightmargin=2cm,
	singleextra={%
		\draw[fill=black!10!white](P)++(0,-1.2em)rectangle(P-|O);
		\node[anchor=north west]
		at(P-|O){\ttfamily\mdfilename};
		%
		\def\l{3em}
		\draw(O-|P)++(-\l,0)--++(\l,\l)--(P)--(P-|O)--(O)--cycle;
		\draw(O-|P)++(-\l,0)--++(0,\l)--++(\l,0);
	},
	nobreak,
}

% Define a custom environment for file contents
\newenvironment{file}[1][File]{ % Set the default filename to "File"
	\medskip
	\newcommand{\mdfilename}{#1}
	\begin{mdframed}[style=file]
	}{
	\end{mdframed}
	\medskip
}

%----------------------------------------------------------------------------------------
%	NUMBERED QUESTIONS ENVIRONMENT
%----------------------------------------------------------------------------------------

% Usage:
% \begin{question}[optional title]
	%	Question contents
	% \end{question}

\mdfdefinestyle{question}{
	innertopmargin=1.2\baselineskip,
	innerbottommargin=0.8\baselineskip,
	roundcorner=5pt,
	nobreak,
	singleextra={%
		\draw(P-|O)node[xshift=1em,anchor=west,fill=white,draw,rounded corners=5pt]{%
			Question \theQuestion\questionTitle};
	},
}

\newcounter{Question} % Stores the current question number that gets iterated with each new question

% Define a custom environment for numbered questions
\newenvironment{question}[1][\unskip]{
	\bigskip
	\stepcounter{Question}
	\newcommand{\questionTitle}{~#1}
	\begin{mdframed}[style=question]
	}{
	\end{mdframed}
	\medskip
}

%----------------------------------------------------------------------------------------
%	WARNING TEXT ENVIRONMENT
%----------------------------------------------------------------------------------------

% Usage:
% \begin{warn}[optional title, defaults to "Warning:"]
	%	Contents
	% \end{warn}

\mdfdefinestyle{warning}{
	topline=false, bottomline=false,
	leftline=false, rightline=false,
	nobreak,
	singleextra={%
		\draw(P-|O)++(-0.5em,0)node(tmp1){};
		\draw(P-|O)++(0.5em,0)node(tmp2){};
		\fill[black,rotate around={45:(P-|O)}](tmp1)rectangle(tmp2);
		\node at(P-|O){\color{white}\scriptsize\bf !};
		\draw[very thick](P-|O)++(0,-1em)--(O);%--(O-|P);
	}
}

% Define a custom environment for warning text
\newenvironment{warn}[1][Warning:]{ % Set the default warning to "Warning:"
	\medskip
	\begin{mdframed}[style=warning]
		\noindent{\textbf{#1}}
	}{
	\end{mdframed}
}

%----------------------------------------------------------------------------------------
%	INFORMATION ENVIRONMENT
%----------------------------------------------------------------------------------------

% Usage:
% \begin{info}[optional title, defaults to "Info:"]
	% 	contents
	% 	\end{info}

\mdfdefinestyle{info}{%
	topline=false, bottomline=false,
	leftline=false, rightline=false,
	nobreak,
	singleextra={%
		\fill[black](P-|O)circle[radius=0.4em];
		\node at(P-|O){\color{white}\scriptsize\bf i};
		\draw[very thick](P-|O)++(0,-0.8em)--(O);%--(O-|P);
	}
}

% Define a custom environment for information
\newenvironment{info}[1][Info:]{ % Set the default title to "Info:"
	\medskip
	\begin{mdframed}[style=info]
		\noindent{\textbf{#1}}
	}{
	\end{mdframed}
}
 % Include the file specifying the document structure and custom commands

%----------------------------------------------------------------------------------------
%	ASSIGNMENT INFORMATION
%----------------------------------------------------------------------------------------

\title{Functional Analysis Homework 2} % Title of the assignment

\author{Gandhar Kulkarni (mmat2304)} % Author name and email address

\date{} % University, school and/or department name(s) and a date

%----------------------------------------------------------------------------------------

\begin{document}

\maketitle % Print the title

%----------------------------------------------------------------------------------------
%	INTRODUCTION
%----------------------------------------------------------------------------------------

\section{} %Problem 1 
We need to check that rules of inner products hold--- 
\begin{enumerate}
	\item For $A=B,$ we have $\langle A,A\rangle=tr(AA^*)= \sum_{i,j}|a_{ij}|^2 \geq 0,$ where $a_{ij}$ denotes the elements of $A.$ 
	Moreover, $||A|| =0 \implies |a_{ij}| =0 $ for all $1 \leq i,j \leq n \implies A=0.$ 
	\item $\langle B,A \rangle = tr(BA^*)= tr(A\overline{B}^T).$ See that $A\overline{B}^T (c_{ij})$ is such that 
	$c_{ij}= \sum_{i=1}^n a_{i1}\overline{b_{j1}}.$ See that $\overline{c_{ij}}=\sum_{i=1}^n \overline{a_{i1} } b_{j1},$ gives us $ \sum_{1 \leq i,j \leq n} 
	a_{ij}\overline{b_{ij}}.$ Note that replacing $A$ and $B$ just gives us the conjugate, which is the desired result, that $$\langle B,A \rangle = 
	\overline{\langle A,B  \rangle}. $$
	\item We have $\langle A+B,C\rangle= tr((A+B)C^*).$ We know that $$tr((A+B)C^*)= \sum_{1 \leq i,j \leq n} (a_{ij}+b_{ij}) \overline{c_{ij}}=\sum_{1 \leq 
	i,j \leq n} a_{ij}\overline{c_{ij}} +\sum_{1 \leq i,j \leq n} b_{ij} \overline{c_{ij}}= tr(AC^*) + tr(BC^*).$$ 
\end{enumerate}
Therefore we have defined an inner product. 
To solve the second part, see that since we can apply the Cauchy Schwarz inequality on inner product spaces, we have $$|\langle A,B\rangle|^2 \leq ||A||^2 
\cdot ||B||^2,$$ which gives us the required answer.
\section{} %Problem 2 
\section{} %Problem 3 
We assume that there is $y \in Y$ such that $||x-y|| = d(x,Y).$ Then we have $x-y \perp Y.$ Thus $\Re \langle x-y,y \rangle =0.$ This implies the other side 
trivially. 

For the converse, we assume that $$ \Re \langle x-y,z \rangle \leq  \Re \langle x-y,y \rangle. $$


\section{} %Problem 4 
\section{} %Problem 5 
\section{} %Problem 6 
We want to construct an isometric isomorphism between $H,$ a separable Hilbert space and $\ell^2,$ the sequence of square summable sequences over a linear 
field. We have $H$ is separable, hence there exists a countable dense subset. This, in fact gives us an orthonormal Schauder basis $\{b_n\}_{n \in 
\mathbb{N}}$. Let the standard orthonormal basis for $\ell^2$ be given by $\{e_n\}_{n \in \mathbb{N}}.$ Define $T:H \to \ell^2$ be such that 
$$T(\sum_{n=1}^{\infty}a_n b_n) = \sum_{n=1}^{\infty}a_n e_n.$$

For $\mathbf{a}= \{k_n\}, \mathbf{b}= \{l_n\} \in H,$ we have 
$$	\langle T\mathbf{a}, T \mathbf{b} \rangle =  \langle \sum_{n=1}^{\infty} k_ne_n, \mathbf{b} \rangle = \sum_{n=1}^{\infty} k_n \langle e_n, \mathbf{b} 
\rangle.$$ We can see that $$\langle e_n, \sum_{m=1}^{\infty} l_n e_n \rangle = \sum_{m=1}^{\infty} \overline{l_m} \langle e_n,e_m \rangle = 
\overline{l}_n.$$  Thus we have $\langle T\mathbf{a}, T \mathbf{b} \rangle = \sum_{n=1}^{\infty} k_n \overline{l}_n= \langle \mathbf{a}, \mathbf{b} 
\rangle.$ Thus our map is an isometry. It is clearly one-one. It is also onto, as the pre-image of any $\sum_{n=1}^{\infty} c_n e_n$ is $\sum_{n=1}^{\infty} 
c_n b_n.$ Therefore we have an isomorphism of Hilbert spaces. 
\section{} %Problem 7 
If $V$ is a finite-dimensional vector space, then any total orthonormal set must be finite as there can be at most some finite number of linearly 
independent elements. Since a total orthonormal set must span the entire space, we have a Hamel basis since any element can be written as a finite linear 
combination of elements from the total orthonormal set.

Conversely, let $V$ be a vector space such that every total orthonormal set is a Hamel basis. 
\section{} %Problem 8 
\section{} %Problem 9

We are given $T: \mathbb{K}^n \to \mathbb{K}^m,$ where $$(Tx)(i)=\sum_{j=1}^{n}k_{ij}x_j,$$
where $i=1,2,\dots,m.$ Let $a_i$ denote the $i$th row of $T.$ Then we have $\langle Tx,y \rangle= \sum_{j=1}^m (Tx)(i)y_j.$
Expanding the entire thing, we have $$ \langle Tx,y\rangle= \sum_{1 \leq i \leq m, 1 \leq j \leq n} k_{ij}x_j \bar{y_i}.$$ We can write this as 
$$\sum_{j=1}^n x_j \overline{\overline{k_{1i}}y_1 + \dots + \overline{k_{mi}}y_m} = \langle x, \overline{T}^{T}y \rangle!$$ Therefore from uniqueness of 
adjoint we must have $T^{*}= \overline{T}^{T}.$
\section{} %Problem 10 
See that for any operator we have $$|\langle Tx,x\rangle| \leq ||Tx|| \cdot ||x|| \leq ||T||,$$
taking $||x|| =1.$ Since the left of the inequality depends on $x$ while the right is independent, we have $\sup_{||x||=1}\langle Tx,x\rangle \leq ||T||.$

\end{document}
