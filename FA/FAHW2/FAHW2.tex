%%%%%%%%%%%%%%%%%%%%%%%%%%%%%%%%%%%%%%%%%
% Lachaise Assignment
% LaTeX Template
% Version 1.0 (26/6/2018)
%
% This template originates from:
% http://www.LaTeXTemplates.com
%
% Authors:
% Marion Lachaise & François Févotte
% Vel (vel@LaTeXTemplates.com)
%
% License:
% CC BY-NC-SA 3.0 (http://creativecommons.org/licenses/by-nc-sa/3.0/)
% 
%%%%%%%%%%%%%%%%%%%%%%%%%%%%%%%%%%%%%%%%%

%----------------------------------------------------------------------------------------
%	PACKAGES AND OTHER DOCUMENT CONFIGURATIONS
%----------------------------------------------------------------------------------------

\documentclass{article}

\input{structure.tex} % Include the file specifying the document structure and custom commands

%----------------------------------------------------------------------------------------
%	ASSIGNMENT INFORMATION
%----------------------------------------------------------------------------------------

\title{Functional Analysis Homework 2} % Title of the assignment

\author{Gandhar Kulkarni (mmat2304)} % Author name and email address

\date{} % University, school and/or department name(s) and a date

%----------------------------------------------------------------------------------------

\begin{document}

\maketitle % Print the title

%----------------------------------------------------------------------------------------
%	INTRODUCTION
%----------------------------------------------------------------------------------------

\section{} %Problem 1 
\section{} %Problem 2 
\section{} %Problem 3 
\section{} %Problem 4 
\section{} %Problem 5 
\section{} %Problem 6 
\section{} %Problem 7 
\section{} %Problem 8 
\section{} %Problem 9
We are given $T: \mathbb{K}^n \to \mathbb{K}^m,$ where $$(Tx)(i)=\sum_{j=1}^{n}k_{ij}x_j,$$
where $i=1,2,\dots,m.$ Let $a_i$ denote the $i$th row of $T.$ Then we have $\langle Tx,y \rangle= \sum_{j=1}^m (Tx)(i)y_j.$
Expanding the entire thing, we have $$ \langle Tx,y\rangle= \sum_{1 \leq i \leq m, 1 \leq j \leq n} k_{ij}x_j \bar{y_i}.$$ We can write this as 
$$\sum_{j=1}^n x_j \overline{\overline{k_{1i}}y_1 + \dots + \overline{k_{mi}}y_m} = \langle x, \overline{T}^{T}y \rangle!$$ Therefore from uniqueness of 
adjoint we must have $T^{*}= \overline{T}^{T}.$
\section{} %Problem 10 

\end{document}
