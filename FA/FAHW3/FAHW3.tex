\documentclass[letterpaper,11pt,twoside]{article}
\usepackage[utf8]{inputenc}
\usepackage{enumitem}
\setlist{nosep}
\usepackage{graphicx}
\usepackage{amsmath,amssymb,amsfonts,amsthm}
\usepackage{tikz-cd}
\usepackage[margin=0.9in,
left=1.25in,%
right=1.25in,%
top=1.25in,%
bottom=1.25in
]{geometry}	
%\usepackage{stmaryrd} %For mapsfrom

%\usepackage{quiver}
\usepackage{bm}
\usepackage{fancyhdr}
\usepackage{mathrsfs}
\usepackage{amsbsy}
\usepackage{titlesec}
%\usepackage{yhmath}
%\usepackage{mathabx,epsfig}


%Hyperref Settings------
\usepackage{hyperref}
\usepackage{xcolor}
\hypersetup{
	colorlinks,
	linkcolor={black},
	citecolor={red!50!black}
	urlcolor={green!80!black}
}

%%%%%% TITLE %%%%%

\title{Algebra 2 Homework 5}
%\date{\today}


%MATH BACKGROUND DECLARATORS-------------------------------------------------------
\theoremstyle{proposition}
\newtheorem{proposition}{Proposition}[section]

\theoremstyle{definition}
\newtheorem{definition}{Definition}[section]

\theoremstyle{theorem}
\newtheorem{theorem}{Theorem}[section]

\theoremstyle{definition}
\newtheorem{remark}{\textbf{Remark}}[section]

\theoremstyle{definition}
\newtheorem{notation}{\textbf{Notation}}[section]

\theoremstyle{definition}
\newtheorem{discussion}{\textbf{Discussion}}[section]

\theoremstyle{lemma}
\newtheorem{lemma}{\textbf{Lemma}}[section]

\theoremstyle{definition}
\newtheorem{example}{\textbf{Example}}[section]

%\theoremstyle{remark}
%\newtheorem*{comment}{\textbf{Comments on Proof Technique}}

\theoremstyle{definition}
\newtheorem{construct}{Construction}[section]

\theoremstyle{corollary}
\newtheorem{corollary}{Corollary}[section]

\theoremstyle{definition}
\newtheorem{caution}{\textbf{Caution}}[section]


\theoremstyle{definition}
\newtheorem{question}{\textbf{Question}}[section]

\theoremstyle{definition}
\newtheorem{para}{}[section]


%--------------------------------------------------------------------------------- SOME USEFUL MACROS.


\newcommand{\N}{\mathbb{N}}
\newcommand{\Z}{\mathbb{Z}}
\newcommand{\C}{\mathbb{C}}
\newcommand{\R}{\mathbb{R}}
\DeclareMathOperator{\GL}{\text{\rm GL}}
\newcommand{\Ker}[1]{{\fontfamily{lmss}\selectfont 
		\text{\rm Ker}\left (#1\right )
}}
\newcommand{\nsg}{\trianglelefteq}
\newcommand{\abs}[1]{\left \vert #1 \right \vert}
\newcommand{\gen}[1]{\left\langle #1\right\rangle}
\newcommand{\norm}[1]{\left \vert \left \vert #1 \right \vert \right \vert}
\renewcommand{\div}{\;\vert\;}
\newcommand{\isom}{\cong}
\DeclareMathOperator{\Stab}{\text{\rm Stab}}
\newcommand{\Image}[1]{{\fontfamily{lmss}\selectfont 
		\text{\rm Im}\left (#1\right )
}}
\DeclareMathOperator{\Bij}{\text{\rm Bij}}
\DeclareMathOperator{\acts}{\rotatebox[origin=c]{-90}{$\circlearrowright$}}
\DeclareMathOperator{\Orb}{\text{\rm Orb}}
\DeclareMathOperator{\lcm}{\text{\rm lcm}}
\newcommand{\floor}[1]{\left \lfloor #1 \right \rfloor}
\DeclareMathOperator{\Aut}{\text{\rm Aut}}
\DeclareMathOperator{\Inn}{\text{\rm Inn}}
\DeclareMathOperator{\id}{\text{\rm id}}
\newcommand{\F}{\mathbb{F}}




\begin{document}
	\maketitle
	%\tableofcontents
	\begin{proof}[Solution of problem $1$:]
		\begin{itemize}
			\item BIT $\implies$ OMT: We first need to show that $\pi: X \to X/M$ is an open map, where $M$ is a closed subspace. Let $U$ be open in $X.$ 
			Then we want to understand $q(U).$ Let $x \in U.$ Then there must be $r >0$ such that $B(x,r) \subsetneq U.$ Let $x'+M$ be such that 
			$\norm{(x+M)-(x'+M)} <r,$ that is, $\inf_{m \in M} \norm{(x-x')+m} <r.$ We must obtain some $m \in M,$ where $ \norm{(x-x')+m} <r,$ since the 
			inequality is strict. But now $x'-m \in B(x,r) \subset U,$ and so $q(x'-m) \in q(U),$ and note that $q(x'-m)=x'+M,$ which lies in $B(x'+M,r),$ a 
			ball in $q(U).$ Thus we have an open map.
			
			Now we assume that bounded inverse theorem. Let $T:X \to Y$ be a bounded surjective map. Then let $M=\ker T.$ Then we have the map $\bar{T}: X/M 
			\to Y$ where $x+M \mapsto T(x).$  This map is clearly well-defined since if $x+M=x'+M$ then $x-x' \in M.$ Then $T(x)=T(x').$ This is clearly a 
			bijection by the first isomorphism theorem.  Now by the Bounded Inverse Theorem, we have that $S= \bar{T}^{-1}: Y \to X/M$ exists and is a 
			bounded and linear map. Hence it is also continuous. 
			\item 
		\end{itemize}
	\end{proof}
\begin{proof}[Solution of problem $2$:]
	Let $T: C^1[-1,1] \to \mathbb{R},$ where $T(f)=f'.$ This is clearly linear. Note that $\norm{f}:= \sup_{x \in [-1,1]} |f(x)|.$ Then see that $\norm{T}= 
	\sup_{\norm{f}=1}|Tf|,$ which is unbounded. Then note for $(u_n),$ a sequence of differentiable functions that converge uniformly to $u,$ and 
	$Tu_n=u_n'$ converges to $f,$ then we have $Tu=u'=f.$ Thus this a discontinuous linear operator than has closed graph.   
\end{proof}
\begin{proof}[Solution of problem $3$:]
	
\end{proof}
\begin{proof}[Solution of problem $4$:]
 \begin{enumerate}
 	\item Let $X$ be a Banach space, and $p: X \to [0,\infty)$ is a semi-norm (a norm, but without the rule that $p(x)=0 \implies x=0$). If we take any 
 	absolutely convergent series $\sum_{n=1}^{\infty}x_n \in X,$ we have $$p\left( \sum_{n=1}^{\infty}x_n  \right) \leq \sum_{n=1}^{\infty}p(x_n)  \in [0, 
 	\infty],$$ then $p$ is continuous. To prove this, we let  $A_n= p^{-1}([0,n])$ and $F_n= \overline{A_n}.$ See that $A_n$ and $F_n$ are symmetric convex 
 	sets since $p$ is a seminorm. We have $X = \cup_{n=1}^{\infty} F_n,$ and by Baire's theorem there must be some $N$ such that $F_N$ has non-empty 
 	interior. Therefore, there exist $x_0 \in X,$ and $R>0$ such that $B_R(x_0) \subset F_n.$ By symmetry of $F_N, B_R(-x_0)= -B_R(x_0) \subset F_N.$ If 
 	$\norm{x} < R,$ then $x+x_0 \in B_R(x_0), x-x_0 \in B_R(-x_0),$ so we have $x \pm x_0 \in F_N.$ Since $F_N$ is convex, we have $\frac{1}{2}(x_0 + 
 	(-x_0)) =0 \in F_N.$ Then we have $B_R(0) \subset F_N.$ We want to show that $B_R(0) \subset A_N.$ Suppose $\norm{x} < r < R.$ Fix $0 < q < 1- 
 	\frac{r}{R},$ so that $ \frac{1}{1-q} \cdot \frac{r}{R} < 1.$ Then $y = \frac{R}{r}x \in B_R(0) \subset F_N = \overline{A_N}.$ Thus there is $y_0 \in 
 	A_N$ such that $\norm{y-y_0} < qR,$ so $q^{-1}(y-y_0) \in B_R.$ Choose a  $y_1 \in A_N$ such that $\norm{q^{-1}(y-y_0)-y_1} < qR,$ so 
 	$\norm{y-y_0-qy_1} < q^2R.$ By induction we have $(y_n)$ such that $$ \norm{y - \sum_{k=0}^{n}q^ky_k } < q^nR,$$ for all $n \geq 0,$  thus we have 
 	$y=\sum_{k=0}^{\infty} q^ky_k.$ We see that $\norm{y_k} \leq R+ qR$ for all $k,$ so $y$ as a series exists since the constructed series is absolutely 
 	convergent. Now, using the subadditivity that was given in the hypothesis, we have 
 	
 	$$p(y) = p \left( \sum_{k=0}^{\infty} q^ky_k \right) \leq \sum_{k=0}^{\infty} q^kp(y_k) \leq \frac{1}{1-q}N, $$
 	and hence $p(x) \leq \frac{N(1+ \varepsilon)}{R}\norm{x},$ which proves the continuity. 
 	
 	\item \begin{enumerate}
 		\item 	$T: \to Y$ is a bounded linear operator. Then suppose the $T(U),$ where $U$ is the open unit ball, is open. In that case, let $V$ be some 
 		open 
 		neighbourhood of $X.$ Then for $x \in V,$ we have that some ball of radius $r $ centered at $x$ is in $V.$ We can then see that $T(rU+x) \subset V,$ 
 		which means that we only need to see that $T(U)$ is open. 
 		
 		Define $p(y) := \inf \{ \norm{x} \div Tx=y \}.$ We need to show that this is a seminorm with countable subadditivity. Let $\alpha \neq 0$ be a 
 		scalar. 
 		Then we have $\{x \div x \in X, Tx=\alpha y\}= \{\alpha x \div x \in X, Tx=y\},$ and taking infimums, we have $p(\alpha y) = \abs{\alpha} p(y).$ For 
 		$\alpha =0,$ this can be easily checked. Let $\sum_n y_n$ be a convergent series. We need to show that $p\left( \sum_n y_n \right) \leq \sum_n 
 		p(y_n),$ 
 		so we assume $\sum_n p(y_n)$ is finite, since if it was infinite there would be nothing to prove. Fixing some $\varepsilon >0,$ we take a sequence 
 		$(x_n)$ in $X$ such that $Tx_n=y_n,$ and $ \norm{x_n} < p(y_n) + 2^{-n} \varepsilon.$ Then we have $\sum_n \norm{x_n} < \sum_n p(y_n) + 
 		\varepsilon,$ 
 		which is finite. Since in Banach spaces absolutely convergent series are also convergent, we have $\sum_n x_n$ converges. Then $T(\sum_n x_n)= 
 		\sum_n Tx_n= \sum_n y_n,$ so $$ p \left( \sum_n y_n \right) \leq  \norm{\sum_n x_n}  \leq \sum_n \norm{x_n} < \sum_n p(x_n) + \varepsilon.$$ 
 		Therefore subadditivity is confirmed, so by Zabreiko's lemma, we have $$T(U)= \{ y: y \in Y, Tx=y \text{ for some } x \in U \}= \{ y: y \in Y, 
 		p(y)=1 \},$$ which is open. This proves the open mapping theorem.
 		\item 	If we have a one-one onto linear mapping from a topological space to another is a homeomorphism if and only if it is continuous and open. 
 		Using the open mapping theorem, we have that this map is open, and continuous. Thus $T^{-1}$ exists and must be bounded, as it is a homeomorphism 
 		too. This proves the bounded inverse theorem.
 		
 		\item 	Let $\mathcal{F}$ be a non-empty family of bounded linear operators from a Banach space $X$ to a normed space $Y,$ where $\sup\{ \norm{Tx} 
 		\div T \in \mathcal{F} \}.$ Now, let  $p(x):= \sup \{\norm{Tx}\div T \in \mathcal{F}\}.$ See that $p(\alpha x) = \abs{\alpha} p(x)$ from definition. 
 		For $\sum_n x_n$ a convergent series, we have $$ \norm{ T \left( \sum_n x_n \right)} = \norm{ \sum_n x_n } \leq \sum_n \norm{Tx_n} \leq \sum_n 
 		p(x_n),$$ which implies that $p \left( \sum_n x_n\right) \leq \sum_n p(x_n).$ In particular, we have $p(x_1+x_2) \leq p(x_1) + p(x_2).$ Now, since 
 		$p$ is continuous, we have $\delta > 0$ such that $p(x) \leq 1$  for $\norm{x} \leq \delta.$ Whenever $x \in X, \norm{x}=1$ we have $p(x) \leq 
 		\delta^{-1}$,  which means that $\norm{T} \leq \delta^{-1}$ for each $T \in \mathcal{F}.$ 
 		
 		\item Let, $T:X \to Y.$ Now pick $p(x) = \norm{Tx}.$ If $p$ was continuous, then there would be a neighbourhood $U$ of $0$ such that the set $p(U)$ 
 		is bounded, which implies that $T(U)$ is bounded, and this implies continuity of $T.$ $p$ is a semi-norm, so we need to check its continuity. We 
 		only need to check that this has countable subadditivity. Take $\sum_n x_n,$ a convergent series, then we can assume that $ \sum_n \norm{Tx_n}$ is 
 		finite without loss of generality. Now see that if $\sum_n \norm{Tx_n}$ is convergent, then so is $\sum_n Tx_n$ is convergent in $Y,$ as it is 
 		complete. Since $\sum_{k=1}^{n}x_k \to \sum_n x_n,$ then we have $T \left( \sum_{k=1}^{n}x_k \right) \to T \left( \sum_n x_n \right).$ Then, from 
 		hypothesis we have $ T\left( \sum_n x_n \right)= \sum_n Tx_n.$ Taking norm, we have the norm subadditivity. Since $p$ is continuous by Zabreiko's 
 		lemma, we have proven the closed graph theorem. 
 	\end{enumerate}
 \end{enumerate}
\end{proof}
\begin{proof}[Solution of problem $5$:]
	We know that $x_n \xrightarrow{w} x,$ so for any $f \in X^*,$ we have $f(x_n) \to f(x).$ 
	Using the Hahn-Banach theorem, we can find a linear functional $f$ where $\norm{f}=1,$ and $f(x)= \norm{x}.$ 
	Then we have $$\norm{x}= \lim_{n \to \infty} \abs{f(x_n)}  \leq \liminf_{n \to \infty} \norm{f}\norm{x_n} = \liminf_{n \to \infty}\norm{x_n}, $$
	as desired.
\end{proof}
\begin{proof}[Solution of problem $6$:]
	$X$ is a normed linear space. We say that $(x_n),$ a sequence in $X$ is \textit{weakly Cauchy} if the sequence $(fx_n)$ converges for all $f \in X^*.$ 
	$X$ is \textit{weakly complete} if all weakly Cauchy sequences converge weakly. 
	
	Let $X$ be reflexive. Let $(x_n)$ be a weakly Cauchy sequence in $X.$ Pick $f \in X^*.$ Then since $ (f(x_n))$ is a Cauchy sequence in $\mathbb{C},$ we 
	have that $f(x_n) \to \alpha(f),$ where $\alpha \in X^{**}.$ We do not know what element in $X$ this element corresponds to, but we know that since $X$ 
	is reflexive we can think of it as an element of the bidual acting on $f.$ For any $f \in X^*$ we have $E_{x_n}(f) = f(x_n) \to \alpha(f).$ We define 
	$\alpha$ as the element of $X^{**},$ as the limit of $(f(x_n))$ as $n \to \infty.$ Now see that $\abs{\alpha(f)}$ is bounded as for each $f \in X^*,$ we 
	have that pointwise the set $(f(x_n))$ is bounded for each $f \in X^{*}.$ Then by the Uniform Bounded Principle, we have $(x_n)$ must be bounded in 
	$X^{**}.$ Since $\norm{x_n}_{X^{**}}= \norm{x_n},$ we know that $(x_n)$ is bounded in $X$ by $M>0.$ 
	Then, $$\abs{f(x_n)} \leq M \norm{f} \implies \alpha(f) \leq M \norm{f}.$$
	
	Since $X$ is reflexive, $\alpha \in X.$ Then by definition for each $f \in X^{*}$ we have $f(x_n) \to \alpha(f)=f(\alpha),$ which confirms weak 
	converges. 
\end{proof}

\end{document}