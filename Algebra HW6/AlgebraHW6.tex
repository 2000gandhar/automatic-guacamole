%%%%%%%%%%%%%%%%%%%%%%%%%%%%%%%%%%%%%%%%%
% Lachaise Assignment
% LaTeX Template
% Version 1.0 (26/6/2018)
%
% This template originates from:
% http://www.LaTeXTemplates.com
%
% Authors:
% Marion Lachaise & François Févotte
% Vel (vel@LaTeXTemplates.com)
%
% License:
% CC BY-NC-SA 3.0 (http://creativecommons.org/licenses/by-nc-sa/3.0/)
% 
%%%%%%%%%%%%%%%%%%%%%%%%%%%%%%%%%%%%%%%%%

%----------------------------------------------------------------------------------------
%	PACKAGES AND OTHER DOCUMENT CONFIGURATIONS
%----------------------------------------------------------------------------------------

\documentclass{article}

\input{structure.tex} % Include the file specifying the document structure and custom commands

%----------------------------------------------------------------------------------------
%	ASSIGNMENT INFORMATION
%----------------------------------------------------------------------------------------

\title{The last Home-work} % Title of the assignment

\author{Gandhar Kulkarni (mmat2304)} % Author name and email address

\date{} % University, school and/or department name(s) and a date

%----------------------------------------------------------------------------------------

\begin{document}

\maketitle % Print the title

%----------------------------------------------------------------------------------------
%	INTRODUCTION
%----------------------------------------------------------------------------------------

\section{} %Problem 1 
\begin{enumerate}
	\item Since $R$ is a PID, we know that $(a,b)=(d),$ for some $d,a,b \in R.$ Then we have $d=am+bn,$ for some $m,n \in R.$ Now we have a vector 
	$v=[a,b]^T \in R^2 \backslash \{0\}.$ Then we show that there exists a $2x2$ matrix that does what we want by constructing one. Let the desired matrix 
	by be given by $X= \begin{pmatrix}
	x_{11}&x_{21}\\
	x_{12}&x_{22} 	
	\end{pmatrix}.$ Now we have $Xv= [x_{11}a + x_{21}b, x_{12}a+x_{22}b]^T=[d,0]^T.$ Comparing terms, we have $x_{12}a+x_{22}b=0.$ Then we have 
	$x_{12}a=-x_{22}b,$ which implies that $ x_{12}| -b,$ and $x_{22}|a.$ It is easy to see that $x_{12}=-a$ and $x_{22}=b$ does the trick. For $x_{11}a + 
	x_{21}b=d,$ see that $x_{11}=m$ and $x_{21}=n$ are good choices, since their linear combination produces $d.$ Thus see that $$X=\begin{pmatrix}
	m&n\\
	-b&a
\end{pmatrix} $$ is a matrix that achieves the intended result.
\end{enumerate}
 
\section{} %Problem 2
\section{} %Problem 3
We have $M$ a $R-$module which has itself as a generating set. Then $\pi: R^{\oplus M}\twoheadrightarrow M$ is the surjective map sending $e_m$ to $m.$ We 
see that $\pi(e_{rm}-re_m)=\pi(e_{rm})-r\pi(e_m)=rm-rm=0.$ Also, $\pi(e_{m_1+m_2}-e_{m_1}-e_{m_2})=\pi(e_{m_1+m_2})-\pi(e_{m_1})-\pi(e_{m_2})= 
(m_1+m_2)-m_1-m_2=0.$ Therefore we have $ e_{rm}-re_m \in \ker \pi, $ and $e_{m_1+m_2}-e_{m_1}-e_{m_2} \in \ker \pi.$ Thus we have 
$$(e_{rm}-re_m,e_{m_1+m_2}-e_{m_1}-e_{m_2}) \subseteq \ker \pi.$$ Let us have $\sum_{m \in M}r_m e_m \in R^{\oplus M}.$ Note that there are only finitely 
many 
terms in the summation. See that $$\pi(\sum_{m \in M}r_m e_m)=\sum_{m \in M}r_m \pi(e_m)= \sum_{m \in M}r_m m = \pi(\sum_{m \in M}e_{r_mm}).$$ This means 
that 
$\pi(\sum_{m \in M}r_m e_m - \sum_{m \in M}e_{r_mm})=0.$ This, in turn implies that $ \sum_{m \in M} r_me_m - e_{r_mm} \in \ker \pi.$ We also see that 
$$\pi(\sum_{m \in M}r_m e_m)=\sum_{m \in M}r_m \pi(e_m)= \sum_{m \in M}r_m m = \pi(e_{\sum_{m \in M}r_mm}).$$ This means that $\sum_{m \in M}r_m 
e_m-e_{\sum_{m \in M}r_mm} \in \ker \pi.$ Given an element in $R^{\oplus M},$ we can choose which summands to clump and which to leave unchanged. Either 
ways, we see that we get a linear combination of $re_m-e_{rm}$ and $e_{m_1+m_2}-e_{m_1}-e_{m_2},$ which implies that $ \ker \pi \subseteq 
(e_{rm}-re_m,e_{m_1+m_2}-e_{m_1}-e_{m_2}).$ This gives us the desired equality.
\section{} %Problem 4 
\begin{enumerate}
	\item We take the module $\mathbb{Z}[q_1,q_2],$ where $q_1,q_2 \in \mathbb{Q}.$ 
\end{enumerate}
\section{} %Problem 5 
\section{} %Problem 6 

\end{document}

















