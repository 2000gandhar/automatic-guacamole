%%%%%%%%%%%%%%%%%%%%%%%%%%%%%%%%%%%%%%%%%
% Lachaise Assignment
% LaTeX Template
% Version 1.0 (26/6/2018)
%
% This template originates from:
% http://www.LaTeXTemplates.com
%
% Authors:
% Marion Lachaise & François Févotte
% Vel (vel@LaTeXTemplates.com)
%
% License:
% CC BY-NC-SA 3.0 (http://creativecommons.org/licenses/by-nc-sa/3.0/)
% 
%%%%%%%%%%%%%%%%%%%%%%%%%%%%%%%%%%%%%%%%%

%----------------------------------------------------------------------------------------
%	PACKAGES AND OTHER DOCUMENT CONFIGURATIONS
%----------------------------------------------------------------------------------------

\documentclass{article}

%%%%%%%%%%%%%%%%%%%%%%%%%%%%%%%%%%%%%%%%%
% Lachaise Assignment
% Structure Specification File
% Version 1.0 (26/6/2018)
%
% This template originates from:
% http://www.LaTeXTemplates.com
%
% Authors:
% Marion Lachaise & François Févotte
% Vel (vel@LaTeXTemplates.com)
%
% License:
% CC BY-NC-SA 3.0 (http://creativecommons.org/licenses/by-nc-sa/3.0/)
% 
%%%%%%%%%%%%%%%%%%%%%%%%%%%%%%%%%%%%%%%%%

%----------------------------------------------------------------------------------------
%	PACKAGES AND OTHER DOCUMENT CONFIGURATIONS
%----------------------------------------------------------------------------------------

\usepackage{amsmath,amsfonts,amssymb, tikz-cd} % Math packages

\usepackage{enumerate} % Custom item numbers for enumerations


\usepackage[framemethod=tikz]{mdframed} % Allows defining custom boxed/framed environments

\usepackage{listings} % File listings, with syntax highlighting
\lstset{
	basicstyle=\ttfamily, % Typeset listings in monospace font
}

%----------------------------------------------------------------------------------------
%	DOCUMENT MARGINS
%----------------------------------------------------------------------------------------

\usepackage{geometry} % Required for adjusting page dimensions and margins

\geometry{
	paper=letterpaper, % Paper size, change to letterpaper for US letter size
	top=2.5cm, % Top margin
	bottom=3cm, % Bottom margin
	left=2.5cm, % Left margin
	right=2.5cm, % Right margin
	headheight=14pt, % Header height
	footskip=1.5cm, % Space from the bottom margin to the baseline of the footer
	headsep=1.2cm, % Space from the top margin to the baseline of the header
	%showframe, % Uncomment to show how the type block is set on the page
}

%----------------------------------------------------------------------------------------
%	FONTS
%----------------------------------------------------------------------------------------

\usepackage[utf8]{inputenc} % Required for inputting international characters
\usepackage[T1]{fontenc} % Output font encoding for international characters


%----------------------------------------------------------------------------------------
%	COMMAND LINE ENVIRONMENT
%----------------------------------------------------------------------------------------

% Usage:
% \begin{commandline}
	%	\begin{verbatim}
		%		$ ls
		%		
		%		Applications	Desktop	...
		%	\end{verbatim}
	% \end{commandline}

\mdfdefinestyle{commandline}{
	leftmargin=10pt,
	rightmargin=10pt,
	innerleftmargin=15pt,
	middlelinecolor=black!50!white,
	middlelinewidth=2pt,
	frametitlerule=false,
	backgroundcolor=black!5!white,
	frametitle={Command Line},
	frametitlefont={\normalfont\sffamily\color{white}\hspace{-1em}},
	frametitlebackgroundcolor=black!50!white,
	nobreak,
}

% Define a custom environment for command-line snapshots
\newenvironment{commandline}{
	\medskip
	\begin{mdframed}[style=commandline]
	}{
	\end{mdframed}
	\medskip
}

%----------------------------------------------------------------------------------------
%	FILE CONTENTS ENVIRONMENT
%----------------------------------------------------------------------------------------

% Usage:
% \begin{file}[optional filename, defaults to "File"]
	%	File contents, for example, with a listings environment
	% \end{file}

\mdfdefinestyle{file}{
	innertopmargin=1.6\baselineskip,
	innerbottommargin=0.8\baselineskip,
	topline=false, bottomline=false,
	leftline=false, rightline=false,
	leftmargin=2cm,
	rightmargin=2cm,
	singleextra={%
		\draw[fill=black!10!white](P)++(0,-1.2em)rectangle(P-|O);
		\node[anchor=north west]
		at(P-|O){\ttfamily\mdfilename};
		%
		\def\l{3em}
		\draw(O-|P)++(-\l,0)--++(\l,\l)--(P)--(P-|O)--(O)--cycle;
		\draw(O-|P)++(-\l,0)--++(0,\l)--++(\l,0);
	},
	nobreak,
}

% Define a custom environment for file contents
\newenvironment{file}[1][File]{ % Set the default filename to "File"
	\medskip
	\newcommand{\mdfilename}{#1}
	\begin{mdframed}[style=file]
	}{
	\end{mdframed}
	\medskip
}

%----------------------------------------------------------------------------------------
%	NUMBERED QUESTIONS ENVIRONMENT
%----------------------------------------------------------------------------------------

% Usage:
% \begin{question}[optional title]
	%	Question contents
	% \end{question}

\mdfdefinestyle{question}{
	innertopmargin=1.2\baselineskip,
	innerbottommargin=0.8\baselineskip,
	roundcorner=5pt,
	nobreak,
	singleextra={%
		\draw(P-|O)node[xshift=1em,anchor=west,fill=white,draw,rounded corners=5pt]{%
			Question \theQuestion\questionTitle};
	},
}

\newcounter{Question} % Stores the current question number that gets iterated with each new question

% Define a custom environment for numbered questions
\newenvironment{question}[1][\unskip]{
	\bigskip
	\stepcounter{Question}
	\newcommand{\questionTitle}{~#1}
	\begin{mdframed}[style=question]
	}{
	\end{mdframed}
	\medskip
}

%----------------------------------------------------------------------------------------
%	WARNING TEXT ENVIRONMENT
%----------------------------------------------------------------------------------------

% Usage:
% \begin{warn}[optional title, defaults to "Warning:"]
	%	Contents
	% \end{warn}

\mdfdefinestyle{warning}{
	topline=false, bottomline=false,
	leftline=false, rightline=false,
	nobreak,
	singleextra={%
		\draw(P-|O)++(-0.5em,0)node(tmp1){};
		\draw(P-|O)++(0.5em,0)node(tmp2){};
		\fill[black,rotate around={45:(P-|O)}](tmp1)rectangle(tmp2);
		\node at(P-|O){\color{white}\scriptsize\bf !};
		\draw[very thick](P-|O)++(0,-1em)--(O);%--(O-|P);
	}
}

% Define a custom environment for warning text
\newenvironment{warn}[1][Warning:]{ % Set the default warning to "Warning:"
	\medskip
	\begin{mdframed}[style=warning]
		\noindent{\textbf{#1}}
	}{
	\end{mdframed}
}

%----------------------------------------------------------------------------------------
%	INFORMATION ENVIRONMENT
%----------------------------------------------------------------------------------------

% Usage:
% \begin{info}[optional title, defaults to "Info:"]
	% 	contents
	% 	\end{info}

\mdfdefinestyle{info}{%
	topline=false, bottomline=false,
	leftline=false, rightline=false,
	nobreak,
	singleextra={%
		\fill[black](P-|O)circle[radius=0.4em];
		\node at(P-|O){\color{white}\scriptsize\bf i};
		\draw[very thick](P-|O)++(0,-0.8em)--(O);%--(O-|P);
	}
}

% Define a custom environment for information
\newenvironment{info}[1][Info:]{ % Set the default title to "Info:"
	\medskip
	\begin{mdframed}[style=info]
		\noindent{\textbf{#1}}
	}{
	\end{mdframed}
}
 % Include the file specifying the document structure and custom commands

%----------------------------------------------------------------------------------------
%	ASSIGNMENT INFORMATION
%----------------------------------------------------------------------------------------

\title{The last Home-work} % Title of the assignment

\author{Gandhar Kulkarni (mmat2304)} % Author name and email address

\date{} % University, school and/or department name(s) and a date

%----------------------------------------------------------------------------------------

\begin{document}

\maketitle % Print the title

%----------------------------------------------------------------------------------------
%	INTRODUCTION
%----------------------------------------------------------------------------------------

\section{} %Problem 1 
\begin{enumerate}
	\item Since $R$ is a PID, we know that $(a,b)=(d),$ for some $d,a,b \in R.$ Then we have $d=am+bn,$ for some $m,n \in R.$ Now we have a vector 
	$v=[a,b]^T \in R^2 \backslash \{0\}.$ Then we show that there exists a $2x2$ matrix that does what we want by constructing one. Let the desired matrix 
	by be given by $X= \begin{pmatrix}
		x_{11}&x_{21}\\
		x_{12}&x_{22} 	
	\end{pmatrix}.$ Now we have $Xv= [x_{11}a + x_{21}b, x_{12}a+x_{22}b]^T=[d,0]^T.$ Comparing terms, we have $x_{12}a+x_{22}b=0.$ Then we have 
	$x_{12}a=-x_{22}b,$ which implies that $ x_{12}| -b,$ and $x_{22}|a.$ It is easy to see that $x_{12}=-a/d$ and $x_{22}=b/d$ does the trick. For $x_{11}a 
	+ 
	x_{21}b=d,$ see that $x_{11}=m$ and $x_{21}=n$ are good choices, since their linear combination produces $d.$ Thus see that 
		$$X=\begin{pmatrix}
			m&n\\
			-b/d&a/d
		\end{pmatrix} $$ is a matrix that achieves the intended result.
	
	\item The above result shall be of much use to us. We see that we want to send $a_{11}$ to $d,$ that is the gcd of $a_{11}$ and $a_{i1},$ and $a_{i1}$ 
	to $0.$ Let $m,n \in R$ such that $d=a_{11}m+a_{i1}n.$ Define the matrix $\tilde{X}=(x_{kl})$ thus--- $x_{11}=m, x_{1i}=n, x_{i1}=-a_{i1}/d, 
	x_{ii}=a_{11}/d.$ Also we have $x_{kk}=1$ if $k \neq 1,i.$ All other elements are $0.$ Then we have $A'=\tilde{X}A=(a'_{kl}),$ where 
	$a'_{11}=a_{11}m+a_{i1}n,$ $a'_{i1}=(-a_{i1})a_{11}+(a_{11})a_{i1}=0,$ and $a'_{kl}=a_{kl}$ for $k \neq 1,i.$   
	We need to see that this here matrix is invertible. For $\tilde{X}$ a $m \times m$ matrix, we want $\det \tilde{X}.$ We expand the determinant along the 
	first row. Then we have $\det \tilde{X}= m \det \tilde{X}[1|1]+ (-1)^{i+1}((-1)^{i+1}\det \tilde{X}[1|i]).$ $\tilde{X}[1|1]$ is a diagonal matrix with 
	$a_{11}$ on the $a_{ii}$th entry, and $1$ otherwise on the diagonal. Thus $\det \tilde{X}[1|1]=a_{11}.$ $\tilde{X}[1|i])$ is a matrix with $x_{i1}$ at 
	the $(i-1,1)$th entry, with every element below and above it zero. We take the determinant along this column, we have $(-1)^{i-1+1}x_{i1}\cdot \det 
	I_{m-2}=(-1)^{i}x_{i1}.$ Thus see that $\det \tilde{X}=ma_{11}+(-1)^{2i+1}(-a_{i1})=1,$ means that $\tilde{X}$ is invertible. 
	
	\item The above result and the result above that shall be of much use to us. If $A=0,$ then there is nothing to do. We then have $A\neq 0.$ Without loss 
	of generality, we take $a_{11}\neq 0.$ This is because we can shift the row with a non-zero element to the top, then send the column with that element 
	to the first column. Now using the above result, there is a $\tilde{X}_1$ such that $a_{21}=0.$ The value of $a_{11}$ changes. Now we have $\tilde{X}_2$ 
	that sends $a_{31}$ to $0.$ We repeat this process till $a_{i1}=0$ for all $i >1.$ Now we have the first column all zero except for $a_{11}.$ Let  
	$X_1:= \tilde{X}_{i-1}\dots \tilde{X}_1.$ Let us denote $X_1A$ by $A'.$ Then consider $A'^T.$ The first row now becomes the first column, and we can do 
	the same thing that we did earlier, to reduce all elements below $a_{11}$ in $A'^T$ to $0.$ Let that operation be given by the matrix $Y_1.$ Naturally, 
	this matrix is the product of matrices obtained from $j-1$ operations as given in the previous part. Then we take the transpose of the matrix $Y_1 A'^T$ 
	to have $$A''=(Y_1A'^T)^T=(Y_1(X_1A)^T)^T=X_1^TAY_1^T.$$ The matrix we have obtained has no non-zero elements below $a_{11}$ or to its right. 
	
	Now note that we can modify our previous result to the second row. Earlier, we reduced all the leading terms of rows other than the first row to zero, 
	then we did the same with columns. Here we reduce the second terms of the $i$th rows for $i >2,$ then do the same for the columns. We can find a $X_2$ 
	and $Y_2$ both invertible such that $X_2A''Y_2$ has all elements of the type $a_{2j}$ and $a_{i2}$ zero, for $i,j \neq 2.$ Now we have a matrix where 
	$a_{11}$ and $a_{22}$ may or may not be zero, but all elements sharing the same row or column with them is zero. We continue this process for the entire 
	matrix, which gives us at every stage two invertible matrices that do the above reduction. To be precise, we have $X_1,\dots,X_t, Y_1,\dots,Y_t$ where 
	$t=\min(m,n).$ We say $X=X_1\dots X_t,$ and $Y=Y_t^{-1}\dots Y_q^{-1}.$ Then putting all of these results together, we get $ D=XAY^{-1}.$ 
	
	We do not know a priori if $a_{11}|a_{22}|\dots |a_{tt},$ but we can ensure this. We first make sure that $a_{11}|a_{22},$ then the general case is easy 
	to see. We execute the elementary column operation $C_1 \mapsto C_1+C_2.$ Now using the previous result we can change $a_{11}$ to $\gcd(a_{11},a_{22},)$ 
	and $a_{21}$ goes to $0.$ All other terms remain unchanged. We know that $d | a_{22}.$ We repeat this procedure for $d_{ii}$ and $d_{(i+1)(i+1)},$ to 
	get the desired result.
\end{enumerate}
 
\section{} %Problem 2

\section{} %Problem 3
We have $M$ a $R-$module which has itself as a generating set. Then $\pi: R^{\oplus M}\twoheadrightarrow M$ is the surjective map sending $e_m$ to $m.$ We 
see that $\pi(e_{rm}-re_m)=\pi(e_{rm})-r\pi(e_m)=rm-rm=0.$ Also, $\pi(e_{m_1+m_2}-e_{m_1}-e_{m_2})=\pi(e_{m_1+m_2})-\pi(e_{m_1})-\pi(e_{m_2})= 
(m_1+m_2)-m_1-m_2=0.$ Therefore we have $ e_{rm}-re_m \in \ker \pi, $ and $e_{m_1+m_2}-e_{m_1}-e_{m_2} \in \ker \pi.$ Thus we have 
$$(e_{rm}-re_m,e_{m_1+m_2}-e_{m_1}-e_{m_2}) \subseteq \ker \pi.$$ 

To see the other inclusion, let there be some $re_m \in \ker \pi.$ Then we have $\pi(re_m)=rm=0.$ See that we can write $re_m$ as $$re_m=-( 
(e_{rm}-re_m)+(e_{0+0}-e_0-e_0) ),$$ as $e_{rm}=e_0.$ Thus for any general element $\sum re_m \in \ker \varphi ,$ we can write the term as a linear 
combination 
of terms in $\left(e_{rm}-re_m, e_{m_1+m_2}-e_{m_1}-e_{m_2}\right).$ Thus $$(e_{rm}-re_m,e_{m_1+m_2}-e_{m_1}-e_{m_2})= \ker \pi, $$ as required. 
\section{} %Problem 4 
\begin{enumerate}
	\item By the structure theorem for finitely generated modules, we have $N \cong \frac{\mathbb{Z}}{(a_1)}\oplus \dots \oplus 
	\frac{\mathbb{Z}}{(a_k)}\oplus \mathbb{Z}^d$, where $a_1|a_2|\dots|a_k$ and $N$ is a finitely generated submodule in $\mathbb{Q}.$ Since no element in 
	$\mathbb{Q}$ is a torsion element, we must have $T(N)=\{0\}.$ Thus $N \cong \mathbb{Z}^d.$ Let $d >1,$ say $d=2.$ Then we have a map $f: \mathbb{Z}^2 
	\rightarrow N$ from $\mathbb{Z}^2$ to $N,$ where $f(1,0)=\frac{p_1}{q_1}$ and $f(0,1)=\frac{p_2}{q_2},$ for some $\frac{p_1}{q_1},\frac{p_2}{q_2} \in 
	\mathbb{Q}.$ Then we have $ f(q_1p_2,-q_2p_1)=0, $ which contradicts the linear independence of $\mathbb{Z}.$ Then $d=0,1,$ which means $N$ is either 
	zero or a cyclic module.
	
	\item We have $N_1,N_2,$ two non-zero submodules of $\mathbb{Q}.$ We have $\frac{p_1}{q_2} \in N_1, \frac{p_2}{q_2} \in N_2.$ See that $ 
	\frac{p_1}{q_1}(q_1p_2)=\frac{p_2}{q_2}(q_2p_1),$ thus $p_1p_2 \in N_1 \cap N_2.$ Thus for any non-zero submodule we can find a non-zero element they 
	have in common. 
	
	\item Let $f: \mathbb{Q} \rightarrow M \cong \bigoplus_{i=1}^k \frac{\mathbb{Z}}{(a_i)} \oplus \mathbb{Z}^d$ be a $\mathbb{Z}-$linear map. By the 
	structure theorem, we can write $M$ as given. Then we have $f(1)=(m_1,\dots,m_k,n_1,\dots,n_d),$ where $m_1 \in \frac{\mathbb{Z}}{(a_1)}, \dots, m_k \in 
	\frac{\mathbb{Z}}{(a_k)},$ and $n_1,\dots,n_d \in \mathbb{Z}^d.$ 
	
\end{enumerate}
\section{} %Problem 5 
$R^{\oplus X}$ is a free module, for some indexing set $X.$ $M$ is some submodule of $R^{\oplus X}.$ We find a subset $Y \subseteq X$ such that $M \cap 
R^{\oplus Y}$ is free and $B$ is a basis for this free module. Let $(B,Y)$ be such a pair with the given partial order. $\mathbb{T}$ is the poset of all 
such submodules in $R^{\oplus X}.$
\begin{enumerate}
	\item $X$ is non-empty. Then we can pick a singleton subset $ \{x\} \subseteq X.$ $R^{\oplus Y}$ must be a finitely generated module (hence the free 
	module generated by a singleton must be $R$), and hence so must $M \cap R,$ as this is merely an ideal in $R,$ which is an ideal generated by one 
	element. Thus the ideal is isomorphic to $R$ as a module. Thus $R \cong Ra,$ where $(a)=I.$ Thus this is an element of $\mathbb{T},$ which means that 
	$\mathbb{T},$ where this above example corresponds to the element $(R,\{1\}).$ 
\end{enumerate}
\section{} %Problem 6 
\begin{enumerate}
	\item Let $f \in \text{Hom}_{\mathbb{Z}}\left(\frac{\mathbb{Z}}{(n)},\frac{\mathbb{Q}}{\mathbb{Z}}\right).$ Then we have an abelian group, as $ 
	\text{Hom}_{\mathbb{Z}}\left(\frac{\mathbb{Z}}{(n)},\frac{\mathbb{Q}}{\mathbb{Z}}\right)$ is a $\mathbb{Z}-$module ($\mathbb{Z}-$modules and abelian 
	groups are the same).  Elements in $\frac{\mathbb{Q}}{\mathbb{Z}}$ are precisely the elements of $\mathbb{Q} \cap [0,1).$ Then due to the 
	$\mathbb{Z}-$linearity of $f,$ we only need to ask where $1$ is sent to. Let us say that $f(1)=\frac{p}{q}.$ We see that $n \cdot f(1)=f(n)=f(0)=0,$ 
	thus $n \cdot \frac{p}{q}=0.$ This then means that $\frac{np}{q} \in \mathbb{Z} \implies q | n.$ This means that $f(1)=\frac{p}{n},$ for some $p<n,$ 
	which means that we have $n$ choices at most. Now note that any map of the form $f_i:\frac{Z}{(n)}\rightarrow \frac{\mathbb{Q}}{\mathbb{Z}}$ where 
	$f_i(k)=\frac{ik}{n},$ where $i=0,1,\dots,n-1$ is in $\text{Hom}_{\mathbb{Z}}\left(\frac{\mathbb{Z}}{(n)},\frac{\mathbb{Q}}{\mathbb{Z}}\right),$ which 
	means that it must have at least $n$ elements. Since the only number that is at least $n$ but is at most $n$ is $n$, we are done.
	
	\item Since $M$ is a finite $\mathbb{Z}-$module, it is finitely generated, hence it must be of the form $\bigoplus_{i=1}^k \frac{\mathbb{Z}}{(a_i)} 
	\oplus \mathbb{Z}^d,$ for some $d \geq 0.$ Note that since the module is finite, we must have $d=0.$ Then we have $M= \bigoplus_{i=1}^k 
	\frac{\mathbb{Z}}{(a_i)},$ where $(a_1) \supseteq (a_2) \supseteq \dots \supseteq (a_k).$ Let us denote $M_i:= \frac{\mathbb{Z}}{(a_i)},$ so we have 
	$M=\bigoplus_{i=1}^k M_i.$ See that $\text{Hom}_{\mathbb{Z}}\left(M_i,\frac{\mathbb{Q}}{\mathbb{Z}}\right)$ is a cyclic group of order $a_i$ as given in 
	the previous result. Moreover, we know explicitly what those maps are. Also see that in 
	$\text{Hom}_{\mathbb{Z}}\left(M_i,\frac{\mathbb{R}}{\mathbb{Z}}\right),$ the map is determined solely by where $1$ is taken to. If it is taken to an 
	irrational number, then this map cannot be a cyclic group, as it will not have a period. Therefore these maps must be rational, and hence we defer to 
	the rational case to see that the module $\text{Hom}_{\mathbb{Z}}\left(M_i,\frac{\mathbb{R}}{\mathbb{Z}}\right)$ must have the same maps as 
	$\text{Hom}_{\mathbb{Z}}\left(M_i,\frac{\mathbb{Q}}{\mathbb{Z}}\right),$ giving us the isomorphism. 
	
	We will state a few facts that will make proving that $M \cong \text{Hom}_{\mathbb{Z}}\left(M,\frac{\mathbb{Q}}{\mathbb{Z}}\right)$ easier. Note that 
	$\text{Hom}_{\mathbb{Z}}\left(M,\frac{\mathbb{Q}}{\mathbb{Z}}\right) \cong 
	\bigoplus_{i=1}^{k}\text{Hom}_{\mathbb{Z}}\left(M_i,\frac{\mathbb{Q}}{\mathbb{Z}}\right),$ and that for modules $M_1,\dots,M_k$ and $N_1, \dots, N_k$ 
	where $M_i \cong N_i$ for all $1 \leq i \leq k,$ we have $\oplus_{i=1}^{k}M_i \cong \oplus_{i=1}^k N_i.$ By the universal property of direct sums, we 
	know that the composition of isomorphic maps shall also be an isomorphic map.
	
	We want to show that $ M_i \cong \text{Hom}_{\mathbb{Z}}\left(M_i,\frac{\mathbb{Q}}{\mathbb{Z}}\right),$ the required result should follow directly. We 
	know that $M_i=\{0,1,\dots,a_i-1\},$ and $\text{Hom}_{\mathbb{Z}}\left(M_i,\frac{\mathbb{Q}}{\mathbb{Z}}\right)=\{\varphi^i_0,\dots, 
	\varphi^i_{a_i-1}\},$ where $\varphi^i_{m}(1)=\frac{m}{a_i}.$ Now define the map $\sigma: M_i \rightarrow 
	\text{Hom}_{\mathbb{Z}}\left(M_i,\frac{\mathbb{Q}}{\mathbb{Z}}\right),$ where $\sigma(m_i)= \varphi^i_{m_i}.$ Our map is a $\mathbb{Z}-$module 
	homomorphism. We define $\theta:\text{Hom}_{\mathbb{Z}}\left(M_i,\frac{\mathbb{Q}}{\mathbb{Z}}\right)  \rightarrow M_i,$ where 
	$\theta(\varphi^i_{m_i})=m_i.$ Then see that $\sigma \circ \theta= \iota_{\text{Hom}_{\mathbb{Z}}\left(M_i,\frac{\mathbb{Q}}{\mathbb{Z}}\right) },$ and 
	$\theta \circ \sigma = \iota_{M_i}.$ Thus we see that  $M_i \cong \text{Hom}_{\mathbb{Z}}\left(M_i,\frac{\mathbb{Q}}{\mathbb{Z}}\right).$ 
	
	Since $\text{Hom}_{\mathbb{Z}}\left(M,\frac{\mathbb{Q}}{\mathbb{Z}}\right) \cong 
	\bigoplus_{i=1}^{k}\text{Hom}_{\mathbb{Z}}\left(M_i,\frac{\mathbb{Q}}{\mathbb{Z}}\right),$ we have $$M \cong \bigoplus_{i=1}^k M_i \cong 
	\bigoplus_{i=1}^{k}\text{Hom}_{\mathbb{Z}}\left(M_i,\frac{\mathbb{Q}}{\mathbb{Z}}\right) \cong 
	\text{Hom}_{\mathbb{Z}}\left(M,\frac{\mathbb{Q}}{\mathbb{Z}}\right) ,$$ as required. 
	
	\item We have the map $\delta: M \rightarrow \text{Hom}_{R}\left(\text{Hom}_{R}\left(M, N\right),N \right)$ where $\delta(m)= (\varphi 
	\mapsto \varphi(m)).$ Let us denote by $f_m \in \text{Hom}_{R}\left(\text{Hom}_{R}\left(M, N\right),N \right),$ where 
	$f_m(\varphi)=\varphi(m).$ So $\delta(m)=f_m.$ We want to see if $\delta$ is a $R-$linear map. 
	
	We will check the behaviour of $\delta$ with respect to addition and scalar multiplication. We have $\delta(m+n)=f_{m+n},$ which means that 
	$f_{m+n}(\varphi)=\varphi(m+n)=\varphi(m)+\varphi(n).$ We know that $f_m(\varphi)=\varphi(m)$ and $f_n(\varphi)=\varphi(n),$ so we have $f_{m+n}=f_m + 
	f_n$.  Also we know that 
	$\delta(m)=f_m$ and $\delta(n)=f_n,$ so putting all this together we have $\delta(m+n)=\delta(m)+\delta(n).$ For scalar multiples, see that 
	$\delta(rm)=f_{rm},$ where $f_{rm}(\varphi)=\varphi(rm)=r\varphi(m).$ See that $f_m(\varphi)=\varphi(m),$ so $r\varphi(m)=rf_m.$ We know that 
	$r\delta(m)=rf_m,$ which is 
	the required result. Thus we have that $\delta$ is a $R-$linear map. 
	
	To see that $\delta$ commutes with finite direct sums, we note that since finite direct sums in the first slot and Hom commute, we can see that 
	$$\text{Hom}_{R}\left(\text{Hom}_{R}\left(\oplus_{i=1}^k M_i, N\right),N \right) \cong \oplus_{i=1}^k 
	\text{Hom}_{R}\left(\text{Hom}_{R}\left(M_i, N\right),N \right).$$ We specify an isomorphism between the two using the terminology 
	defined earlier. For the ease of typing and the eyes of the reader, let $$H_1=\text{Hom}_{R}\left(\text{Hom}_{R}\left(\oplus_{i=1}^k 
	M_i, N\right),N \right),$$ and $$H_2=\oplus_{i=1}^k \text{Hom}_{R}\left(\text{Hom}_{R}\left(M_i, N\right),N \right).$$ 
	
	We have $\delta_{\oplus_{i=1}^k M_i}: \oplus_{i=1}^k M_i \rightarrow H_1,$ where $\delta_{\oplus_{i=1}^k M_i}(m_1,\dots,m_k)=(\varphi \mapsto 
	\phi(m_1,\dots,m_k))$. Denote the map on the right by $f_{m_1,\dots,m_k}.$ We have $\delta_{M_i} \rightarrow \text{Hom}_{R}\left(\text{Hom}_{R}\left(M, 
	N\right),N \right),$ where $\delta_{M_i}(m_i)=(\varphi \mapsto \varphi(m_i)).$ We denote the map on the right by $f^i_{m_i}.$ We extend this map to 
	$H_2,$ where $\delta|_{M_i}=\delta_{M_i}.$ Here, we abuse notation since more accurately $\delta_{M_i}$ acts on $M_k,$ while $\delta$ acts on 
	$\oplus_{i=1}^k M_i.$ This product of maps is denoted by $\oplus_{i=1}^k\delta_{M_i}.$ 
	
	Now we wish to show that the given diagram in the problem commutes. Define $\psi_1:H_1\rightarrow H_2,$ and $\psi_2: H_2 \rightarrow H_1,$ two module 
	maps. We know a map between the two exists such that their composition is the identity maps for $H_1$ and $H_2.$ Then we have $\psi_1 \circ 
	\delta_{\oplus_{i=1}^k M_i}: \oplus_{i=1}^k M_i \rightarrow H_2,$ where $\psi_1 \circ \delta_{\oplus_{i=1}^k M_i}(m_1,\dots,m_k)= 
	\psi_1(f_{(m_1,\dots,m_k)})= (f^1_{m_1},\dots,f^k_{m_k}).$ Similarly, we have the map $\psi_2 \circ \oplus_{i=1}^k \delta_{M_i}: \oplus_{i=1}^k M_i 
	\rightarrow H_1$ where $\psi_2 \circ \oplus_{i=1}^k \delta_{M_i}(m_1,\dots,m_k)=\psi_2(f^1_{m_1},\dots,f^k_{m_k})=f_{(m_1,\dots,m_k)}.$ See that this 
	diagram commutes, which is the desired result.
	
	\item We will use the previous two results liberally to get our result. We know $M \cong  
	\text{Hom}_{\mathbb{Z}}\left(M,\frac{\mathbb{Q}}{\mathbb{Z}}\right),$ Thus 
	$\text{Hom}_{\mathbb{Z}}\left(\text{Hom}_{\mathbb{Z}}\left(M,\frac{\mathbb{Q}}{\mathbb{Z}}\right),\frac{\mathbb{Q}}{\mathbb{Z}}\right) \cong 
	\text{Hom}_{\mathbb{Z}}\left(M,\frac{\mathbb{Q}}{\mathbb{Z}}\right) \cong M.$ Now we need to check that $\delta_{\oplus_{i=1}^k M_i}$ is an isomorphism. 
	Previously we established a correspondence between the elements of each $M_i$ with its Hom module, we extend it to all of $M.$ A general element of $M$ 
	is $(m_1,\dots,m_k),$ and a general map from $M$ to $\frac{\mathbb{Q}}{\mathbb{Z}}$ is $(\phi^1_{m_1},\dots,\phi^k_{m_k}),$ where $\phi^i_{m_i}:M_i 
	\rightarrow \frac{\mathbb{Q}}{\mathbb{Z}}$ is the map where $\phi^i_{m_i}(1)=\frac{m_i}{a_i}.$ 
	
	Now see that $\delta_{\oplus_{i=1}^k M_i}(m_1,\dots, m_k)=f_{m_1,\dots,m_k}.$ This map takes any map in 
	$\text{Hom}_{\mathbb{Z}}\left(M,\frac{\mathbb{Q}}{\mathbb{Z}}\right)$ and sends it to that map evaluated at $(m_1,\dots,m_k).$ That is, 
	$\delta_{\oplus_{i=1}^k M_i}(m_1,\dots,m_k)=(f_{m_1},\dots,f_{m_k}),$ which is the map where each component sends any map in 
	$\text{Hom}_{\mathbb{Z}}\left(M_i,\frac{\mathbb{Q}}{\mathbb{Z}}\right) $ to that map evaluated at $m_i,$ for $i=1,\dots, k.$ We wish to have a map from 
	$\text{Hom}_{\mathbb{Z}}\left(\text{Hom}_{\mathbb{Z}}\left(M,\frac{\mathbb{Q}}{\mathbb{Z}}\right),\frac{\mathbb{Q}}{\mathbb{Z}}\right)$ to $M,$ such 
	that their composition is the identity map. Any element of $\text{Hom}_{\mathbb{Z}}\left(M_i,\frac{\mathbb{Q}}{\mathbb{Z}}\right)$ is of the form 
	$\phi^i_{\ell},$ where $\ell= 0,1,\dots,a_i-1.$ As mentioned before, it can be mapped to $\ell \in M_i.$ Then we specify a map from $M_i$ to 
	$\frac{\mathbb{Q}}{\mathbb{Z}},$ which sends $\ell$ to $\phi^i(\ell).$ Thus any element of 
	$\text{Hom}_{\mathbb{Z}}\left(\text{Hom}_{\mathbb{Z}}\left(M,\frac{\mathbb{Q}}{\mathbb{Z}}\right),\frac{\mathbb{Q}}{\mathbb{Z}}\right)$ can be 
	represented by $(\phi^1_{\ell_1},\dots,\phi^k_{\ell_k}).$ We naturally see that we can send this to $(\ell_1,\dots,\ell_k) \in M,$ which is the required 
	inverse map.  
	 
\end{enumerate}
\end{document}

















