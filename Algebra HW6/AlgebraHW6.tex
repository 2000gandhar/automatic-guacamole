%%%%%%%%%%%%%%%%%%%%%%%%%%%%%%%%%%%%%%%%%
% Lachaise Assignment
% LaTeX Template
% Version 1.0 (26/6/2018)
%
% This template originates from:
% http://www.LaTeXTemplates.com
%
% Authors:
% Marion Lachaise & François Févotte
% Vel (vel@LaTeXTemplates.com)
%
% License:
% CC BY-NC-SA 3.0 (http://creativecommons.org/licenses/by-nc-sa/3.0/)
% 
%%%%%%%%%%%%%%%%%%%%%%%%%%%%%%%%%%%%%%%%%

%----------------------------------------------------------------------------------------
%	PACKAGES AND OTHER DOCUMENT CONFIGURATIONS
%----------------------------------------------------------------------------------------

\documentclass{article}

%%%%%%%%%%%%%%%%%%%%%%%%%%%%%%%%%%%%%%%%%
% Lachaise Assignment
% Structure Specification File
% Version 1.0 (26/6/2018)
%
% This template originates from:
% http://www.LaTeXTemplates.com
%
% Authors:
% Marion Lachaise & François Févotte
% Vel (vel@LaTeXTemplates.com)
%
% License:
% CC BY-NC-SA 3.0 (http://creativecommons.org/licenses/by-nc-sa/3.0/)
% 
%%%%%%%%%%%%%%%%%%%%%%%%%%%%%%%%%%%%%%%%%

%----------------------------------------------------------------------------------------
%	PACKAGES AND OTHER DOCUMENT CONFIGURATIONS
%----------------------------------------------------------------------------------------

\usepackage{amsmath,amsfonts,amssymb, tikz-cd} % Math packages

\usepackage{enumerate} % Custom item numbers for enumerations


\usepackage[framemethod=tikz]{mdframed} % Allows defining custom boxed/framed environments

\usepackage{listings} % File listings, with syntax highlighting
\lstset{
	basicstyle=\ttfamily, % Typeset listings in monospace font
}

%----------------------------------------------------------------------------------------
%	DOCUMENT MARGINS
%----------------------------------------------------------------------------------------

\usepackage{geometry} % Required for adjusting page dimensions and margins

\geometry{
	paper=letterpaper, % Paper size, change to letterpaper for US letter size
	top=2.5cm, % Top margin
	bottom=3cm, % Bottom margin
	left=2.5cm, % Left margin
	right=2.5cm, % Right margin
	headheight=14pt, % Header height
	footskip=1.5cm, % Space from the bottom margin to the baseline of the footer
	headsep=1.2cm, % Space from the top margin to the baseline of the header
	%showframe, % Uncomment to show how the type block is set on the page
}

%----------------------------------------------------------------------------------------
%	FONTS
%----------------------------------------------------------------------------------------

\usepackage[utf8]{inputenc} % Required for inputting international characters
\usepackage[T1]{fontenc} % Output font encoding for international characters


%----------------------------------------------------------------------------------------
%	COMMAND LINE ENVIRONMENT
%----------------------------------------------------------------------------------------

% Usage:
% \begin{commandline}
	%	\begin{verbatim}
		%		$ ls
		%		
		%		Applications	Desktop	...
		%	\end{verbatim}
	% \end{commandline}

\mdfdefinestyle{commandline}{
	leftmargin=10pt,
	rightmargin=10pt,
	innerleftmargin=15pt,
	middlelinecolor=black!50!white,
	middlelinewidth=2pt,
	frametitlerule=false,
	backgroundcolor=black!5!white,
	frametitle={Command Line},
	frametitlefont={\normalfont\sffamily\color{white}\hspace{-1em}},
	frametitlebackgroundcolor=black!50!white,
	nobreak,
}

% Define a custom environment for command-line snapshots
\newenvironment{commandline}{
	\medskip
	\begin{mdframed}[style=commandline]
	}{
	\end{mdframed}
	\medskip
}

%----------------------------------------------------------------------------------------
%	FILE CONTENTS ENVIRONMENT
%----------------------------------------------------------------------------------------

% Usage:
% \begin{file}[optional filename, defaults to "File"]
	%	File contents, for example, with a listings environment
	% \end{file}

\mdfdefinestyle{file}{
	innertopmargin=1.6\baselineskip,
	innerbottommargin=0.8\baselineskip,
	topline=false, bottomline=false,
	leftline=false, rightline=false,
	leftmargin=2cm,
	rightmargin=2cm,
	singleextra={%
		\draw[fill=black!10!white](P)++(0,-1.2em)rectangle(P-|O);
		\node[anchor=north west]
		at(P-|O){\ttfamily\mdfilename};
		%
		\def\l{3em}
		\draw(O-|P)++(-\l,0)--++(\l,\l)--(P)--(P-|O)--(O)--cycle;
		\draw(O-|P)++(-\l,0)--++(0,\l)--++(\l,0);
	},
	nobreak,
}

% Define a custom environment for file contents
\newenvironment{file}[1][File]{ % Set the default filename to "File"
	\medskip
	\newcommand{\mdfilename}{#1}
	\begin{mdframed}[style=file]
	}{
	\end{mdframed}
	\medskip
}

%----------------------------------------------------------------------------------------
%	NUMBERED QUESTIONS ENVIRONMENT
%----------------------------------------------------------------------------------------

% Usage:
% \begin{question}[optional title]
	%	Question contents
	% \end{question}

\mdfdefinestyle{question}{
	innertopmargin=1.2\baselineskip,
	innerbottommargin=0.8\baselineskip,
	roundcorner=5pt,
	nobreak,
	singleextra={%
		\draw(P-|O)node[xshift=1em,anchor=west,fill=white,draw,rounded corners=5pt]{%
			Question \theQuestion\questionTitle};
	},
}

\newcounter{Question} % Stores the current question number that gets iterated with each new question

% Define a custom environment for numbered questions
\newenvironment{question}[1][\unskip]{
	\bigskip
	\stepcounter{Question}
	\newcommand{\questionTitle}{~#1}
	\begin{mdframed}[style=question]
	}{
	\end{mdframed}
	\medskip
}

%----------------------------------------------------------------------------------------
%	WARNING TEXT ENVIRONMENT
%----------------------------------------------------------------------------------------

% Usage:
% \begin{warn}[optional title, defaults to "Warning:"]
	%	Contents
	% \end{warn}

\mdfdefinestyle{warning}{
	topline=false, bottomline=false,
	leftline=false, rightline=false,
	nobreak,
	singleextra={%
		\draw(P-|O)++(-0.5em,0)node(tmp1){};
		\draw(P-|O)++(0.5em,0)node(tmp2){};
		\fill[black,rotate around={45:(P-|O)}](tmp1)rectangle(tmp2);
		\node at(P-|O){\color{white}\scriptsize\bf !};
		\draw[very thick](P-|O)++(0,-1em)--(O);%--(O-|P);
	}
}

% Define a custom environment for warning text
\newenvironment{warn}[1][Warning:]{ % Set the default warning to "Warning:"
	\medskip
	\begin{mdframed}[style=warning]
		\noindent{\textbf{#1}}
	}{
	\end{mdframed}
}

%----------------------------------------------------------------------------------------
%	INFORMATION ENVIRONMENT
%----------------------------------------------------------------------------------------

% Usage:
% \begin{info}[optional title, defaults to "Info:"]
	% 	contents
	% 	\end{info}

\mdfdefinestyle{info}{%
	topline=false, bottomline=false,
	leftline=false, rightline=false,
	nobreak,
	singleextra={%
		\fill[black](P-|O)circle[radius=0.4em];
		\node at(P-|O){\color{white}\scriptsize\bf i};
		\draw[very thick](P-|O)++(0,-0.8em)--(O);%--(O-|P);
	}
}

% Define a custom environment for information
\newenvironment{info}[1][Info:]{ % Set the default title to "Info:"
	\medskip
	\begin{mdframed}[style=info]
		\noindent{\textbf{#1}}
	}{
	\end{mdframed}
}
 % Include the file specifying the document structure and custom commands

%----------------------------------------------------------------------------------------
%	ASSIGNMENT INFORMATION
%----------------------------------------------------------------------------------------

\title{The last Home-work} % Title of the assignment

\author{Gandhar Kulkarni (mmat2304)} % Author name and email address

\date{} % University, school and/or department name(s) and a date

%----------------------------------------------------------------------------------------

\begin{document}

\maketitle % Print the title

%----------------------------------------------------------------------------------------
%	INTRODUCTION
%----------------------------------------------------------------------------------------

\section{} %Problem 1 
\begin{enumerate}
	\item Since $R$ is a PID, we know that $(a,b)=(d),$ for some $d,a,b \in R.$ Then we have $d=am+bn,$ for some $m,n \in R.$ Now we have a vector 
	$v=[a,b]^T \in R^2 \backslash \{0\}.$ Then we show that there exists a $2x2$ matrix that does what we want by constructing one. Let the desired matrix 
	by be given by $X= \begin{pmatrix}
		x_{11}&x_{21}\\
		x_{12}&x_{22} 	
	\end{pmatrix}.$ Now we have $Xv= [x_{11}a + x_{21}b, x_{12}a+x_{22}b]^T=[d,0]^T.$ Comparing terms, we have $x_{12}a+x_{22}b=0.$ Then we have 
	$x_{12}a=-x_{22}b,$ which implies that $ x_{12}| -b,$ and $x_{22}|a.$ It is easy to see that $x_{12}=-a/d$ and $x_{22}=b/d$ does the trick. For $x_{11}a 
	+ 
	x_{21}b=d,$ see that $x_{11}=m$ and $x_{21}=n$ are good choices, since their linear combination produces $d.$ Thus see that 
		$$X=\begin{pmatrix}
			m&n\\
			-b/d&a/d
		\end{pmatrix} $$ is a matrix that achieves the intended result.
	
	\item The above result shall be of much use to us. We see that we want to send $a_{11}$ to $d,$ that is the gcd of $a_{11}$ and $a_{i1},$ and $a_{i1}$ 
	to $0.$ Let $m,n \in R$ such that $d=a_{11}m+a_{i1}n.$ Define the matrix $\tilde{X}=(x_{kl})$ thus--- $x_{11}=m, x_{1i}=n, x_{i1}=-a_{i1}/d, 
	x_{ii}=a_{11}/d.$ Also we have $x_{kk}=1$ if $k \neq 1,i.$ All other elements are $0.$ Then we have $A'=\tilde{X}A=(a'_{kl}),$ where 
	$a'_{11}=a_{11}m+a_{i1}n,$ $a'_{i1}=(-a_{i1})a_{11}+(a_{11})a_{i1}=0,$ and $a'_{kl}=a_{kl}$ for $k \neq 1,i.$   
	We need to see that this here matrix is invertible. For $\tilde{X}$ a $m \times m$ matrix, we want $\det \tilde{X}.$ We expand the determinant along the 
	first row. Then we have $\det \tilde{X}= m \det \tilde{X}[1|1]+ (-1)^{i+1}((-1)^{i+1}\det \tilde{X}[1|i]).$ $\tilde{X}[1|1]$ is a diagonal matrix with 
	$a_{11}$ on the $a_{ii}$th entry, and $1$ otherwise on the diagonal. Thus $\det \tilde{X}[1|1]=a_{11}.$ $\tilde{X}[1|i])$ is a matrix with $x_{i1}$ at 
	the $(i-1,1)$th entry, with every element below and above it zero. We take the determinant along this column, we have $(-1)^{i-1+1}x_{i1}\cdot \det 
	I_{m-2}=(-1)^{i}x_{i1}.$ Thus see that $\det \tilde{X}=ma_{11}+(-1)^{2i+1}(-a_{i1})=1,$ means that $\tilde{X}$ is invertible. 
	
	\item The above result and the result above that shall be of much use to us. If $A=0,$ then there is nothing to do. We then have $A\neq 0.$ Without loss 
	of generality, we take $a_{11}\neq 0.$ This is because we can shift the row with a non-zero element to the top, then send the column with that element 
	to the first column. Now using the above result, there is a $\tilde{X}_1$ such that $a_{21}=0.$ The value of $a_{11}$
\end{enumerate}
 
\section{} %Problem 2
\section{} %Problem 3
We have $M$ a $R-$module which has itself as a generating set. Then $\pi: R^{\oplus M}\twoheadrightarrow M$ is the surjective map sending $e_m$ to $m.$ We 
see that $\pi(e_{rm}-re_m)=\pi(e_{rm})-r\pi(e_m)=rm-rm=0.$ Also, $\pi(e_{m_1+m_2}-e_{m_1}-e_{m_2})=\pi(e_{m_1+m_2})-\pi(e_{m_1})-\pi(e_{m_2})= 
(m_1+m_2)-m_1-m_2=0.$ Therefore we have $ e_{rm}-re_m \in \ker \pi, $ and $e_{m_1+m_2}-e_{m_1}-e_{m_2} \in \ker \pi.$ Thus we have 
$$(e_{rm}-re_m,e_{m_1+m_2}-e_{m_1}-e_{m_2}) \subseteq \ker \pi.$$ Let us have $\sum_{m \in M}r_m e_m \in R^{\oplus M}.$ Note that there are only finitely 
many 
terms in the summation. See that $$\pi(\sum_{m \in M}r_m e_m)=\sum_{m \in M}r_m \pi(e_m)= \sum_{m \in M}r_m m = \pi(\sum_{m \in M}e_{r_mm}).$$ This means 
that 
$\pi(\sum_{m \in M}r_m e_m - \sum_{m \in M}e_{r_mm})=0.$ This, in turn implies that $ \sum_{m \in M} r_me_m - e_{r_mm} \in \ker \pi.$ We also see that 
$$\pi(\sum_{m \in M}r_m e_m)=\sum_{m \in M}r_m \pi(e_m)= \sum_{m \in M}r_m m = \pi(e_{\sum_{m \in M}r_mm}).$$ This means that $\sum_{m \in M}r_m 
e_m-e_{\sum_{m \in M}r_mm} \in \ker \pi.$ Given an element in $R^{\oplus M},$ we can choose which summands to clump and which to leave unchanged. Either 
ways, we see that we get a linear combination of $re_m-e_{rm}$ and $e_{m_1+m_2}-e_{m_1}-e_{m_2},$ which implies that $ \ker \pi \subseteq 
(e_{rm}-re_m,e_{m_1+m_2}-e_{m_1}-e_{m_2}).$ This gives us the desired equality.
\section{} %Problem 4 
\begin{enumerate}
	\item We take the module $\mathbb{Z}[q_1,q_2],$ where $q_1,q_2 \in \mathbb{Q}.$ 
\end{enumerate}
\section{} %Problem 5 
\section{} %Problem 6 

\end{document}

















