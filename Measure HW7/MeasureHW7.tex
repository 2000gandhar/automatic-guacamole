%%%%%%%%%%%%%%%%%%%%%%%%%%%%%%%%%%%%%%%%%
% Lachaise Assignment
% LaTeX Template
% Version 1.0 (26/6/2018)
%
% This template originates from:
% http://www.LaTeXTemplates.com
%
% Authors:
% Marion Lachaise & François Févotte
% Vel (vel@LaTeXTemplates.com)
%
% License:
% CC BY-NC-SA 3.0 (http://creativecommons.org/licenses/by-nc-sa/3.0/)
% 
%%%%%%%%%%%%%%%%%%%%%%%%%%%%%%%%%%%%%%%%%

%----------------------------------------------------------------------------------------
%	PACKAGES AND OTHER DOCUMENT CONFIGURATIONS
%----------------------------------------------------------------------------------------

\documentclass{article}

%%%%%%%%%%%%%%%%%%%%%%%%%%%%%%%%%%%%%%%%%
% Lachaise Assignment
% Structure Specification File
% Version 1.0 (26/6/2018)
%
% This template originates from:
% http://www.LaTeXTemplates.com
%
% Authors:
% Marion Lachaise & François Févotte
% Vel (vel@LaTeXTemplates.com)
%
% License:
% CC BY-NC-SA 3.0 (http://creativecommons.org/licenses/by-nc-sa/3.0/)
% 
%%%%%%%%%%%%%%%%%%%%%%%%%%%%%%%%%%%%%%%%%

%----------------------------------------------------------------------------------------
%	PACKAGES AND OTHER DOCUMENT CONFIGURATIONS
%----------------------------------------------------------------------------------------

\usepackage{amsmath,amsfonts,amssymb, tikz-cd} % Math packages

\usepackage{enumerate} % Custom item numbers for enumerations


\usepackage[framemethod=tikz]{mdframed} % Allows defining custom boxed/framed environments

\usepackage{listings} % File listings, with syntax highlighting
\lstset{
	basicstyle=\ttfamily, % Typeset listings in monospace font
}

%----------------------------------------------------------------------------------------
%	DOCUMENT MARGINS
%----------------------------------------------------------------------------------------

\usepackage{geometry} % Required for adjusting page dimensions and margins

\geometry{
	paper=letterpaper, % Paper size, change to letterpaper for US letter size
	top=2.5cm, % Top margin
	bottom=3cm, % Bottom margin
	left=2.5cm, % Left margin
	right=2.5cm, % Right margin
	headheight=14pt, % Header height
	footskip=1.5cm, % Space from the bottom margin to the baseline of the footer
	headsep=1.2cm, % Space from the top margin to the baseline of the header
	%showframe, % Uncomment to show how the type block is set on the page
}

%----------------------------------------------------------------------------------------
%	FONTS
%----------------------------------------------------------------------------------------

\usepackage[utf8]{inputenc} % Required for inputting international characters
\usepackage[T1]{fontenc} % Output font encoding for international characters


%----------------------------------------------------------------------------------------
%	COMMAND LINE ENVIRONMENT
%----------------------------------------------------------------------------------------

% Usage:
% \begin{commandline}
	%	\begin{verbatim}
		%		$ ls
		%		
		%		Applications	Desktop	...
		%	\end{verbatim}
	% \end{commandline}

\mdfdefinestyle{commandline}{
	leftmargin=10pt,
	rightmargin=10pt,
	innerleftmargin=15pt,
	middlelinecolor=black!50!white,
	middlelinewidth=2pt,
	frametitlerule=false,
	backgroundcolor=black!5!white,
	frametitle={Command Line},
	frametitlefont={\normalfont\sffamily\color{white}\hspace{-1em}},
	frametitlebackgroundcolor=black!50!white,
	nobreak,
}

% Define a custom environment for command-line snapshots
\newenvironment{commandline}{
	\medskip
	\begin{mdframed}[style=commandline]
	}{
	\end{mdframed}
	\medskip
}

%----------------------------------------------------------------------------------------
%	FILE CONTENTS ENVIRONMENT
%----------------------------------------------------------------------------------------

% Usage:
% \begin{file}[optional filename, defaults to "File"]
	%	File contents, for example, with a listings environment
	% \end{file}

\mdfdefinestyle{file}{
	innertopmargin=1.6\baselineskip,
	innerbottommargin=0.8\baselineskip,
	topline=false, bottomline=false,
	leftline=false, rightline=false,
	leftmargin=2cm,
	rightmargin=2cm,
	singleextra={%
		\draw[fill=black!10!white](P)++(0,-1.2em)rectangle(P-|O);
		\node[anchor=north west]
		at(P-|O){\ttfamily\mdfilename};
		%
		\def\l{3em}
		\draw(O-|P)++(-\l,0)--++(\l,\l)--(P)--(P-|O)--(O)--cycle;
		\draw(O-|P)++(-\l,0)--++(0,\l)--++(\l,0);
	},
	nobreak,
}

% Define a custom environment for file contents
\newenvironment{file}[1][File]{ % Set the default filename to "File"
	\medskip
	\newcommand{\mdfilename}{#1}
	\begin{mdframed}[style=file]
	}{
	\end{mdframed}
	\medskip
}

%----------------------------------------------------------------------------------------
%	NUMBERED QUESTIONS ENVIRONMENT
%----------------------------------------------------------------------------------------

% Usage:
% \begin{question}[optional title]
	%	Question contents
	% \end{question}

\mdfdefinestyle{question}{
	innertopmargin=1.2\baselineskip,
	innerbottommargin=0.8\baselineskip,
	roundcorner=5pt,
	nobreak,
	singleextra={%
		\draw(P-|O)node[xshift=1em,anchor=west,fill=white,draw,rounded corners=5pt]{%
			Question \theQuestion\questionTitle};
	},
}

\newcounter{Question} % Stores the current question number that gets iterated with each new question

% Define a custom environment for numbered questions
\newenvironment{question}[1][\unskip]{
	\bigskip
	\stepcounter{Question}
	\newcommand{\questionTitle}{~#1}
	\begin{mdframed}[style=question]
	}{
	\end{mdframed}
	\medskip
}

%----------------------------------------------------------------------------------------
%	WARNING TEXT ENVIRONMENT
%----------------------------------------------------------------------------------------

% Usage:
% \begin{warn}[optional title, defaults to "Warning:"]
	%	Contents
	% \end{warn}

\mdfdefinestyle{warning}{
	topline=false, bottomline=false,
	leftline=false, rightline=false,
	nobreak,
	singleextra={%
		\draw(P-|O)++(-0.5em,0)node(tmp1){};
		\draw(P-|O)++(0.5em,0)node(tmp2){};
		\fill[black,rotate around={45:(P-|O)}](tmp1)rectangle(tmp2);
		\node at(P-|O){\color{white}\scriptsize\bf !};
		\draw[very thick](P-|O)++(0,-1em)--(O);%--(O-|P);
	}
}

% Define a custom environment for warning text
\newenvironment{warn}[1][Warning:]{ % Set the default warning to "Warning:"
	\medskip
	\begin{mdframed}[style=warning]
		\noindent{\textbf{#1}}
	}{
	\end{mdframed}
}

%----------------------------------------------------------------------------------------
%	INFORMATION ENVIRONMENT
%----------------------------------------------------------------------------------------

% Usage:
% \begin{info}[optional title, defaults to "Info:"]
	% 	contents
	% 	\end{info}

\mdfdefinestyle{info}{%
	topline=false, bottomline=false,
	leftline=false, rightline=false,
	nobreak,
	singleextra={%
		\fill[black](P-|O)circle[radius=0.4em];
		\node at(P-|O){\color{white}\scriptsize\bf i};
		\draw[very thick](P-|O)++(0,-0.8em)--(O);%--(O-|P);
	}
}

% Define a custom environment for information
\newenvironment{info}[1][Info:]{ % Set the default title to "Info:"
	\medskip
	\begin{mdframed}[style=info]
		\noindent{\textbf{#1}}
	}{
	\end{mdframed}
}
 % Include the file specifying the document structure and custom commands

%----------------------------------------------------------------------------------------
%	ASSIGNMENT INFORMATION
%----------------------------------------------------------------------------------------

\title{Algebra HW5} % Title of the assignment

\author{Gandhar Kulkarni (mmat2304)} % Author name and email address

\date{} % University, school and/or department name(s) and a date

%----------------------------------------------------------------------------------------

\begin{document}

\maketitle % Print the title

%----------------------------------------------------------------------------------------
%	INTRODUCTION
%----------------------------------------------------------------------------------------

\section{} %Problem 1 
If we have $\nu=0,$ then see that $\nu(E)=\int_{E}0 d\mu,$ thus $\mu(E)=0 \implies \nu(E)=0$ trivially, so $\nu << \mu.$ Also see that $\nu \perp \mu$, as 
$X= X \sqcup \phi,$ and see that $\mu(E)=\mu(E \cap X),$ while $\nu(E)=\nu(E \int \phi).$ 

Now let us assume that $\nu << \mu$ and $\nu \perp \mu.$ Then we have $X=A \sqcup B,$ where $\mu(E)=\mu(E \cap A),$ while $\nu(E)=\nu(E \int B).$ Let us 
pick a measurable set $E \subseteq B.$ Then we have $\mu(E)=\mu(E \cap B)=\mu(\phi)=0.$ Since $\nu << \mu,$ we have $\nu(E)=0.$ Thus $\nu$ is zero on every 
measurable subset $E$ in $B.$ For a general measurable set $E,$ we have $E=(E \cap A) \sqcup (E \cap B).$ We already know that $\nu(E\cap A)=0,$ now we see 
that $\nu(E \cap B)=0$ also. Thus $\nu(E)=\nu(E \cap A)+\nu(E \cap B)=0$ for all measurable $E \subset X.$
\section{} %Problem 2
If $\nu \perp \mu,$ then there exists we have $X=A \sqcup B,$ where $\mu(B)=0$ and $\nu(A)=0.$ Then $\{E_n\}$ is a sequence such that $E_n=B$ for all $n \in 
\mathbb{N}.$ Then see that $\mu(E_n), \nu(X\backslash A)=0,$ as required.

Conversely, we assume that $\{E_n\}$ is a sequence of measurable subsets such that $\mu(E_n)\to \infty$ and $\nu(X \backslash E_n) \to \infty$ as $n \to 
\infty.$ 
\section{} %Problem 3 
We can check that $m << m$ trivially, as $m(E)=0 \implies m(E)=0.$ We can split $\mathbb{R}$ into two disjoint subsets, that is, $\mathbb{R}= \{0\} \sqcup 
((\infty,0)\cup (0,\infty)).$ We denote the two sets as $A$ and $B.$ Then observe that $\delta_0(E)= \delta_0(E \cap A),$ that is, the Dirac measure at $0$ 
only cares if it intersects $\{0\},$ and nothing else. Also, we have $m(E)=m(E \cap B),$ as $A$ is a $m-$null set, hence $m(E)=m(E\cap A \sqcup E \cap 
B)=m(E \cap A)+ m(E \cap B)=m(E \cap B),$ seeing as $E \cap A$ is also a $m-$null set. Thus we have $m \perp \delta_0.$ Thus $\nu=m+\delta_0$ is already in 
the Lebesgue decomposition. 
\section{} %Problem 4 
\begin{itemize}
	\item We find the positive and negative parts of $f.$ Note that the roots of this polynomial are $3+2\sqrt{2}$ and 
	$3-\sqrt{2}.$ Let us call them $\alpha_1$ and $\alpha_2$ for sake of convenience. Then $$p^+=\begin{cases}
		x^2-6x+1 & x \in (-\infty,\alpha_2] \cup [\alpha_1, \infty)\\
		0 & \text{ else},
	\end{cases}$$ 
	and $$p^-= \begin{cases} 
		-(x^2-6x+1) & x \in (\alpha_2,\alpha_1)\\
		0 & \text{ else}.
	\end{cases}$$
	Then $\nu(E)=\int_{E}p^+d\mu - \int_{E}p^-d\mu=\nu^+-\nu^-,$ where $\nu^+:=\int_{E}p^+d\mu$ and $\nu^-:=\int_{E}p^-d\mu$ are two positive measures. Note 
	that it is not possible for both of them to attain $\infty$ together, since $\nu^-$ is a finite measure. Thus it is trivial to see that $\nu$ must be a 
	signed measure.
	\item Let $\mathbb{R}= A \sqcup B,$ where $A= (-\infty,\alpha_2] \cup [\alpha_1, \infty)$ and $B=(\alpha_2,\alpha_1).$ See that since both the positive 
	measures have their usual properties, we have that for $E \subseteq A$ measurable, we have $\nu(E)=\nu^+(E)-\nu^-(E)=\nu^+(E)-0 \geq 0,$ and likewise 
	for $E \subseteq B$ measurable, we have $\nu(E)=\nu^+(E)-\nu^-(E)=0-\nu^-(E) \leq 0.$ Thus the above construction is a Hahn decomposition.
	\item See that $\nu^+$ lives on $A,$ while $\nu^-$ lives on $B.$ That is, $\nu^+(E)=\nu(E \cap A),$ and $\nu^-(E)=-\nu(E \cap B).$ This is easy to see, 
	as $E= (E \cap A) \sqcup (C \cap B).$ Then $\nu(E)= \int_{(E\cap A) \sqcup (E \cap B)}p^+d\mu - \int_{(E\cap A) \sqcup (E \cap B)}p^-d\mu= \int_{(E\cap 
	A)}p^+ d\mu + \int_{(E\cap B)}p^+ d\mu - \int_{(E\cap A)}p^- d\mu - \int_{(E\cap B)}p^- d\mu.$ Since $p^+$ is $0$ on $B,$ and $p^-$  is $0$ on $A,$ we 
	have that $\nu^+ \perp \nu^-.$ Thus we have the Jordan decomposition.  
\end{itemize}

\section{} %Problem 5 
Let us assume that $\mu(E)=0.$ As $E$ is $\mu-$null, then $\mu(E\cap E_n)=0$ for all $n,$ by monotonicity of the measure. Then we have 
$\nu(E)=\sum_{n=1}^{N}c_n \mu(E \cap E_n)=0.$ Thus $\nu << \mu.$ See that the function $f:= \sum_{n=1}^{N}c_n \chi_{E_n}$ is a good candidate for the 
Radon-Nikodym derivative. $$\int_{E} f d\mu= \int_{E}\sum_{n=1}^{N}c_n \chi_{E_n} d\mu= \sum_{n=1}^{N}c_n \int_{X}\chi_{E} \chi_{E_n} d\mu= 
\sum_{n=1}^{N}c_n \int_{X}\chi_{E \cap E_n} d\mu=\sum_{n=1}^{N}c_n \mu(E \cap E_n),$$ which is the desired result. Thus $\frac{d\nu}{d\mu}=f.$
\section{} %Problem 6 
\begin{enumerate}
	\item Since $\nu << \mu,$ there exists $f \in L^1(\mu)$ such that $\nu(E)=\int_E f d\mu.$ We know that $f>0 \mu-$almost everywhere, then assume that 
	$\nu(E)=0.$ Thus $\int_{E}f d\mu=0.$ Assume that $\mu(E)>0.$ Then $f$ is greater than zero on all of $E,$ thus $\int_E f d\mu > 0.$ However, since 
	$\nu(E)=0$ this forces $\mu(E)$ to be $0.$ Thus $\mu << \nu.$
	
	\item 
\end{enumerate}
\section{} %Problem 7 
Let $\theta:= \mu + \nu.$ Then $f=\frac{d\nu}{d\theta}.$ See that since $\theta(E)=\int_{E}1 d\theta= \mu(E)+ \int_{E} f d\theta.$ Therefore we have 
$\mu(E)=\int_{E}(1-f)d\theta.$ Thus we have $\frac{d\mu}{d\theta}=1-f.$  
\section{} %Problem 8 
In $(\mathbb{N},\mathbb{P(N)}),$ $\mu$ is the counting measure. Note that the empty set is the only $\mu-$null set, since every non-empty set has 
cardinality more than zero. Then somewhat trivially we have $\mu(E)=0 \implies E=\phi \implies \nu(E)=0.$ So $\nu << \mu.$ We have $\nu$ is $\sigma-$finite, 
thus $\mathbb{N}=\sum_{n=1}^{\infty}\{n\},$ where $\nu(\{n\})<\infty.$ Thus define $f: \mathbb{N} \rightarrow \mathbb{R},$ where $f(n)=\nu(\{n\}).$ Then we 
have $\nu(E)=\int_E f d\mu=\sum_{n \in E}f(n),$ is the required function. Thus $f=\frac{d\nu}{d\mu}.$
\section{} %Problem 9 
If we assume that $\frac{d\mu}{d\lambda}\cdot \frac{d\nu}{d\lambda}=0$ $\lambda$ almost everywhere, we know from the previous assignment that $\mu(E)$ and 
$\nu(E)$ are mutually singular. To see the converse, let us assume that $\mu \perp \nu.$ Then we have $X=A \sqcup B$, where $\mu$ lives on $A$ while $\nu$ 
lives on $B.$ Then let $f:= \frac{d\mu}{d\lambda},$ and $g:=\frac{d\nu}{d\lambda}.$ See that $$\int_{E}fg d\lambda=\int_{(E\cap A)}fg d\lambda + 
\int_{(E\cap B)} fg d\lambda.$$ We know that $ \int_{(E\cap A)}fg d\lambda=\int_{(E\cap A)}f d\nu,$ and $ \int_{(E\cap B)}fg d\lambda=\int_{(E\cap B)}g 
d\mu.$ As $E \cap A$ is a $ \nu-$null set and $E \cap B$ is a $\mu-$null set, we have  $\int_{E}fg d\lambda=0 \implies fg=0$  $\lambda-$almost everywhere.
\section{} %Problem 10 
$\nu=\nu^+-\nu^-$ is a signed measure, where we have $x=P \sqcup N,$ that is, $\nu^+(E)=\nu(E \cap P),$ and $\nu^-(E)=\nu(E \cap N).$
Taking $|\nu|=\nu^+ +\nu^-,$ see that $ \nu^+ << |\nu|$ and $\nu^- << |\nu|,$ and thus there must exist $\frac{d\nu^+}{d|\nu|}$ and 
$$\frac{d\nu^-}{d|\nu|},$$ the Radon-Nikodym derivatives. Then see that $\frac{d\nu^+}{d|\nu|}=\chi_P.$ To see this, see that for $E$ measurable in $P,$
\begin{align*}
	\int_E \chi_{P}d|\nu| &= \int_E \chi_P d\nu^+ + \int_E \chi_P d\nu^-\\
	&= \nu^+(E\cap P)  + 0 = \nu^+(E),
\end{align*}
and similarly for $\nu^-,$ $\frac{d\nu^-}{d|\nu|}=\chi_N.$ 
\section{} %Problem 11
See that \begin{align*}
	\left| \int_X f d\nu \right| &\leq \left|\int_{X}f d\nu^+ - \int_{X}f d\nu^-  \right|\\
	&= \left|\int_{X}f d\nu^+\right| + \left|\int_{X}f d\nu^-\right| \leq \int_{X}|f| d\nu^+ + \int_{X}|f| d\nu^-\\
	&= \int_{X}|f| d|\nu|,
\end{align*}
as desired. 
\section{} %Problem 12
We know that the counting measure $\mu$ only has one $\nu-$null set, that is $\phi.$ Let us say that $\mu << \nu.$ Then for $\nu(E)=0$ for some measurable 
$E \in \mathcal{A},$ we have $\mu(E)=0.$ To take the contrapositive, we see that if $\mu(E)\neq 0,$ then $\nu(E) \neq 0.$ Since we know that $\mu$ has no 
non-empty null sets, non $\mu-$null sets and non-empty subsets are synonymous. Thus, if $E \neq \phi,$ $\nu(E) \neq 0.$

For the Dirac measure at $x_0 \in X,$ we know that $\delta_{x_0} << \nu$ means that $\nu(E)=0$ for some $E \in \mathcal{A}$ implies that 
$\delta_{x_0}(E)=0.$ The Dirac measure is zero if the measurable subset $E$ does not have $x_0.$ Thus we can take the contrapositive of the absolute 
continuity to say that if $x_0 \in E,$ then we must necessarily have $\nu(E)\neq 0.$ 
\end{document}

















