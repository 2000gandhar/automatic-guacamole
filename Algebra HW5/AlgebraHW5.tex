%%%%%%%%%%%%%%%%%%%%%%%%%%%%%%%%%%%%%%%%%
% Lachaise Assignment
% LaTeX Template
% Version 1.0 (26/6/2018)
%
% This template originates from:
% http://www.LaTeXTemplates.com
%
% Authors:
% Marion Lachaise & François Févotte
% Vel (vel@LaTeXTemplates.com)
%
% License:
% CC BY-NC-SA 3.0 (http://creativecommons.org/licenses/by-nc-sa/3.0/)
% 
%%%%%%%%%%%%%%%%%%%%%%%%%%%%%%%%%%%%%%%%%

%----------------------------------------------------------------------------------------
%	PACKAGES AND OTHER DOCUMENT CONFIGURATIONS
%----------------------------------------------------------------------------------------

\documentclass{article}

%%%%%%%%%%%%%%%%%%%%%%%%%%%%%%%%%%%%%%%%%
% Lachaise Assignment
% Structure Specification File
% Version 1.0 (26/6/2018)
%
% This template originates from:
% http://www.LaTeXTemplates.com
%
% Authors:
% Marion Lachaise & François Févotte
% Vel (vel@LaTeXTemplates.com)
%
% License:
% CC BY-NC-SA 3.0 (http://creativecommons.org/licenses/by-nc-sa/3.0/)
% 
%%%%%%%%%%%%%%%%%%%%%%%%%%%%%%%%%%%%%%%%%

%----------------------------------------------------------------------------------------
%	PACKAGES AND OTHER DOCUMENT CONFIGURATIONS
%----------------------------------------------------------------------------------------

\usepackage{amsmath,amsfonts,amssymb, tikz-cd} % Math packages

\usepackage{enumerate} % Custom item numbers for enumerations


\usepackage[framemethod=tikz]{mdframed} % Allows defining custom boxed/framed environments

\usepackage{listings} % File listings, with syntax highlighting
\lstset{
	basicstyle=\ttfamily, % Typeset listings in monospace font
}

%----------------------------------------------------------------------------------------
%	DOCUMENT MARGINS
%----------------------------------------------------------------------------------------

\usepackage{geometry} % Required for adjusting page dimensions and margins

\geometry{
	paper=letterpaper, % Paper size, change to letterpaper for US letter size
	top=2.5cm, % Top margin
	bottom=3cm, % Bottom margin
	left=2.5cm, % Left margin
	right=2.5cm, % Right margin
	headheight=14pt, % Header height
	footskip=1.5cm, % Space from the bottom margin to the baseline of the footer
	headsep=1.2cm, % Space from the top margin to the baseline of the header
	%showframe, % Uncomment to show how the type block is set on the page
}

%----------------------------------------------------------------------------------------
%	FONTS
%----------------------------------------------------------------------------------------

\usepackage[utf8]{inputenc} % Required for inputting international characters
\usepackage[T1]{fontenc} % Output font encoding for international characters


%----------------------------------------------------------------------------------------
%	COMMAND LINE ENVIRONMENT
%----------------------------------------------------------------------------------------

% Usage:
% \begin{commandline}
	%	\begin{verbatim}
		%		$ ls
		%		
		%		Applications	Desktop	...
		%	\end{verbatim}
	% \end{commandline}

\mdfdefinestyle{commandline}{
	leftmargin=10pt,
	rightmargin=10pt,
	innerleftmargin=15pt,
	middlelinecolor=black!50!white,
	middlelinewidth=2pt,
	frametitlerule=false,
	backgroundcolor=black!5!white,
	frametitle={Command Line},
	frametitlefont={\normalfont\sffamily\color{white}\hspace{-1em}},
	frametitlebackgroundcolor=black!50!white,
	nobreak,
}

% Define a custom environment for command-line snapshots
\newenvironment{commandline}{
	\medskip
	\begin{mdframed}[style=commandline]
	}{
	\end{mdframed}
	\medskip
}

%----------------------------------------------------------------------------------------
%	FILE CONTENTS ENVIRONMENT
%----------------------------------------------------------------------------------------

% Usage:
% \begin{file}[optional filename, defaults to "File"]
	%	File contents, for example, with a listings environment
	% \end{file}

\mdfdefinestyle{file}{
	innertopmargin=1.6\baselineskip,
	innerbottommargin=0.8\baselineskip,
	topline=false, bottomline=false,
	leftline=false, rightline=false,
	leftmargin=2cm,
	rightmargin=2cm,
	singleextra={%
		\draw[fill=black!10!white](P)++(0,-1.2em)rectangle(P-|O);
		\node[anchor=north west]
		at(P-|O){\ttfamily\mdfilename};
		%
		\def\l{3em}
		\draw(O-|P)++(-\l,0)--++(\l,\l)--(P)--(P-|O)--(O)--cycle;
		\draw(O-|P)++(-\l,0)--++(0,\l)--++(\l,0);
	},
	nobreak,
}

% Define a custom environment for file contents
\newenvironment{file}[1][File]{ % Set the default filename to "File"
	\medskip
	\newcommand{\mdfilename}{#1}
	\begin{mdframed}[style=file]
	}{
	\end{mdframed}
	\medskip
}

%----------------------------------------------------------------------------------------
%	NUMBERED QUESTIONS ENVIRONMENT
%----------------------------------------------------------------------------------------

% Usage:
% \begin{question}[optional title]
	%	Question contents
	% \end{question}

\mdfdefinestyle{question}{
	innertopmargin=1.2\baselineskip,
	innerbottommargin=0.8\baselineskip,
	roundcorner=5pt,
	nobreak,
	singleextra={%
		\draw(P-|O)node[xshift=1em,anchor=west,fill=white,draw,rounded corners=5pt]{%
			Question \theQuestion\questionTitle};
	},
}

\newcounter{Question} % Stores the current question number that gets iterated with each new question

% Define a custom environment for numbered questions
\newenvironment{question}[1][\unskip]{
	\bigskip
	\stepcounter{Question}
	\newcommand{\questionTitle}{~#1}
	\begin{mdframed}[style=question]
	}{
	\end{mdframed}
	\medskip
}

%----------------------------------------------------------------------------------------
%	WARNING TEXT ENVIRONMENT
%----------------------------------------------------------------------------------------

% Usage:
% \begin{warn}[optional title, defaults to "Warning:"]
	%	Contents
	% \end{warn}

\mdfdefinestyle{warning}{
	topline=false, bottomline=false,
	leftline=false, rightline=false,
	nobreak,
	singleextra={%
		\draw(P-|O)++(-0.5em,0)node(tmp1){};
		\draw(P-|O)++(0.5em,0)node(tmp2){};
		\fill[black,rotate around={45:(P-|O)}](tmp1)rectangle(tmp2);
		\node at(P-|O){\color{white}\scriptsize\bf !};
		\draw[very thick](P-|O)++(0,-1em)--(O);%--(O-|P);
	}
}

% Define a custom environment for warning text
\newenvironment{warn}[1][Warning:]{ % Set the default warning to "Warning:"
	\medskip
	\begin{mdframed}[style=warning]
		\noindent{\textbf{#1}}
	}{
	\end{mdframed}
}

%----------------------------------------------------------------------------------------
%	INFORMATION ENVIRONMENT
%----------------------------------------------------------------------------------------

% Usage:
% \begin{info}[optional title, defaults to "Info:"]
	% 	contents
	% 	\end{info}

\mdfdefinestyle{info}{%
	topline=false, bottomline=false,
	leftline=false, rightline=false,
	nobreak,
	singleextra={%
		\fill[black](P-|O)circle[radius=0.4em];
		\node at(P-|O){\color{white}\scriptsize\bf i};
		\draw[very thick](P-|O)++(0,-0.8em)--(O);%--(O-|P);
	}
}

% Define a custom environment for information
\newenvironment{info}[1][Info:]{ % Set the default title to "Info:"
	\medskip
	\begin{mdframed}[style=info]
		\noindent{\textbf{#1}}
	}{
	\end{mdframed}
}
 % Include the file specifying the document structure and custom commands

%----------------------------------------------------------------------------------------
%	ASSIGNMENT INFORMATION
%----------------------------------------------------------------------------------------

\title{Algebra HW5} % Title of the assignment

\author{Gandhar Kulkarni (mmat2304)} % Author name and email address

\date{} % University, school and/or department name(s) and a date

%----------------------------------------------------------------------------------------

\begin{document}

\maketitle % Print the title

%----------------------------------------------------------------------------------------
%	INTRODUCTION
%----------------------------------------------------------------------------------------

\section{} %Problem 1 
\begin{enumerate}
	\item See that $bx-a \in \ker \pi,$ since $b \cdot \left( \frac{a}{b} \right)-a=0.$ Therefore $(bx-a) \subseteq \ker \pi.$ Now consider $f(x) \in R[x]$ where $f(x) \in \ker \pi.$ We consider the polynomials $f(x)$ and $x-\frac{a}{b}$ as elements of $Q[x]$ the ring of polynomials with coefficients from the fraction field of $R.$ Then this is a PID, which is why we can apply the division algorithm to see that $f(x)=q(x)\left(x-\frac{a}{b}\right)+c,$ where $c \in Q, q(x) \in Q[x].$ Setting $x=\frac{a}{b}$ we get $c=0.$ Thus we have $f(x)=q(x)\left(x-\frac{a}{b}\right).$ We rewrite all polynomials are primitive polynomials in $R[x];$ thus we have $f(x)=a_1 \cdot f_0(x),$ $q(x)=a_2 \cdot q_0(x),$ and $x-\frac{a}{b}=b^{-1} \cdot (bx-a).$ Then we have $ a_1 \cdot f_0(x)= (a_2b^{-1}) q_0(x) (bx-a).$ We multiply on both sides by some $k \in R$ such that $ka_1b \in R $ and $ ka_2 \in R$ and the two are coprime. The constant cannot divide the polynomials as they are all primitive, hence we must have $ka_1b  |  ka_2,$ and by Gauss' lemma we can say that $f_0(x) | q_0(x)(bx-a),$ that is, $f \in (bx-a).$ Thus we have $\ker \pi=(bx-a).$
	
	\item Note that $(1+\sqrt{-3})\cdot (1-\sqrt{-3})=2 \cdot 2 = 4,$ which means that $R$ is not a UFD. Therefore the above result needn't apply. To show that the above result strictly does not apply, we need to find $f \in \ker \pi$ such that $f \notin (2x-(1+\sqrt{-3})).$ See that $f(x)=x^2-x+1$ does the trick well. It is in fact the minimal polynomial, but it is not in $(2x-(1+\sqrt{-3})).$ It is easy to prove, as the ideal of leading coefficients $R \cap (2x-(1+\sqrt{-3}))=(2),$ and this clearly does not include the leading coefficient of $f(x).$ Thus $f \in \ker \pi \backslash (2x-(1+\sqrt{-3})).$
	
	The underlying reason for why this fails stems from the fact that the ring of integers of the number field $\mathbb{Q}(\sqrt{-3})$ is $\mathbb{Z}\left[ \frac{1+\sqrt{-3}}{2}\right] \supsetneq \mathbb{Z}[\sqrt{-3}];$ that is, the ring of integers is strictly larger than $R$ as given in this problem. This has to do with the fact that $-3 \cong 1 \mod 4,$ which introduces interesting additional algebraic integers into the number field. 
\end{enumerate}
\section{} %Problem 2
\section{} %Problem 3 
Let $R=\frac{F[x,y]}{xy},$ and $I=(\bar{x},\bar{y}).$ Then $$\frac{R}{I} \equiv \frac{\frac{F[x,y]}{(xy)}}{(\bar{x},\bar{y})} \equiv \frac{F[x,y]}{(xy,x,y)} \equiv \frac{F[x]}{0 \cdot x,x} \equiv F,$$ which is a domain. Thus $I$ is prime. To see that it is not principal, we assume for the sake of contradiction that $I=(f_0),$ where $f_0 \in F[x,y].$ Note that once seen modulo $(xy),$ we have $\bar{f_0}=f_1(x)+f_2(y),$ where $f_1 \in F[x], f_2 \in F[y].$ If we say that $(f_0)=(\bar{x},\bar{y}),$ then we have $f_0 | x$ and $f_0 | y.$ $f_0(x,y)=f_1(x)+f_2(y)$ must have degree less than  or equal to $1,$ with no term of $y,$ hence $f_2(y)=c_2$ and $f_1(x)=c_1+dx.$ Set $c=c_1+c_2,$ then see that we must have $x=t(c+dx)$ for some $t \in F[x,y].$ Comparing degrees, we must have $t \in F \backslash \{0\}.$ Comparing the two sides, see that $c=0, d=1/t$ which is the only possibility. However, $x \nmid y,$ so such a $f_0$ cannot exist. Thus $I$ is not principal.

 
\section{} %Problem 4 
Let $$A=\begin{pmatrix}
	1 & 2 & 2 &3 \\
	5 & 5 & 4 & 4\\
	6 & 7 & 7 & 8\\
	10 & 10 & 9 & 9
\end{pmatrix}.$$ Through a myriad set of row and column operations shall we reduce our matrix $A$ to form that will generate an alike cokernel. We shall use $R_1,R_2,$ and $R_3$ to denote the rows, while $C_1,C_2,$ and $C_3$ shall denote the columns of $A.$ First we execute $C_2 \mapsto C_2-C_1, C_4 \mapsto C_4 - C_3.$ Then we execute $C_4 \mapsto C_4 - C_2$ to get $$A'=\begin{pmatrix}
1 & 1 & 2 & 0\\
5 & 0 & 4 & 0\\
6 & 1 & 7 & 0\\
10 & 0 & 9 & 0
\end{pmatrix}.$$
Execute $C_2 \mapsto C_2 - C_1, C_3 \mapsto C_3- 2 C_1$ to get $$A''=\begin{pmatrix}
	1 & 0 & 0 & 0\\
	5 & -5 & -6 & 0\\
	6 & -5 & -5 & 0\\
	10 & -10 & -11 &0
\end{pmatrix}. $$
Execute $R_2 \mapsto R_2 -R_1, R_3 \mapsto R_3- 6R_1,$ and $R_4 \mapsto R_4 - 10R_1.$ After this, execute $R_3 \mapsto R_3-R_2$ and $R_4 \mapsto R_4 - 2R_2$ to get $$A'''=\begin{pmatrix}
	1 & 0 & 0 & 0\\
	0 & -5 & -6 & 0\\
	0 & 0 & 1 & 0\\
	0 & 0 & 1 & 0
\end{pmatrix}.$$ Execute $R_2 \mapsto -R_2,$ and $R_4 \mapsto R_4 - R_3.$ After this execute $R_2 \mapsto R_2 - 6R_3$ to get $$A''''=\begin{pmatrix}
	1 & 0 & 0 & 0\\
0 & 5 & 0 & 0\\
0 & 0 & 1 & 0\\
0 & 0 & 0 & 0
\end{pmatrix}.$$

We are well aware that the cokernel is left unchanged due to our row and column operations. Thus it can be seen that the image of this matrix is $A''''(x_1 x_2 x_3 x_4)^T=(x_1 5x_2 x_3 0).$ Therefore the image is $\mathbb{Z} \oplus 5\mathbb{Z} \oplus \mathbb{Z} \oplus 0.$ Then $\text{coker} A=\frac{Z^4}{\mathbb{Z} \oplus 5\mathbb{Z} \oplus \mathbb{Z} \oplus 0}= \frac{\mathbb{Z}}{5\mathbb{Z}}.$

\section{} %Problem 5 
\begin{enumerate}
	\item Since $f \circ g=0,$ we have $f(g(p))=0 \forall p \in P.$ Then we have $g(p) \in \ker f \forall p \in P \implies g(P) \subseteq \ker f.$ Thus we can define $h: P \rightarrow \ker f$ as $h(p)=g(p).$ If another $h': P \rightarrow \ker f$ exists such that $g=i \circ h',$ then we have $g=i \circ h= i \circ h'.$ Since $i$ is injective, for all $p \in P$ we have $i(h(p))=i(h'(p)) \implies h(p)=h'(p),$ thus we have $h=h',$ proving the uniqueness of $h.$
	
	\item Let $h(\bar{n})=g\circ \pi^{-1}(\bar{n}),$ for $\bar{n} \in \text{coker} f.$ We propose that this is the desired map. We need to see that this map is well defined. $\pi^{-1}(\bar{n})= n + f(M),$ for some $n \in N.$ We need to see that the choice of representative does not matter. We can see that since $g$ is $R-$linear we have $g(n + f(M))= g(n) + g \circ f(M)=g(n) \in P.$ Thus this map is well defined. To see that this map is unique, for another such map $h': \text{coker} f \rightarrow P$ such that $ g=h \circ \pi,$ we have $g=h \circ \pi= h' \circ \pi,$ which implies that $h=h'$ is surjective, where right cancellation is possible. Thus this map is unique. 
\end{enumerate}
\section{} %Problem 6 
We can see that $(0) \subseteq \ker f \subseteq \ker f^2 \subseteq \dots$ which is an ascending chain of submodules of $M.$ This clearly must stabilise as the Noetherian condition is equivalent to the ascending chain condition. That is, for some $n \in \mathbb{N}$ we have $\ker f^n=\ker f^{n+1}=\ker f^{n+2}=\dots.$ Now see that for some $m \in \ker f$ we have $f(m)=0.$ Since $f$ is surjective, we can find a $m' \in M$ such that $f(m')=m.$ Repeating this process, see that there must exist some $m_n \in M$ such that $f^n(m_n)=m.$ Applying $f$ on both sides, we have $f^{n+1}(m_n)=f(m)=0.$ Thus $m_n \in \ker f^{n+1}=\ker f^n,$ we must have $f^n(m_n)=m=0.$ Thus $m$ must necessarily be zero, meaning that a surjective endomorphism on a Noetherian module must necessarily be injective, and thus an isomorphism. 
\end{document}

















