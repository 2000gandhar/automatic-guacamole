\documentclass[letterpaper,11pt,twoside]{article}
\usepackage[utf8]{inputenc}
\usepackage{enumitem}
\setlist{nosep}
\usepackage{graphicx}
\usepackage{amsmath,amssymb,amsfonts,amsthm}
\usepackage{tikz-cd}
\usepackage[margin=0.9in,
left=1.25in,%
right=1.25in,%
top=1.25in,%
bottom=1.25in
]{geometry}	
%\usepackage{stmaryrd} %For mapsfrom

%\usepackage{quiver}
\usepackage{bm}
\usepackage{fancyhdr}
\usepackage{mathrsfs}
\usepackage{amsbsy}
\usepackage{titlesec}
%\usepackage{yhmath}
%\usepackage{mathabx,epsfig}


%Hyperref Settings------
\usepackage{hyperref}
\usepackage{xcolor}
\hypersetup{
	colorlinks,
	linkcolor={black},
	citecolor={red!50!black}
	urlcolor={green!80!black}
}

%%%%%% TITLE %%%%%

\title{Algebra 2 Homework 8}
%\date{\today}


%MATH BACKGROUND DECLARATORS-------------------------------------------------------
\theoremstyle{proposition}
\newtheorem{proposition}{Proposition}[section]

\theoremstyle{definition}
\newtheorem{definition}{Definition}[section]

\theoremstyle{theorem}
\newtheorem{theorem}{Theorem}[section]

\theoremstyle{definition}
\newtheorem{remark}{\textbf{Remark}}[section]

\theoremstyle{definition}
\newtheorem{notation}{\textbf{Notation}}[section]

\theoremstyle{definition}
\newtheorem{discussion}{\textbf{Discussion}}[section]

\theoremstyle{lemma}
\newtheorem{lemma}{\textbf{Lemma}}[section]

\theoremstyle{definition}
\newtheorem{example}{\textbf{Example}}[section]

%\theoremstyle{remark}
%\newtheorem*{comment}{\textbf{Comments on Proof Technique}}

\theoremstyle{definition}
\newtheorem{construct}{Construction}[section]

\theoremstyle{corollary}
\newtheorem{corollary}{Corollary}[section]

\theoremstyle{definition}
\newtheorem{caution}{\textbf{Caution}}[section]


\theoremstyle{definition}
\newtheorem{question}{\textbf{Question}}[section]

\theoremstyle{definition}
\newtheorem{para}{}[section]


%--------------------------------------------------------------------------------- SOME USEFUL MACROS.


\newcommand{\N}{\mathbb{N}}
\newcommand{\Z}{\mathbb{Z}}
\newcommand{\C}{\mathbb{C}}
\newcommand{\R}{\mathbb{R}}
\DeclareMathOperator{\GL}{\text{\rm GL}}
\newcommand{\Ker}[1]{{\fontfamily{lmss}\selectfont 
		\text{\rm Ker}\left (#1\right )
}}
\newcommand{\nsg}{\trianglelefteq}
\newcommand{\abs}[1]{\left \vert #1 \right \vert}
\newcommand{\gen}[1]{\left\langle #1\right\rangle}
\newcommand{\norm}[1]{\left \vert \left \vert #1 \right \vert \right \vert}
\renewcommand{\div}{\;\vert\;}
\newcommand{\isom}{\cong}
\DeclareMathOperator{\Stab}{\text{\rm Stab}}
\newcommand{\Image}[1]{{\fontfamily{lmss}\selectfont 
		\text{\rm Im}\left (#1\right )
}}
\DeclareMathOperator{\Bij}{\text{\rm Bij}}
\DeclareMathOperator{\acts}{\rotatebox[origin=c]{-90}{$\circlearrowright$}}
\DeclareMathOperator{\Orb}{\text{\rm Orb}}
\DeclareMathOperator{\lcm}{\text{\rm lcm}}
\newcommand{\floor}[1]{\left \lfloor #1 \right \rfloor}
\DeclareMathOperator{\Aut}{\text{\rm Aut}}
\DeclareMathOperator{\Inn}{\text{\rm Inn}}
\DeclareMathOperator{\id}{\text{\rm id}}
\newcommand{\F}{\mathbb{F}}




\begin{document}
	\maketitle
	%\tableofcontents
	\begin{proof}[Solution of problem $1$:]
		Since $x^3+ax+b$ is irreducible, then the discriminant $\Delta= -4a^3-27b^3$ is a square if and only if the Galois group is $A_3.$ The splitting 
		field of $F_{p^n}$ must be isomorphic to $F_{p^{3n}},$ since our polynomial is irreducible. Then since $[F_{p^{3n}}:F_{p^{n}}]=3,$ which is the 
		order of the Galois group of the splitting field, we have that the Galois group is $A_3,$ hence $\Delta$ is a square. 
	\end{proof}
	\begin{proof}[Solution of problem $2$:]
		The resolvent of the polynomial $x^4+2x^2+x+3$ is $ x^3-4x^2-8x-1.$ See that modulo $3$ the polynomial $x^3-x^2+x-1$ is irreducible, and thus it 
		must be irreducible in $\mathbb{Q}.$ The discriminant of this polynomial is $3877,$ which is not a square, thus the Galois group is $S_4.$ 
	\end{proof}
	\begin{proof}[Solution of problem $3$:]
	If $K$ has $x^4+ax^2+b$ as its minimal polynomial, then we can do some calculations to see that $$K= \mathbb{Q} \left[ \sqrt{ \frac{-a + 
	\sqrt{a^2-4b}}{2} }, \sqrt{ \frac{-a - \sqrt{a^2-4b}}{2} } \right].$$ (If $a^2-4b$ is not a square, and the two elements adjoined to $\mathbb{Q}$ aren't 
	squares, then we can do the next steps). We can then see that $\sqrt{a^2-4b} \in K,$ since if $\alpha= \sqrt{ \frac{-a + \sqrt{a^2-4b}}{2} },$ then $ 
	\sqrt{a^2-4b} = 2\alpha^2 + a \in K.$ Clearly, $\mathbb{Q}( \sqrt{a^2-4b} )$ is a quadratic extension that will lie in $K.$ 

	Let $F$ contain $\mathbb{Q}(\sqrt{\alpha}),$ a field of degree $2.$ Then we must have that $F$ is a quadratic extension of this field, hence we must 
	have $ F= \mathbb{Q}( \sqrt{a+ \sqrt{\alpha}}).$ It can now be seen that the minimal polynomial for $\sqrt{a+ \sqrt{\alpha}}$ must be a biquadratic 
	polynomial.
	\end{proof}
	\begin{proof}[Solution of problem $4$:]
		\begin{enumerate}
			\item The automorphisms of $\text{Gal}(K/F)$ are cyclic of order $n.$ Let $\sigma$ be an automorphism .Then we only need to see where $\sigma$ 
			sends $\sqrt[n]{a}.$ Clearly $\sigma( \sqrt[n]{a} )= (\zeta_n)^i \sqrt[n]{a},$ where $\zeta_n$ is a primitive $n$th root of unity, and $i \in 
			\mathbb{Z}.$ Since $\sigma^d = i,$ $\zeta_n^i$ is a $d$th root of unity.
			\item See that $ \frac{\sigma(\sqrt[n]{a})}{\sqrt[n]{a}}$ and $\frac{\sigma(\sqrt[n]{b})}{\sqrt[n]{b}}$ both are primitive $d$th roots of unity. 
			Then we must have that the two are such that  $ \frac{\sigma(\sqrt[n]{a})}{\sqrt[n]{a}}= 
			\left(\frac{\sigma(\sqrt[n]{b})}{\sqrt[n]{b}}\right)^i,$ for some $i $ coprime to $d.$ See that $ \sigma \left( 
			\frac{\sqrt[n]{a}}{\sqrt[n]{b}^i} \right)= \frac{\sqrt[n]{a}}{\sqrt[n]{b}^i},$ which means that this element lies in the fixed field of the 
			automorphism, which is $F.$ Thus it lies in $F.$ 
		
			\item If $K=F(\sqrt[n]{a})=F(\sqrt[n]{b}),$ then by the previous problem we have $a= b^i \left( \frac{\sqrt[n]{a}}{\sqrt[n]{b}^i} \right)^n,$ 
			and a similar expression for $b.$ The term in the brackets is in $F,$ which is the required result.
		\end{enumerate}
	\end{proof}
	\begin{proof}[Solution of problem $5$:]
		By Cauchy's theorem, $G$ has a subgroup $H$ of order $p,$ which gives us a subfield $F$ of $L$ such that $[L:F]=p.$ If we say that for all $\sigma 
		\in G$ we have $\sigma(\alpha) \in F,$ then we would have $F=L.$ This is not possible, hence there is some $\sigma \in G$ where $\sigma(\alpha) 
		\notin F.$ Since $P$ is prime and degree multiplies then $F(\sigma(\alpha))=L.$ See that $F'= \sigma^{-1}(F)$ is our required field.  
	\end{proof}
	\begin{proof}[Solution of problem $6$:]
		Any Galois extension of $F$ in $K=F(\sqrt[n]{a})$ is trivial if $n$ is odd and if $n$ is even then the only non-trivial Galois extension. In either 
		case, $[K:F] \leq 2.$ 
	\end{proof}
	\begin{proof}[Solution of problem $7$:]
	We know that $S_p$ is generated by a $p-$cycle and a transposition. To show this, see that we just need to check that any transposition can be generated 
	using these two. Without loss of generality, assume that the two permutations are $(1, 2)$ and $(1,2, \dots, p).$ Now see that $(m,k)= (1,m)(1,k)(1,m),$ 
	and $(1,k+1)=(1,k)(k, k+1)(1,k).$ Thus if we could generate $(k,k+1)$ for all $k$ then we are done since we could generate $(1,k )$ inductively. Now see 
	that $(k,k+1)= (1,2,\dots,p)(k-1, k) (1,2,\dots,p)^{-1},$ so inductively using $(1,2)$ we can generate the entire group $S_p.$ 
	
	We want to show that a polynomial with exactly $2$ non-real roots has its Galois group as $S_p.$ 
	Let $E$ be the splitting field of $f$ in $\mathbb{C},$ and let $\alpha \in E$ be a root of $f.$ $[\mathbb{Q}(\alpha): \mathbb{Q}]=p,$ so $p \div [E: 
	\mathbb{Q}].$ Thus the Galois group must contain an element of order $p$ by Cauchy's theorem, which gives us $p$ cycles in $S_p.$ If we consider 
	$\sigma,$ complex conjugation, then it must flip the two non-real roots, and fix the others. Then that gives us an element of order $2$ in the Galois 
	group, which generates $S_p.$
	
	
	Now pick a polynomial $f(x)=(x^2+m)(x-n_1)\dots (x- n_{p-2}),$ where $m >0,$ and $n_i = n_j \implies i=j,$ all even. Consider the polynomial $g(x)=f(x)- 
	2/n,$ where $n$ is such that $2/n < \varepsilon,$ where $\varepsilon = \min_{f'(x)=0}= \abs{ f(x) } >0.$ This must also have $p-2$ roots, and $2$ 
	non-real roots. Now see that $ng(x)$ fulfills Eisenstein's criterion since all coefficients of $x^i$ for $i<p$ are even, and the constant term does not 
	divide $4.$  This must have $S_p$ as its Galois group.
	\end{proof}
	
	
	
	

\end{document}