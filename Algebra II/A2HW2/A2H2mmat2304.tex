%%%%%%%%%%%%%%%%%%%%%%%%%%%%%%%%%%%%%%%%%
% Lachaise Assignment
% LaTeX Template
% Version 1.0 (26/6/2018)
%
% This template originates from:
% http://www.LaTeXTemplates.com
%
% Authors:
% Marion Lachaise & François Févotte
% Vel (vel@LaTeXTemplates.com)
%
% License:
% CC BY-NC-SA 3.0 (http://creativecommons.org/licenses/by-nc-sa/3.0/)
% 
%%%%%%%%%%%%%%%%%%%%%%%%%%%%%%%%%%%%%%%%%

%----------------------------------------------------------------------------------------
%	PACKAGES AND OTHER DOCUMENT CONFIGURATIONS
%----------------------------------------------------------------------------------------

\documentclass{article}

%%%%%%%%%%%%%%%%%%%%%%%%%%%%%%%%%%%%%%%%%
% Lachaise Assignment
% Structure Specification File
% Version 1.0 (26/6/2018)
%
% This template originates from:
% http://www.LaTeXTemplates.com
%
% Authors:
% Marion Lachaise & François Févotte
% Vel (vel@LaTeXTemplates.com)
%
% License:
% CC BY-NC-SA 3.0 (http://creativecommons.org/licenses/by-nc-sa/3.0/)
% 
%%%%%%%%%%%%%%%%%%%%%%%%%%%%%%%%%%%%%%%%%

%----------------------------------------------------------------------------------------
%	PACKAGES AND OTHER DOCUMENT CONFIGURATIONS
%----------------------------------------------------------------------------------------

\usepackage{amsmath,amsfonts,amssymb, tikz-cd} % Math packages

\usepackage{enumerate} % Custom item numbers for enumerations


\usepackage[framemethod=tikz]{mdframed} % Allows defining custom boxed/framed environments

\usepackage{listings} % File listings, with syntax highlighting
\lstset{
	basicstyle=\ttfamily, % Typeset listings in monospace font
}

%----------------------------------------------------------------------------------------
%	DOCUMENT MARGINS
%----------------------------------------------------------------------------------------

\usepackage{geometry} % Required for adjusting page dimensions and margins

\geometry{
	paper=letterpaper, % Paper size, change to letterpaper for US letter size
	top=2.5cm, % Top margin
	bottom=3cm, % Bottom margin
	left=2.5cm, % Left margin
	right=2.5cm, % Right margin
	headheight=14pt, % Header height
	footskip=1.5cm, % Space from the bottom margin to the baseline of the footer
	headsep=1.2cm, % Space from the top margin to the baseline of the header
	%showframe, % Uncomment to show how the type block is set on the page
}

%----------------------------------------------------------------------------------------
%	FONTS
%----------------------------------------------------------------------------------------

\usepackage[utf8]{inputenc} % Required for inputting international characters
\usepackage[T1]{fontenc} % Output font encoding for international characters


%----------------------------------------------------------------------------------------
%	COMMAND LINE ENVIRONMENT
%----------------------------------------------------------------------------------------

% Usage:
% \begin{commandline}
	%	\begin{verbatim}
		%		$ ls
		%		
		%		Applications	Desktop	...
		%	\end{verbatim}
	% \end{commandline}

\mdfdefinestyle{commandline}{
	leftmargin=10pt,
	rightmargin=10pt,
	innerleftmargin=15pt,
	middlelinecolor=black!50!white,
	middlelinewidth=2pt,
	frametitlerule=false,
	backgroundcolor=black!5!white,
	frametitle={Command Line},
	frametitlefont={\normalfont\sffamily\color{white}\hspace{-1em}},
	frametitlebackgroundcolor=black!50!white,
	nobreak,
}

% Define a custom environment for command-line snapshots
\newenvironment{commandline}{
	\medskip
	\begin{mdframed}[style=commandline]
	}{
	\end{mdframed}
	\medskip
}

%----------------------------------------------------------------------------------------
%	FILE CONTENTS ENVIRONMENT
%----------------------------------------------------------------------------------------

% Usage:
% \begin{file}[optional filename, defaults to "File"]
	%	File contents, for example, with a listings environment
	% \end{file}

\mdfdefinestyle{file}{
	innertopmargin=1.6\baselineskip,
	innerbottommargin=0.8\baselineskip,
	topline=false, bottomline=false,
	leftline=false, rightline=false,
	leftmargin=2cm,
	rightmargin=2cm,
	singleextra={%
		\draw[fill=black!10!white](P)++(0,-1.2em)rectangle(P-|O);
		\node[anchor=north west]
		at(P-|O){\ttfamily\mdfilename};
		%
		\def\l{3em}
		\draw(O-|P)++(-\l,0)--++(\l,\l)--(P)--(P-|O)--(O)--cycle;
		\draw(O-|P)++(-\l,0)--++(0,\l)--++(\l,0);
	},
	nobreak,
}

% Define a custom environment for file contents
\newenvironment{file}[1][File]{ % Set the default filename to "File"
	\medskip
	\newcommand{\mdfilename}{#1}
	\begin{mdframed}[style=file]
	}{
	\end{mdframed}
	\medskip
}

%----------------------------------------------------------------------------------------
%	NUMBERED QUESTIONS ENVIRONMENT
%----------------------------------------------------------------------------------------

% Usage:
% \begin{question}[optional title]
	%	Question contents
	% \end{question}

\mdfdefinestyle{question}{
	innertopmargin=1.2\baselineskip,
	innerbottommargin=0.8\baselineskip,
	roundcorner=5pt,
	nobreak,
	singleextra={%
		\draw(P-|O)node[xshift=1em,anchor=west,fill=white,draw,rounded corners=5pt]{%
			Question \theQuestion\questionTitle};
	},
}

\newcounter{Question} % Stores the current question number that gets iterated with each new question

% Define a custom environment for numbered questions
\newenvironment{question}[1][\unskip]{
	\bigskip
	\stepcounter{Question}
	\newcommand{\questionTitle}{~#1}
	\begin{mdframed}[style=question]
	}{
	\end{mdframed}
	\medskip
}

%----------------------------------------------------------------------------------------
%	WARNING TEXT ENVIRONMENT
%----------------------------------------------------------------------------------------

% Usage:
% \begin{warn}[optional title, defaults to "Warning:"]
	%	Contents
	% \end{warn}

\mdfdefinestyle{warning}{
	topline=false, bottomline=false,
	leftline=false, rightline=false,
	nobreak,
	singleextra={%
		\draw(P-|O)++(-0.5em,0)node(tmp1){};
		\draw(P-|O)++(0.5em,0)node(tmp2){};
		\fill[black,rotate around={45:(P-|O)}](tmp1)rectangle(tmp2);
		\node at(P-|O){\color{white}\scriptsize\bf !};
		\draw[very thick](P-|O)++(0,-1em)--(O);%--(O-|P);
	}
}

% Define a custom environment for warning text
\newenvironment{warn}[1][Warning:]{ % Set the default warning to "Warning:"
	\medskip
	\begin{mdframed}[style=warning]
		\noindent{\textbf{#1}}
	}{
	\end{mdframed}
}

%----------------------------------------------------------------------------------------
%	INFORMATION ENVIRONMENT
%----------------------------------------------------------------------------------------

% Usage:
% \begin{info}[optional title, defaults to "Info:"]
	% 	contents
	% 	\end{info}

\mdfdefinestyle{info}{%
	topline=false, bottomline=false,
	leftline=false, rightline=false,
	nobreak,
	singleextra={%
		\fill[black](P-|O)circle[radius=0.4em];
		\node at(P-|O){\color{white}\scriptsize\bf i};
		\draw[very thick](P-|O)++(0,-0.8em)--(O);%--(O-|P);
	}
}

% Define a custom environment for information
\newenvironment{info}[1][Info:]{ % Set the default title to "Info:"
	\medskip
	\begin{mdframed}[style=info]
		\noindent{\textbf{#1}}
	}{
	\end{mdframed}
}
 % Include the file specifying the document structure and custom commands

%----------------------------------------------------------------------------------------
%	ASSIGNMENT INFORMATION
%----------------------------------------------------------------------------------------

\title{} % Title of the assignment

\author{Gandhar Kulkarni (mmat2304)} % Author name and email address

\date{} % University, school and/or department name(s) and a date

%----------------------------------------------------------------------------------------

\begin{document}

\maketitle % Print the title

%----------------------------------------------------------------------------------------
%	INTRODUCTION
%----------------------------------------------------------------------------------------

\section{} %Problem 1 
To prove the result, we will decompose an arbitrary permutation $\sigma \in S_n$ into transpositions. We will then show that each transposition can be 
written as a product of transpositions of the form $(i i+1).$ We have $\sigma=\tau_1\dots\tau_r,$ where $\tau_i$ is some transposition of the form $(k_1 
k+1+k_2)_i.$ Note that we can assume the first element of $\tau_i$ is strictly lesser than the second since if it weren't we could just invert the order 
without any loss of generality. We state that $(k_1 k_1+k_2)$ can written as a product of finite transpositions of the form $(t t+1).$ See that $(k_1 
k_1+k_2-1)=(k_1+k_2-1 k_1+k_2)(k_1 k_1+k_2)(k_1+k_2-1 k_1+k_2).$ Since we have $(k_1 k_1+k_2-1),$ we have lowered the second entry of the transposition by 
one. In $k_2-1$ steps, we will get $(k_1 k_1+1)=(k_1+1 k_1+2)(k_1 k_1+2)(k_1+1 k_1+2),$ which means that we can stop. We have $\tau_i$ as a product of 
$2(k_2-1)+1$ transpositions of the desired type. We can do this for all transpositions to get our result. Thus we can generate any permutation in $S_n$ by 
exchanging adjacent elements. 
The bubble sort algorithm works this way too, which accepts a permutation then returns the list of $n$ numbers. This is essentially the same problem.
\section{} %Problem 2
We want to find an isomorphism between $S_{n-2}$ and $A_n.$ We define $\varphi:S_{n-2} \to A_n$ thus:
$$\varphi(\sigma)= \begin{cases}
	\sigma  \text{ if } \sigma \text{ is even}\\
	\sigma \circ (n-1,n)  \text{ if } \sigma \text{ is odd}.
\end{cases} $$
We want to see that this is a homomorphism. If both $\sigma, \tau \in S_{n-2}$
are even, then there is nothing to prove. If both are odd, then 
$$\varphi(\sigma)\varphi(\tau)= \sigma \circ (n-1,n) \circ \tau \circ (n-1,n)= \sigma \circ \tau = \varphi(\sigma \circ \tau). $$
If $\sigma$ is odd and $\tau$ is even, we have $$\varphi(\sigma) \sigma(\tau)= \sigma \circ (n-1, n) \circ \tau = \sigma \circ \tau \circ (n-1,n)= 
\varphi(\sigma \circ \tau).$$

To find the kernel, see that $\varphi(\sigma)=e$ implies that either $\sigma=e$ if $\sigma$ is even, and if $\sigma$ is odd, $\sigma \circ (n-1,n)$ can 
never be the identity element. Thus the kernel is trivial.

This homomorphism restricted to the image gives us an isomorphism of $S_{n-2}$ to a subgroup of $A_n.$
\section{} %Problem 3 
\begin{enumerate}
	\item For some $1 \leq i \leq n,$ $G)i=\{\sigma \in S_n: \sigma(i)=i\} \cong S_{n-1}.$ Consider $G_i$ acting on $\{1,2,\dots,i-1,i+1,\dots,n\}.$
	If $n=2,$ this set will be a singleton, hence trivially transitive. For $n \geq 3,$ the set $\{1,2,\dots,n\}\backslash\{i\}$ has at least two elements. 
	Then $k,\ell \in \{1,2,\dots,i-1,i+1,\dots,n\}$ such that they are distinct (If $k=\ell,$ then $i \in G_i$ works). See that $ (k \ell) \in G_i,$ as it 
	does not affect $i.$ Then we have $(k \ell)k=\ell,$ which means that $G_i$ is transitive. 
	\item For a doubly transitive action of $G$ on $A,$ let us claim that this action is doubly transitive. Let $B \subseteq A$ be a proper block, then we 
	have elements $b \in B$ and $a \in A\backslash B.$ We have $\text{stab}(B):= \{g \in G: gB=B\}.$ See that we have $\text{stab}(b) \leq \text{stab}(B).$ 
	To prove this, we consider the map $f: \text{stab}(b) \to \text{stab}(B)$ that sends $g$ to itself. It is evident that $\text{stab}(b) \subseteq 
	\text{stab}(B)$ as sets. Thus we have this relation. So if we have $g \in \text{stab}(b),$ then $gB=B.$ Suppose there is an element $c \in B, c \neq b,$ 
	by double transitivity of $G$ on $A,$ there is a $\tau \in \text{stab}(b)$ such that $\tau(c)=a.$ But then $\tau B \neq B,$ a contradiction. Thus either 
	the block is a singleton or the entire set $A.$
\end{enumerate}
\section{} %Problem 4 
Let us denote all elements of $Q_8$ thus: $1,i,j,k,-1,-i,-j,-k$ are assigned the numbers from $1$ to $8.$ 
Then see that left multiplication by $1$ is the identity permutation on $S_8.$ Left multiplication by $i$ is $(1 2 5 6)(3 4 7 8),$ by $j$ is $(1 3 5 7)(2 8 
6 4),$ and by $k$ is $(1 4 5 8)(2 3 6 7).$ Their negatives also have a left regular representation. Now see that $i,j$ can generate $Q_8$ as a group, then 
it must stand to reason that their corresponding left regular representations will behave in the same way! Thus, see that $G=\langle (1 2 5 6)(3 4 7 8), (1 
3 5 7)(2 8 6 4) \rangle \cong Q_8.$
\section{} %Problem 5 
Consider the action of $G$ on the cosets $G/H$ by left multiplication. $\lambda: G \to S_{|G/H|}$ is the permutation representation of this action, and let 
$K$ be its kernel. $K$ is normal in $G,$ and we have $K \leq \text{stab}(H)=H.$ By the first isomorphism theorem, we have an injective homomorphism 
$\bar{\lambda}: G/K \to S_{|G/H|}.$ Since $|S_{|G/H|}|=n!,$ we have $[G:K] \leq n!.$
\section{} %Problem 6 
We shall prove a result that for $|G| =n$ and $p$ the smallest prime that divides $n,$ then a subgroup of order $p$ must be normal.
Let some $H \leq G,$ with $|[G:H]|=p.$ Consider the group action of left multiplication on left cosets of $H.$ This group action has a permutation 
representation, let us denote that by $\pi_H.$ Take $K=\ker \pi_H,$ and $|[H:K]| =k.$ 
Then we have $|[G:K]|=|[G:H]||H:K|=pk.$ We know that H has $p$ many left cosets, hence $\frac{G}{K}$ is isomorphic to some subgroup of $S_p$ which is the 
image of $G$ under $ \pi_H.$ Thus we must have $pk|p! \implies k|(p-1)!.$ But since $k$ can only have prime factors greater than or equal to $p$ and 
$(p-1)!$ has no prime factors greater than $p,$ we must have $k=1.$ Thus $H=K \trianglelefteq G$ is normal.
 
Let $p$ be the smallest prime dividing $n.$ We know that $p < n,$ since $n$ is composite. Then see that there must exist a subgroup of order $\frac{n}{p}$ 
as given in the problem, hence this subgroup has index $\frac{n}{\frac{n}{p}}=p,$ hence this is a normal subgroup. Thus $G$ cannot be simple. 
\section{} %Problem 7 
We know that $|[G:Z(G)]| =n.$ For some $g \in G,$ let $\text{Cl}(g)$ be the conjugacy class of $g.$ By orbit stabiliser theorem, we have $|\text{Cl}(g)| 
=\left|\frac{G}{C_{G}(g)}\right|,$ where $C_G(g)$ is the centraliser. But since $Z(G) \leq C_G(g),$ we have $\left|\frac{G}{C_{G}(g)}\right| \leq 
\left|\frac{G}{Z(G)}\right|=n, $ which is the required result.
\section{} %Problem 8 
Given a permutation $\sigma \in S_n,$ let $m_1,\dots, m_s$ be the distinct integers that appear in the cycle type of $\sigma $, including $1-$cycles. Let 
$k_1,\dots,k_s$ be the number of times the above cycles appear. Then see that for another $\tau \in S_{n}$ to be conjugate to it, we must have the exact 
cycle breakup. For this, we choose a permutation of $n$ integers, then determine where to `draw' brackets. For each repetition, we factor it out. Then from 
$n!,$ we have for any $m_i,$ and $k_i$ the number of times the cycle of length $m_i$ appears, we get $k_i!m_i^{k_i}$ many such equivalent cycles. For 
example, $(1 2 3)$ and $(2 3 1)$ are the same permutation. Thus, we end up with $$ \frac{n!}{(k_1!m_1^{k_1})\dots (k_s!m_s^{k_s})}.$$

Take any $\sigma \in S_n.$ Then every cycle must be of order $p,$ else the permutation will not have order $p.$ Thus $\sigma$ consists up any number of 
$p-$cycles, up to $\lfloor \frac{n}{p} \rfloor.$ If $\sigma$ has $k$ $p-$cycles, its conjugacy must have size $$\frac{n!}{k!p^k(n-kp)!}.$$
Summing this over all $k,$ we get $$\sum_{k=0}^{\lfloor \frac{n}{p} \rfloor} \frac{n!}{k!p^k(n-kp)!} ,$$ the required solution.
\section{} %Problem 9 
We know that $r$ is the greatest prime. Then let $n_r=rk+1$ be the number of Sylow $r-$subgroups. Since $rk_1+1|pq,$ we can have $rk_1+1=1,p,q,$ or $pq.$ If 
$rk_1+1$ is $p$ or $q,$ it would be absurd to have $k_1\neq 0$ as for $k_1=1$ $r+1 > p,q.$ Assume that $n_r=pq.$
Let $ qk_2+1$ be the number of Sylow $q-$subgroups. Then $qk_2+1=1,r,pr.$ Assume that $n_q=r.$ We have $n_p=pk_3+1$ as the number of Sylow $p-$subgroups. 
Then we have $n_p=1, q, r ,qr.$ Assume that $n_p=q.$ Let us count the number of elements in $G$ of order $p,q,r.$ Since $n_r=pq,$ the number of elements of 
order $r$ is $pq(r-1).$ The number of elements of order $q$ is $(q-1)r$ and the number of elements of order $p$ is $(p-1)q.$ Adding these up, for $q>1,$ we 
have a number that is greater than $pqr,$ which is not possible. Thus at least one of $n_p,n_q,n_r$ is one.    
\section{} %Problem 10 
We have $|G|=42 \cdot 11.$ Let $n_{11}=11k+1$ be the number of Sylow $11-$subgroups. Then $11k+1 | 42,$ which essentially forces $k=0.$ As $n_{11}=1,$ there 
is only one such subgroup which must be normal. Thus $G$ is not simple.
\end{document}
