%%%%%%%%%%%%%%%%%%%%%%%%%%%%%%%%%%%%%%%%%
% Lachaise Assignment
% LaTeX Template
% Version 1.0 (26/6/2018)
%
% This template originates from:
% http://www.LaTeXTemplates.com
%
% Authors:
% Marion Lachaise & François Févotte
% Vel (vel@LaTeXTemplates.com)
%
% License:
% CC BY-NC-SA 3.0 (http://creativecommons.org/licenses/by-nc-sa/3.0/)
% 
%%%%%%%%%%%%%%%%%%%%%%%%%%%%%%%%%%%%%%%%%

%----------------------------------------------------------------------------------------
%	PACKAGES AND OTHER DOCUMENT CONFIGURATIONS
%----------------------------------------------------------------------------------------

\documentclass{article}

%%%%%%%%%%%%%%%%%%%%%%%%%%%%%%%%%%%%%%%%%
% Lachaise Assignment
% Structure Specification File
% Version 1.0 (26/6/2018)
%
% This template originates from:
% http://www.LaTeXTemplates.com
%
% Authors:
% Marion Lachaise & François Févotte
% Vel (vel@LaTeXTemplates.com)
%
% License:
% CC BY-NC-SA 3.0 (http://creativecommons.org/licenses/by-nc-sa/3.0/)
% 
%%%%%%%%%%%%%%%%%%%%%%%%%%%%%%%%%%%%%%%%%

%----------------------------------------------------------------------------------------
%	PACKAGES AND OTHER DOCUMENT CONFIGURATIONS
%----------------------------------------------------------------------------------------

\usepackage{amsmath,amsfonts,amssymb, tikz-cd} % Math packages

\usepackage{enumerate} % Custom item numbers for enumerations


\usepackage[framemethod=tikz]{mdframed} % Allows defining custom boxed/framed environments

\usepackage{listings} % File listings, with syntax highlighting
\lstset{
	basicstyle=\ttfamily, % Typeset listings in monospace font
}

%----------------------------------------------------------------------------------------
%	DOCUMENT MARGINS
%----------------------------------------------------------------------------------------

\usepackage{geometry} % Required for adjusting page dimensions and margins

\geometry{
	paper=letterpaper, % Paper size, change to letterpaper for US letter size
	top=2.5cm, % Top margin
	bottom=3cm, % Bottom margin
	left=2.5cm, % Left margin
	right=2.5cm, % Right margin
	headheight=14pt, % Header height
	footskip=1.5cm, % Space from the bottom margin to the baseline of the footer
	headsep=1.2cm, % Space from the top margin to the baseline of the header
	%showframe, % Uncomment to show how the type block is set on the page
}

%----------------------------------------------------------------------------------------
%	FONTS
%----------------------------------------------------------------------------------------

\usepackage[utf8]{inputenc} % Required for inputting international characters
\usepackage[T1]{fontenc} % Output font encoding for international characters


%----------------------------------------------------------------------------------------
%	COMMAND LINE ENVIRONMENT
%----------------------------------------------------------------------------------------

% Usage:
% \begin{commandline}
	%	\begin{verbatim}
		%		$ ls
		%		
		%		Applications	Desktop	...
		%	\end{verbatim}
	% \end{commandline}

\mdfdefinestyle{commandline}{
	leftmargin=10pt,
	rightmargin=10pt,
	innerleftmargin=15pt,
	middlelinecolor=black!50!white,
	middlelinewidth=2pt,
	frametitlerule=false,
	backgroundcolor=black!5!white,
	frametitle={Command Line},
	frametitlefont={\normalfont\sffamily\color{white}\hspace{-1em}},
	frametitlebackgroundcolor=black!50!white,
	nobreak,
}

% Define a custom environment for command-line snapshots
\newenvironment{commandline}{
	\medskip
	\begin{mdframed}[style=commandline]
	}{
	\end{mdframed}
	\medskip
}

%----------------------------------------------------------------------------------------
%	FILE CONTENTS ENVIRONMENT
%----------------------------------------------------------------------------------------

% Usage:
% \begin{file}[optional filename, defaults to "File"]
	%	File contents, for example, with a listings environment
	% \end{file}

\mdfdefinestyle{file}{
	innertopmargin=1.6\baselineskip,
	innerbottommargin=0.8\baselineskip,
	topline=false, bottomline=false,
	leftline=false, rightline=false,
	leftmargin=2cm,
	rightmargin=2cm,
	singleextra={%
		\draw[fill=black!10!white](P)++(0,-1.2em)rectangle(P-|O);
		\node[anchor=north west]
		at(P-|O){\ttfamily\mdfilename};
		%
		\def\l{3em}
		\draw(O-|P)++(-\l,0)--++(\l,\l)--(P)--(P-|O)--(O)--cycle;
		\draw(O-|P)++(-\l,0)--++(0,\l)--++(\l,0);
	},
	nobreak,
}

% Define a custom environment for file contents
\newenvironment{file}[1][File]{ % Set the default filename to "File"
	\medskip
	\newcommand{\mdfilename}{#1}
	\begin{mdframed}[style=file]
	}{
	\end{mdframed}
	\medskip
}

%----------------------------------------------------------------------------------------
%	NUMBERED QUESTIONS ENVIRONMENT
%----------------------------------------------------------------------------------------

% Usage:
% \begin{question}[optional title]
	%	Question contents
	% \end{question}

\mdfdefinestyle{question}{
	innertopmargin=1.2\baselineskip,
	innerbottommargin=0.8\baselineskip,
	roundcorner=5pt,
	nobreak,
	singleextra={%
		\draw(P-|O)node[xshift=1em,anchor=west,fill=white,draw,rounded corners=5pt]{%
			Question \theQuestion\questionTitle};
	},
}

\newcounter{Question} % Stores the current question number that gets iterated with each new question

% Define a custom environment for numbered questions
\newenvironment{question}[1][\unskip]{
	\bigskip
	\stepcounter{Question}
	\newcommand{\questionTitle}{~#1}
	\begin{mdframed}[style=question]
	}{
	\end{mdframed}
	\medskip
}

%----------------------------------------------------------------------------------------
%	WARNING TEXT ENVIRONMENT
%----------------------------------------------------------------------------------------

% Usage:
% \begin{warn}[optional title, defaults to "Warning:"]
	%	Contents
	% \end{warn}

\mdfdefinestyle{warning}{
	topline=false, bottomline=false,
	leftline=false, rightline=false,
	nobreak,
	singleextra={%
		\draw(P-|O)++(-0.5em,0)node(tmp1){};
		\draw(P-|O)++(0.5em,0)node(tmp2){};
		\fill[black,rotate around={45:(P-|O)}](tmp1)rectangle(tmp2);
		\node at(P-|O){\color{white}\scriptsize\bf !};
		\draw[very thick](P-|O)++(0,-1em)--(O);%--(O-|P);
	}
}

% Define a custom environment for warning text
\newenvironment{warn}[1][Warning:]{ % Set the default warning to "Warning:"
	\medskip
	\begin{mdframed}[style=warning]
		\noindent{\textbf{#1}}
	}{
	\end{mdframed}
}

%----------------------------------------------------------------------------------------
%	INFORMATION ENVIRONMENT
%----------------------------------------------------------------------------------------

% Usage:
% \begin{info}[optional title, defaults to "Info:"]
	% 	contents
	% 	\end{info}

\mdfdefinestyle{info}{%
	topline=false, bottomline=false,
	leftline=false, rightline=false,
	nobreak,
	singleextra={%
		\fill[black](P-|O)circle[radius=0.4em];
		\node at(P-|O){\color{white}\scriptsize\bf i};
		\draw[very thick](P-|O)++(0,-0.8em)--(O);%--(O-|P);
	}
}

% Define a custom environment for information
\newenvironment{info}[1][Info:]{ % Set the default title to "Info:"
	\medskip
	\begin{mdframed}[style=info]
		\noindent{\textbf{#1}}
	}{
	\end{mdframed}
}
 % Include the file specifying the document structure and custom commands

%----------------------------------------------------------------------------------------
%	ASSIGNMENT INFORMATION
%----------------------------------------------------------------------------------------

\title{} % Title of the assignment

\author{Gandhar Kulkarni (mmat2304)} % Author name and email address

\date{} % University, school and/or department name(s) and a date

%----------------------------------------------------------------------------------------

\begin{document}

\maketitle % Print the title

%----------------------------------------------------------------------------------------
%	INTRODUCTION
%----------------------------------------------------------------------------------------

\section{} %Problem 1 
To prove the result, we will decompose an arbitrary permutation $\sigma \in S_n$ into transpositions. We will then show that each transposition can be 
written as a product of transpositions of the form $(i i+1).$ We have $\sigma=\tau_1\dots\tau_r,$ where $\tau_i$ is some transposition of the form $(k_1 
k+1+k_2)_i.$ Note that we can assume the first element of $\tau_i$ is strictly lesser than the second since if it weren't we could just invert the order 
without any loss of generality. We state that $(k_1 k_1+k_2)$ can written as a product of finite transpositions of the form $(t t+1).$ See that $(k_1 
k_1+k_2-1)=(k_1+k_2-1 k_1+k_2)(k_1 k_1+k_2)(k_1+k_2-1 k_1+k_2).$ Since we have $(k_1 k_1+k_2-1),$ we have lowered the second entry of the transposition by 
one. In $k_2-1$ steps, we will get $(k_1 k_1+1)=(k_1+1 k_1+2)(k_1 k_1+2)(k_1+1 k_1+2),$ which means that we can stop. We have $\tau_i$ as a product of 
$2(k_2-1)+1$ transpositions of the desired type. We can do this for all transpositions to get our result. Thus we can generate any permutation in $S_n$ by 
exchanging adjacent elements. 
The bubble sort algorithm works this way too, which accepts a permutation then returns the list of $n$ numbers. This is essentially the same problem.
\section{} %Problem 2
\section{} %Problem 3 
\begin{enumerate}
	\item For some $1 \leq i \leq n,$ $G)i=\{\sigma \in S_n: \sigma(i)=i\} \cong S_{n-1}.$ Consider $G_i$ acting on $\{1,2,\dots,i-1,i+1,\dots,n\}.$
	If $n=2,$ this set will be a singleton, hence trivially transitive. For $n \geq 3,$ the set $\{1,2,\dots,n\}\backslash\{i\}$ has at least two elements. 
	Then $k,\ell \in \{1,2,\dots,i-1,i+1,\dots,n\}$ such that they are distinct (If $k=\ell,$ then $i \in G_i$ works). See that $ (k \ell) \in G_i,$ as it 
	does not affect $i.$ Then we have $(k \ell)k=\ell,$ which means that $G_i$ is transitive. 
	\item For a doubly transitive action of $G$ on $X,$ see that if we choose a proper subset then since 
\end{enumerate}
\section{} %Problem 4 
Let us denote all elements of $Q_8$ thus: $1,i,j,k,-1,-i,-j,-k$ are assigned the numbers from $1$ to $8.$ 
Then see that left multiplication by $1$ is the identity permutation on $S_8.$ Left multiplication by $i$ is $(1 2 5 6)(3 4 7 8),$ by $j$ is $(1 3 5 7)(2 8 
6 4),$ and by $k$ is $(1 4 5 8)(2 3 6 7).$ Their negatives also have a left regular representation. Now see that $i,j$ can generate $Q_8$ as a group, then 
it must stand to reason that their corresponding left regular representations will behave in the same way! Thus, see that $G=\langle (1 2 5 6)(3 4 7 8), (1 
3 5 7)(2 8 6 4) \rangle \cong Q_8.$
\section{} %Problem 5 
We know that $|[G:H]| = n.$ Consider the group action of left multiplication on left cosets of $H.$ This group action has a permutation 
representation, let us denote that by $\pi_H.$ Take $K=\ker \pi_H,$ and $|[H:K]| =k.$ Then we have $|[G:K]|=|[G:H]||H:K|=nk.$ 
Since $H$ has $n$ many cosets, we must have $\frac{G}{K}$ is isomorphic to some subgroup of $S_n.$ Clearly $K \leq H,$ and $K \trianglelefteq G,$ and since 
$nk | n!,$ we have 
\section{} %Problem 6 
We shall prove a result that for $|G| =n$ and $p$ the smallest prime that divides $n,$ then a subgroup of order $p$ must be normal.
Let some $H \leq G,$ with $|[G:H]|=p.$ Consider the group action of left multiplication on left cosets of $H.$ This group action has a permutation 
representation, let us denote that by $\pi_H.$ Take $K=\ker \pi_H,$ and $|[H:K]| =k.$ 
Then we have $|[G:K]|=|[G:H]||H:K|=pk.$ We know that H has $p$ many left cosets, hence $\frac{G}{K}$ is isomorphic to some subgroup of $S_p$ which is the 
image of $G$ under $ \pi_H.$ Thus we must have $pk|p! \implies k|(p-1)!.$ But since $k$ can only have prime factors greater than or equal to $p$ and 
$(p-1)!$ has no prime factors greater than $p,$ we must have $k=1.$ Thus $H=K \trianglelefteq G$ is normal.
 
Let $p$ be the smallest prime dividing $n.$ We know that $p < n,$ since $n$ is composite. Then see that there must exist a subgroup of order $\frac{n}{p}$ 
as given in the problem, hence this subgroup has index $\frac{n}{\frac{n}{p}}=p,$ hence this is a normal subgroup. Thus $G$ cannot be simple. 
\section{} %Problem 7 
We know that $|[G:Z(G)]| =n.$ We see that in the class equation we have 
$$|G| =|Z(G)|+\sum_{x \notin Z(G)}|[G:C(x)]|.$$ 
Dividing on both sides by $|Z(G)|$ we have $$n= 1+\sum_{x \notin Z(G)}\frac{n}{|C(x)|}.$$
Thus $$\frac{1}{n}+\sum_{x \notin Z(G)}\frac{1}{|C(x)|}=1.$$  
\section{} %Problem 8 
\section{} %Problem 9 
\section{} %Problem 10 

\end{document}
