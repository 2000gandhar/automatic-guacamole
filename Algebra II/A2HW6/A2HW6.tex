\documentclass[letterpaper,11pt,twoside]{article}
\usepackage[utf8]{inputenc}
\usepackage{enumitem}
\setlist{nosep}
\usepackage{graphicx}
\usepackage{amsmath,amssymb,amsfonts,amsthm}
\usepackage{tikz-cd}
\usepackage[margin=0.9in,
left=1.25in,%
right=1.25in,%
top=1.25in,%
bottom=1.25in
]{geometry}	
%\usepackage{stmaryrd} %For mapsfrom

%\usepackage{quiver}
\usepackage{bm}
\usepackage{fancyhdr}
\usepackage{mathrsfs}
\usepackage{amsbsy}
\usepackage{titlesec}
%\usepackage{yhmath}
%\usepackage{mathabx,epsfig}


%Hyperref Settings------
\usepackage{hyperref}
\usepackage{xcolor}
\hypersetup{
	colorlinks,
	linkcolor={black},
	citecolor={red!50!black}
	urlcolor={green!80!black}
}

%%%%%% TITLE %%%%%

\title{Algebra 2 Homework 6}
%\date{\today}


%MATH BACKGROUND DECLARATORS-------------------------------------------------------
\theoremstyle{proposition}
\newtheorem{proposition}{Proposition}[section]

\theoremstyle{definition}
\newtheorem{definition}{Definition}[section]

\theoremstyle{theorem}
\newtheorem{theorem}{Theorem}[section]

\theoremstyle{definition}
\newtheorem{remark}{\textbf{Remark}}[section]

\theoremstyle{definition}
\newtheorem{notation}{\textbf{Notation}}[section]

\theoremstyle{definition}
\newtheorem{discussion}{\textbf{Discussion}}[section]

\theoremstyle{lemma}
\newtheorem{lemma}{\textbf{Lemma}}[section]

\theoremstyle{definition}
\newtheorem{example}{\textbf{Example}}[section]

%\theoremstyle{remark}
%\newtheorem*{comment}{\textbf{Comments on Proof Technique}}

\theoremstyle{definition}
\newtheorem{construct}{Construction}[section]

\theoremstyle{corollary}
\newtheorem{corollary}{Corollary}[section]

\theoremstyle{definition}
\newtheorem{caution}{\textbf{Caution}}[section]


\theoremstyle{definition}
\newtheorem{question}{\textbf{Question}}[section]

\theoremstyle{definition}
\newtheorem{para}{}[section]


%--------------------------------------------------------------------------------- SOME USEFUL MACROS.


\newcommand{\N}{\mathbb{N}}
\newcommand{\Z}{\mathbb{Z}}
\newcommand{\C}{\mathbb{C}}
\newcommand{\R}{\mathbb{R}}
\DeclareMathOperator{\GL}{\text{\rm GL}}
\newcommand{\Ker}[1]{{\fontfamily{lmss}\selectfont 
		\text{\rm Ker}\left (#1\right )
}}
\newcommand{\nsg}{\trianglelefteq}
\newcommand{\abs}[1]{\left \vert #1 \right \vert}
\newcommand{\gen}[1]{\left\langle #1\right\rangle}
\newcommand{\norm}[1]{\left \vert \left \vert #1 \right \vert \right \vert}
\renewcommand{\div}{\;\vert\;}
\newcommand{\isom}{\cong}
\DeclareMathOperator{\Stab}{\text{\rm Stab}}
\newcommand{\Image}[1]{{\fontfamily{lmss}\selectfont 
		\text{\rm Im}\left (#1\right )
}}
\DeclareMathOperator{\Bij}{\text{\rm Bij}}
\DeclareMathOperator{\acts}{\rotatebox[origin=c]{-90}{$\circlearrowright$}}
\DeclareMathOperator{\Orb}{\text{\rm Orb}}
\DeclareMathOperator{\lcm}{\text{\rm lcm}}
\newcommand{\floor}[1]{\left \lfloor #1 \right \rfloor}
\DeclareMathOperator{\Aut}{\text{\rm Aut}}
\DeclareMathOperator{\Inn}{\text{\rm Inn}}
\DeclareMathOperator{\id}{\text{\rm id}}
\newcommand{\F}{\mathbb{F}}




\begin{document}
	\maketitle
	%\tableofcontents
	\begin{proof}[Solution of problem $1$:]
		\begin{enumerate}
			\item Let $f(x)= a_0 + a_1x + a_2x^2 + \dots + a_n x^n,$ where $a_i$ is the coefficient of the term of degree $i,$ and $a_n \neq 0.$ See that 
			the reverse of this polynomial will have degree $n,$ Since $x^nf(1/x)= x^n\left(a_0 + a_1x^{-1} + \dots + a_n x^{-n} \right)$
			\item Let the constant coefficient and leading term both be non-zero (If not, then one could have $x^2+x,$ which is reducible, while its 
			reverse, $x+1$ is irreducible). It is easy to see that the reverse of the reverse is just the original polynomial. That is, $x^n 
			(x^{-n}f(x))=f(x).$ Thus we only need to show that if $f$ is reducible, $g$ is reducible. If $f(x)=p(x)q(x),$ where $p,q$ are not units, and 
			$d_p := \deg p(x)>1$ and $d_q := \deg q(x)>1.$ Since the constant term of $f$ is non-zero, the constant term of $p$ and $q$ must also be 
			non-zero. Replacing $x$ by $\frac{1}{x},$ and multiplying on both sides by $x^n,$ we get 
			$$x^nf\left(\frac{1}{x}\right)= x^{d_p}p\left(\frac{1}{x}\right) \cdot x^{d_q}q\left(\frac{1}{x}\right) = \ell (x) m(x),$$ which gives us a 
			factorisation for the reverse of $f.$  
		\end{enumerate}
	\end{proof}
	\begin{proof}[Solution of problem $2$:]
	We begin by enumerating all irreducible polynomials of degree $1, 2$ and $4.$ See that $x$ and $x+1,$ the only degree one polynomials, are irreducible. 
	For degree $2,$ we have four choices. Of these, $x^2, x^2+x$ and $x^2+1$ are reducible. $x^2+x+1$ is irreducible since it has no roots, plugging in $0$ 
	and $1$. For degree $4,$ we have sixteen choices. Of these, we must weed out the reducible polynomials. We can also calculate the irreducibles of degree 
	$3$ easily, since we can use them to find the reducible polynomials of degree $4.$ We have eight possibilities for polynomials of degree four, we can 
	eliminate six of them easily, with a mixture of clever thinking and brute force ($x^3,x^3+1,x^3+x^2+x+1, x^3+x,x^3+x^2, x^3+x^2+x$ are reducible) we can 
	see that $x^3+x+1$ and $x^3+x^2+1$ are irreducible. A clever trick that shall aid us in our effort to weed out reducible polynomials is to notice that 
	if there is a polynomial which has evenly many non-zero terms, than it must be reducible since then we have $1+ \dots +1$ even number of times. 
	Therefore an irreducible polynomial must necessarily have odd number of terms with constant term $1.$ This gives us $x^4+x^3+x^2+x+1, x^4+x^3+1, 
	x^4+x^2+1,$ and $x^4+x+1.$ In the case of  $x^4+x^2+1,$ see that it is $(x^2+x+1)^2,$ so this must be excluded. Thus we only have three irreducible 
	polynomials in $\mathbb{F}_2[x]$ of degree $4.$ The polynomials all have $16$ distinct roots, and $15$ non-zero roots. The only possibility in 
	$\mathbb{F}_2[x]$ is $x^{15}-1.$ Multiplying by $x,$ we have $x^{16}-x,$ which is the required polynomial.  
\end{proof}
	\begin{proof}[Solution of problem $3$:]
	See that $f(x+1)-f(x)= (x+1)^p - (x+1) + a - x^p +x -a = 1 -1 =0.$
	Thus either the polynomial has a root for all $x \in \mathbb{F}_p,$ or it has no roots in $\mathbb{F}_p$. Assuming it has a 
	root, then we must have that $0$ is also a root, which forces $a=0,$ which is not possible. Thus $f$ has no linear factors.  Now see that $F_p(\alpha)= 
	F_p(\alpha'),$ where $\alpha, \alpha'$ are two roots of the polynomial, both of whom are not in $\mathbb{F}_p.$ Then their irreducible polynomials must 
	be equal, which must divide $f.$ Then the degree of $f$ is some multiple of $q,$ which is the degree of $\alpha.$ However, since $p$ is prime, $q$ is 
	either $1$ or $p.$ Since the first is not possible, the polynomial must be irreducible.
	
	We see that $f'(x)= -1.$ Then the gcd can only be a constant, which means that there can be no common root between $f$ and $f'.$ Thus this polynomial is 
	separable.  
\end{proof}
\begin{proof}[Solution of problem $4$:]
	Let $L$ be an extension of $K$ that contains all the roots of $f(x).$ If $f(x)$ had repeated irreducible factors in $K[x],$ there would be multiple 
	roots of $f(x)$ in $L.$ However, since $L$ is also an extension of $F,$ it contradicts the separability of $f(x)$ over $F,$ which is a contradiction. 
	Thus $f(x)$ cannot have repeated irreducible factors in $K[x].$ 
\end{proof}
	\begin{proof}[Solution of problem $5$:]
	If we had an isomorphism $f: \mathbb{Q}(\sqrt{2}) \to \mathbb{Q}(\sqrt{3}),$ then we consider the map 
	\[\begin{tikzcd}
		& {\mathbb{Q}[x]} \\
		{\mathbb{Q}(\sqrt{2})} && {\mathbb{Q}(\sqrt{3})}
		\arrow["{x \mapsto \sqrt{2}}"', two heads, from=1-2, to=2-1]
		\arrow["{x \mapsto \sqrt{3}}", two heads, from=1-2, to=2-3]
		\arrow["f", no head, from=2-1, to=2-3]
	\end{tikzcd}\]
	This map must commute, so $x \mapsto \sqrt{2} \mapsto f(\sqrt{2}).$ But since the above diagram must commute, we have $f(\sqrt{2})= \sqrt{3}.$ (It does 
	not matter if we send $\sqrt{2}$ to $\sqrt{3}$ or its conjugate, the end result is the same). If $f(\sqrt{2})=\sqrt{3},$ then we have $2=f(2)=f(\sqrt{2} 
	\cdot \sqrt{2})= f(\sqrt{2})f(\sqrt{2})\sqrt{3} \cdot \sqrt{3}=3.$ This means that $$2=3 \implies 2-3 =0 \implies -1=0 \implies -1 \cdot -1 = -1 \cdot 0 
	\implies 1=0,$$ which is known to be not possible. Thus the two fields do not exist.

\end{proof}
	\begin{proof}[Solution of problem $6$:]
	\begin{enumerate}
		\item We need to check that this is a homomorphism that is one-one and onto. Note that $k$ is not affected, since the mapping affects only $x.$ Then 
		is map is $k-$linear. Thus acting on the vector space of all polynomials on $k,$ this is clearly a linear map. To see that it respect the ring 
		operation as well, we want to see that $\varphi( x^m \cdot x^n ) = \varphi(x^m)\varphi(x^n).$ The two sides are clearly equal, so we can extend this 
		to all polynomials, claiming that $\varphi(f(t)\cdot g(t))= \varphi(f(t))\cdot \varphi(g(t)).$ See that the degree of $\varphi(f(t))$ is the same as 
		the degree of $f(t),$ since for each monomial the degree is preserved. Now see that the kernel must be trivial, since we say that $0$ has an 
		undefined degree and it is the only such element. Thus any polynomial that goes to $0$ under $\varphi$ must also have that same degree, which means 
		that our map is onto. To see that our map is onto, we propose that the map $\tau(f(x)):= f\left(\frac{x-b}{a}\right),$ which is the inverse of 
		$\varphi.$ Since $a$ is non-zero, it is invertible. Since this is also a map just like $\varphi,$ we see that this also has all of the properties of 
		$\varphi.$ Now see that $\tau ( \varphi(f(t)))= \tau( f(at+b))= \tau \left( \frac{at+b-b}{a}\right)=f(t),$ and $ \varphi( \tau(f(t)) ) = \varphi 
		\left( f\left( \frac{t-b}{a}\right) \right)= f\left( a\frac{t-b}{a}+b \right)=f(t).$ Thus $\varphi$ is onto, and hence an automorphism.
		
		\item We have $\varphi,$ an automorphism on $k[t].$ We only need to know where $t$ is sent. Let $x \mapsto p(t).$ If $p(t)$ is constant, it is not 
		an automorphism. Let the inverse map be such that $x \mapsto g(t).$ Then $f(g(t))=t.$ This implies that $\deg f \cdot \deg g=1 \implies \deg f= \deg 
		g=1.$ Then $f(t)=at+b,$ a linear polynomial ( $a \neq 0$). 
	\end{enumerate}
\end{proof}
	\begin{proof}[Solution of problem $7$:]
	\begin{enumerate}
		\item Fix a $\sigma \in \Aut(\mathbb{R}/\mathbb{Q}).$ Then for some number $k^2 \in \mathbb{R},$ we have $\sigma(k^2)= \sigma(k)^2.$ Therefore 
		square numbers are taken to square numbers. If $r >0,$ then $r=k^2,$ for some $k \in \mathbb{R},$ then we have the squares are taken to squares and 
		hence $\sigma(r)=\sigma(k^2)=\sigma(k)^2 > 0.$
		
		Now see that if $a< b,$ $b-a > 0.$ Thus $\sigma(b-a) >0,$ that is $\sigma(b) > \sigma(a).$ 
		
		\item If $\frac{-1}{m} < a-b < \frac{1}{m},$ then applying $\sigma$ yields $\frac{-1}{m} < \sigma(a) - \sigma(b) < \frac{1}{m}.$ The bounds are 
		unchanged since the automorphism is identity over the rationals. Pick any $\varepsilon > 0.$ Then if we want $|\sigma(b-a)| < \varepsilon,$ we just 
		find the least $N(\varepsilon) \in \mathbb{N}$ such that $\frac{1}{N(\varepsilon)} < \varepsilon.$ Then let $\delta= \frac{1}{N(\varepsilon)},$ 
		which means all automorphisms are continuous. 
		
		\item We know that $\sigma$ is the identity on the rationals, which is a dense subset of $\mathbb{R}.$ Pick a sequence $\{q_n\}$ such that $q_n \to 
		r$ as $n \to \infty.$ Then by continuity of $\sigma$ we have $\sigma(q_n) \to \sigma(r).$ The sequence $\{\sigma(q_n)\}$ is just the sequence 
		$\{q_n\},$ which can only converge to $r.$ Since limits are unique in $\mathbb{R},$ we have $\sigma(r)=r.$ Thus $\sigma$ is the identity on 
		$\mathbb{R}.$ Since our choice of $\sigma$ was arbitrary, we have that $\Aut(\mathbb{R}/\mathbb{Q})=0.$ 
		
		
	\end{enumerate}
\end{proof}








\end{document}