\documentclass[letterpaper,11pt,twoside]{article}
\usepackage[utf8]{inputenc}
\usepackage{enumitem}
\setlist{nosep}
\usepackage{graphicx}
\usepackage{amsmath,amssymb,amsfonts,amsthm}
\usepackage{tikz-cd}
\usepackage[margin=0.9in,
left=1.25in,%
right=1.25in,%
top=1.25in,%
bottom=1.25in
]{geometry}	
%\usepackage{stmaryrd} %For mapsfrom

%\usepackage{quiver}
\usepackage{bm}
\usepackage{fancyhdr}
\usepackage{mathrsfs}
\usepackage{amsbsy}
\usepackage{titlesec}
%\usepackage{yhmath}
%\usepackage{mathabx,epsfig}


%Hyperref Settings------
\usepackage{hyperref}
\usepackage{xcolor}
\hypersetup{
	colorlinks,
	linkcolor={black},
	citecolor={red!50!black}
	urlcolor={green!80!black}
}

%%%%%% TITLE %%%%%

\title{Algebra 2 Homework 5}
%\date{\today}


%MATH BACKGROUND DECLARATORS-------------------------------------------------------
\theoremstyle{proposition}
\newtheorem{proposition}{Proposition}[section]

\theoremstyle{definition}
\newtheorem{definition}{Definition}[section]

\theoremstyle{theorem}
\newtheorem{theorem}{Theorem}[section]

\theoremstyle{definition}
\newtheorem{remark}{\textbf{Remark}}[section]

\theoremstyle{definition}
\newtheorem{notation}{\textbf{Notation}}[section]

\theoremstyle{definition}
\newtheorem{discussion}{\textbf{Discussion}}[section]

\theoremstyle{lemma}
\newtheorem{lemma}{\textbf{Lemma}}[section]

\theoremstyle{definition}
\newtheorem{example}{\textbf{Example}}[section]

%\theoremstyle{remark}
%\newtheorem*{comment}{\textbf{Comments on Proof Technique}}

\theoremstyle{definition}
\newtheorem{construct}{Construction}[section]

\theoremstyle{corollary}
\newtheorem{corollary}{Corollary}[section]

\theoremstyle{definition}
\newtheorem{caution}{\textbf{Caution}}[section]


\theoremstyle{definition}
\newtheorem{question}{\textbf{Question}}[section]

\theoremstyle{definition}
\newtheorem{para}{}[section]


%--------------------------------------------------------------------------------- SOME USEFUL MACROS.


\newcommand{\N}{\mathbb{N}}
\newcommand{\Z}{\mathbb{Z}}
\newcommand{\C}{\mathbb{C}}
\newcommand{\R}{\mathbb{R}}
\DeclareMathOperator{\GL}{\text{\rm GL}}
\newcommand{\Ker}[1]{{\fontfamily{lmss}\selectfont 
		\text{\rm Ker}\left (#1\right )
}}
\newcommand{\nsg}{\trianglelefteq}
\newcommand{\abs}[1]{\left \vert #1 \right \vert}
\newcommand{\gen}[1]{\left\langle #1\right\rangle}
\newcommand{\norm}[1]{\left \vert \left \vert #1 \right \vert \right \vert}
\renewcommand{\div}{\;\vert\;}
\newcommand{\isom}{\cong}
\DeclareMathOperator{\Stab}{\text{\rm Stab}}
\newcommand{\Image}[1]{{\fontfamily{lmss}\selectfont 
		\text{\rm Im}\left (#1\right )
}}
\DeclareMathOperator{\Bij}{\text{\rm Bij}}
\DeclareMathOperator{\acts}{\rotatebox[origin=c]{-90}{$\circlearrowright$}}
\DeclareMathOperator{\Orb}{\text{\rm Orb}}
\DeclareMathOperator{\lcm}{\text{\rm lcm}}
\newcommand{\floor}[1]{\left \lfloor #1 \right \rfloor}
\DeclareMathOperator{\Aut}{\text{\rm Aut}}
\DeclareMathOperator{\Inn}{\text{\rm Inn}}
\DeclareMathOperator{\id}{\text{\rm id}}
\newcommand{\F}{\mathbb{F}}




\begin{document}
	\maketitle
	%\tableofcontents
	\begin{proof}[Solution of problem $1$:]
		\begin{enumerate}
			\item Let $f(x)= a_0 + a_1x + a_2x^2 + \dots + a_n x^n,$ where $a_i$ is the coefficient of the term of degree $i,$ and $a_n \neq 0.$ See that 
			the reverse of this polynomial will have degree $n,$ Since $x^nf(1/x)= x^n\left(a_0 + a_1x^{-1} + \dots + a_n x^{-n} \right)$
			\item Let the constant coefficient and leading term both be non-zero (If not, then one could have $x^2+x,$ which is reducible, while its 
			reverse, $x+1$ is irreducible). It is easy to see that the reverse of the reverse is just the original polynomial. That is, $x^n 
			(x^{-n}f(x))=f(x).$ Thus we only need to show that if $f$ is reducible, $g$ is reducible. If $f(x)=p(x)q(x),$ where $p,q$ are not units, and 
			$d_p := \deg p(x)>1$ and $d_q := \deg q(x)>1.$ Since the constant term of $f$ is non-zero, the constant term of $p$ and $q$ must also be 
			non-zero. Replacing $x$ by $\frac{1}{x},$ and multiplying on both sides by $x^n,$ we get 
			$$x^nf\left(\frac{1}{x}\right)= x^{d_p}p\left(\frac{1}{x}\right) \cdot x^{d_q}q\left(\frac{1}{x}\right) = \ell (x) m(x),$$ which gives us a 
			factorisation for the reverse of $f.$  
		\end{enumerate}
	\end{proof}
	\begin{proof}[Solution of problem $2$:]
	We begin by enumerating all irreducible polynomials of degree $1, 2$ and $4.$ See that $x$ and $x+1,$ the only degree one polynomials, are irreducible. 
	For degree $2,$ we have four choices. Of these, $x^2, x^2+x$ and $x^2+1$ are reducible. $x^2+x+1$ is irreducible since it has no roots, plugging in $0$ 
	and $1$. For degree $4,$ we have sixteen choices. Of these, we must weed out the reducible polynomials. We can also calculate the irreducibles of degree 
	$3$ easily, since we can use them to find the reducible polynomials of degree $4.$ We have eight possibilities for polynomials of degree four, we can 
	eliminate six of them easily, with a mixture of clever thinking and brute force ($x^3,x^3+1,x^3+x^2+x+1, x^3+x,x^3+x^2, x^3+x^2+x$ are reducible) we can 
	see that $x^3+x+1$ and $x^3+x^2+1$ are irreducible. A clever trick that shall aid us in our effort to weed out reducible polynomials is to notice that 
	if there is a polynomial which has evenly many non-zero terms, than it must be reducible since then we have $1+ \dots +1$ even number of times. 
	Therefore an irreducible polynomial must necessarily have odd number of terms with constant term $1.$ This gives us $x^4+x^3+x^2+x+1, x^4+x^3+1, 
	x^4+x^2+1,$ and $x^4+x+1.$ In the case of  $x^4+x^2+1,$ see that it is $(x^2+x+1)^2,$ so this must be excluded. Thus we only have three irreducible 
	polynomials in $\mathbb{F}_2[x].$ $x(x+1)(x^2+x+1)=(x^2+x)(x^2+x+1)= x^4 +x^2 +x^2+x= x^4+x.$ Then multiplying the remaining polynomials, we have 
	\begin{align*}
		(x^4+x)(x^4+x^3+x^2+x+1)(x^4+x^3+1)(x^4+x+1)&=((x^4+x^3+1)^2 +(x^4+x^3+1)(x^2+x) ) ((x^4+x)^2 + (x^4+x))\\
		&= (x^8+x^6+1 + x^6 + x^5 +x^2 +x^5+x^4+x )(x^8 +x^4 +x^4+x)\\
		&=  (x^8 + x^4 + x^2+x +1)(x^8+x)\\
		&= x^{16}+ x^{12}+ x^{10} +x^{9} + x^8 + x^9 + x^5 +x^3+ x^2 + x\\
		&=x^{16}+ x^{12} + x^{10} + x^8+ x^5+x^5+x^3+x^2+x
	\end{align*}
\end{proof}
	\begin{proof}[Solution of problem $3$:]
	
\end{proof}
	\begin{proof}[Solution of problem $4$:]
	
\end{proof}
	\begin{proof}[Solution of problem $5$:]
	
\end{proof}
	\begin{proof}[Solution of problem $6$:]
	
\end{proof}

\end{document}