%%%%%%%%%%%%%%%%%%%%%%%%%%%%%%%%%%%%%%%%%
% Lachaise Assignment
% LaTeX Template
% Version 1.0 (26/6/2018)
%
% This template originates from:
% http://www.LaTeXTemplates.com
%
% Authors:
% Marion Lachaise & François Févotte
% Vel (vel@LaTeXTemplates.com)
%
% License:
% CC BY-NC-SA 3.0 (http://creativecommons.org/licenses/by-nc-sa/3.0/)
% 
%%%%%%%%%%%%%%%%%%%%%%%%%%%%%%%%%%%%%%%%%

%----------------------------------------------------------------------------------------
%	PACKAGES AND OTHER DOCUMENT CONFIGURATIONS
%----------------------------------------------------------------------------------------

\documentclass{article}

%%%%%%%%%%%%%%%%%%%%%%%%%%%%%%%%%%%%%%%%%
% Lachaise Assignment
% Structure Specification File
% Version 1.0 (26/6/2018)
%
% This template originates from:
% http://www.LaTeXTemplates.com
%
% Authors:
% Marion Lachaise & François Févotte
% Vel (vel@LaTeXTemplates.com)
%
% License:
% CC BY-NC-SA 3.0 (http://creativecommons.org/licenses/by-nc-sa/3.0/)
% 
%%%%%%%%%%%%%%%%%%%%%%%%%%%%%%%%%%%%%%%%%

%----------------------------------------------------------------------------------------
%	PACKAGES AND OTHER DOCUMENT CONFIGURATIONS
%----------------------------------------------------------------------------------------

\usepackage{amsmath,amsfonts,amssymb, tikz-cd} % Math packages

\usepackage{enumerate} % Custom item numbers for enumerations


\usepackage[framemethod=tikz]{mdframed} % Allows defining custom boxed/framed environments

\usepackage{listings} % File listings, with syntax highlighting
\lstset{
	basicstyle=\ttfamily, % Typeset listings in monospace font
}

%----------------------------------------------------------------------------------------
%	DOCUMENT MARGINS
%----------------------------------------------------------------------------------------

\usepackage{geometry} % Required for adjusting page dimensions and margins

\geometry{
	paper=letterpaper, % Paper size, change to letterpaper for US letter size
	top=2.5cm, % Top margin
	bottom=3cm, % Bottom margin
	left=2.5cm, % Left margin
	right=2.5cm, % Right margin
	headheight=14pt, % Header height
	footskip=1.5cm, % Space from the bottom margin to the baseline of the footer
	headsep=1.2cm, % Space from the top margin to the baseline of the header
	%showframe, % Uncomment to show how the type block is set on the page
}

%----------------------------------------------------------------------------------------
%	FONTS
%----------------------------------------------------------------------------------------

\usepackage[utf8]{inputenc} % Required for inputting international characters
\usepackage[T1]{fontenc} % Output font encoding for international characters


%----------------------------------------------------------------------------------------
%	COMMAND LINE ENVIRONMENT
%----------------------------------------------------------------------------------------

% Usage:
% \begin{commandline}
	%	\begin{verbatim}
		%		$ ls
		%		
		%		Applications	Desktop	...
		%	\end{verbatim}
	% \end{commandline}

\mdfdefinestyle{commandline}{
	leftmargin=10pt,
	rightmargin=10pt,
	innerleftmargin=15pt,
	middlelinecolor=black!50!white,
	middlelinewidth=2pt,
	frametitlerule=false,
	backgroundcolor=black!5!white,
	frametitle={Command Line},
	frametitlefont={\normalfont\sffamily\color{white}\hspace{-1em}},
	frametitlebackgroundcolor=black!50!white,
	nobreak,
}

% Define a custom environment for command-line snapshots
\newenvironment{commandline}{
	\medskip
	\begin{mdframed}[style=commandline]
	}{
	\end{mdframed}
	\medskip
}

%----------------------------------------------------------------------------------------
%	FILE CONTENTS ENVIRONMENT
%----------------------------------------------------------------------------------------

% Usage:
% \begin{file}[optional filename, defaults to "File"]
	%	File contents, for example, with a listings environment
	% \end{file}

\mdfdefinestyle{file}{
	innertopmargin=1.6\baselineskip,
	innerbottommargin=0.8\baselineskip,
	topline=false, bottomline=false,
	leftline=false, rightline=false,
	leftmargin=2cm,
	rightmargin=2cm,
	singleextra={%
		\draw[fill=black!10!white](P)++(0,-1.2em)rectangle(P-|O);
		\node[anchor=north west]
		at(P-|O){\ttfamily\mdfilename};
		%
		\def\l{3em}
		\draw(O-|P)++(-\l,0)--++(\l,\l)--(P)--(P-|O)--(O)--cycle;
		\draw(O-|P)++(-\l,0)--++(0,\l)--++(\l,0);
	},
	nobreak,
}

% Define a custom environment for file contents
\newenvironment{file}[1][File]{ % Set the default filename to "File"
	\medskip
	\newcommand{\mdfilename}{#1}
	\begin{mdframed}[style=file]
	}{
	\end{mdframed}
	\medskip
}

%----------------------------------------------------------------------------------------
%	NUMBERED QUESTIONS ENVIRONMENT
%----------------------------------------------------------------------------------------

% Usage:
% \begin{question}[optional title]
	%	Question contents
	% \end{question}

\mdfdefinestyle{question}{
	innertopmargin=1.2\baselineskip,
	innerbottommargin=0.8\baselineskip,
	roundcorner=5pt,
	nobreak,
	singleextra={%
		\draw(P-|O)node[xshift=1em,anchor=west,fill=white,draw,rounded corners=5pt]{%
			Question \theQuestion\questionTitle};
	},
}

\newcounter{Question} % Stores the current question number that gets iterated with each new question

% Define a custom environment for numbered questions
\newenvironment{question}[1][\unskip]{
	\bigskip
	\stepcounter{Question}
	\newcommand{\questionTitle}{~#1}
	\begin{mdframed}[style=question]
	}{
	\end{mdframed}
	\medskip
}

%----------------------------------------------------------------------------------------
%	WARNING TEXT ENVIRONMENT
%----------------------------------------------------------------------------------------

% Usage:
% \begin{warn}[optional title, defaults to "Warning:"]
	%	Contents
	% \end{warn}

\mdfdefinestyle{warning}{
	topline=false, bottomline=false,
	leftline=false, rightline=false,
	nobreak,
	singleextra={%
		\draw(P-|O)++(-0.5em,0)node(tmp1){};
		\draw(P-|O)++(0.5em,0)node(tmp2){};
		\fill[black,rotate around={45:(P-|O)}](tmp1)rectangle(tmp2);
		\node at(P-|O){\color{white}\scriptsize\bf !};
		\draw[very thick](P-|O)++(0,-1em)--(O);%--(O-|P);
	}
}

% Define a custom environment for warning text
\newenvironment{warn}[1][Warning:]{ % Set the default warning to "Warning:"
	\medskip
	\begin{mdframed}[style=warning]
		\noindent{\textbf{#1}}
	}{
	\end{mdframed}
}

%----------------------------------------------------------------------------------------
%	INFORMATION ENVIRONMENT
%----------------------------------------------------------------------------------------

% Usage:
% \begin{info}[optional title, defaults to "Info:"]
	% 	contents
	% 	\end{info}

\mdfdefinestyle{info}{%
	topline=false, bottomline=false,
	leftline=false, rightline=false,
	nobreak,
	singleextra={%
		\fill[black](P-|O)circle[radius=0.4em];
		\node at(P-|O){\color{white}\scriptsize\bf i};
		\draw[very thick](P-|O)++(0,-0.8em)--(O);%--(O-|P);
	}
}

% Define a custom environment for information
\newenvironment{info}[1][Info:]{ % Set the default title to "Info:"
	\medskip
	\begin{mdframed}[style=info]
		\noindent{\textbf{#1}}
	}{
	\end{mdframed}
}
 % Include the file specifying the document structure and custom commands

%----------------------------------------------------------------------------------------
%	ASSIGNMENT INFORMATION
%----------------------------------------------------------------------------------------

\title{Algebra 2 HW1} % Title of the assignment

\author{Gandhar Kulkarni (mmat2304)} % Author name and email address

\date{} % University, school and/or department name(s) and a date

%----------------------------------------------------------------------------------------

\begin{document}

\maketitle % Print the title

%----------------------------------------------------------------------------------------
%	INTRODUCTION
%----------------------------------------------------------------------------------------

\section{} %Problem 1 
Assume for the sake of contradiction that there exists an isomorphism $\varphi: \mathbb{C}\backslash \{0\} \rightarrow \mathbb{R}\backslash \{0\}.$ Then we 
must have $$\varphi(i^4)=\varphi(i)^4=1.$$ Thus we must have $\varphi(i)=\pm 1,$ since $\varphi(i) \in \mathbb{R}\backslash \{0\}.$ If $\varphi(i)=1,$ then 
$\varphi$ is not one-one. If $\varphi(i)=-1,$ then $\varphi(i^2)=-1^2=1,$ which also means that $\varphi$ is not one-one. Thus no such isomorphism exists.

\section{} %Problem 2
\begin{enumerate}
	\item To characterise a linear transformation, it is enough to understand its action on the basis elements, that is $(1,0)^T$ and $(0,1)^T.$ Looking at 
	the point on the unit circle that has an angle $\theta$ to the x-axis, we can see that it has the coordinate $(\cos \theta, \sin \theta).$ Similarly, we 
	want to see the coordinates of the point that has an angle of $\frac{\pi}{2}+ \theta$ to the x-axis. Its coordinates are $(-\sin \theta, \cos \theta).$ 
	Putting it together, we get the required rotation matrix that describes the linear transformation.
	\item To confirm that $\varphi:D_{2n} \rightarrow GL_2(\mathbb{R})$ is a homomorphism, we need to confirm that $\varphi(r)^n=\varphi(s)^2=I_{2},$ and 
	that $\phi(r)\phi$
\end{enumerate}
\section{} %Problem 3
$$D_{2n}= \left\{ r^is^j : 0 \leq i \leq n-1, 0 \leq j \leq 1, rs=sr^{n-1} \right\}.$$
For any two elements $r^{i_1}s^{j_1}, r^{i_2}s^{j_2} \in D_{2n}.$ % we have $$r^{i_1}s^{j_1} \cdot r^{i_2}s^{j_2}=r^{n+i_1+ (-1)^{j_1}i_2} s^{j_1+j_2}.$$
Let $r^{i_1}s^{j_1} \in Z(D_{2n})$ commute with $r^{i_2}s^{j_2} \in D_{2n}.$ Then $$r^{i_1}s^{j_1} \cdot r^{i_2}s^{j_2}= 
r^{i_2}s^{j_2} \cdot r^{i_1}s^{j_1}.$$
Working this out, we get $$r^{n+i_1+(-1)^{j_1}i_2} s^{j_1+j_2}=r^{n+i_2+(-1)^{j_2}i_1} s^{j_2+j_1}, $$ which implies that 
\begin{enumerate}
	\item If $n$ is odd, 
\end{enumerate}
\section{} %Problem 4
Let $x \in G$ be such that $xZ(G)$ generates $G/Z(G).$ Thus any term in $G/Z(G)$ is of the form $x^aZ(G)$ for some $a \in \mathbb{Z}.$ Consider the 
canonical quotient map $\pi: G \twoheadrightarrow G/Z(G)$ where $\pi(g)=gZ(G).$ Its kernel is $Z(G),$ so we have $G \cong Z(G) \times G/Z(G).$ Thus we can 
write $g \in G$ as $(z,x^a),$ such that $g=x^az.$ Now take $g_1,g_2 \in G,$ and consider $g_1 \cdot g_2=x^{a_1}z_1 \cdot x^{a_2}z_2=g_2=x^{a_1} 
x^{a_2}z_1z_2=g_2 \cdot g_1,$ as the order of multiplication of $z_1$ and $z_2$ can be switched as it is in the centre. Thus we have $G$ is abelian.  
\section{} %Problem 5
Let $n=|G|,$ $k=|[G:N]|.$ 
\section{} %Problem 6
$G=MN,$ where $M,N \trianglelefteq G.$ Define the map $f: G \rightarrow (G/M) \times (G/N),$ where $f(g)=(gM,gN).$ To see that this map is well-defined, see 
that for $g=g'$ in $G$, we have $gM=g'M$ and $gN=g'N$ as the canonical projections from $G$ to $G/M$ and $G/N$ are well-defined. From these two maps it can 
be seen that the map $f$ also respects the group operation, hence this is also a homomorphism. Note that for all $g \in G,$ $g=mn,$ for $m \in M, n \in N.$ 
Then an arbitrary element of $(G/M) \times (G/N)$ is of the form $(nM,mN).$ Thus we can see that this corresponds to an element $mn \in G,$ which can cover 
all of $G.$ Thus $f$ is surjective. To compute the kernel of $f,$ see that $gM=e_{G/M} \implies g \in M,$ and $gN=e_{G/N} \implies g \in N$. Thus $g \in M 
\cap N,$ thus $\ker f=M \cap N.$ Using the first isomorphism theorem gives us our result. 
\end{document}
