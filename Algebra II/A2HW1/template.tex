%%%%%%%%%%%%%%%%%%%%%%%%%%%%%%%%%%%%%%%%%
% Lachaise Assignment
% LaTeX Template
% Version 1.0 (26/6/2018)
%
% This template originates from:
% http://www.LaTeXTemplates.com
%
% Authors:
% Marion Lachaise & François Févotte
% Vel (vel@LaTeXTemplates.com)
%
% License:
% CC BY-NC-SA 3.0 (http://creativecommons.org/licenses/by-nc-sa/3.0/)
% 
%%%%%%%%%%%%%%%%%%%%%%%%%%%%%%%%%%%%%%%%%

%----------------------------------------------------------------------------------------
%	PACKAGES AND OTHER DOCUMENT CONFIGURATIONS
%----------------------------------------------------------------------------------------

\documentclass{article}

\input{structure.tex} % Include the file specifying the document structure and custom commands

%----------------------------------------------------------------------------------------
%	ASSIGNMENT INFORMATION
%----------------------------------------------------------------------------------------

\title{Algebra 2 HW1} % Title of the assignment

\author{Gandhar Kulkarni (mmat2304)} % Author name and email address

\date{} % University, school and/or department name(s) and a date

%----------------------------------------------------------------------------------------

\begin{document}

\maketitle % Print the title

%----------------------------------------------------------------------------------------
%	INTRODUCTION
%----------------------------------------------------------------------------------------

\section{} %Problem 1 
Assume for the sake of contradiction that there exists an isomorphism $\varphi: \mathbb{C}\backslash \{0\} \rightarrow \mathbb{R}\backslash \{0\}.$ Then we 
must have $$\varphi(i^4)=\varphi(i)^4=1.$$ Thus we must have $\varphi(i)=\pm 1,$ since $\varphi(i) \in \mathbb{R}\backslash \{0\}.$ If $\varphi(i)=1,$ then 
$\varphi$ is not one-one. If $\varphi(i)=-1,$ then $\varphi(i^2)=-1^2=1,$ which also means that $\varphi$ is not one-one. Thus no such isomorphism exists.

\section{} %Problem 2
\begin{enumerate}
	\item To characterise a linear transformation, it is enough to understand its action on the basis elements, that is $(1,0)^T$ and $(0,1)^T.$ Looking at 
	the point on the unit circle that has an angle $\theta$ to the $x$-axis, we can see that it has the coordinate $(\cos \theta, \sin \theta).$ Similarly, 
	we 
	want to see the coordinates of the point that has an angle of $\frac{\pi}{2}+ \theta$ to the x-axis. Its coordinates are $(-\sin \theta, \cos \theta).$ 
	Putting it together, we get the required rotation matrix that describes the linear transformation.
	\item To confirm that $\varphi:D_{2n} \rightarrow GL_2(\mathbb{R})$ is a homomorphism, we need to see that $\varphi$ respects multiplication, and  
	$\varphi(r)^n=\varphi(s)^2=I_{2}.$ The latter is easy to check, as $$\varphi(r)^n= \begin{pmatrix}
		\cos \theta && -\sin \theta \\
		\sin \theta && \cos \theta
	\end{pmatrix}^n=\begin{pmatrix}
	\cos n\theta && -\sin n\theta \\
	\sin n\theta && \cos n\theta
\end{pmatrix}=I_2,$$
and $$\varphi(s)^2= \begin{pmatrix}
	0 && 1 \\
	1 && 0
\end{pmatrix}^2=I_2.$$
Now, is $\varphi(r)\varphi(s)=\varphi(s)\varphi(r)^{-1}?$ See that $$\varphi(r)\varphi(s)= \begin{pmatrix}
	\cos \theta && -\sin \theta \\
	\sin \theta && \cos \theta
\end{pmatrix}\cdot \begin{pmatrix}
0 && 1\\
1 && 0
\end{pmatrix}= \begin{pmatrix}
-\sin \theta && \cos \theta \\
\cos \theta && \sin \theta
\end{pmatrix}.$$

We can see that $$\varphi(s)\varphi(r)^{-1}= \begin{pmatrix}
	0 && 1\\
	1 && 0
\end{pmatrix}\cdot \begin{pmatrix}
	\cos \theta && -\sin \theta \\
	\sin \theta && \cos \theta
\end{pmatrix}^{-1}=\begin{pmatrix}
0 && 1\\
1 && 0
\end{pmatrix}\cdot \begin{pmatrix}
\cos \theta && \sin \theta \\
-\sin \theta && \cos \theta
\end{pmatrix} = \begin{pmatrix}
	-\sin \theta && \cos \theta \\
	\cos \theta && \sin \theta
\end{pmatrix},$$ which means that $\varphi(r)\varphi(s)=\varphi(s)\varphi(r)^{-1}.$ Thus we can see that $\varphi(r)$ and $\varphi(s)$ generated $D_{2n}$ in 
$GL_2(\mathbb{R}).$
\item To check injectivity, we wish to find the kernel of this homomorphism. We know that $\varphi(r^ks^\ell)=\varphi(r)^k\varphi(s)^\ell.$ To find the 
kernel, we say that $\varphi(r^ks^\ell)=\varphi(r)^k\varphi(s)^\ell=I_2.$ If $\ell=1,$ then we have $$\varphi(r)^k\varphi(s)= \begin{pmatrix}
	\cos k\theta && -\sin k\theta \\
	\sin k\theta && \cos k\theta
\end{pmatrix}\cdot \begin{pmatrix}
0 && 1 \\
1 && 0
\end{pmatrix} = \begin{pmatrix}
-\sin k\theta && -\cos k\theta \\
\cos k\theta && \sin k\theta
\end{pmatrix}= I_2.$$ This means that $\cos k\theta=0 \implies \frac{2k\pi}{n}= \frac{4z \pm 1}{4}2n\pi,$ which means that $k$ cannot be an integer, which 
is absurd. Thus we must have $\ell=0.$ Then see that  $$\varphi(r)^k= \begin{pmatrix}
\cos k\theta && -\sin k\theta \\
\sin k\theta && \cos k\theta
\end{pmatrix}=I_2.$$ For this, we already know that $n$ is the smallest possible positive solution, since $\cos \frac{2\pi k}{n}=1 \implies k=n.$ But since 
we know that $\varphi(r)^n=I_2,$ we can pick $k=0$ as well. Thus we have $\ker \varphi = \{r^0s^0\},$ that is to say that $\varphi$ is injective.  
\end{enumerate}
\section{} %Problem 3
$$D_{2n}= \left\{ r^is^j : 0 \leq i \leq n-1, 0 \leq j \leq 1, rs=sr^{n-1} \right\}.$$
For any two elements $r^{i_1}s^{j_1}, r^{i_2}s^{j_2} \in D_{2n}.$ % we have $$r^{i_1}s^{j_1} \cdot r^{i_2}s^{j_2}=r^{n+i_1+ (-1)^{j_1}i_2} s^{j_1+j_2}.$$
Let $r^{i_1}s^{j_1} \in Z(D_{2n})$ commute with $r^{i_2}s^{j_2} \in D_{2n}.$ Then $$r^{i_1}s^{j_1} \cdot r^{i_2}s^{j_2}= 
r^{i_2}s^{j_2} \cdot r^{i_1}s^{j_1}.$$
Working this out, we get $$r^{n+i_1+(-1)^{j_1}i_2} s^{j_1+j_2}=r^{n+i_2+(-1)^{j_2}i_1} s^{j_2+j_1}. $$ Let $r^{i_2}s^{j_2}$ be arbitrary, then we can divide 
all cases into the case where $j_1=0$ and $j_1=1.$ 

If $j_1=1,$ then comparing exponents of $r$ on both sides, we have $$i_1-i_2\equiv i_2+ (-1)^{j_2}i_1 \mod n,$$ which means that the answer for $i_1$ must 
depend on $i_2$ and $j_2,$ which means we cannot find an element that commutes with all elements of $D_{2n}.$ 

If $j_1=0,$ then comparing terms we have $$i_1-i_2 \equiv i_2 + (-1)^{j_2}i_1 \mod n \implies i_1(1-(-1)^{j_2})\equiv 0 \mod n.$$
Then see that the term in brackets could be either $0$ or $2,$ so we need to find $i_1$ such that the above equation is satisfied only in the case that 
the term is $2$, as in the case of $0$ there is no need to check. 
\begin{enumerate}
	\item If $n$ is odd, then we must necessarily have $i_1|n$ as $2 \nmid n,$ implying that $i_1=0.$ Thus $i,$ the identity rotation is the only element in 
	the centre of $D_{2n}$ for $n$ odd.
	\item If $n$ is even, then $2|n,$ so we can have that $i_1|\frac{n}{2}.$ Thus we can have $i_1=0, \frac{n}{2}.$ Thus we see that the centre of $D_{2n}$ 
	for $n$ even is $i$ and $r^k,$ where $k=\frac{n}{2}.$
\end{enumerate}
\section{} %Problem 4
Let $x \in G$ be such that $xZ(G)$ generates $G/Z(G).$ Thus any term in $G/Z(G)$ is of the form $x^aZ(G)$ for some $a \in \mathbb{Z}.$ Consider the 
canonical quotient map $\pi: G \twoheadrightarrow G/Z(G)$ where $\pi(g)=gZ(G).$ Its kernel is $Z(G),$ so we have $G \cong Z(G) \times G/Z(G).$ Thus we can 
write $g \in G$ as $(z,x^a),$ such that $g=x^az.$ Now take $g_1,g_2 \in G,$ and consider $g_1 \cdot g_2=x^{a_1}z_1 \cdot x^{a_2}z_2=g_2=x^{a_1} 
x^{a_2}z_1z_2=g_2 \cdot g_1,$ as the order of multiplication of $z_1$ and $z_2$ can be switched as it is in the centre. Thus we have $G$ is abelian.  
\section{} %Problem 5
Let $n=|G|,$, $|H| =n_1,$ and $|N| = n_2.$ For an $x \in G,$ let $k$ be the smallest positive number such that $x^k \in H.$ We can write $n=kq+r$ by the 
division algorithm. Then see that $$1=x^{n}=x^{kq+r}=(x^k)^qx^r \in H.$$ But since $(x^k)^q \in H,$ we must also have $x^r \in H,$ which contradicts the 
minimality of $k.$ Thus $r=0,$ and $k|n.$ 
For some element $x \in H,$ let $k$ be that smallest positive integer such that $x^k \in N.$ This is guaranteed as $H$ is finite, so $k$ is at most $n_1.$ 
See that the element $xN \in G/N$ has order $k,$ which just means that the cyclic subgroup of $x$ generated in $H$ has exactly $k$ elements, since 
$x^kN=eN.$ Then we have $|xN| = k| n_1,$ and also $k \mid [G:N] =\frac{n}{n_2},$ since as an element of $G/N$ it must divide its order as well. However, 
since the two are coprime by hypothesis, we must have $k|1.$ This means that $x^1 \in N$ for $x \in H,$ which implies that $H \leq N,$ as required. 
\section{} %Problem 6
$G=MN,$ where $M,N \trianglelefteq G.$ Define the map $f: G \rightarrow (G/M) \times (G/N),$ where $f(g)=(gM,gN).$ To see that this map is well-defined, see 
that for $g=g'$ in $G$, we have $gM=g'M$ and $gN=g'N$ as the canonical projections from $G$ to $G/M$ and $G/N$ are well-defined. From these two maps it can 
be seen that the map $f$ also respects the group operation, hence this is also a homomorphism. Note that for all $g \in G,$ $g=mn,$ for $m \in M, n \in N.$ 
Then an arbitrary element of $(G/M) \times (G/N)$ is of the form $(nM,mN).$ Thus we can see that this corresponds to an element $mn \in G,$ which can cover 
all of $G.$ Thus $f$ is surjective. To compute the kernel of $f,$ see that $gM=e_{G/M} \implies g \in M,$ and $gN=e_{G/N} \implies g \in N$. Thus $g \in M 
\cap N,$ thus $\ker f=M \cap N.$ Using the first isomorphism theorem gives us our result. 
\end{document}
