%%%%%%%%%%%%%%%%%%%%%%%%%%%%%%%%%%%%%%%%%
% Lachaise Assignment
% LaTeX Template
% Version 1.0 (26/6/2018)
%
% This template originates from:
% http://www.LaTeXTemplates.com
%
% Authors:
% Marion Lachaise & François Févotte
% Vel (vel@LaTeXTemplates.com)
%
% License:
% CC BY-NC-SA 3.0 (http://creativecommons.org/licenses/by-nc-sa/3.0/)
% 
%%%%%%%%%%%%%%%%%%%%%%%%%%%%%%%%%%%%%%%%%

%----------------------------------------------------------------------------------------
%	PACKAGES AND OTHER DOCUMENT CONFIGURATIONS
%----------------------------------------------------------------------------------------

\documentclass{article}

\input{structure.tex} % Include the file specifying the document structure and custom commands
\usepackage{tikz-cd}
%----------------------------------------------------------------------------------------
%	ASSIGNMENT INFORMATION
%----------------------------------------------------------------------------------------

\title{Algebra 2 Homework 4} % Title of the assignment

\author{Gandhar Kulkarni (mmat2304)} % Author name and email address

\date{} % University, school and/or department name(s) and a date

%----------------------------------------------------------------------------------------

\begin{document}
 
\maketitle % Print the title

%----------------------------------------------------------------------------------------
%	INTRODUCTION
%----------------------------------------------------------------------------------------

\section{} %Problem 1 
We know that the commutator subgroup of $F,$ the free group on two generators, will be generated by all elements of the form $w_1w_2w_1^{-1}w_2^{-1},$ where 
$w_1,w_2$ are words in the free group. Let the set of generators be denoted by $S.$ We consider $DF,$ the commutator subgroup of $F.$ By the universal 
property of free groups, we have the commutative diagram 
% 
%https://q.uiver.app/#q=WzAsMyxbMCwyLCJTIl0sWzQsMiwiREYiXSxbMiwwLCJGKFMpIl0sWzAsMSwiaSIsMCx7InN0eWxlIjp7InRhaWwiOnsibmFtZSI6Imhvb2siLCJzaWRlIjoidG9wIn19fV0sWzAsMiwiaSIsMCx7InN0eWxlIjp7InRhaWwiOnsibmFtZSI6Imhvb2siLCJzaWRlIjoidG9wIn19fV0sWzIsMSwiXFxleGlzdHMgISBcXFBoaSIsMCx7InN0eWxlIjp7ImhlYWQiOnsibmFtZSI6ImVwaSJ9fX1dXQ==
\[\begin{tikzcd}
	&& {F(S)} \\
	\\
	S &&&& DF
	\arrow["i", hook, from=3-1, to=3-5]
	\arrow["i", hook, from=3-1, to=1-3]
	\arrow["{\exists ! \Phi}", two heads, from=1-3, to=3-5]
\end{tikzcd}\]
where $F(S)$ surjects onto $DF.$ Then see that for any element $w \in \ker \Phi,$ the action on the map is given by the inclusion of each letter of the 
word, and where it is sent. Due to this, we must have that $F(S) \cong DF.$ Since $DF$ is isomorphic to a free group of infinite rank, it is not finitely 
generated. 
\section{} %Problem 2
See that for all $i \in \{1,2,3,4,5\}$ we have $t'_i$ is a product of disjoint transpositions. We know that the order of $t'_i$ is the least common multiple 
of all the cycles, which is clearly $2$ in this case. Then we have ${t'}_i^2=1.$

Now see that $t_1't_2'= (1 ,2)(3 ,4)(5 ,6)\circ (1 ,4)(2, 5)(3, 6)=(1 ,3 ,5)(2, 6 ,4),$ $t_2't_3'=(1 ,4)(2, 5)(3, 6) \circ (1, 3)(2, 4)(5, 6)=(1, 6 ,2)(3, 
4, 5),$
$t_3't_4'= (1, 3)(2 ,4)(5 ,6) \circ (1, 2) (3, 6) (4, 5)=(1 ,4 ,6)(2, 3 ,5),$ and $t_4' t_5'= (1, 2) (3, 6) (4, 5) \circ (1, 4)(2, 3)(5, 6)=(1 ,5, 3)(2, 6, 
4).$ These are all 
products of disjoint $3-$cycles, thus they have order $3.$ 

It is also interesting to see that $t_1't_3'=(1 ,2)(3, 4)(5, 6)\circ (1, 3)(2, 4)(5 ,6)=(1 ,4)(2 ,3), t_1't_4'=(1 ,2)(3, 4)(5, 6)\circ (1 ,2) (3, 6) (4 ,5)= 
(3, 5 )(4 ,6), 
t_1't_5'=(1 ,2)(3, 4)(5, 6)\circ (1 ,4)(2, 3)(5, 6)= (1, 3)(2, 4), t_2't_4'=(1, 4)(2 ,5)(3, 6) \circ (1, 2) (3, 6) (4 ,5)=(1, 5)(2, 4), t_2't_5'=(1, 4)(2, 
5)(3, 6) \circ (1, 
4)(2, 3)(5 ,6)= (2 ,6)(3, 5),$ and $t_3't_5'= (1 ,3)(2, 4)(5, 6) \circ  (1 ,4)(2, 3)(5, 6)=(1 ,2)(3, 4),$ which is a product of disjoint $2-$cycles. 

See that $i_3't_5't_1'=(6,5),$ and $t_1't_2't_4'=(6,5,4,3,2,1).$ These two can generate $S_6,$ since a transposition and a $n-$cycle can generate $S_n.$
Thus we have a presentation for $S_n.$ We also know that transpositions that change consecutive elements generate $S_n,$ hence the map defined for the five 
such elements of $S_6$ suffice to determine an automorphism. This automorphism is clearly not inner.

\section{} %Problem 3 
Let $\alpha \in \mathbb{Q}.$ It satisfies a monic polynomial $f(x) \in \mathbb{Z}[x].$ Then we have $$\alpha^n + c_{n-1}\alpha^{n-1} + \dots + c_0=0.$$
Let $\alpha=\frac{p}{q},$ with $(p,q)=1.$ Then we have $$p^n + c_{n-1}p^{n-1}q + \dots + c_0q^n=0.$$ Reducing this equation modulo $q,$ we have 
$$p^n \equiv 0 \mod q.$$ Since $(p,q)=1,$ we have $q=1.$ Thus $\alpha \in \mathbb{Z}.$
\section{} %Problem 4 
We have $f(x)=x^5 -ax -1.$ For $a=0,$ $f(x)=(x-1)(x^4+x^3+x^2+x+1),$ which is a non-trivial reduction. For $a=2,$ we have $f(-1)=0,$ so this is also 
reducible. For $a=-1,$ we have $f(x)=x^5-x-1$ which can be factored as given by the problem. Now assume $a \neq -1,0,2.$  If $f(x)$ factors, it will either 
have a linear factor or a quadratic factor. Then we shall see that there can never be zero remainder, to prove that it is irreducible. 

If there was a linear factor, then $f$ would have an integer root, say $k$. By the rational roots theorem, we must have $k | -1.$ Thus we have $k= 1,-1.$ In 
either case, $f$ is not zero. Thus there is no linear factor. 

If there was a quadratic factor, then we have $x^5-ax-1 = (x^2 + a_1 x + a_0)(x^3 + b_2 x^2+ b_1 x + b_0).$ See that $a_0b_0 = -1,$ so we have two cases 
$a_0 =1, b_0 = -1$ and $a_0 = -1, b_0=1.$ In the first, we substitute this in the other constraints 
\begin{align*}
	b_2 + a_1 &= 0\\
	a_1b_2+ a_1 +b_1&= 0\\
	a_0b_2+a_1b_1+b_0 &=0\\
	a_0b_1 + a_1b_0 &= -a \\
\end{align*}  
Many substitutions later, we have $a_1^3 + a_1^2 -a_1 + 1=0.$ If it has an integer root, we must have $a_1 | 1.$ In either case $a_1 =1,-1,$ this equation 
is not satisfied. Thus there is no solution for $a_1.$

In the other case where $a_0=-1, b_0=1,$ we have the equation $ a_1^3 -a_1^2 + a_1 +1=0.$ The same way as above, if it is possible, $1, -1$ are the only 
roots. However, it is not zero at those points, hence there are no solutions for this case. 

Thus this polynomial is irreducible.
\section{} %Problem 5 
See that $x^2-4x+1$ is a polynomial that $2 + \sqrt{3}$ satisfies. The minimal polynomial if it is any smaller would have degree $1.$ But since $2 + 
\sqrt{3}$ is not rational, the degree of its minimal polynomial must be at least $2.$ Thus we have that $2+\sqrt{3}$ has exactly degree $2$ over 
$\mathbb{Q}.$ 

Consider the number field $\mathbb{Q}(\sqrt[3]{2})/\mathbb{Q},$ which is a degree $3$ extension. It is a $\mathbb{Q}$ vector space, so it clearly contains 
the element $1 + \sqrt[3]{2} + \sqrt[3]{4}.$ Therefore it must have degree at most $3$ in $\mathbb{Q}.$ Since it is not rational, its degree over 
$\mathbb{Q}$ must be at least $2.$ Let us see if any quadratic polynomial can satisfy it. Let $x^2+ax+b \in \mathbb{Q}[x]$ be some rational polynomial. 
Assume $\alpha = \sqrt[3]{2}.$ Assume that $(1+ \alpha + \alpha^2)$ satisfies this quadratic polynomial, so we must have 
\begin{align*}
	(1+\alpha+\alpha^2)^2 + a (1+\alpha+\alpha^2) + b &= 3\alpha^2 + 4 \alpha + 5 + a + a\alpha + a\alpha^2 + b\\
	&= (3+a)\alpha^2 + (4+a)\alpha + (5+a+b). 
\end{align*}
If this is to be $0,$ then we must have $a=-3$ and $a=-4,$ which is absurd. 

Therefore $1 + \sqrt[3]{2} + \sqrt[3]{4}$ must be of degree $3.$
\section{} %Problem 6 
If we can find a $a+bi \in \mathbb{Q}(i),$ such that $x^3-q$ vanishes for $q \in \{2,3\} \in \mathbb{Q}$, then we can reduce the polynomial. If this is 
possible, then we must have $$ (a+bi)^3= (a^3-3ab^2) + i(3a^2b-b^3) \in \mathbb{Q},$$ which forces $3a^2b=b^3.$ If $b=0,$ then it is equivalent to asking if 
a rational root for  $q$ exists, which is not true. Thus we must have $b \neq 0.$ Then we have $3a^2=b^2,$ which has no rational solution since $\sqrt{3}$ 
is not rational. Therefore we must have that $x^3-2$ and $x^3-3$ are both irreducible, since they have no solutions on $\mathbb{Q}(i).$
\section{} %Problem 7 
Let $q(x)$ be an irreducible polynomial dividing $f(g(x)).$ Then we define $K \cong \frac{F[x]}{(q(x))}.$ We consider the canonical projection $\pi: F[x] 
\to K.$ We consider the image of $ g(x)$ under $\pi.$ We get $\overline{g(x)} = g(x)+ (q(x)).$ We claim that $f(\overline{g(x)})$ is $0.$ We have 
$$f(\overline{g(x)}) = \sum_{i=0}^{n} a_i (g(x)+ (q(x)))^i = \sum_{i=0}^{n} a_i (g(x)^i+ (q(x))) =\sum_{i=0}^{n} a_i g(x)^i+ (q(x)) = f(g(x)) + (q(x)).$$ 
We know that $f(g(x))$ is a multiple of $q(x),$ thus $\overline{g(x)}$ is a root of $f(g(x)).$ Let $\alpha= \overline{g(x)}.$ 
We have a root of $f$ over $K,$ and we know $f$ is irreducible on $F,$ thus $[F(\alpha): F]= \deg f= n.$ Note that $q $ is irreducible over $F,$ thus we 
have $[K:F]=\deg q.$ Now we have $$[K:F] = [K:F(\alpha)][F(\alpha):F],$$ which gives the required result that $n | \deg q.$.  
\end{document}
