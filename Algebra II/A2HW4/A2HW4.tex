%%%%%%%%%%%%%%%%%%%%%%%%%%%%%%%%%%%%%%%%%
% Lachaise Assignment
% LaTeX Template
% Version 1.0 (26/6/2018)
%
% This template originates from:
% http://www.LaTeXTemplates.com
%
% Authors:
% Marion Lachaise & François Févotte
% Vel (vel@LaTeXTemplates.com)
%
% License:
% CC BY-NC-SA 3.0 (http://creativecommons.org/licenses/by-nc-sa/3.0/)
% 
%%%%%%%%%%%%%%%%%%%%%%%%%%%%%%%%%%%%%%%%%

%----------------------------------------------------------------------------------------
%	PACKAGES AND OTHER DOCUMENT CONFIGURATIONS
%----------------------------------------------------------------------------------------

\documentclass{article}

%%%%%%%%%%%%%%%%%%%%%%%%%%%%%%%%%%%%%%%%%
% Lachaise Assignment
% Structure Specification File
% Version 1.0 (26/6/2018)
%
% This template originates from:
% http://www.LaTeXTemplates.com
%
% Authors:
% Marion Lachaise & François Févotte
% Vel (vel@LaTeXTemplates.com)
%
% License:
% CC BY-NC-SA 3.0 (http://creativecommons.org/licenses/by-nc-sa/3.0/)
% 
%%%%%%%%%%%%%%%%%%%%%%%%%%%%%%%%%%%%%%%%%

%----------------------------------------------------------------------------------------
%	PACKAGES AND OTHER DOCUMENT CONFIGURATIONS
%----------------------------------------------------------------------------------------

\usepackage{amsmath,amsfonts,amssymb, tikz-cd} % Math packages

\usepackage{enumerate} % Custom item numbers for enumerations


\usepackage[framemethod=tikz]{mdframed} % Allows defining custom boxed/framed environments

\usepackage{listings} % File listings, with syntax highlighting
\lstset{
	basicstyle=\ttfamily, % Typeset listings in monospace font
}

%----------------------------------------------------------------------------------------
%	DOCUMENT MARGINS
%----------------------------------------------------------------------------------------

\usepackage{geometry} % Required for adjusting page dimensions and margins

\geometry{
	paper=letterpaper, % Paper size, change to letterpaper for US letter size
	top=2.5cm, % Top margin
	bottom=3cm, % Bottom margin
	left=2.5cm, % Left margin
	right=2.5cm, % Right margin
	headheight=14pt, % Header height
	footskip=1.5cm, % Space from the bottom margin to the baseline of the footer
	headsep=1.2cm, % Space from the top margin to the baseline of the header
	%showframe, % Uncomment to show how the type block is set on the page
}

%----------------------------------------------------------------------------------------
%	FONTS
%----------------------------------------------------------------------------------------

\usepackage[utf8]{inputenc} % Required for inputting international characters
\usepackage[T1]{fontenc} % Output font encoding for international characters


%----------------------------------------------------------------------------------------
%	COMMAND LINE ENVIRONMENT
%----------------------------------------------------------------------------------------

% Usage:
% \begin{commandline}
	%	\begin{verbatim}
		%		$ ls
		%		
		%		Applications	Desktop	...
		%	\end{verbatim}
	% \end{commandline}

\mdfdefinestyle{commandline}{
	leftmargin=10pt,
	rightmargin=10pt,
	innerleftmargin=15pt,
	middlelinecolor=black!50!white,
	middlelinewidth=2pt,
	frametitlerule=false,
	backgroundcolor=black!5!white,
	frametitle={Command Line},
	frametitlefont={\normalfont\sffamily\color{white}\hspace{-1em}},
	frametitlebackgroundcolor=black!50!white,
	nobreak,
}

% Define a custom environment for command-line snapshots
\newenvironment{commandline}{
	\medskip
	\begin{mdframed}[style=commandline]
	}{
	\end{mdframed}
	\medskip
}

%----------------------------------------------------------------------------------------
%	FILE CONTENTS ENVIRONMENT
%----------------------------------------------------------------------------------------

% Usage:
% \begin{file}[optional filename, defaults to "File"]
	%	File contents, for example, with a listings environment
	% \end{file}

\mdfdefinestyle{file}{
	innertopmargin=1.6\baselineskip,
	innerbottommargin=0.8\baselineskip,
	topline=false, bottomline=false,
	leftline=false, rightline=false,
	leftmargin=2cm,
	rightmargin=2cm,
	singleextra={%
		\draw[fill=black!10!white](P)++(0,-1.2em)rectangle(P-|O);
		\node[anchor=north west]
		at(P-|O){\ttfamily\mdfilename};
		%
		\def\l{3em}
		\draw(O-|P)++(-\l,0)--++(\l,\l)--(P)--(P-|O)--(O)--cycle;
		\draw(O-|P)++(-\l,0)--++(0,\l)--++(\l,0);
	},
	nobreak,
}

% Define a custom environment for file contents
\newenvironment{file}[1][File]{ % Set the default filename to "File"
	\medskip
	\newcommand{\mdfilename}{#1}
	\begin{mdframed}[style=file]
	}{
	\end{mdframed}
	\medskip
}

%----------------------------------------------------------------------------------------
%	NUMBERED QUESTIONS ENVIRONMENT
%----------------------------------------------------------------------------------------

% Usage:
% \begin{question}[optional title]
	%	Question contents
	% \end{question}

\mdfdefinestyle{question}{
	innertopmargin=1.2\baselineskip,
	innerbottommargin=0.8\baselineskip,
	roundcorner=5pt,
	nobreak,
	singleextra={%
		\draw(P-|O)node[xshift=1em,anchor=west,fill=white,draw,rounded corners=5pt]{%
			Question \theQuestion\questionTitle};
	},
}

\newcounter{Question} % Stores the current question number that gets iterated with each new question

% Define a custom environment for numbered questions
\newenvironment{question}[1][\unskip]{
	\bigskip
	\stepcounter{Question}
	\newcommand{\questionTitle}{~#1}
	\begin{mdframed}[style=question]
	}{
	\end{mdframed}
	\medskip
}

%----------------------------------------------------------------------------------------
%	WARNING TEXT ENVIRONMENT
%----------------------------------------------------------------------------------------

% Usage:
% \begin{warn}[optional title, defaults to "Warning:"]
	%	Contents
	% \end{warn}

\mdfdefinestyle{warning}{
	topline=false, bottomline=false,
	leftline=false, rightline=false,
	nobreak,
	singleextra={%
		\draw(P-|O)++(-0.5em,0)node(tmp1){};
		\draw(P-|O)++(0.5em,0)node(tmp2){};
		\fill[black,rotate around={45:(P-|O)}](tmp1)rectangle(tmp2);
		\node at(P-|O){\color{white}\scriptsize\bf !};
		\draw[very thick](P-|O)++(0,-1em)--(O);%--(O-|P);
	}
}

% Define a custom environment for warning text
\newenvironment{warn}[1][Warning:]{ % Set the default warning to "Warning:"
	\medskip
	\begin{mdframed}[style=warning]
		\noindent{\textbf{#1}}
	}{
	\end{mdframed}
}

%----------------------------------------------------------------------------------------
%	INFORMATION ENVIRONMENT
%----------------------------------------------------------------------------------------

% Usage:
% \begin{info}[optional title, defaults to "Info:"]
	% 	contents
	% 	\end{info}

\mdfdefinestyle{info}{%
	topline=false, bottomline=false,
	leftline=false, rightline=false,
	nobreak,
	singleextra={%
		\fill[black](P-|O)circle[radius=0.4em];
		\node at(P-|O){\color{white}\scriptsize\bf i};
		\draw[very thick](P-|O)++(0,-0.8em)--(O);%--(O-|P);
	}
}

% Define a custom environment for information
\newenvironment{info}[1][Info:]{ % Set the default title to "Info:"
	\medskip
	\begin{mdframed}[style=info]
		\noindent{\textbf{#1}}
	}{
	\end{mdframed}
}
 % Include the file specifying the document structure and custom commands

%----------------------------------------------------------------------------------------
%	ASSIGNMENT INFORMATION
%----------------------------------------------------------------------------------------

\title{} % Title of the assignment

\author{Gandhar Kulkarni (mmat2304)} % Author name and email address

\date{} % University, school and/or department name(s) and a date

%----------------------------------------------------------------------------------------

\begin{document}
 
\maketitle % Print the title

%----------------------------------------------------------------------------------------
%	INTRODUCTION
%----------------------------------------------------------------------------------------

\section{} %Problem 1 
\section{} %Problem 2
See that for all $i \in \{1,2,3,4,5\}$ we have $t'_i$ is a product of disjoint transpositions. We know that the order of $t'_i$ is the least common multiple 
of all the cycles, which is clearly $2$ in this case. Then we have ${t'}_i^2=1.$

Now see that $t_1't_2'= (1 ,2)(3 ,4)(5 ,6)\circ (1 ,4)(2, 5)(3, 6)=(1 ,3 ,5)(2, 6 ,4),$ $t_2't_3'=(1 ,4)(2, 5)(3, 6) \circ (1, 3)(2, 4)(5, 6)=(1, 6 ,2)(3, 
4, 5),$
$t_3't_4'= (1, 3)(2 ,4)(5 ,6) \circ (1, 2) (3, 6) (4, 5)=(1 ,4 ,6)(2, 3 ,5),$ and $t_4' t_5'= (1, 2) (3, 6) (4, 5) \circ (1, 4)(2, 3)(5, 6)=(1 ,5, 3)(2, 6, 
4).$ These are all 
products of disjoint $3-$cycles, thus they have order $3.$ 

It is also interesting to see that $t_1't_3'=(1 ,2)(3, 4)(5, 6)\circ (1, 3)(2, 4)(5 ,6)=(1 ,4)(2 ,3), t_1't_4'=(1 ,2)(3, 4)(5, 6)\circ (1 ,2) (3, 6) (4 ,5)= 
(3, 5 )(4 ,6), 
t_1't_5'=(1 ,2)(3, 4)(5, 6)\circ (1 ,4)(2, 3)(5, 6)= (1, 3)(2, 4), t_2't_4'=(1, 4)(2 ,5)(3, 6) \circ (1, 2) (3, 6) (4 ,5)=(1, 5)(2, 4), t_2't_5'=(1, 4)(2, 
5)(3, 6) \circ (1, 
4)(2, 3)(5 ,6)= (2 ,6)(3, 5),$ and $t_3't_5'= (1 ,3)(2, 4)(5, 6) \circ  (1 ,4)(2, 3)(5, 6)=(1 2,)(3, 4),$ which is a product of disjoint $2-$cycles. 

We know from a previous assignment that $S_n$ can be generated by elements of the form $(n, n+1).$ Then see that $(1,2)= $
\section{} %Problem 3 
Let $\alpha \in \mathbb{Q}.$ It satisfies a monic polynomial $f(x) \in \mathbb{Z}[x].$ Then we have $$\alpha^n + c_{n-1}\alpha^{n-1} + \dots + c_0=0.$$
Let $\alpha=\frac{p}{q},$ with $(p,q)=1.$ Then we have $$p^n + c_{n-1}p^{n-1}q + \dots + c_0q^n=0.$$ Reducing this equation modulo $q,$ we have 
$$p^n \equiv 0 \mod q.$$ Since $(p,q)=1,$ we have $q=1.$ Thus $\alpha \in \mathbb{Z}.$
\section{} %Problem 4 
We have $f(x)=x^5 -ax -1.$ For $a=0,$ $f(x)=(x-1)(x^4+x^3+x^2+x+1),$ which is a non-trivial reduction. For $a=2,$ we have $f(-1)=0,$ so this is also 
reducible. For $a=-1,$ we have $f(x)=x^5-x-1$ which can be factored as given by the problem. Now assume $a \neq -1,0,2.$  
\section{} %Problem 5 
See that $x^2-4x+1$ is a polynomial that $2 + \sqrt{3}$ satisfies. The minimal polynomial if it is any smaller would have degree $1.$ But since $2 + 
\sqrt{3}$ is not rational, the degree of its minimal polynomial must be at least $2.$ Thus we have that $2+\sqrt{3}$ has exactly degree $2$ over 
$\mathbb{Q}.$ 

Consider the number field $\mathbb{Q}(\sqrt[3]{2})/\mathbb{Q},$ which is a degree $3$ extension. It is a $\mathbb{Q}$ vector space, so it clearly contains 
the element $1 + \sqrt[3]{2} + \sqrt[3]{4}.$ Therefore it must have degree at most $3$ in $\mathbb{Q}.$ Since it is not rational, its degree over 
$\mathbb{Q}$ must be at least $2.$ Let us see if any quadratic polynomial can satisfy it. Let $x^2+ax+b \in \mathbb{Q}[x]$ be some rational polynomial. 
Assume $\alpha = \sqrt[3]{2}.$ Assume that $(1+ \alpha + \alpha^2)$ satisfies this quadratic polynomial, so we must have 
\begin{align*}
	(1+\alpha+\alpha^2)^2 + a (1+\alpha+\alpha^2) + b &= 3\alpha^2 + 4 \alpha + 5 + a + a\alpha + a\alpha^2 + b\\
	&= (3+a)\alpha^2 + (4+a)\alpha + (5+a+b). 
\end{align*}
If this is to be $0,$ then we must have $a=-3$ and $a=-4,$ which is absurd. 

Therefore $1 + \sqrt[3]{2} + \sqrt[3]{4}$ must be of degree $3.$
\section{} %Problem 6 
If we can find a $a+bi \in \mathbb{Q}(i),$ such that $x^3-q$ vanishes for $q \in \{2,3\} \in \mathbb{Q}$, then we can reduce the polynomial. If this is 
possible, then we must have $$ (a+bi)^3= (a^3-3ab^2) + i(3a^2b-b^3) \in \mathbb{Q},$$ which forces $3a^2b=b^3.$ If $b=0,$ then it is equivalent to asking if 
a rational root for  $q$ exists, which is not true. Thus we must have $b \neq 0.$ Then we have $3a^2=b^2,$ which has no rational solution since $\sqrt{3}$ 
is not rational. Therefore we must have that $x^3-2$ and $x^3-3$ are both irreducible, since they have no solutions on $\mathbb{Q}(i).$
\section{} %Problem 7 

\end{document}
