%%%%%%%%%%%%%%%%%%%%%%%%%%%%%%%%%%%%%%%%%
% Lachaise Assignment
% LaTeX Template
% Version 1.0 (26/6/2018)
%
% This template originates from:
% http://www.LaTeXTemplates.com
%
% Authors:
% Marion Lachaise & François Févotte
% Vel (vel@LaTeXTemplates.com)
%
% License:
% CC BY-NC-SA 3.0 (http://creativecommons.org/licenses/by-nc-sa/3.0/)
% 
%%%%%%%%%%%%%%%%%%%%%%%%%%%%%%%%%%%%%%%%%

%----------------------------------------------------------------------------------------
%	PACKAGES AND OTHER DOCUMENT CONFIGURATIONS
%----------------------------------------------------------------------------------------

\documentclass{article}

%%%%%%%%%%%%%%%%%%%%%%%%%%%%%%%%%%%%%%%%%
% Lachaise Assignment
% Structure Specification File
% Version 1.0 (26/6/2018)
%
% This template originates from:
% http://www.LaTeXTemplates.com
%
% Authors:
% Marion Lachaise & François Févotte
% Vel (vel@LaTeXTemplates.com)
%
% License:
% CC BY-NC-SA 3.0 (http://creativecommons.org/licenses/by-nc-sa/3.0/)
% 
%%%%%%%%%%%%%%%%%%%%%%%%%%%%%%%%%%%%%%%%%

%----------------------------------------------------------------------------------------
%	PACKAGES AND OTHER DOCUMENT CONFIGURATIONS
%----------------------------------------------------------------------------------------

\usepackage{amsmath,amsfonts,amssymb, tikz-cd} % Math packages

\usepackage{enumerate} % Custom item numbers for enumerations


\usepackage[framemethod=tikz]{mdframed} % Allows defining custom boxed/framed environments

\usepackage{listings} % File listings, with syntax highlighting
\lstset{
	basicstyle=\ttfamily, % Typeset listings in monospace font
}

%----------------------------------------------------------------------------------------
%	DOCUMENT MARGINS
%----------------------------------------------------------------------------------------

\usepackage{geometry} % Required for adjusting page dimensions and margins

\geometry{
	paper=letterpaper, % Paper size, change to letterpaper for US letter size
	top=2.5cm, % Top margin
	bottom=3cm, % Bottom margin
	left=2.5cm, % Left margin
	right=2.5cm, % Right margin
	headheight=14pt, % Header height
	footskip=1.5cm, % Space from the bottom margin to the baseline of the footer
	headsep=1.2cm, % Space from the top margin to the baseline of the header
	%showframe, % Uncomment to show how the type block is set on the page
}

%----------------------------------------------------------------------------------------
%	FONTS
%----------------------------------------------------------------------------------------

\usepackage[utf8]{inputenc} % Required for inputting international characters
\usepackage[T1]{fontenc} % Output font encoding for international characters


%----------------------------------------------------------------------------------------
%	COMMAND LINE ENVIRONMENT
%----------------------------------------------------------------------------------------

% Usage:
% \begin{commandline}
	%	\begin{verbatim}
		%		$ ls
		%		
		%		Applications	Desktop	...
		%	\end{verbatim}
	% \end{commandline}

\mdfdefinestyle{commandline}{
	leftmargin=10pt,
	rightmargin=10pt,
	innerleftmargin=15pt,
	middlelinecolor=black!50!white,
	middlelinewidth=2pt,
	frametitlerule=false,
	backgroundcolor=black!5!white,
	frametitle={Command Line},
	frametitlefont={\normalfont\sffamily\color{white}\hspace{-1em}},
	frametitlebackgroundcolor=black!50!white,
	nobreak,
}

% Define a custom environment for command-line snapshots
\newenvironment{commandline}{
	\medskip
	\begin{mdframed}[style=commandline]
	}{
	\end{mdframed}
	\medskip
}

%----------------------------------------------------------------------------------------
%	FILE CONTENTS ENVIRONMENT
%----------------------------------------------------------------------------------------

% Usage:
% \begin{file}[optional filename, defaults to "File"]
	%	File contents, for example, with a listings environment
	% \end{file}

\mdfdefinestyle{file}{
	innertopmargin=1.6\baselineskip,
	innerbottommargin=0.8\baselineskip,
	topline=false, bottomline=false,
	leftline=false, rightline=false,
	leftmargin=2cm,
	rightmargin=2cm,
	singleextra={%
		\draw[fill=black!10!white](P)++(0,-1.2em)rectangle(P-|O);
		\node[anchor=north west]
		at(P-|O){\ttfamily\mdfilename};
		%
		\def\l{3em}
		\draw(O-|P)++(-\l,0)--++(\l,\l)--(P)--(P-|O)--(O)--cycle;
		\draw(O-|P)++(-\l,0)--++(0,\l)--++(\l,0);
	},
	nobreak,
}

% Define a custom environment for file contents
\newenvironment{file}[1][File]{ % Set the default filename to "File"
	\medskip
	\newcommand{\mdfilename}{#1}
	\begin{mdframed}[style=file]
	}{
	\end{mdframed}
	\medskip
}

%----------------------------------------------------------------------------------------
%	NUMBERED QUESTIONS ENVIRONMENT
%----------------------------------------------------------------------------------------

% Usage:
% \begin{question}[optional title]
	%	Question contents
	% \end{question}

\mdfdefinestyle{question}{
	innertopmargin=1.2\baselineskip,
	innerbottommargin=0.8\baselineskip,
	roundcorner=5pt,
	nobreak,
	singleextra={%
		\draw(P-|O)node[xshift=1em,anchor=west,fill=white,draw,rounded corners=5pt]{%
			Question \theQuestion\questionTitle};
	},
}

\newcounter{Question} % Stores the current question number that gets iterated with each new question

% Define a custom environment for numbered questions
\newenvironment{question}[1][\unskip]{
	\bigskip
	\stepcounter{Question}
	\newcommand{\questionTitle}{~#1}
	\begin{mdframed}[style=question]
	}{
	\end{mdframed}
	\medskip
}

%----------------------------------------------------------------------------------------
%	WARNING TEXT ENVIRONMENT
%----------------------------------------------------------------------------------------

% Usage:
% \begin{warn}[optional title, defaults to "Warning:"]
	%	Contents
	% \end{warn}

\mdfdefinestyle{warning}{
	topline=false, bottomline=false,
	leftline=false, rightline=false,
	nobreak,
	singleextra={%
		\draw(P-|O)++(-0.5em,0)node(tmp1){};
		\draw(P-|O)++(0.5em,0)node(tmp2){};
		\fill[black,rotate around={45:(P-|O)}](tmp1)rectangle(tmp2);
		\node at(P-|O){\color{white}\scriptsize\bf !};
		\draw[very thick](P-|O)++(0,-1em)--(O);%--(O-|P);
	}
}

% Define a custom environment for warning text
\newenvironment{warn}[1][Warning:]{ % Set the default warning to "Warning:"
	\medskip
	\begin{mdframed}[style=warning]
		\noindent{\textbf{#1}}
	}{
	\end{mdframed}
}

%----------------------------------------------------------------------------------------
%	INFORMATION ENVIRONMENT
%----------------------------------------------------------------------------------------

% Usage:
% \begin{info}[optional title, defaults to "Info:"]
	% 	contents
	% 	\end{info}

\mdfdefinestyle{info}{%
	topline=false, bottomline=false,
	leftline=false, rightline=false,
	nobreak,
	singleextra={%
		\fill[black](P-|O)circle[radius=0.4em];
		\node at(P-|O){\color{white}\scriptsize\bf i};
		\draw[very thick](P-|O)++(0,-0.8em)--(O);%--(O-|P);
	}
}

% Define a custom environment for information
\newenvironment{info}[1][Info:]{ % Set the default title to "Info:"
	\medskip
	\begin{mdframed}[style=info]
		\noindent{\textbf{#1}}
	}{
	\end{mdframed}
}
 % Include the file specifying the document structure and custom commands

%----------------------------------------------------------------------------------------
%	ASSIGNMENT INFORMATION
%----------------------------------------------------------------------------------------

\title{} % Title of the assignment

\author{Gandhar Kulkarni (mmat2304)} % Author name and email address

\date{} % University, school and/or department name(s) and a date

%----------------------------------------------------------------------------------------

\begin{document}

\maketitle % Print the title

%----------------------------------------------------------------------------------------
%	INTRODUCTION
%----------------------------------------------------------------------------------------

\section{} %Problem 1 
\begin{enumerate}
	\item Let $\mathcal{K}$ be a conjugacy class, let $x \in G$ be a representative. Then $\mathcal{K}=\{gxg^{-1}: x \in G  \},$ and $$\sigma(\mathcal{K})= 
	\{ \sigma(gxg^{-1})= \sigma(g) \sigma(x) \sigma(g)^{-1}: g \in G\}.$$ Since $\sigma$ is an automorphism, it is a bijection. Then $\sigma(g)$ spans all 
	of $G.$ Then we have $\sigma(\mathcal{K})=\sigma(\text{Cl}(x))= \text{Cl}(\sigma(x)).$ Thus this is also a conjugacy class.
	\item See that in $S_n$ $\mathcal{K}=\text{Cl}((12)),$ that is, all elements that switch two and only two elements. 
	However, in general, the product of multiple transpositions also has order $2,$ but they are not transpositions. For example, in $S_n,$ where $n \geq 
	4,$ $(1 2)(3 4)$ has order $2,$ but are not $2-$cycles. For $n=2,$ there are no order $2$ permutations that aren't $2-$cycles, so this is vacuously true 
	for $n=2.$ For $n=3,$ the same principle applies since it is not possible to find two disjoint $2-$cycles. We have $ |\mathcal{K}| = \binom{n}{2},$ 
	while $|\mathcal{K'}| = \binom{n}{2}\binom{n-2}{2} + \dots + \binom{n}{2} \dots \binom{n- 2 \lfloor\frac{n}{2}\rfloor}{2},$ which is greater than 
	$|\mathcal{K}|$ for $n \geq 2, n \neq 6.$ Thus $|\mathcal{K}| \neq |\mathcal{K'}|.$ 
	
	See that for any transposition $(a b),$ it swaps two elements. When an automorphism $\sigma$ acts on $(a b),$ we know that $a \mapsto a',$ $b \mapsto 
	b',$ where $a' \neq b',$ since automorphisms are bijective. Then $\sigma(a b)$ is also a transposition.
	\item  We have $\sigma((1 2))= (a b_2).$ See that $\sigma((1 2 k))= \sigma ((1 k ) (1 2)  )= \sigma((1 k))\sigma((1 2))=\sigma((1 k)) (a b_2).$ If 
	$\sigma((1 k))$ is disjoint with $(a b_2),$ then the resultant element is of order $2.$ Thus $\sigma((1 k))$ must overlap with $(a b_2)$ for $k.$ 
	On the other hand, we have $$\sigma((1 k)(i j))= \sigma((1 k)(1 j)(1 i))= \sigma((1 k))(a b_i)(a b_j)(a b_i)= \sigma((1 k))(b_i b_j).$$
	Now, if $\sigma((1 k))$ has elements in common with $(b_i b_j)$ then we get a $3-$cycle, which cannot happen. Thus all elements of $\sigma((1 k))$ 
	overlap with $(1 2)$ have the same element $a$. Thus $\sigma((1 k))=(a b_k).$
	
	\item Any permutation in $S_n$ can be written as $(a_1 \dots a_k)= (1 a_k)\dots (1 a_1).$ Then any automorphism is decided by where it sends its 
	generators, which is $(1 k)$. First to choose $a$ we have $n$ choices. We have $n-1$ choices for $b_2,$ and so on. Thus we have at most $n!$ 
	automorphisms of $n.$ However, we know that $S_n$ has trivial centre which means that 
	$$\text{Inn}(S_n) \cong \frac{S_n}{\{e\}} \cong S_n,$$ and $\text{Inn}(S_n) \leq \text{Aut}(S_n),$ which means that we must have at least $n!$ 
	automorphisms! This means that $\text{Aut}(S_n)=\text{Inn}(S_n)=S_n.$
\end{enumerate}
\section{} %Problem 2 
\begin{enumerate}
	\item Let $H$ be a non-abelian simple group. Consider $DH \trianglelefteq H.$ Then since $H$ is simple, we must have $DH= H$ or $DH=0.$ If $DH=0,$ then 
	$H$ would be abelian, which is not possible, thus $H$ is perfect.
	\item We know that $DH=H, DK=K.$ Thus any term in $H$ and $K$ can be seen as an element of the type $h_1h_2h_1^{-1}h_2^{-1}$ for $H$ and respectively 
	for $K.$ Thus we consider any word in $\langle H,K \rangle.$ Then it is of the form $h_1k_1 \dots h_n k_n,$ where $h_1,\dots, h_n \in H, k_1,\dots, k_n 
	\in K.$ We know that we can write any $h \in H$ and $k \in K$ can be written as an element of the commutator. The commutator is generated by such 
	elements. Thus we have $\langle H,K \rangle= D\langle H,K \rangle.$ 
	
	\item We propose that $D(g^{-1}Hg)=g^{-1}DHg.$ Take any element of $D(g^{-1}Hg),$ which is of the form 
	$(g^{-1}h_1g)(g^{-1}h_2g)(g^{-1}h_1^{-1}g)(g^{-1}h_2^{-1}g)=g^{-1}(h_1h_2h_1^{-1}h_2^{-1})g \in g^{-1}DHg.$ These operations are all if and only if 
	statements, hence $D(g^{-1}Hg)=g^{-1}DHg.$ If $H=DH,$ $D(g^{-1}Hg)=g^{-1}Hg.$
	\item If $G$ is simple, then either $DG=0$ or $DG=G.$ Then for abelian simple groups the maximal perfect subgroup is $0,$ and in the non-abelian case it 
	is $G$ itself. Both are clearly normal in $G.$ 
\end{enumerate}
\section{} %Problem 3 
Since $K$ is cyclic, it is generated by some element, say $x \in K,$ and we know that the map $\varphi_1$ and $\varphi_2$ are decided entirely by where it 
sends the element $x.$ We know that there exists $\sigma \in \text{Aut}(H)$ such that $\sigma \varphi_1(K) \sigma^{-1}= \varphi_2(K),$ so we have
$\sigma \varphi_1(x) \sigma^{-1} = \varphi_2(x)^l,$ for some $l \in \mathbb{Z}.$ Then $k=x^t \in K,$ we have $\sigma \varphi_1(k) \sigma^{-1}=(\sigma 
\varphi_1(x) \sigma^{-1})^t= \varphi_2(x)^{a},$ where $t \in \mathbb{Z}.$  

Take the map $\psi: H \rtimes_{\varphi_1} K \to H \rtimes_{\varphi_2} K$ where $\psi(h,k)=(\sigma(h),k^a).$ To see that this is a homomorphism, we have 

\begin{align*}
	\psi((h_1,k_1)\circ_{1}(h_2,k_2)) &= \psi((h_1 \varphi_1(k_1)(h_2),k_1k_2))\\
	&= (\sigma(h_1)\sigma(\varphi_1(k_1)(h_2)), (k_1k_2)^a).
\end{align*}
Also, see that \begin{align*}
	\psi((h_1,k_1))\circ_{2}\psi((h_2,k_2)) &= (\sigma(h_1),k_1^a) \circ_{2} (\sigma(h_2),k_2^a)\\
	&=(\sigma(h_1) \varphi_2(k_1)^a(\sigma(h_2)), (k_1k_2)^a)\\
	&=(\sigma(h_1)\sigma\varphi_1(k_1)\sigma^{-1}\sigma(h_2), (k_1k_2)^a)\\
	&= (\sigma(h_1)\sigma(\varphi_1(k_1)(h_2)), (k_1k_2)^a). 
\end{align*}

This shows that we have a homomorphism. We need to find the kernel of $\psi.$ See that if $\sigma(h)=e_H,$ then we must have $h=e_H,$ since $\sigma \in 
\text{Aut}(H),$ thus it is bijective and preserves the identity. If $k^a=e_K,$ then applying $\varphi_2$ on both sides, we have $\varphi_2(k^a)= \sigma 
\varphi_1(k)\sigma^{-1}=e_K \implies \varphi_1(k)=e_K.$ Since $\varphi_1$ is injective, we must have $k=e_K.$ Thus $\psi$ is injective since the kernel is 
trivial.

We know that $x$ is a generator of $K,$ so $\varphi_1(x)$ generates $\varphi_1(K),$ and hence $\varphi_2(x)^a$ generates $\varphi_2(K).$ Thus there is some 
power $a'\in \mathbb{Z}$ such that $ (x^a)^{a'}=e$ and $\sigma^{-1} \varphi_2(x) \sigma=\varphi_1(x^{a'}).$ Then see that for any $(h,k) \in H 
\rtimes_{\varphi_2} K,$ we have $(\sigma^{-1}(h),k^{a'})$ as the corresponding pre-image, which means that $\psi$ is onto, as required. 

\section{} %Problem 4 
We have $|G| =75.$ We clearly have a $5-$Sylow subgroup of order $25,$ as well as a $3-$Sylow group of order $3.$ Since the number of $5-$Sylow subgroups 
must be $5k+1$ and $5k+1|3,$ we are forced to have a normal subgroup of order $25.$ Then let $H \trianglelefteq G$ where $|H|=25.$ Let $K$ be some $3-$Sylow 
subgroup of order $3.$ Then by Lagrange's theorem, $|H \cap K|=1,$ so we can construct a semi-direct product of the groups.  

We have that the number of $3-$Sylow subgroups of $G$ are $n_3=3k+1,$ where $3k+1|25.$ The only choices are $n_3=1,$ which would give us an abelian group, 
so assume that $n_3=25,$ the only other choice. These subgroups are all conjugate to each other. 

We have $H$ of order $25,$ hence it is abelian as it is the square of a prime. Then it is either $\mathbb{Z}^{25}$ or $\mathbb{Z}_5 \times \mathbb{Z}_5 .$ 
Since the units of $\mathbb{Z}^{25}$ have order $25-5=20,$ we cannot have any non trivial map from $K$ to $H.$ Therefore we must have $H=\mathbb{Z}_5 \times 
\mathbb{Z}_5.$ The automorphisms of this group are basically the invertible maps on $\mathbb{Z}_5 \times \mathbb{Z}_5,$ which means $\text{Aut}(\mathbb{Z}_5 
\times \mathbb{Z}_5)=\text{GL}_2(\mathbb{Z}_5).$ This group has order $(p^2-1)(p^2-p)=480.$ 

See that $|K|||H|,$ but only once. Using the previous result we want to show that $H \rtimes_{\varphi_1} K \cong H \rtimes_{\varphi_2} K,$ for some 
$\varphi_1, \varphi_2: K \to \text{Aut}(H).$ See that any map from $k \to \text{Aut}(H)$ is determined solely by where it sends $a \neq e_K.$ Then we have 
$480$ choices, with one choices being where $a$ is sent to the trivial automorphism. This corresponds to the direct product of $H$ and $K.$ 

For $\varphi_1, \varphi_2: K \to \text{Aut}(H),$ both non-trivial, we have $\varphi_1(K),$ and $\varphi_2(K)$ with three elements each. Hence there can only 
be one non-abelian group. We pick $a,$ the generator of $K$ to be sent to some non-trivial automorphism of $H.$ That gives us $H \rtimes_{\varphi} K,$ where 
$\varphi(a) \in \text{GL}_2(\mathbb{Z}_5)\backslash \{0\}.$ 
\section{} %Problem 5
Let $n= p_1^{\alpha_1}\dots p_r^{\alpha_r}$ be such that $\alpha_i=1,2$ for all $1 \leq i \leq r$ and $p_i \nmid p_j^{\alpha_j}-1$ for all $i,j.$ 
Assume WLOG that $p_1 < \dots < p_r.$ We will proceed by induction on $r,$ the number of prime factor For $r=1,$ we have order $p$ and $p^2,$ which are 
always abelian. For $r=2,$ we have four possibilities--- $p_1p_2, p_1p_2^2, p_1^2p_2, p_1^2p_2^2.$ The number of $p-$Sylow groups in any case must divide 
$p_j^{\alpha}$ for the other prime, and be $1$ modulo $p_i$ which is only possible if the number is $1,$ since $p_i \nmid p_j^{\alpha_j}-1.$
So we have that all these groups are the direct product of two abelian groups, which is thus abelian.

For $r \geq 3,$ we consider the result that $G$ is solvable. To see this, we use our assumption that all proper subgroups of $G$ are abelian. Thus any 
normal series we have should have abelian quotient. Then $DG,$ the commutator subgroup of $G$ is abelian. We suppose that $|DG|$ isn't the power of a prime. 
Then we have $DG=H \times K,$ where $H,K$ are non-trivial with coprime order. Then we have $G/H$ is abelian since it is a proper quotient of $G,$ which 
contradicts the fact that $DG$ is the smallest normal subgroup of $G$ such that the quotient is abelian. Thus $DG$ is a $p-$group. We pick a $p_i-$Sylow 
subgroup (for some $p_i$), say $P,$ that contains $DG.$ It contains $DG,$ so $P$ is normal. 

For $p_j \neq p_i,$ and $P_j$ is a $p_j-$Sylow subgroup of $G,$ then $DG \cdot P_j$ is an abelian normal subgroup of $G$ (Since it is a subgroup containing 
$DG$). We have $\gcd(|DG|,|P_j|)=1,$ thus $ P_j \trianglelefteq DG \cdot P_j.$ Then we have that the $p_j-$Sylow subgroup is normal when $p_j \neq p_i,$ and 
we already know that the $p_i-$Sylow subgroup is normal. Then we have $G= P_1 \times \dots \times P_r,$ where these are all normal. This is a product of 
groups of the order $p_i$ or $p_i^2,$ which is abelian, giving us the result.

The converse can be shown by proving the contrapositive. See that if we have $n=p_1^{\alpha_1}\dots p_r^{\alpha_r},$ and $\alpha_i>2$ for some $i,$ or $p_i 
| p_j^{\alpha_j}-1$ for some $i,j,$ then we are done. 

To check the first case, take $G,$ which has order $p^3.$ Let $H=\mathbb{Z}_{p^2}$ and $K=\mathbb{Z}_p.$ Then we can find a non-trivial map $\varphi$ from 
$K$ to $\text{Aut}(H)$ since $p| (p^2-1)(p^2-p).$ Thus $H \rtimes_{\varphi} K$ will be a non-abelian group of order $p^3.$ 

Let us assume then, that $\alpha_i=1$ or $2,$ but that $p_i | p_j-1,$ where $\alpha_j=1.$ Then there is a non trivial map $\varphi:K \to \text{Aut}(H),$ 
where $K$ is the group of order $p_i$ and the group of automorphisms of $H$, the group of order $p_j.$ Then we have $\mathbb{Z}_{\frac{n}{p_ip_j}} \times (H 
\rtimes_{\varphi} K)$ is a non-abelian group.  

Now for $p_i | p_j^2-1,$, take the group of order $75$ as we showed in the previous question was non-abelian. Generalising this, we have a non-trivial map 
$\varphi:K \to \text{Aut}(H),$ where $K$ is the group of order $p_i$ and the group of automorphisms of $H$, the group of order $p_j^2.$ The group of 
automorphisms will be $\text{GL}_2(\mathbb{Z}_{p_j}),$ of order $(p_j^2-1)(p_j^2-p_j),$ which $p_i$ clearly divides. Then we have 
$\mathbb{Z}_{\frac{n}{p_ip_j^2}} \times (H \rtimes_{\varphi} K),$ where $\varphi$ is the non-trivial map. 

Thus if $n$ is abelian, the aforementioned condition must apply.   
\end{document}




