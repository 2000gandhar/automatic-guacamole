%%%%%%%%%%%%%%%%%%%%%%%%%%%%%%%%%%%%%%%%%
% Lachaise Assignment
% LaTeX Template
% Version 1.0 (26/6/2018)
%
% This template originates from:
% http://www.LaTeXTemplates.com
%
% Authors:
% Marion Lachaise & François Févotte
% Vel (vel@LaTeXTemplates.com)
%
% License:
% CC BY-NC-SA 3.0 (http://creativecommons.org/licenses/by-nc-sa/3.0/)
% 
%%%%%%%%%%%%%%%%%%%%%%%%%%%%%%%%%%%%%%%%%

%----------------------------------------------------------------------------------------
%	PACKAGES AND OTHER DOCUMENT CONFIGURATIONS
%----------------------------------------------------------------------------------------

\documentclass{article}

\input{structure.tex} % Include the file specifying the document structure and custom commands

%----------------------------------------------------------------------------------------
%	ASSIGNMENT INFORMATION
%----------------------------------------------------------------------------------------

\title{} % Title of the assignment

\author{Gandhar Kulkarni (mmat2304)} % Author name and email address

\date{} % University, school and/or department name(s) and a date

%----------------------------------------------------------------------------------------

\begin{document}

\maketitle % Print the title

%----------------------------------------------------------------------------------------
%	INTRODUCTION
%----------------------------------------------------------------------------------------

\section{} %Problem 1 
\section{} %Problem 2 
\begin{enumerate}
	\item Let $H$ be a non-abelian simple group. Consider $DH \trianglelefteq H.$ Then since $H$ is simple, we must have $DH= H$ or $DH=0.$ If $DH=0,$ then 
	$H$ would be abelian, which is not possible, thus $H$ is perfect.
	\item We need to calculate $D(\langle H,K\rangle).$ Any element of $\langle H,K \rangle$ can be realised as a word $w=h_1k_1\dots h_nk_n$ for some $n 
	\in \mathbb{N}$ and $h_1,\dots,h_n \in H, k_1,\dots,k_n \in K.$ We also assume that the terms are reduced. We consider two such words $w_1$ and $w_2,$ 
	where $w_1=h_1k_1\dots h_nk_n, w_2=h_1k_1\dots h'_{n'}k'_{n'}.$ Then any term in $D\langle H,K\rangle$ is of the form $w_1w_2w_1^{-1}w_2^{-1}= 
	(h_1k_1\dots h_nk_n)(h_1k_1\dots h'_{n'}k'_{n'})(k^{-1}_nh^{-1}_n\dots k^{-1}_{1}h^{-1}_{1})(k^{-1}_{n'}h^{-1}_{n'}\dots k'^{-1}_{1}h'^{-1}_{1}),$ which 
	is another word in $\langle H,K \rangle.$ Now take any word $w=h_1k_1\dots h_nk_n$ in $\langle H,K \rangle.$ 
	
	\item We propose that $D(g^{-1}Hg)=g^{-1}DHg.$ Take any element of $D(g^{-1}Hg),$ which is of the form 
	$(g^{-1}h_1g)(g^{-1}h_2g)(g^{-1}h_1^{-1}g)(g^{-1}h_2^{-1}g)=g^{-1}(h_1h_2h_1^{-1}h_2^{-1})g \in g^{-1}DHg.$ These operations are all if and only if 
	statements, hence $D(g^{-1}Hg)=g^{-1}DHg.$ If $H=DH,$ $D(g^{-1}Hg)=g^{-1}Hg.$
	\item If $G$ is simple, then either $DG=0$ or $DG=G.$ Then for abelian simple groups the maximal perfect subgroup is $0,$ and in the non-abelian case it 
	is $G$ itself. Both are clearly normal in $G.$ 
\end{enumerate}
\section{} %Problem 3 
\section{} %Problem 4 
We have $|G| =75.$ We clearly have a $5-$Sylow subgroup of order $25,$ as well as a $3-$Sylow group of order $3.$ Since the number of $5-$Sylow subgroups 
must be $5k+1$ and $5k+1|3,$ we are forced to have a normal subgroup of order $25.$ Then let $H \trianglelefteq G$ where $|H|=25.$ Let $K$ be some $3-$Sylow 
subgroup of order $3.$ Then by Lagrange's theorem, $|H \cap K|=1,$ so we can construct a semi-direct product of the groups.  
\section{} %Problem 5
Let $n= p_1^{\alpha_1}\dots p_r^{\alpha_r}$ be such that $\alpha_i=1,2$ for all $1 \leq i \leq r$ and $p_i \nmid p_j^{\alpha_j}-1$ for all $i,j.$ 
Assume WLOG that $p_1 < \dots < p_r.$ Then we have that the number of $p_r-$Sylow subgroups is of the form $p_rk+1,$ and since $p_rk+1 | p_1^{\alpha_1}\dots 
p_{r-1}^{\alpha_{r-1}},$ we have $p_rk+1|p_i^{\alpha_i},$ for $1 \leq i \leq r-1.$ Since $p_r$ is the largest prime, we 

The converse can be shown by proving the contrapositive. See that if we have $n=p_1^{\alpha_1}\dots p_r^{\alpha_r},$ and $\alpha_i>2$ for some $i,$ or $p_i 
| p_j^{\alpha_j}-1$ for some $i,j,$ then we are done. To check the first case, take $D_8,$ which has order $8=2^3.$ This group has centre $\{1,r^2\},$ which 
implies that this group is not abelian.

Now for the other case, take the group of order $75$ as we showed in the previous question was non-abelian. In that case, we had $3|25-1.$ Thus we have that 
a  $n$
\end{document}




