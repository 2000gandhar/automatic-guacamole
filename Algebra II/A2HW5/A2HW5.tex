\documentclass[letterpaper,11pt,twoside]{article}
\usepackage[utf8]{inputenc}
\usepackage{enumitem}
\setlist{nosep}
\usepackage{graphicx}
\usepackage{amsmath,amssymb,amsfonts,amsthm}
\usepackage{tikz-cd}
\usepackage[margin=0.9in,
left=1.25in,%
right=1.25in,%
top=1.25in,%
bottom=1.25in
]{geometry}	
%\usepackage{stmaryrd} %For mapsfrom

%\usepackage{quiver}
\usepackage{bm}
\usepackage{fancyhdr}
\usepackage{mathrsfs}
\usepackage{amsbsy}
\usepackage{titlesec}
%\usepackage{yhmath}
%\usepackage{mathabx,epsfig}


%Hyperref Settings------
\usepackage{hyperref}
\usepackage{xcolor}
\hypersetup{
	colorlinks,
	linkcolor={black},
	citecolor={red!50!black}
	urlcolor={green!80!black}
}

%%%%%% TITLE %%%%%

\title{Algebra 2 Homework 5}
%\date{\today}


%MATH BACKGROUND DECLARATORS-------------------------------------------------------
\theoremstyle{proposition}
\newtheorem{proposition}{Proposition}[section]

\theoremstyle{definition}
\newtheorem{definition}{Definition}[section]

\theoremstyle{theorem}
\newtheorem{theorem}{Theorem}[section]

\theoremstyle{definition}
\newtheorem{remark}{\textbf{Remark}}[section]

\theoremstyle{definition}
\newtheorem{notation}{\textbf{Notation}}[section]

\theoremstyle{definition}
\newtheorem{discussion}{\textbf{Discussion}}[section]

\theoremstyle{lemma}
\newtheorem{lemma}{\textbf{Lemma}}[section]

\theoremstyle{definition}
\newtheorem{example}{\textbf{Example}}[section]

%\theoremstyle{remark}
%\newtheorem*{comment}{\textbf{Comments on Proof Technique}}

\theoremstyle{definition}
\newtheorem{construct}{Construction}[section]

\theoremstyle{corollary}
\newtheorem{corollary}{Corollary}[section]

\theoremstyle{definition}
\newtheorem{caution}{\textbf{Caution}}[section]


\theoremstyle{definition}
\newtheorem{question}{\textbf{Question}}[section]

\theoremstyle{definition}
\newtheorem{para}{}[section]


%--------------------------------------------------------------------------------- SOME USEFUL MACROS.


\newcommand{\N}{\mathbb{N}}
\newcommand{\Z}{\mathbb{Z}}
\newcommand{\C}{\mathbb{C}}
\newcommand{\R}{\mathbb{R}}
\DeclareMathOperator{\GL}{\text{\rm GL}}
\newcommand{\Ker}[1]{{\fontfamily{lmss}\selectfont 
		\text{\rm Ker}\left (#1\right )
}}
\newcommand{\nsg}{\trianglelefteq}
\newcommand{\abs}[1]{\left \vert #1 \right \vert}
\newcommand{\norm}[1]{\left \vert \left \vert #1 \right \vert \right \vert}
\newcommand{\gen}[1]{\left\langle #1\right\rangle}
\renewcommand{\div}{\;\vert\;}
\newcommand{\isom}{\cong}
\DeclareMathOperator{\Stab}{\text{\rm Stab}}
\newcommand{\Image}[1]{{\fontfamily{lmss}\selectfont 
		\text{\rm Im}\left (#1\right )
}}
\DeclareMathOperator{\Bij}{\text{\rm Bij}}
\DeclareMathOperator{\acts}{\rotatebox[origin=c]{-90}{$\circlearrowright$}}
\DeclareMathOperator{\Orb}{\text{\rm Orb}}
\DeclareMathOperator{\lcm}{\text{\rm lcm}}
\newcommand{\floor}[1]{\left \lfloor #1 \right \rfloor}
\DeclareMathOperator{\Aut}{\text{\rm Aut}}
\DeclareMathOperator{\Inn}{\text{\rm Inn}}
\DeclareMathOperator{\id}{\text{\rm id}}
\newcommand{\F}{\mathbb{F}}




\begin{document}
	\maketitle
	%\tableofcontents
\begin{proof}[Solution of problem $1$:]
		If we have $[K_1K_2:F]=[K_1:F][K_2:F],$ then for $K_1=F(\{\alpha_m\}_{m \in I}),$ and $K_2=F(\{\beta_n\}_{n \in I'}).$ Then we have 
		$K_1K_2=F(\{\alpha_m \beta_n\}_{m \in I,n \in I'}).$ We have that $K_1 \otimes_F K_2$ is the $F-$module generated by the generating elements 
		$\alpha_m \otimes_F  \beta_n.$ We have an obvious $F-$module homomorphism from $ K_1 \otimes_F K_2$ to $K_1K_2$ where $\alpha_m \otimes_F  \beta_n 
		\mapsto \alpha_m\beta_n.$ This clearly is an isomorphism of modules, and since $K_1K_2$ is a field, so is $ K_1 \otimes_F K_2.$
		
		Conversely, we have $K_1 \otimes_F K_2$ is a field. Then we have $[K_1 \otimes_F K_2:F]=[K_1:F][K_2:F],$ since as $F-$modules this must happen. Now 
		define $\varphi: K_1 \times K_2 \to K_1K_2$ where $(a,b) \mapsto ab.$ This is a map that distributes over addition and scalar multiplication over 
		both $a$ and $b.$ Then by the universal property of tensor products we have a unique map $\Omega: K_1 \otimes_F K_2 \to K_1K_2$ where $\Omega(a 
		\otimes b)=ab.$ This homomorphism preserves multiplication, and it is necessarily injective. Moreover, for any $ab \in K_1K_2,$ we have a 
		corresponding $a \otimes b \in K_1 \otimes_F K_2.$ Thus we have an isomorphism of fields. 
		
		Thus we have $[K_1K_2:F]=[K_1 \otimes_F K_2:F]=[K_1:F][K_2:F].$  
\end{proof}
\begin{proof}[Solution of problem $2$:]
	We have a quadratic equation in $x^2,$ which gives us $x^2= \pm \omega.$ Solving this further, we get $x^4+x^2+1=(x-\omega)(x+ \omega)(x - i\omega)(x + 
	i\omega).$ Then clearly $\mathbb{Q}(i,\omega)$ contains the splitting field. Also, adjoining all the roots to $\mathbb{Q}$ gives us $\mathbb{Q}(\omega, 
	-\omega, i \omega, -i \omega)$ which certainly contains $\mathbb{Q}(i,\omega).$ Thus the splitting field is $\mathbb{Q}(i,\omega).$
\end{proof}
\begin{proof}[Solution of problem $3$:]
	The polynomial $x^6-4$ splits into linear factors in $\mathbb{C},$ where the roots $ \pm \zeta_3 \alpha,$ where $\zeta_3 \in \{1,\omega,\omega^2\},$ and 
	$\alpha=\sqrt[3]{2}.$ We propose that the splitting field is $\mathbb{Q}(\alpha, \omega).$ This clearly contains all the roots of this polynomial, thus 
	it contains the splitting field. Also, we get the splitting field by adjoining all the roots to $\mathbb{Q},$ which clearly contains $\mathbb{Q}(\alpha, 
	\omega),$ thus it is the splitting field. 
\end{proof}
\begin{proof}[Solution of problem $4$:]
Let us assume $K$ is a splitting field over $F$ for some polynomial $f(x) \in F[x].$ Then we take some irreducible polynomial $p(x) \in F[x]$ such that some 
root of $p$ is in $K.$ Let $K=K(\alpha).$ We also have another root $\beta.$ We know that since $F(\alpha) \cong F(\beta)$ we can extend this isomorphism to 
their splitting fields, that is $K(\alpha) \cong K(\beta).$ Then we have that $K/F$ and $K(\beta)/F$ have the same degree. Thus 
$[K(\alpha):K(\beta)][K(\beta):F]=[K(\alpha):F] \implies [K(\alpha):K(\beta)]=1.$ Thus $K(\beta) \cong K(\alpha) \cong K.$

Conversely, assume that any irreducible polynomial over $F$ either contains all its roots in $K$ or none of its roots in $K.$ We can assume 
$K=F(\alpha_1,\dots, \alpha_n).$ Take $m(x),$ the minimal polynomial of $K$ over $F.$ This is irreducible, and it is clearly all in $K.$ Then we must have 
the splitting field of $m(x)$ is contained in $K.$ However, the splitting field must also contain all the roots of $m(x),$ and thus must also contain $K.$ 
Thus we have $K$ is a splitting field. 
\end{proof}
\end{document}