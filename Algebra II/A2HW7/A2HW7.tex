\documentclass[letterpaper,11pt,twoside]{article}
\usepackage[utf8]{inputenc}
\usepackage{enumitem}
\setlist{nosep}
\usepackage{graphicx}
\usepackage{amsmath,amssymb,amsfonts,amsthm}
\usepackage{tikz-cd}
\usepackage[margin=0.9in,
left=1.25in,%
right=1.25in,%
top=1.25in,%
bottom=1.25in
]{geometry}	
%\usepackage{stmaryrd} %For mapsfrom

%\usepackage{quiver}
\usepackage{bm}
\usepackage{fancyhdr}
\usepackage{mathrsfs}
\usepackage{amsbsy}
\usepackage{titlesec}
%\usepackage{yhmath}
%\usepackage{mathabx,epsfig}


%Hyperref Settings------
\usepackage{hyperref}
\usepackage{xcolor}
\hypersetup{
	colorlinks,
	linkcolor={black},
	citecolor={red!50!black}
	urlcolor={green!80!black}
}

%%%%%% TITLE %%%%%

\title{Algebra 2 Homework 7}
%\date{\today}


%MATH BACKGROUND DECLARATORS-------------------------------------------------------
\theoremstyle{proposition}
\newtheorem{proposition}{Proposition}[section]

\theoremstyle{definition}
\newtheorem{definition}{Definition}[section]

\theoremstyle{theorem}
\newtheorem{theorem}{Theorem}[section]

\theoremstyle{definition}
\newtheorem{remark}{\textbf{Remark}}[section]

\theoremstyle{definition}
\newtheorem{notation}{\textbf{Notation}}[section]

\theoremstyle{definition}
\newtheorem{discussion}{\textbf{Discussion}}[section]

\theoremstyle{lemma}
\newtheorem{lemma}{\textbf{Lemma}}[section]

\theoremstyle{definition}
\newtheorem{example}{\textbf{Example}}[section]

%\theoremstyle{remark}
%\newtheorem*{comment}{\textbf{Comments on Proof Technique}}

\theoremstyle{definition}
\newtheorem{construct}{Construction}[section]

\theoremstyle{corollary}
\newtheorem{corollary}{Corollary}[section]

\theoremstyle{definition}
\newtheorem{caution}{\textbf{Caution}}[section]


\theoremstyle{definition}
\newtheorem{question}{\textbf{Question}}[section]

\theoremstyle{definition}
\newtheorem{para}{}[section]


%--------------------------------------------------------------------------------- SOME USEFUL MACROS.


\newcommand{\N}{\mathbb{N}}
\newcommand{\Z}{\mathbb{Z}}
\newcommand{\C}{\mathbb{C}}
\newcommand{\R}{\mathbb{R}}
\DeclareMathOperator{\GL}{\text{\rm GL}}
\newcommand{\Ker}[1]{{\fontfamily{lmss}\selectfont 
		\text{\rm Ker}\left (#1\right )
}}
\newcommand{\nsg}{\trianglelefteq}
\newcommand{\abs}[1]{\left \vert #1 \right \vert}
\newcommand{\gen}[1]{\left\langle #1\right\rangle}
\newcommand{\norm}[1]{\left \vert \left \vert #1 \right \vert \right \vert}
\renewcommand{\div}{\;\vert\;}
\newcommand{\isom}{\cong}
\DeclareMathOperator{\Stab}{\text{\rm Stab}}
\newcommand{\Image}[1]{{\fontfamily{lmss}\selectfont 
		\text{\rm Im}\left (#1\right )
}}
\DeclareMathOperator{\Bij}{\text{\rm Bij}}
\DeclareMathOperator{\acts}{\rotatebox[origin=c]{-90}{$\circlearrowright$}}
\DeclareMathOperator{\Orb}{\text{\rm Orb}}
\DeclareMathOperator{\lcm}{\text{\rm lcm}}
\newcommand{\floor}[1]{\left \lfloor #1 \right \rfloor}
\DeclareMathOperator{\Aut}{\text{\rm Aut}}
\DeclareMathOperator{\Inn}{\text{\rm Inn}}
\DeclareMathOperator{\id}{\text{\rm id}}
\newcommand{\F}{\mathbb{F}}




\begin{document}
	\maketitle
	%\tableofcontents
	\begin{proof}[Solution of problem $1$:]
		
	\end{proof}
\begin{proof}[Solution of problem $2$:]
	Since $A^k = I$ for some $k \geq 1,$ we have that $x^k-1=0$ must be the minimal polynomial. Solving this over $\mathbb{C},$ we have the $k$th roots of 
	unity, which are exactly $k$ many. The characteristic polynomial must have $n$ roots, all of whom must be $k$th roots of unity. 
	
	For $A= \begin{pmatrix}
		1 & \alpha \\
		0 & 1
	\end{pmatrix},$ see that $A^k=  \begin{pmatrix}
	1 & k\alpha \\
	0 & 1
\end{pmatrix},$ for any $k \in \mathbb{N}.$ This can easily be checked by induction. See that for $k=p,$ we must have $I$ since the field has characteristic 
$p.$ See that the characteristic of $A$ is just $x^2-1=0,$ which gives us $\pm 1$ as the eigenvalues. We see that the $\ker \dim (A-I)= \{ (x,y) \in 
\mathbb{R}^2 \div A-I(x,y)^T=0  \}= \dim(\text{ \textrm span } (1,0))=1,$ while $\ker \dim (A+I)= \{ (x,y) \in \mathbb{R}^2 \div A+I(x,y)^T=0  \}= 
\dim(\text{ \textrm span } (0,0) )=0, $ which clearly does not add up to $2,$ since $1$ and $2$ are not the same number. Thus for $\alpha \neq 0,$ this 
matrix is not diagonalisable. If $\alpha=0,$ in the eigenvalue $-1,$ we would have the null space as $ \text{ \textrm span } (0,1),$ which would give us a 
basis for $\mathbb{R}^2.$
\end{proof}
\begin{proof}[Solution of problem $3$:]
	If we see $K/F_1,$ then it is clear that since $K= F_1(\sqrt[8]{2}),$ we have a degree $8$ extension. To understand $\Aut(K/F_1),$ see that we only need 
	to understand where $\sqrt[8]{2}$ goes. We have eight choices. We can send $\sqrt[8]{2}$ to any 
\end{proof}
\begin{proof}[Solution of problem $4$:]
		We can see that $x^4-14x^2+0=0$ is solved as a quadratic in $x^2,$ that is, $x^2= 7 \pm 2 \sqrt{10}.$ 
	We can write $7 -2 \sqrt{10}= 7-2\sqrt{10}\frac{7+2\sqrt{10}}{7+2\sqrt{10}}= \frac{9}{7+2\sqrt{10}}.$ 
	Then we can see that $x= \pm \sqrt{7+2\sqrt{10}}, \pm \frac{3}{\sqrt{7+2\sqrt{10}}}.$ All of these roots lie in $K=\mathbb{Q}(\sqrt{7+2\sqrt{10}}).$
	
	We propose that this field is Galois. The minimal polynomial for $\sqrt{7+2\sqrt{10}}$ is all in $K,$ thus it is normal, since it contains all of its 
	conjugates. It is also clear that none of the roots are repeated, so the extension is separable. Thus $K$ is the Galois splitting field.
\end{proof}
\begin{proof}[Solution of problem $5$:]
	\begin{enumerate}
		\item $[K:F]=n < \infty,$ then we say that $K$ is represented as by a finite $F-$basis, by $\{b_1,\dots, b_n\}.$ 
		Let $x \in K,$ where  $x= \sum_{i=1}^{n}c_ib_i,$ where $c_i \in F$ for $i=1,\dots,n.$
		Now we have $\alpha \cdot x= \sum_{i=1}^{n}c_i (\alpha \cdot b_i).$ Since $\alpha \cdot b_i \in K,$ $\alpha \cdot b_i= \sum_{j=1}^{n} a_{ji} b_j.$ 
		Then we can put this in matrix form, by $T_{\alpha}:= (a_{ji}).$ By this, it is clear that $T_{\alpha}$ is a linear transformation, since it can be 
		written as a matrix.
		
		\item Let $m(x)$ be the minimal polynomial for $\alpha$ with degree $d_1,$ and $m'(x),$ be the minimal polynomial for $A$ with degree $d_2.$ If $d_1 
		\geq d_2,$ then $$m(x) = q_1(x) m'(x) + r_1(x),$$
		
	\end{enumerate}
\end{proof}
\begin{proof}[Solution of problem $6$:]
	We have proven in the previous assignment that $f(x)=x^p-x-a$ is irreducible over $\mathbb{F}_p,$ and that it is separable.
	We know that if $\alpha$ is a root, then so is $\alpha + 1.$ So it is easy to see that the 
\end{proof}
\begin{proof}[Solution of problem $7$:]
	In $E=\mathbb{Q}(\sqrt{1+\sqrt{2}}),$let $x= \sqrt{1 + \sqrt{2}}.$ Then $x(x^2-1)^2=2.$ We get that $x^4-2x^2-1=0.$ We can explicitly see that none of 
	its roots are rational, so the polynomial we have must be the minimal polynomial. We can calculate the conjugates, which are $$\pm \sqrt{1+ \sqrt{2}}, 
	\pm i \frac{1}{\sqrt{1+ \sqrt{2}}}.$$ So it is easy to see that $i$ is missing. Thus, by adjoining $i,$ we can have a normal and separable extension 
	containing $\sqrt{1+ \sqrt{2}}.$ We propose that $K=\mathbb{Q}(\sqrt{1+ \sqrt{2}},i)$ is the Galois extension. We know $K$ contains all the conjugates, 
	so the Galois field must be sandwiched between $K$ and $F.$ But since $[K:F]=2,$ which is prime, we must have that either the field is equal to $E$ or 
	$F,$ and it cannot be $F,$ since the imaginary conjugates are not in $F,$ the Galois closure must be $E.$
\end{proof}
\begin{proof}[Solution of problem $8$:]
	
\end{proof}
\begin{proof}[Solution of problem $9$:]
	
\end{proof}

\end{document}