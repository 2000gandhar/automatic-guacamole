\documentclass[letterpaper,11pt,twoside]{article}
\usepackage[utf8]{inputenc}
\usepackage{enumitem}
\setlist{nosep}
\usepackage{graphicx}
\usepackage{amsmath,amssymb,amsfonts,amsthm}
\usepackage{tikz-cd}
\usepackage[margin=0.9in,
left=1.25in,%
right=1.25in,%
top=1.25in,%
bottom=1.25in
]{geometry}	
%\usepackage{stmaryrd} %For mapsfrom

%\usepackage{quiver}
\usepackage{bm}
\usepackage{fancyhdr}
\usepackage{mathrsfs}
\usepackage{amsbsy}
\usepackage{titlesec}
%\usepackage{yhmath}
%\usepackage{mathabx,epsfig}


%Hyperref Settings------
\usepackage{hyperref}
\usepackage{xcolor}
\hypersetup{
	colorlinks,
	linkcolor={black},
	citecolor={red!50!black}
	urlcolor={green!80!black}
}

%%%%%% TITLE %%%%%

\title{Complex  Analysis 3}
%\date{\today}


%MATH BACKGROUND DECLARATORS-------------------------------------------------------
\theoremstyle{proposition}
\newtheorem{proposition}{Proposition}[section]

\theoremstyle{definition}
\newtheorem{definition}{Definition}[section]

\theoremstyle{theorem}
\newtheorem{theorem}{Theorem}[section]

\theoremstyle{definition}
\newtheorem{remark}{\textbf{Remark}}[section]

\theoremstyle{definition}
\newtheorem{notation}{\textbf{Notation}}[section]

\theoremstyle{definition}
\newtheorem{discussion}{\textbf{Discussion}}[section]

\theoremstyle{lemma}
\newtheorem{lemma}{\textbf{Lemma}}[section]

\theoremstyle{definition}
\newtheorem{example}{\textbf{Example}}[section]

%\theoremstyle{remark}
%\newtheorem*{comment}{\textbf{Comments on Proof Technique}}

\theoremstyle{definition}
\newtheorem{construct}{Construction}[section]

\theoremstyle{corollary}
\newtheorem{corollary}{Corollary}[section]

\theoremstyle{definition}
\newtheorem{caution}{\textbf{Caution}}[section]


\theoremstyle{definition}
\newtheorem{question}{\textbf{Question}}[section]

\theoremstyle{definition}
\newtheorem{para}{}[section]


%--------------------------------------------------------------------------------- SOME USEFUL MACROS.


\newcommand{\N}{\mathbb{N}}
\newcommand{\Z}{\mathbb{Z}}
\newcommand{\C}{\mathbb{C}}
\newcommand{\R}{\mathbb{R}}
\DeclareMathOperator{\GL}{\text{\rm GL}}
\newcommand{\Ker}[1]{{\fontfamily{lmss}\selectfont 
		\text{\rm Ker}\left (#1\right )
}}
\newcommand{\nsg}{\trianglelefteq}
\newcommand{\abs}[1]{\left \vert #1 \right \vert}
\newcommand{\gen}[1]{\left\langle #1\right\rangle}
\newcommand{\norm}[1]{\left \vert \left \vert #1 \right \vert \right \vert}
\renewcommand{\div}{\;\vert\;}
\newcommand{\isom}{\cong}
\newcommand{\type}[1]{\text{\rm #1}}
\DeclareMathOperator{\Stab}{\text{\rm Stab}}
\newcommand{\Image}[1]{{\fontfamily{lmss}\selectfont 
		\text{\rm Im}\left (#1\right )
}}
\DeclareMathOperator{\Bij}{\text{\rm Bij}}
\DeclareMathOperator{\acts}{\rotatebox[origin=c]{-90}{$\circlearrowright$}}
\DeclareMathOperator{\Orb}{\text{\rm Orb}}
\DeclareMathOperator{\lcm}{\text{\rm lcm}}
\newcommand{\floor}[1]{\left \lfloor #1 \right \rfloor}
\DeclareMathOperator{\Aut}{\text{\rm Aut}}
\DeclareMathOperator{\Inn}{\text{\rm Inn}}
\DeclareMathOperator{\id}{\text{\rm id}}
\newcommand{\F}{\mathbb{F}}




\begin{document}
	\maketitle
	%\tableofcontents
	\begin{proof}[Solution of problem $1$:]
		\begin{enumerate}
			\item Let $f(z)= z^n,$ and $g(z)= -\abs{a_{n-1}}z^{n-1} - \dots - \abs{a_1}z - \abs{a_0}.$ Since for large enough $M>0 (M= \abs{a_0}+ \dots + 
			\abs{a_n} \type{ works, for example})$ we have $|f(z)| > 
			\abs{g(z)},$ we have that all the roots of $f$ and $f+g$ (the polynomial whose roots we want) are all contained within the disc $\abs{z} \leq 
			M,$ which has $n$ roots since $f$ has $n$ roots. 
			
			If we consider the real version of the same polynomial, see that the polynomial at $x=0,$ where it is negative. For $x=M,$ we must have 
			$\abs{(f+g)(M)} \geq \abs{\abs{f(M)}-\abs{g(M)}} > 0.$ Thus by intermediate value theorem, we must have that there is a root between $0$ and 
			$M.$ 
			We can refine $M$ to be smaller to get closer to $0,$ which we can then set as the desired root. 
			\item Consider the disc of radius $\frac{1}{M+1}$ centered at the origin. Let $g(z)= \frac{a_n}{a_0}z^n + \dots + \frac{a_1}{a_0}z+1,$ and $f(z) 
			=1.$ Then $$ \abs{f(z)-g(z)}= \abs{\frac{a_n}{a_0}z^n + \dots + \frac{a_1}{a_0}z } \leq M \abs{z} (1+\abs{z}+ \dots) = \frac{M}{ 
			\abs{z}^{-1}-1}.$$ 
			Since $\abs{z} < \frac{1}{M+1},$ then we must have $\abs{z}^{-1}-1 > M,$ thus $ \frac{M}{ \abs{z}^{-1}-1} < 1. $ 
			
			Thus by Rouch\'{e}'s theorem, we must have that $f$ and $g$ have the same number of roots in this disc. However, a constant function cannot 
			have any roots, thus $f,$ and by extension our original polynomial must have all of its roots outside that disc. 
		\end{enumerate}
	\end{proof}
	\begin{proof}[Solution of problem $2$:]
	Let us assume that our polynomial has Rolle's property. Then let not all zeroes be collinear. We say that $z_1,z_2,z_3 $ form a triangle with the 
	smallest area. We can say that there cannot be any other roots on this triangle, since if it were it would contradict our assumption that the above 
	triangle had minimal area. We pick a way to connect the remaining roots $z_4, \dots, z_r$ to one of the vertices of the triangle. Since Rolle's property 
	holds here, there are at least $r-3$ zeroes of $f'$ outside the triangle and different from the roots of $f.$ Looking at the edges of the triangle, we 
	see that there are three more zeroes of $f'.$ However, this gives us $r$ roots for $f',$ which is not possible since the derivative must lower the 
	number of roots strictly. Thus we have at least $\sum_{i=1}^r (m_i-1)+r > \deg f$ roots, which is not possible for a polynomial of degree $r-1.$ Thus 
	the roots must be collinear.
	
	If all the roots fall on a line, then $z_i= x_i + i (mx_i + c),$ where $m,c$ are real constants (The roots could also be all vertical, that will be 
	tackled later). Then see that $f((1+mi)z + (ic))$ should produce the desired effect, since $-z_i= (1+mi)z + (ic) - (1_mi)x_i - ic = (1+mi)(z-x_i).$ 
	Since this applies for all $i= 1,\dots, r,$ we have that $f((1+mi)z + (ic))$ has all real roots. In fact it is a complex number times a real polynomial. 
	Trivially real polynomials have Rolle's theorem, which proves the statement. If all the roots lie on a vertical line, then we just replace by the 
	substitution $z \mapsto iz_i + c,$ where $y=c$ is the line on which the roots of $f$ lie. 
	\end{proof}
	\begin{proof}[Solution of problem $3$:]
	\begin{enumerate}
		\item Our function is clearly holomorphic (entire, in fact) hence it must attain its maximum modulus on the square with vertices $a,b,c,d.$ Without 
		loss of generality, assume that the maximum is attained somewhere on the line segment joining $a$ and $b.$ Then let the maxima be at some point, 
		$x.$ Then we see that $$g(x,y) = x^2y^2 (y^2+l^2)(x^2+l^2)$$ where $x+y=l.$ We use Lagrange's multiplier here to get 
		\begin{align*}
			y^2(l^2+y^2)(4x^3 + 2x l^2) &= \lambda\\
			x^2(l^2+x^2)(4y^3 + 2y l^2) &= \lambda.
		\end{align*}
	Eliminating $\lambda,$ we have $$2xy (y(l^2+y^2)(2x^2+ l^2)) = 2xy (x(l^2+x^2)(2y^2+ l^2)).$$ Assuming neither $x,y =0,$ so we have after a tedious 
	calculation that   $$(y-x) \left( (l^4 + 2x^2y^2 -2xyl^2) + (x^2+xy+y^2)l^2 \right) =0.$$ See that the term in the brackets can be written as $l^4 + 
	2x^2y^2 + l^2 ((x+y)^2-y^2),$ which must be greater than zero. Thus we must have that $x=y,$ thus $x=y= \frac{l}{2}.$ The maximum will be 
	$\frac{5l^4}{16}.$ 
 		\item We have $f(z)= z^2 +2az -1,$ with $a \in \R.$ See that $f$ is a polynomial, thus it is entire. Specifically, it is holomorphic on the unit 
		disc, thus by maximum modulus principle, the maximum modulus must be attained on the boundary, that is on $\{z: \abs{z}=1\}. $
		Then we have $f(e^{i\theta})= e^{2i\theta}+2a e^{i \theta} -1,$ that is $ e^{i \theta}\left( e^{i\theta} - e^{-i\theta} +2a \right).$
		Now see that $e^{i\theta} - e^{-i\theta} = 2i \sin (\theta).$ Thus, we have 
		\begin{align*}
			\abs{f(e^{i\theta})}&= \abs{e^{2i\theta}+2a e^{i \theta} -1}\\
			&= \abs{e^{i \theta}} \cdot \abs{e^{i\theta} - e^{-i\theta} +2a}\\
			&= \abs{2a + 2i \sin (\theta)}= 2 \sqrt{a^2+\sin^2(\theta)}.
		\end{align*}
	Since $\sin^2(\theta)$ is at most $1$ for $\theta= \pi/2, 3 \pi/2,$ we have $|f|$ is at most $2 \sqrt{a^2+1}.$ 
	\end{enumerate}
	\end{proof}
	\begin{proof}[Solution of problem $4$:]
	\begin{enumerate}
		\item To see a function on the open disc that has one fixed point, consider the function $f_{\varphi}$ where $z \mapsto ze^{i\varphi},$ for $\varphi 
		\neq 0.$ For this, $$re^{i(\theta+ \varphi)} = re^{i(\theta)} \implies r=0 \type{ or } \varphi=0,$$ which implies that $r=0,$ hence $z=0$ is the one 
		and only fixed point. 
		
		Now assume that a function $f: \mathcal{D} \to \mathcal{D}$ is such that it has two fixed points. If one of the fixed points is $0,$ then it 
		satisfies all conditions of Schwarz's lemma. Since $\abs{f(z)}= \abs{z}$ for $0$ and another point in the disc, we must have that $f(z)=cz,$ where 
		$\abs{c}=1.$ But since $f(a)=a,$ where $a \neq 0$ is the other fixed point of $f,$ we must have $c=1.$ Thus in this special case the only such 
		function must be the identity. 
		
		Now we consider a more general function, where there is no restriction of where the fixed point is. Let $a \in \mathbb{D}$ be one of the fixed 
		point. Then $T_a: \mathbb{D} \to \mathbb{D}$ be a bijective map, where $$T_a(z)= \frac{z-a}{1- \overline{a}z}$$. This is clearly a bijection on a 
		open unit disc, since $T_a^{-1}=T_{-a}.$ Let $f$ be such that $a,$ and $b$ are fixed points. Now see that $T_afT^{-1}_a(0)=T_af(a)=T_a(a)=0.$ Thus 
		we have that $T_afT^{-1}_a$ has a fixed point at the origin. Also, let $b$ be the other fixed point of $f.$ Let $x=T_a(b).$ Then 
		$T_afT^{-1}_a(x)=T_a(f(b))=x,$ which is another fixed point of $T_afT^{-1}_a.$ By the previous result, we must have that $T_afT^{-1}_a(z)=z,$ for 
		all $z \in \mathbb{D}.$ Thus $f(T_{-a}(z))=T_{-a}(z).$ By setting $z=T_a(z),$ we get $f(z)=z.$
		\item The solution to this problem emulates the proof for Schwarz's lemma. Define $g(z)= \begin{cases}
			\frac{f(z)}{z^2} & n \neq 0\\
			        f''(z)/2 & z=0;
		\end{cases}$
	See that this is holomorphic since $f(0)=f'(0)=0,$ hence this function has only a removable singularity at $z=0.$ See that by the maximum modulus 
	principle the maximum must lie on the boundary, that is the unit circle. Thus there is some $c,$ with $\abs{c}=1$ where $$\abs{g(z)} \leq \abs{g(c)} 
	\leq \frac{\abs{f(c)}}{\abs{c^2}} \leq \frac{1}{r^2},$$ for some $r<1.$ As $r \to 1,$ fixing $z=0$ on the left, we see that $\frac{\abs{f''(0)}}{2} \leq 
	1,$ which gives us our required answer. 
	
	If the equality is true for any point on the disc, then $g(z)$ must be constant by the maximum modulus principle. Then we must have $\abs{f(z)}= 
	\abs{z^2},$ which means we must have some $\lambda \in S^1$ where $f(z)=\lambda z^2.$ 
	
	The converse is trivial, since if $f(z)=\lambda z^2,$ then $|f''(0)| = 2,$ which means it attains the equality.
	\item Consider the function $ g:\mathbb{D} \to \mathbb{H}, g(z)= i\frac{1-z}{1+z}.$ See that 
	\begin{align*}
		g(z) &= i\frac{1-x + iy}{1+x + iy} = i \frac{((1-x)+iy)((1+x)-iy)}{(1+x)^2+ y^2}\\
		\implies \Im{g(z)}= \frac{1-x^2 + y^2}{(1+x)^2+ y^2},
	\end{align*}
		which is greater than zero as $ 1-x^2 + y^2 > 1- x^2 -y^2 > 0,$ so this indeed falls in the upper half plane. Now see that $g(0)=i,$ and $f(i)=0,$ 
		as given. Thus $f\circ g$ has $0$ as a fixed point. Thus we can apply Schwarz's lemma here to see that $\abs{f(g(z))} \leq \abs{z}.$ See that 
		$g\left(\frac{-1}{3}\right)= 2i.$ Plugging this into the above inequality, we get $\abs{f(2i)} \leq \frac{1}{3}.$ 
 	\end{enumerate}
	\end{proof}
	\begin{proof}[Solution of problem $5$:]
	\begin{enumerate}
		\item Let $f(z)=16z^7+16^3+1,$ and $g(z)=16z^3+1.$ Then $$|f(z)-g(z)| = \abs{16z^7}.$$ On $|z|=\frac{1}{2},$ we have $\abs{f(z)-g(z)} < 
		\frac{16}{128}=\frac{1}{8}.$ Also, $$\abs{g(z)} \geq \abs{16 \abs{z}^3-1}= 2-1=1. $$ Thus by Rouch\'{e}'s theorem, we have that $f(z)$ has the same 
		number of roots as $g(z)$ in the disc $\abs{z} < \frac{1}{2}.$ See that $g(z)$ has three roots in that region, seeing as its roots are $ k \cdot 
		\frac{-1}{2^{4/3}},$ which all lie in the disc. Thus $f$ also has three roots.
		\item Let $f(z)= z+e^{-z}-2024$ and $g(z)=z-2024.$ Now see that if $z$ is a root of $f$ with positive real part, then we have 
		\begin{align*}
			z-2024 &= e^{-z}\\
			&= e^{-x-iy}\\
			&= e^{-x}e^{-iy}\\
			\implies \abs{z-2024} &= \abs{e^{-x}} < 1.
		\end{align*}
	Thus from a previous problem we can say that there is some $M$ for which such a root exists. 
	Now consider the unit circle centered at $(2024,0).$ See that $$\abs{f(z)-g(z)} = \abs{e^{-z}} = \abs{e^{-(2024+ e^{i\theta})}} \leq e^{-2023}.$$
	Now, $\abs{g(z)}= \abs{2024 + e^{i\theta} -2024}=1.$ By applying Rouch\'{e}'s theorem, the two polynomials must have the same number of roots. But we 
	know that $g$ has one root. Thus $f$ has one root. Since $\abs{z-2024}<1$ contains only one zero, there is only zero with positive real part. 
	\item Let $f(z)= \sum_{i=0}^n \frac{z^i}{i!},$ and $g(z)=e^z.$ Then for some $R>0,$ we see that $\abs{g(z)} = \abs{e^{R\cos(\theta)+iR\sin (\theta)}} 
	\geq e^{-R}.$ We know that as $n \to \infty,$ $f(z)$ converges to $g(z)$ uniformly. Thus there exists for $\varepsilon = \frac{e^{-R}}{2},$ a $N \in 
	\mathbb{N}$ such that for all $n \geq N,$ we have $\abs{f(z)-g(z)} < \frac{e^{-R}}{2},$ and $\frac{e^{-R}}{2} < e^{-R} \leq \abs{g(z)}.$
	Thus by Rouch\'{e}s theorem $f$ and $g$ must have the same number of roots. However, $g$ has no roots in all of $\mathbb{C}$. Thus all the roots that 
	$f$ has must be outside this disc!
	\item Let $f(z)=cz^n-e^z$, and $g(z)=cz^n.$ We have $\abs{g(z)-f(z)}=\abs{e^{z}}.$ On the unit circle, we have $$\abs{g(z)-f(z)} = \abs{e^{e^{i\theta}}} 
	\geq e^{-1}.$$ See that $\abs{g(z)} = \abs{c} > e.$ Thus $\abs{g(z)-f(z)} < \abs{g(z)}$ on the unit circle, and hence by Rouch\'{e}s theorem, the two 
	must have the same number of roots. Clearly, $g$ has $n$ roots, so must $f.$
	\item Since $f(z)=z^4+iz^2+2.$ Solving this, we get $f(z)=(z^2-i)(z^2+2i).$ Thus $z= \pm \left( \frac{1+i}{\sqrt{2}} \right), i-1, 1-i. $ The first two 
	clearly lie on the unit circle, and the other two lie on the circle $\abs{z}= \sqrt{2}.$ It is also easy to see that they all lie in different 
	quadrants. $\frac{1+i}{\sqrt{2}}, -1+i, \frac{-1-i}{\sqrt{2}}, 1-i $ lie in the four quadrants, in that order respectively. 
	
	To do this using Rouch\'{e}'s theorem, see that for $f(z)=z^4, g(z)=z^4+iz^2+2,$ and a disc of radius $2+ \varepsilon$ centered at the origin, for some 
	$\varepsilon >0,$ we can see that the modulus of the difference is lesser than $z^4,$ thus we can say that all four roots are in this circle. We can see 
	that on the unit circle this changes, and the modulus of $f$ is bounded by modulus of $g-f,$ which means there are two roots in a circle of radius $1 + 
	\varepsilon$. We then need to confirm that it is only true for $\varepsilon>0,$ which confirms that the roots must lie on the two circles. 
	\end{enumerate}
	\end{proof}
	\begin{proof}[Solution of problem $6$:]
	We have $\mathcal{J}: z \mapsto \frac{1}{2} \left(z+ \frac{1}{z}\right).$ If we write $z=re^{i\theta},$ then we have $$J\left(re^{i\theta}\right)= 
	\frac{1}{2}\left(r + \frac{1}{r}\right)\cos \theta + i \left( \frac{1}{2}\left(r - \frac{1}{r}\right)\sin \theta \right)=u+iv.$$ See by some 
	manipulation that $$\frac{u}{\left(r+ \frac{1}{r}\right)}^2 + \frac{v}{\left(r- \frac{1}{r}\right)}^2 =4.$$ We will call this ellipse $E_r.$ Thus for a 
	fixed $r,$ the range falls on an 
	ellipse. Note that as $r$ approaches zero, the modulus of the function is grows unbounded. Fixing our $r,$ we guess that all points in the unbounded 
	region also are in the range.
	\begin{enumerate}
		\item For $|z| =r, r<1,$ we say that the closure of the unbounded region of $E_r$ is the image. To see this, we need to check that $E_c$ is in the 
		image, for $c < r.$ Since $c<r<1,$ See that the functions $r+ \frac{1}{r}$ and $r+ \frac{1}{r}$ are both increasing, since their derivative is 
		lesser than $0$ in $0 < r < 1.$ Thus $c+ \frac{1}{c} > r+ \frac{1}{r},$ and $c- \frac{1}{c}< r- \frac{1}{r},$ which means that the ellipse $E_c$ is 
		contained in the closure of the unbounded part of $E_r.$ 
		\item If $|z| > r,$ where $r>1,$ then we claim that the closure of the unbounded region of the ellipse $E_r$ is the region spanned, just like the 
		previous case, but for a different reason. See that any ellipse $E_c,$ where $c>r>1,$ is in this region since the function $r+ \frac{1}{r}$ is 
		increasing when $r>1,$ and $\abs{r+ \frac{1}{r}}$ is increasing in the same region.
		\item This case is a limiting case of the first case. If we let $r \to 1,$ we see that we get all of the complex plane except $ [-1 ,1] \times 0,$ 
		which is the range of the cosine. 
		\item If we let $\theta \in [0,\pi],$ then we can split our region into three: $|z| >1, \abs{z} < 1,$ and $\abs{z}=1.$ For the last one, it can be 
		seen that when $r=1,$ the Joukowsky map becomes the cosine function. Thus that portion maps to $[-1,1] \times 0.$ When $\abs{z}<1,$ $v<0,$ since 
		$\sin \theta \geq 0$ in the upper half plane, thus it maps to the lower half plane, without the unit disc in the lower half plane. The portion 
		$\abs{z}>1$ is such that $v>0,$ so it gets mapped to the upper half plane, without the unit disc in the upper half plane. 
 		
		\item Wherever $J'(z) \neq 0,$ the map is conformal. We see that $J'(z)= 1 - \frac{1}{z^2},$ which is zero only at $\pm 1.$ Excluding those points, 
		the map is clearly conformal from $\mathbb{C} $ to $\mathbb{C} \backslash \{\pm 1\}.$ If we map the open disc, we see that we get the complex plane 
		excluding the real line. 
		
	\end{enumerate}
	\end{proof}
\end{document}