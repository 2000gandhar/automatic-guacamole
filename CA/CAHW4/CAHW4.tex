\documentclass[letterpaper,11pt,twoside]{article}
\usepackage[utf8]{inputenc}
\usepackage{enumitem}
\setlist{nosep}
\usepackage{graphicx}
\usepackage{amsmath,amssymb,amsfonts,amsthm}
\usepackage{tikz-cd}
\usepackage[margin=0.9in,
left=1.25in,%
right=1.25in,%
top=1.25in,%
bottom=1.25in
]{geometry}	
%\usepackage{stmaryrd} %For mapsfrom

%\usepackage{quiver}
\usepackage{bm}
\usepackage{fancyhdr}
\usepackage{mathrsfs}
\usepackage{amsbsy}
\usepackage{titlesec}
%\usepackage{yhmath}
%\usepackage{mathabx,epsfig}


%Hyperref Settings------
\usepackage{hyperref}
\usepackage{xcolor}
\hypersetup{
	colorlinks,
	linkcolor={black},
	citecolor={red!50!black}
	urlcolor={green!80!black}
}

%%%%%% TITLE %%%%%

\title{Complex Analysis Homework 4}
%\date{\today}


%MATH BACKGROUND DECLARATORS-------------------------------------------------------
\theoremstyle{proposition}
\newtheorem{proposition}{Proposition}[section]

\theoremstyle{definition}
\newtheorem{definition}{Definition}[section]

\theoremstyle{theorem}
\newtheorem{theorem}{Theorem}[section]

\theoremstyle{definition}
\newtheorem{remark}{\textbf{Remark}}[section]

\theoremstyle{definition}
\newtheorem{notation}{\textbf{Notation}}[section]

\theoremstyle{definition}
\newtheorem{discussion}{\textbf{Discussion}}[section]

\theoremstyle{lemma}
\newtheorem{lemma}{\textbf{Lemma}}[section]

\theoremstyle{definition}
\newtheorem{example}{\textbf{Example}}[section]

%\theoremstyle{remark}
%\newtheorem*{comment}{\textbf{Comments on Proof Technique}}

\theoremstyle{definition}
\newtheorem{construct}{Construction}[section]

\theoremstyle{corollary}
\newtheorem{corollary}{Corollary}[section]

\theoremstyle{definition}
\newtheorem{caution}{\textbf{Caution}}[section]


\theoremstyle{definition}
\newtheorem{question}{\textbf{Question}}[section]

\theoremstyle{definition}
\newtheorem{para}{}[section]


%--------------------------------------------------------------------------------- SOME USEFUL MACROS.


\newcommand{\N}{\mathbb{N}}
\newcommand{\Z}{\mathbb{Z}}
\newcommand{\C}{\mathbb{C}}
\newcommand{\R}{\mathbb{R}}
\DeclareMathOperator{\GL}{\text{\rm GL}}
\newcommand{\Ker}[1]{{\fontfamily{lmss}\selectfont 
		\text{\rm Ker}\left (#1\right )
}}
\newcommand{\nsg}{\trianglelefteq}
\newcommand{\abs}[1]{\left \vert #1 \right \vert}
\newcommand{\gen}[1]{\left\langle #1\right\rangle}
\newcommand{\norm}[1]{\left \vert \left \vert #1 \right \vert \right \vert}
\renewcommand{\div}{\;\vert\;}
\newcommand{\isom}{\cong}
\DeclareMathOperator{\Stab}{\text{\rm Stab}}
\newcommand{\Image}[1]{{\fontfamily{lmss}\selectfont 
		\text{\rm Im}\left (#1\right )
}}
\DeclareMathOperator{\Bij}{\text{\rm Bij}}
\DeclareMathOperator{\acts}{\rotatebox[origin=c]{-90}{$\circlearrowright$}}
\DeclareMathOperator{\Orb}{\text{\rm Orb}}
\DeclareMathOperator{\lcm}{\text{\rm lcm}}
\newcommand{\floor}[1]{\left \lfloor #1 \right \rfloor}
\DeclareMathOperator{\Aut}{\text{\rm Aut}}
\DeclareMathOperator{\Inn}{\text{\rm Inn}}
\DeclareMathOperator{\id}{\text{\rm id}}
\newcommand{\F}{\mathbb{F}}




\begin{document}
	\maketitle
	%\tableofcontents
	\begin{proof}[Solution of problem $1$:]
		\begin{enumerate}
			\item Plugging $z= \frac{1}{2}$ into the equation, we have $f\left(\frac{1}{2}\right)= \frac{1}{2}.$ Now we differentiate the function $n$ times 
			to see that $f^{(n)}(z) + (-1)^{n}f^{(n)}(1-z)=0,$ for $n \geq 0.$ Thus we must have $f^{(n)}\left(\frac{1}{2}\right)=0$ for $n$ odd. Thus our 
			function can be written with a power series expansion at $z= \frac{1}{2},$ that is, $$f(z)= \sum_{n=0}^{\infty} \frac{c_n}{n!}\left(z- 
			\frac{1}{2}\right)^n $$ where $c_{2n+1}=0$ for $n \in \mathbb{N},$ and $c_0= \frac{1}{2}.$
			
			\item   Consider a rectangle that passes $(-1,0)$ and $(1,0)$ vertically, and with height $2\varepsilon>0.$ That is, the endpoints of this 
			rectangle are $(1, \varepsilon ),(1, -\varepsilon), (-1, \varepsilon),$ and $(-1, -\varepsilon).$ Let this rectangle be $\gamma.$  Since $g$ is 
			continuous and $g$ is holomorphic in $ \mathbb{C} \backslash [-1,1] \times 0,$ hence we can say the same of $g'.$ We use Stokes' theorem to say 
			that $\int_{\gamma}g = \int_{\text{Int}(\gamma)}g'.$ We can claim that the function is bounded within the rectangle. If we pick any point, then 
			either it is in the holomorphic region (in which case it must be bounded in that region), or it is on the segment. Since any neighbourhood of 
			such a point contains points in the holomorphic region, each neighbourhood must itself be bounded. Thus in the rectangle there is some upper 
			bound $M.$ Thus $ \int_{\gamma}g = \int_{\text{Int}(\gamma)}g' \leq M 2 \cdot 2 \varepsilon. $ Thus as $\varepsilon \to 0,$ we must have 
			$\int_{\gamma}g \to 0.$ 
					
			Now, if we take any rectangle that intersects the line segment, we can write it as the sum of integrals of rectangles entirely in the 
			holomorphic region and a rectangle cutting through the line segment. We can use the above special case to see that it must be zero.    
			
		\end{enumerate}
	\end{proof}
	\begin{proof}[Solution of problem $2$:]
		Let us assume that a function $f$ holomorphic on some domain $D$ exists, and we take $a \in D,$ and take some $r>0$ such that the ball of radius $r$
		centered at $a$ is in $D.$ Considering the Taylor Series expansion of $f(z)$ at $a,$ we have $$f(z)= \sum_{n=0}^{\infty} 
		\frac{f^{(n)}(a)}{n!}(z-a)^n.$$ We want to see if this series converges. See that $a_n= \frac{f^{(n)}(a)}{n!}.$ We want to examine $\abs{a_n}$ as $n 
		\to \infty.$ See that $\abs{a_n} \geq \frac{n^nn!}{n!} = n^n.$ See that $\limsup_{n \to \infty} (a_n)^{\frac{1}{n}}= \lim_{n \to \infty} n$ which 
		diverges to $+\infty.$ Then this means that the radius of convergence of this function is $0,$ which means our function cannot exist.
			
 	\end{proof}
	\begin{proof}[Solution of problem $3$:]
		\begin{enumerate}
			\item Let $z_0$ be a root such that $\Re(z_0) >0.$ Then we have $$\abs{z_0-3}= \abs{e^{-z_0}} = \abs{e^{-x_0} \cdot e^{-y_0}} < 1.$$ Thus $z_0$ 
			lies in the unit circle centered at $3 (B_1(3)).$ Let $f(z)= e^{-z}+z-3,$ and $g(z)= z-3.$ On $C_1(3),$ $\abs{f(z)-g(z)} = \abs{e^z} < e^{-2}.$ 
			The bounded on the right comes because the circle has $z=2$ as the smallest value, hence $e^{-2}$ is the largest such number. Since 
			$\abs{g(z)}1 > e^{-2},$ thus $\abs{f(z)-g(z)} < \abs{g(z)}.$ Now by Rouch\'{e}'s theorem, we have that there is exactly one root in $B_1(3).$ If 
			$z$ is any root in $B_1(3),$ then  $1 > \abs{z-3} = \abs{e^{z}} \implies \Re(z) >0.$ Thus this root is unique. 
			
			\item Let $f(z)=1 + z+ z^4 $ and $g(z)=z^4.$ On $C_0(3/2),$ we have $\abs{f(z)-g(z)} = \abs{1+z}< 1 + \frac{3}{2}  \frac{5}{2}.$ See that 
			$\abs{g(z)}< \frac{3}{2}^4 = \frac{81}{16},$ thus $\abs{f(z)-g(z)}< \abs{g(z)},$ and thus by Rouch\'{e}'s theorem, $f(z)$ has exactly four roots 
			in $B_{0}(3/2).$ Now for the second part, we consider the curve $\gamma$ that is the circle $C_{0}(3/2)$ in the first quadrant with the lines on 
			the $x$ and $y$ axes that join the origin to the curve. We plan to use a more general version of Rouch\'{e}'s theorem on $f(z) =z^4+ z+ 1,$ and 
			$g(z)= z^4+1$. We now see that 
			on the circle $\abs{f(z)-g(z)}= \abs{z} = \frac{3}{2}$ and $\abs{z^4+1} < \frac{65}{16} > 4$, thus $\abs{f(z)-g(z)}< \abs{g(z)}.$ On the axes, 
			see that $\abs{f(z)-g(z)}= \abs{z} < \abs{x^4+1},$ so we can apply the theorem. Thus there is exactly one root in the first quadrant. This 
			strategy works perfectly well for every other quadrant. 
			
			\item Let us find solutions for $z\sin z =0.$ This is because $z=0$ or $\sin z=0.$ The latter equation has solutions $n \pi, n \in \mathbb{Z}$ 
			which are all real. Now let $f(z)= z\sin z -1$ and $, g(z)= z\sin z.$ Let $\gamma_n$ be the circle of radius $\left( n + \frac{1}{2} \right)\pi$ 
			centred at $0.$ We claim that Rouch\'{e}'s theorem will apply here. See that $\abs{\sin z}= \frac{1}{2} \sqrt{ e^{2y}+e^{-2y} - 2\cos 2x }.$ 
			Then, $\abs{z \sin z}= \left(n + \frac{1}{2}\right) \frac{1}{2} \sqrt{ e^{2y}+e^{-2y} - 2\cos 2x }.$ Now see that $1< \abs{g(z)}$ on $\gamma_n.$ 
			As $\abs{z} = \left( n + \frac{1}{2} \right)\pi > 4.$ We want to show that $\cos 2x < 1/2,$ then we would have $\abs{\sin z} > 1/4.$ Then see 
			that if $ y^2<1,$ then $$ x^2 \in \left( \left( \left(n + \frac{1}{2}\right)\pi - \pi/6 \right)^2, \left( \left(n + \frac{1}{2}\right)\pi + 
			\pi/6 \right)^2\right).$$ This can be checked to be always true, and hence $\abs{g(z)}>1.$ 
			
			Now see that $g(z)$ has only real roots. Then see that the roots in $C_{\left(n+ \frac{1}{2}\right)\pi}(0)$ are $0, \pm \pi, \dots, \pm n\pi,$ 
			where $0$ is a double root. Then, there are $2n+2$ roots. 
			
			Now see that by identity principle $z\sin z - 1$ is the unique holomorphic extension of the real function $x\sin x -1$ on $\mathbb{C}.$ If we 
			can show that $x \sin x -1$ has at least $2(n+1)$ roots in $ \text{Int }(\gamma_n),$ we would be done. Then see that for $n=1,$ by intermediate 
			value theorem we can say that there is one root in $(0, \pi)$ and one root in $(-\pi,0).$ At $0$ there is a root of multiplicity $2.$ By 
			induction, we can  deduce that there are at at least $2(n+1)$ real roots, proving our result. 
 		\end{enumerate}
	\end{proof}
	\begin{proof}[Solution of problem $4$:]
	\begin{enumerate}
		\item \textit{This section is left blank}
		\item $f:(0,1) \to \mathbb{C} $ is bounded and holomorphic. Let $\{\zeta_n\}$ be the zeroes of $f$ in $ B(0,1).$ Let $Z= \{\zeta_n\},$ which is 
		countable discrete by identity theorem. 
		
		\begin{theorem}[Jensen's Theorem]
			Let $f: \Omega \to \mathbb{C}$ be holomorphic on a domain, and let $\overline{B_R(0)} \subseteq \Omega $ be such that $f$ has no zeroes on its 
			boundary. Then if $\zeta_1, \dots ,\zeta_N$ are zeroes in the disc, then we have $$ \log \abs{f(0)} = \sum_{k=1}^{N}\log \frac{\abs{z_k}}{R} + 
			\frac{1}{2R} \int_{0}^{2\pi} \log \abs{f(re^{i\theta})}d\theta.$$ 
		\end{theorem}
	Pick $r \in (0,1)$ such that $f(z)$ is non-zero for all $\abs{z}=r,$ which is possible since the set of roots is discrete. Thus the closure of 
	$\overline{B_r(0)}$ has finitely many roots due to compactness. Let $\zeta_1, \dots \zeta_{m_r}$ be those roots. By Jensen's theorem, we have 
	$$ \log \abs{f(0)} = \sum_{k=1}^{m_r} \frac{\abs{z_k}}{R} + 
	\frac{1}{2R} \int_{0}^{2\pi} \log \abs{f(re^{i\theta})}d\theta.$$ Pick a sequence $R_n \in (0,1)- \{\abs{\zeta_i}, \zeta_i \in Z\},$ and $R_n \to 1.$ 
	In the above expression, we have $\abs{f}$ is bounded, by some $M>0,$ then $\abs{fe^{i\theta}}$ and $\abs{f(0)}$ are bounded above by $M.$ 
	
	Then it follows that $$ \sum_{k=1}^{m_{R_n}} \log \frac{\abs{\zeta_k}}{R_n} \geq \abs{f(0)} - M $$ for all $R_n.$ When $n \to \infty,$ we have that all 
	zeroes will be covered in this disc.  Since $\log \frac{\abs{\zeta_k}}{R_n} <0,$ we have $\abs{\zeta_k} < R_n,$ and we can let $n \to \infty$ to see 
	that $$ \sum_{k=1}^{\infty} \log \abs{\zeta_k} < \infty. $$  
	
	\item $g: R \to \mathbb{C}$ is a bounded holomorphic function on the right half plane and $\abs{g(z)} \leq M.$ Let $\{w_n\}$ be the zeroes in $R.$ 
	
	We claim that the maximum modulus of the function $f$ is on the imaginary axis, that is the boundary of $R.$ If we assume this, then consider $h(z)=g(z) 
	\cdot \prod_{r=1}^{n} \abs{\frac{z+\overline{w_r}}{z- w_r}}$ for $z \in \overline{R}.$ $h$ is holomorphic on $R \backslash \{w_1, \dots, w_n\}.$ See 
	that these are removable singularities as $h(z)= \prod_{r=1}^{n} \abs{z- w_r} \ell(z)$ is holomorphic. Then 
	\begin{align*}
		\abs{h(z)} & = \abs{g(z)} \cdot \prod_{r=1}^{n} \abs{\frac{z+\overline{w_r}}{z- w_r}} \\
		& \leq M \prod_{r=1}^{n} \abs{\frac{z+\overline{w_r}}{z- w_r}},
	\end{align*}  
and as $n \to \infty,$ we have $ \lim_{n \to \infty} \prod_{r=1}^{n} \abs{\frac{z+\overline{w_r}}{z- w_r}} \leq 1 $  thus it is bounded on the exterior 
$\abs{z} > R$ as shown, and it is bounded on its complement as it is continuous. Thus $h$ is bounded. 

Now we claim that $h$ is bounded on the imaginary axis. When $z=iy,$ we have $  \abs{\frac{z+\overline{w_r}}{z- w_r}} =1,$ and thus $\abs{h(z)} = \abs{g(z)} 
\leq M.$ If we assume that the maximum modulus is attained on the boundary. This gives us $\abs{g(z)} \leq M \prod_{r=1}^{n} \abs{\frac{z+\overline{w_r}}{z- 
w_r}}.$ 

\item See that for $g : R \to \mathbb{C},$ we have $\Re \left( \frac{1}{w_r} \right) = \frac{1}{w_r} + \frac{1}{w_r} = \frac{2 \Re ( w_r)}{\abs{w_r}^2}.$ 
We claim that for $z_0 \neq w_1, \dots, w_n, \dots,$ we have $$ \sum_{r=1}^{\infty} \frac{2\Re (w_r)}{\abs{z_0-w_r}^2} $$ converges. With the result from 
the previous section, we have that this is convergent by taking $\log$ and seeing that we get the product term as in the previous section. We have that $$ 
\sum_{r=1}^{n} \abs{\frac{z+\overline{w_r}}{z- w_r}}$$ is bounded, and rearranging terms we have that $$ \sum_{r} \left(  1+ \frac{4 \Re (w_r) 
\Re(z_0)}{\abs{z_0-r}^2}  \right),$$ thus we have proven our claim. 

We also will show that there is some $z_0$ and $N \in \mathbb{N}$ such that $\abs{w_0-r} \leq \abs{w_r}$ for $r \geq N.$ Since $\{w_r\}$ is infinite 
discrete, it must be unbounded else it would have a limit point. For $\Re w_r \geq \frac{1}{4}$, we have  $\abs{ w_r - \frac{1}{2} }= \sqrt{ \abs{  \Re 
\left( w_r - \frac{1}{2} \right)^2 + \Im (w_r)^2  }} \geq \abs{w_r}.$ Then we have 

$$ \infty > \sum_{r, \Re(w_r) \geq 1/4} \frac{2\Re (w_r)}{\abs{1/2 - w_r}^2} \geq  \sum_{r, \Re(w_r) \geq 1/4} \frac{2\Re (w_r)}{\abs{w_r}^2},$$ 

so we have that $ \sum_{r, \Re(w_r) \geq 1/4} \Re \left( \frac{1}{2_r} \right) $ is bounded, hence we have our result.  
	\end{enumerate}
	
	
	\end{proof}
	\begin{proof}[Solution of problem $5$:]
	For the sake of contradiction, let $f: \mathbb{C} \backslash \{0,1,2\} \to \mathbb{C} \backslash \{0,1,2024\}$ be a conformal isomorphism. 
	
	Claim 1: Neither of $0,1,2$ is an essential singularity for $f.$ If $0$ is an essential singularity, then there exists a neighbourhood $0 \in P$ such 
	that $f(P)$ is dense in $\mathbb{C}.$ Assuming $\delta \in (0,1),$ as $f$ is a conformal isomorphism, for any open set $U,$ where $U$ does not contain 
	$0,1, 2024 ,$ we have that $U \cap f(P) \neq \phi \implies f^{-1}(U) \cap P \neq \phi. $ Every open set in $\mathbb{C} \backslash \{0,1,2\}$ is 
	$f^{-1}(U)$ for some open $U \in \mathbb{C} \backslash \{0,1,2024\}.$ Thus we have that $ P $ is dense in $\mathbb{C} \backslash \{0,1,2\}$ which is a 
	contradiction. Thus none of the points are essential singularities. Thus $f$ is meromorphic from $\mathbb{C} \to \mathbb{C} \backslash \{0,1,2024\}$. 
	
	Now we claim that any such map should be constant. Let $g(z)= \frac{1}{f(z)}.$ This is entire, and misses two points, $1$ and $1/2024.$ This is 
	impossible for entire functions by Little Picard's theorem, hence $f$ must be constant.  Thus $f$ must map to at least one of $0,$ $1,$ or $2024.$ Since 
	our $f$ is bijective, we must have $0,1,2$ going to $0,$ $1$,  or $2024.$ We want to show none are possible. In particular, we show that $0$ can't go to 
	any of the three points. Let's say $f(0)=0.$ Then $f$ is a conformal isomorphism between $\mathbb{C} \backslash \{1,2\}$ and $\mathbb{C} \backslash 
	\{1,2024\}.$ If we do some trickery (send $z$ to $z-1$, and send $f(z)$ to $ \frac{f(z)-1}{2023}$) and let $g(z)= \frac{f(z-1)-1}{2023}$, we have a 
	conformal isomorphism from $\mathbb{C} \backslash \{1,2\}$  to itself. 
	
	We want to show that the only such isomorphisms are---
	$g(z)=z,1-z, 1/z, z/z-1,1/1-z$ and $1- 1/z.$ If we assume this to be true, then $ g(z) $ is either of these six, so by some calculation we can see that 
	none of these maps can work.
	
	If $g(z)=z,$ then $ f(0)=2024,$ which is a contradiction. If $g(z)= z/z-1,$ then $f(z)= \frac{2023}{z} +1$ is unbounded in a neighbourhood of $0,$ which 
	is a contradiction. 
	
	If $g(z) = 1/z,$ then $f(0)= 2024,$ and if $g(z)=1-z,$ $f(0)=1.$ If $g(z)= 1/1-z,$ then $f(z)$ is unbounded in a neighbourhood of $0,$ same as 
	previously seen, and if $g(z)= 1- 1/z,$ then $f(0)=1,$ which is not possible. Now to prove the claim, we can construct a function $g(z)=z(z-1)f(z),$ 
	which we claim is entire. We just need to show that $f(z)$ has poles of order $1$ at $0$ and $1.$ First, if there was an essential singularity at $z=0,$ 
	we can show by the first claim that we reach a contradiction. So there are poles at $0$ and $1.$ If the order was $>1,$ then in a punctured 
	neighbourhood of $0,$ we would have $f(z)= \frac{1}{z^n}g(z),$ where $g$ is holomorphic at isn't $0$ at $0.$ If the order was more than $1,$ then in 
	this neighbourhood there would exist a $n$ to $1$ mapping, which contradicts the injectivity of $g.$ Thus both poles must have order $1.$ 
	
	Now, since $g(z)=z(z-1)f(z)$ is entire, it has a pole at $\infty,$ which means that $g(1/z)$ has a pole at $0.$ If that is the case, then we can deduce 
	that $g(z)$ must be a polynomial, since the order of the root at $0$ of $g(1/z)$ is finite. Thus we have $f(z)= \frac{p(z)}{z(z-1)},$ where $p(z)$ is 
	some polynomial. 
	
	Now, since $f(z)$ must be a fractional linear transformation, it is determined solely by its value at $0,1, \infty,$ we get the six transformations 
	mentioned previously. 	
	 
	\end{proof}
	\begin{proof}[Solution of problem $6$:]
	We have points, labelled $p_0, p_1, \dots, p_{n-1}$ on the unit circle. Let $\ell_k$ be the line joining $p_0$ to $p_k.$ We want to show that $ 
	\prod_{\gcd(k,n)=1} = \begin{cases}
		p & \text{ if } n=p^m, \text{ a prime,} \\
		1 & \text{ else.}
	\end{cases}$ 
We can assume that $p_k$ is a $n$th root of unity, as the lengths would be unchanged. Then the points $p_0, \dots, p_{n-1}$ are $1, \zeta, \dots, 
\zeta^{n-1}.$ Let $\varphi(n)=m,$ then let $ \{ \zeta_1, \dots, \zeta_m \} $ be the primitive $n$th roots of unity. We want to find $ \prod_{k=1}^{m}(1- 
\zeta_k).$ If $n=p,$ then $m=p-1,$ then we get $ \prod_{k=1}^{m}(1- \zeta_k) = \Phi_p(1) = 1^{p-1} + \dots + 1^{0} =p.$ We want to prove that this holds for 
all powers of prime. We have just proven the base case of induction. 

Now for $n=p^l,$ $m= p^l - p^{l-1}.$ Then we want to find $\prod_{k=1}^{m}(1- \zeta_k).$ See that $$\Phi_{n}(x) = \frac{ x^n -1 }{\prod_{d|n, d \neq n 
}\Phi_{d}(x)} = \frac{ x^n -1 }{\prod_{k=0}^{l-1}\Phi_{p^k}(x)}.$$ Putting $x=1,$ we have 

$$ \Phi_n(x) = \frac{x^{p^l-1}+ \dots + 1}{(x-1) \prod_{k=0}^{l-1}\Phi_{p^k}(x)},$$ so we have $$\Phi_n(1)= \frac{p^l -1 + 1}{\prod_{k=1}^{n}\Phi_{p^k}(1)}= 
\frac{p^l}{p^{l-1}}=p, $$ as desired. 

If $n \neq p^m, $ then we first check that for $n=6$ (this is the smallest such number) we know $\Phi_2$ and $\Phi_3$ explicitly. Then $\phi_6(1)= 
\frac{6}{2 \cdot 3}=1.$ Assume now that this holds for all integers less than $n.$ Thus, we have 

$$\Phi_n(1) = \frac{n}{\prod_{d|n, d \neq n}|\Phi_d(1)|}.$$ Let $n= p_1^{k_1}\dots p_l^{k_l},$ then we have $ \Phi_n(1) = \frac{n}{\prod_{d|n, d \neq 
n}|\Phi_d(1)|} = \frac{n}{(p_1)^{k_1} \dots (p_l)^{k_l}}=1,$ as desired.  

	\end{proof}
	\begin{proof}[Solution of problem $7$:]
	Let $a(n):= \min\{k \div f_1^n+ \dots + f_k^n = x, f_i \in \mathbb{C}[x], n \geq 3\}.$ First, we show that $a(n) \geq 3.$ Assume that $a(n) \leq 2,$ for 
	sake of contradiction. Clearly $a(n) \neq 1,$ since if it were, comparing degrees we'd get $\deg f_1 \cdot n = 1,$ which is absurd. If $a(n)=2,$ then we 
	have $x= f_1^n + f_2^n.$ Using a nifty trick we can write $x= f_1^n + f_2^n = f_2^n\left( \frac{f_1^n}{f_2^n}+ 1 \right)= f_2^n \prod_{k=0}^{n-1}\left( 
	\frac{f_1^n}{f_2^n} - \zeta^k \right)= \prod_{k=0}^{n-1}\left({f_1} - \zeta^kf_2 \right).$ Here $\zeta$ is a primitive $n$th root of unity. 
	
	Now, since $\mathbb{C}[x]$ is a UFD, we have for some $k_0,$ $f_1+ \zeta^{k_0}f_2 = x$ and for all $k \neq k_0,$ $f_1+\zeta^kf_2 \in \mathbb{C}.$ This 
	forces $f_1$ and $f_2$ to be constant, which means we cannot have a solution!. Thus $a(n) \geq 3.$
	
	Now define $\Delta: \mathbb{C}[x] \to \mathbb{C}[x],$ where$\Delta(f(x))= f(x+1)-f(x).$ Then we can see that $\Delta^n(f(x))= f(x+n) + c_1f(x+n-1) + 
	\dots + c_nf(x),$ where $c_1, \dots, c_n$ are some integers. This is easily proven by computation. Also see that applying $Delta$ reduces the degree of 
	a polynomial by $1,$ which can again be checked quite easily. Thus we must have $\Delta^{n-1}(x^n)$ is a linear polynomial, which after a linear 
	transformation becomes $x.$ Thus $ ax+b = (x+n-1)^n + c_1 (c+n-2)^{n-1}+ \dots + c_nx^n,$ which gives us a sureshot answer to the Waring problem, at $n$.
    Thus $a(n) \leq n.$
    
    Now for the last bound, we need to show that $n < a(n)^2 - a(n).$ Let $f_1, \dots , f_k$ be such that $k=a(n)$ and $f_1^n+ \dots + f_k^n = x.$ Consider 
    $$W_1=\text{ Wron} (f_1^n, \dots, f_k^n)= \det \begin{pmatrix}
    	f_1^n & \dots & f_k^n\\
    	\vdots & & \vdots \\
    	f_1^{(k-1)} & \dots & f_k^{(k-1)}.
    \end{pmatrix}.$$ 

We claim that $ \deg W_1 \geq (n-k+1)\sum_{i=1}^{k} \deg f_i $ and $ \deg W_1 \leq 1 + n \sum_{i=1}^{k}\deg f_i - \frac{k(k-1)}{2}.$ 

If we do prove this, then comparing the two bounds we get that $$n \deg f_1  \leq  k(k-1) \deg f_1 + 1 - \frac{k(k-1)}{2} <0 $$ for $k \geq 3,$ so $\lneq 
k(k-1),$ as desired. 

Thus we now only need to check our two claims. For the first claim, we need to show that the $i$th column of $W_1$ is divisible by $f_i^{n-k+1}$ We can 
prove this by induction. We have $f_i^{n-k+1} \div f_1^n,$ and $f_i^{n-k+1} \div f_1^{n(1)},$ since $f_1^{n(1)}= nf_1^{n-1} f_1^{(1)},$ and further steps 
follow. We now show that $W_1 \neq 0,$ and since $ \prod_{i=1}^k f_i^{n-k+1} \div W_1, $ we have $ \deg W_1 \geq (n-k+1) \sum_{i=1}^k \deg f_i.$ The 
Wronskian is non-zero iff the columns are linearly independent over some open interval. We shall use the minimality of $k$ here. Since $x= f_1^n + \dots + 
f_k^n,$ then we have $W_1 = \text{Wron}(x, f_2^n,\dots, f_k^n) =: W_2.$ Now we just want to see that $W_2 \neq 0.$ If not, then the columns must be linearly 
dependent. That is, $x = l_2 f_2^n + \dots + l_k f_k^n,$ which contradicts the minimality of $k,$ a contradiction. Thus we prove the first claim.

Now consider $W_2$ as previously defined. We will get the desired bound by expanding $W_2.$ We divide and multiply $W_2$ by $x^j$ in each row to get 

$$W_2= \frac{1}{x^{1+ 2+ \dots + k-1}} \cdot \det \begin{pmatrix}
	x & f_2^n & \dots & f_k^n\\
	x & xf_2^{n(1)} & \dots & x f_k^{n(1)}\\
	0 & & & \vdots \\
	\vdots & & & \vdots\\
	0 & x^{k-1}f_2^{n(k-1)}& \dots & x^{k-1} f_k^{n(k-1)}
\end{pmatrix}, $$
whose degree is at most $n \sum_{i=1}^{k}\deg f_i +1.$ This gives us the required bound, proving the result.  

 
	\end{proof}
	\begin{proof}[Solution of problem $8$:]
	First proof: We will prove a claim--- If $\abs{f(z)}< \abs{g(z)}$ on the unit circle, $f$ and $g$ are same degree polynomials and $g$ has no roots on 
	the unit circle, then $\abs{f'(z)}< \abs{g'(z)}$ for $\abs{z} \geq 1.$ If we prove this, the required result follows since let $g(z)= cz^n,$ where $n = 
	\deg f, $and $c > \norm{f}.$ Then we have $\abs{g(z)}=c > \abs{f(z)}$ on the unit circle. By the above claim, $\abs{f'(z)}< nc \abs{z}^{n-1}$ for all 
	$\abs{z} \geq 1,$ so we have $$\norm{f'}= \sup \{\abs{f'(z)} \div \abs{z}=1\} \leq nc,$$ which is enough. 
	
	Now for the claim, consider the polynomial $\varphi_t(z)= f(z)+ tg(z),$ for $\abs{t} \geq 1.$ By Rouch\'{e}'s theorem, $\varphi_t$ and $tg(z)$ have the 
	same number of roots in the unit ball. As $\deg f= \deg g,$ we have $\deg \varphi_t= \deg f,$ and no other roots of it are in $\abs{z} \geq 1.$ 
	By Gauss-Lucas theorem, we have $\varphi_t'(z)= f'(z) + tg'(z)$ has all of its roots in the unit ball. If not, then let $\abs{z_0} \geq 1$ such that 
	$\abs{f'(z_0)} \geq \abs{g'(z_0)}.$ Observe that where $t = \frac{-f'(z_0)}{g'(z_0)},$ so by our assumption $\abs{t} \geq 1.$ Then $\varphi_t(z_0)=0$ at 
	this point, which is a contradiction. This concludes the first proof.
	
	Second proof: We want a $z_0 \in \mathbb{C}$ such that $\abs{f'(z_0)} \leq n \norm{f}.$ We want to construct a polynomial $g_0(z),$ where $g_0(z) := 
	\frac{f(z_0z)-f(z_0)}{z-1}.$ This is a polynomial since $f(z_0z)-f(z_0)$ has $z=1$ as a root. Thus $\deg g_0 \leq n-1.$ By L'Hopital's rule we have 
	$g_0(1)=z_0f'(z_0).$ Consider $\varphi(z)= \sum{r=1}^{n} \frac{z^n+1}{(z-z_r)nz_r^{n-1}}g_0(z_r).$ See that $\varphi(1)= \frac{1}{n} \sum{r=1}^{n} 
	\frac{2z_r}{(z_r-1)}g_0(z_r)= \frac{1}{n} \sum{r=1}^{n} \frac{2z_r (f(zz_0)-f(z_0))}{(z_r-1)^2}.$ As $\varphi(z)=g_0(z),$ we have $g_0(1)= z_0 f'(z_0),$ 
	which gives us $$z_0f'(z_0)= \frac{1}{n} \sum{r=1}^{n} \frac{2z_r (f(zz_0)-f(z_0))}{(z_r-1)^2}.$$ 
	
	Claim 1: $$ \frac{1}{n} \sum{r=1}^{n} \frac{2z_r}{(z_r-1)^2}= \frac{-n^2}{2}.$$ Assuming this, we have $$z_0f'(z_0)= \frac{n}{2}f(z_0)+  \frac{1}{n} 
	\sum{r=1}^{n} \frac{2z_r f(zz_0)}{(z_r-1)^2}.$$ Since $z_0 $ is arbitrary, let it be on the unit circle. Then, 
	
	$$ \abs{f'(z_0)} \leq \frac{n}{2}\abs{f(z_0)}+  \frac{1}{n} \sum_{r=1}^{n} \abs{\frac{2z_r}{(z_r-1)^2}} \norm{f}.$$
	This gives us the required constant. 
	
	
	\end{proof}

\end{document}