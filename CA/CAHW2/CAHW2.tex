%%%%%%%%%%%%%%%%%%%%%%%%%%%%%%%%%%%%%%%%%
% Lachaise Assignment
% LaTeX Template
% Version 1.0 (26/6/2018)
%
% This template originates from:
% http://www.LaTeXTemplates.com
%
% Authors:
% Marion Lachaise & François Févotte
% Vel (vel@LaTeXTemplates.com)
%
% License:
% CC BY-NC-SA 3.0 (http://creativecommons.org/licenses/by-nc-sa/3.0/)
% 
%%%%%%%%%%%%%%%%%%%%%%%%%%%%%%%%%%%%%%%%%

%----------------------------------------------------------------------------------------
%	PACKAGES AND OTHER DOCUMENT CONFIGURATIONS
%----------------------------------------------------------------------------------------

\documentclass{article}

\input{structure.tex} % Include the file specifying the document structure and custom commands

%----------------------------------------------------------------------------------------
%	ASSIGNMENT INFORMATION
%----------------------------------------------------------------------------------------

\title{Complex Analysis Homework 2} % Title of the assignment

\author{Gandhar Kulkarni (mmat2304)} % Author name and email address

\date{} % University, school and/or department name(s) and a date

%----------------------------------------------------------------------------------------

\begin{document}

\maketitle % Print the title

%----------------------------------------------------------------------------------------
%	INTRODUCTION
%----------------------------------------------------------------------------------------

\section{} %Problem 1 
We will use Cauchy's integral formula here. The point $z_0=2$ is in the circle of radius $3,$ so we can use the formula. We have $$\int_C 
\frac{2z^2-z-2}{z-z_0}dz= f(z_0)=2 \pi i g(z_0),$$ where $g(z)=2z^2-z-2$ for all $z_0 \in D= \int(C).$ Then we have $f(2)=2 \pi i (2(2)^2-2-2)=8\pi i.$    
\section{} %Problem 2
\begin{enumerate}
	\item We will use Cauchy's integral formula here. Since the specified curve is a square, and it contains $z=\frac{-1}{2}.$ 
	Thus we have $$\int_C \frac{3z}{2z+1}dz =\pi i \frac{1}{2 \pi i}\int_C \frac{3z}{z+\frac{1}{2}}dz= \pi i \frac{-3}{2},$$ where the formula 
	is used on $3z.$
	\item We will use the Cauchy's integral formula for higher derivatives. We have $z=0$ in the curve, thus we have 
	$$ \int_C \frac{\cosh (z)}{z^{2024}}dz =\frac{2 \pi i}{2023!}\frac{d^{2023}\cosh(z)}{dz^{2023}} \mid_{z=0},$$
	which requires us to evaluate repeated derivatives of $\cosh(z).$ We know that $\cosh(z)= \frac{e^z + e^{-z}}{2},$ 
	so $\frac{d^n \cosh(z)}{dz^{n}}= \frac{e^z + (-1)^n e^{-z}}{2}.$ For $n=2023, z=0,$ we get 
	$ \frac{e^{0}+ (-1)^{2023}e^{-0}}{2}=0.$ Thus we have $\int_C \frac{\cosh (z)}{z^{2024}}dz=0.$
\end{enumerate}
\section{} %Problem 3 
\begin{enumerate}
	\item We have the annulus $ 0< |z-1|< 2.$ The function $f(z)=\frac{z}{(z-1)(z-3)}$ can be written as $f(z)=\frac{A}{(z-1)}+ \frac{B}{(z-3)},$ where 
	$A=-0.5$ and $B=1.5.$ The term on the left can be ignored for the time being. We have 
	\begin{align*}
		\frac{B}{(z-3)} &= \frac{-B}{2} \cdot \frac{1}{1- \frac{(z-1)}{2}}\\
		&= \frac{-B}{2} \sum_{n=0}^{\infty} \frac{(z-1)^n}{2^n} \text{ for $0 < |z-1| < 2.$}\\
	\end{align*}
Since we are in the annulus, we can define the Laurent series $\sum_{n=-\infty}^{\infty}a_n(x-1)^n$ as follows--- $a_{-1}= \frac{-1}{2}$, 
$a_n=\frac{-3}{2^{n+2}}$ for $n \geq 0,$ and $0$ otherwise. 
\item In its disc of convergence, $\sec (z)$ must have a power series expansion. We know that $\cos (z)= \sum_{n=0}^{\infty} b_{2n}\frac{z^{2n}}{(2n)!},$ 
where $b_{2n}={(-1)^n}.$ We must have that $\sec(z)$ is even, since it is just the reciprocal of $\cos (z),$ so we let the power series of $\sec(z)$ be 
$\sum_{n=0}^{\infty} E_{2n}\frac{z^{2n}}{(2n)!}.$ We have $1= \sec(z)\cos (z),$ which by multiplying gives us $ E_0 \cdot 1 = 1,$ and $$ \sum_{r=0}^n 
\frac{E_{2r}}{(2r)!}\cdot \frac{b_{2n-2r}}{(2n-2r)!} =0$$ for $n > 0.$ Working it out, we have $$ \sum_{n=0}^{\infty} (-1)^{n-r} \binom{2n}{2r}E_{2r} =0.$$ 
It is possible to work out any such number by starting from $n=1$ and working up till the desired number. 

To determine the radius of convergence, we want the disc to be large enough to not have any singularities. Thus we need to first identify the poles of $\cos 
(z)= \frac{e^{z}+ e^{-z}}{2}.$ We have $e^{2z}=-1 \implies 2z = \pi + 2n\pi  \implies z = \frac{\pi}{2}+ n\pi,$ for $n \in \mathbb{Z}$ The least value of 
$|z|$ where $\cos (z)=0$ 
is $\frac{\pi}{2},$ so we must have radius of convergence as $\frac{\pi}{2}.$ 

\item We have by some trigonometric magic $z \cot(z) = \frac{z \cos (z)}{\sin (z)}$  See that 
$z \cot (z)$ is even, as $z$ and $\cot(z)$ are both odd. Then $z\cot (z)= \sum_{n=-\infty}^{\infty} a_{2n}{x^{2n}},$where the power series on the right 
converges on some punctured disc of radius $R$ centered at the origin (The origin is excluded because the function is not defined at that point). Note that 
our power series is a Laurent series, since it is defined on an annulus.  We can fix the value of the function at $z=0$ to be $1,$ which means that we can 
analytically extend the series to a Taylor series. The power 
series of $$z \cdot \cos (z)= \sum_{n=0}^{\infty} (-1)^n \frac{z^{2n+1}}{(2n)!}.$$ On the 
other side, we have 
\begin{align*}
	\left(\sum_{n=0}^{\infty}a_{2n} z^{2n}\right) \cdot \left(\sum_{n=0}^{\infty} \frac{(-1)^n}{(2n+1)!}z^{2n+1}\right) &= z 
	\left(\sum_{n=0}^{\infty}a_{2n} z^{2n}\right) \cdot \left(\sum_{n=0}^{\infty} \frac{(-1)^n}{(2n+1)!}z^{2n}\right)\\
	&= \sum_{r=0}^{\infty} c_{2n+1} z^{2n+1}, 
\end{align*}
where $$c_{2n+1}= \sum_{r=0}^{n} a_{2r}\cdot \frac{(-1)^{n-r}}{(2n-2r+1)!}.$$ Equating the two power series, we have $a_0=1$ and $$\frac{(-1)^n}{(2n)!} =  
\sum_{r=0}^{n} a_{2r}\cdot \frac{(-1)^{n-r}}{(2n-2r+1)!}. $$ This can be written as $$\sum_{r=0}^n (-1)^r a_{2r}  \binom{2n}{2r-1} (2r-1)! =1.$$

To find its radius of convergence, we need to avoid singularities. Thus we need to avoid points where $\tan (z)=0;$ this is the same as finding where $\sin 
(z)=0.$ For this, see that $\sin (z) = \frac{e^z - e^{-z}}{2i}=0 \implies e^{2z}=1.$ Thus we have $z= n \pi,$ for $n \in \mathbb{Z}.$ By keeping the radius 
of convergence as $\pi,$ we can avoid these problem points.  
\end{enumerate}
\section{} %Problem 4 
Let $D=C_R(0)$ be the open disc of radius $R$ centered around $0.$ We claim that $f(z)=2z^2+z$ is injective on this disc. We must have that for $z_1,z_2 \in 
D,$ we have $f(z_1)=f(z_2) \implies z_1=z_2.$ Then $f(z_1)-f(z_2)= 2z_1^2+z_1 - 2z_2^2 - z_2 = 2(z_1-z_2)(2(z_1+z_2)+1).$ If this expression is $0,$ we 
claim that $2(z_1+z_2)+1 \neq 0$ for any $z_1, z_2 \in D.$ Clearly $\frac{-1}{2} \notin D,$ so $R < \frac{1}{2}.$ If we try to solve $z_1+z_2=\frac{1}{2},$ 
then for $z_1=x_1+iy_1, z_2=x_2+iy_2,$ we have $y_1=-y_2$ and $x_1+x_2=\frac{-1}{2}.$ Thus see that we need to avoid the point $z=\frac{-1}{4}$ as well. 
Thus $R < \frac{1}{4}.$ If $R>\frac{1}{4},$ it is possible to have points such that the above equations are satisfied. Thus $R=\frac{1}{4}.$ On the 
boundary, any point except $z=\frac{-1}{4}$ works because of the above reason. However, if we choose $z_1=z_2=\frac{-1}{4},$ it would be not possible, but 
note that here $f(z_1)-f(z_2)$ trivially. Thus this works even on the boundary.  
\section{} %Problem 5 
We decompose $f$ as $u+iv,$ where $u,v$ are real valued functions. Then we have $$(u+iv)^2=u-iv \implies u^2-u-v^2 + (2uv+v)i =0.$$
This implies that $2uv+v=0,$ thus either $v=0$ on $D$ or $u= \frac{-1}{2}$ on $D.$ Substituting the two in the real part, we have $u^2=u$ which means 
$u=0,1$ in the first case, and that $ v^2=\frac{3}{4}$ which implies that $v = \pm \frac{\sqrt{3}}{2}$ for the second case. 
Thus there are four constant functions that satisfy the given condition--- $f=0, 1, \omega, \omega^2,$ where $\omega$ is a primitive third root of unity. 
\section{} %Problem 6 
We have $f$ is entire, and its image is in the upper half plane. Then we have $f=u+iv,$ where $v \geq 0.$ Then $g=\exp(if)$ is also an entire function. We 
have $|g| = \left| e^{i(u +iv)} \right|= \left| e^{iu} \right| \cdot \left|e^{-v}\right|,$ which means that $|g| = \left|e^{-v}\right|,$ and since $v \geq 
0,$ we must have that $g$ is a bounded function. This means that $g$ must be constant by Liouville's theorem, which implies that $f$ is constant.  
\section{} %Problem 7 
Let us assume that $f,$ an entire function has non-dense image. Thus there exists $\lambda \in \mathbb{C}, r>0$ such that $f(\mathbb{C}) \cap B_r(\lambda) + 
phi.$ Then see that $g(z)= \frac{1}{f(z)-\lambda}$ is an entire function, since it cannot be zero. Moreover, since $|f(z)- \lambda| > r \implies |g(z)|< 
\frac{1}{r} $ which implies that $g$ is constant, hence $f$ is constant. This is a contradiction, hence we must have dense image. 

Let us assume $g$ is as given. Let $B_M$ be the ball around the origin of radius $M.$ Then pick an $\varepsilon$ neighbourhood of $z \in \mathbb{C}$ that is 
not in $f(B_M)$ and $\Re(z') > \Im(z')$ for all $z' \in B_{\varepsilon}(z).$ This cannot be in the image of $g,$ which contradicts the result that the image 
of a non-constant entire function is dense in $\mathbb{C}.$ Thus we must have that $g$ is constant.   
\section{} %Problem 8 
We are asked to find the maximum modulus of a polynomial on the unit disc $D.$ By the maximum modulus principle, we know that the maximum must be attained 
on the boundary of the circle, that is the set of points $z \in D$ such that $|z|=1.$ Thus let $z=e^{i\theta},$ and then see that $$f(e^{i\theta})= 
ae^{2i\theta} + 2(|a|^2-1)e^{i\theta} -\bar{a}= e^{i\theta} \left( ae^{i\theta} + 2(|a|^2-1) -\overline{ae^{i\theta}} \right).$$ 
Now we can say that 
\begin{align*}
	|p(z)|^2 &= |e^{i\theta}|^2 \left(\left(2|a|^2-2\right)^2+ \left( ae^{i\theta}-\overline{ae^{i\theta}} \right)^2 \right).
\end{align*}
Now see that the imaginary part is the only part dependent on $\theta.$ Let us only look at that. Let $a= re^{i\varphi}.$ Here, $r=|a|.$ Then we have 
$g(z)= \left( ae^{i\theta}-\overline{ae^{i\theta}} \right)= 2 \Im\left(re^{i(\theta + \varphi)}\right)= 2r \sin (\theta + \varphi).$ We can differentiate 
this with respect to $\theta$ to see that $g'(z)= 2r \cos(\theta + \varphi).$ Thus $g$ has a critical point at $\theta= - \varphi.$ Plugging this into 
$|p(z)|^2,$ we get $$	|p(z)|^2 = \left(2(r^2-1)\right)^2 .$$ This means that $|p(z)| \leq 2(r^2-1) \leq 2,$ as required.
\end{document}
