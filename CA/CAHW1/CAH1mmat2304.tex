%%%%%%%%%%%%%%%%%%%%%%%%%%%%%%%%%%%%%%%%%
% Lachaise Assignment
% LaTeX Template
% Version 1.0 (26/6/2018)
%
% This template originates from:
% http://www.LaTeXTemplates.com
%
% Authors:
% Marion Lachaise & François Févotte
% Vel (vel@LaTeXTemplates.com)
%
% License:
% CC BY-NC-SA 3.0 (http://creativecommons.org/licenses/by-nc-sa/3.0/)
% 
%%%%%%%%%%%%%%%%%%%%%%%%%%%%%%%%%%%%%%%%%

%----------------------------------------------------------------------------------------
%	PACKAGES AND OTHER DOCUMENT CONFIGURATIONS
%----------------------------------------------------------------------------------------

\documentclass{article}

\input{structure.tex} % Include the file specifying the document structure and custom commands

%----------------------------------------------------------------------------------------
%	ASSIGNMENT INFORMATION
%----------------------------------------------------------------------------------------

\title{} % Title of the assignment

\author{Gandhar Kulkarni (mmat2304)} % Author name and email address

\date{} % University, school and/or department name(s) and a date

%----------------------------------------------------------------------------------------

\begin{document}

\maketitle % Print the title

%----------------------------------------------------------------------------------------
%	INTRODUCTION
%----------------------------------------------------------------------------------------

\section{} %Problem 1  
We are given $f(z)=z^2-z\bar{z}^2-2|z|^2.$ Since $z=x+iy,$ we can expand it in $f$ to get 
$$f(x,y)=-2(x^2+y^2)+i4xy.$$ Thus $u=-2(x^2+y^2),v=4xy.$ Then we have $u_x=-4x,u_y=-4y,v_x=4y,v_y=4x.$
If $f$ is holomorphic, then we must have $u_x=v_y \implies -4x=4x \implies x=0.$ Also we must have $u_y=-v_x \implies -4y=-4y \implies y \in \mathbb{R}.$
Thus $f$ satisfies the Cauchy Riemann equations on $\{0\} \times \mathbb{R},$ which is not a domain since it is not open. Thus it is complex differentiable 
at each point of the type $(0,y)$ where $y \in \mathbb{R},$ but not holomorphic at any point in $\mathbb{C}$ since the points at which it satisfies the 
Cauchy Riemann equations is not open in $\mathbb{C}.$ 
\section{} %Problem 2
Let us assume that there exists a holomorphic function on a domain $D$ such that its image lies entirely on a vertical line, say $x=\frac{1}{2}$.
Thus for $f=u+iv,$ we must have that $u=\frac{1}{2},$ a constant. Then $u_x=u_y=0,$ and by the Cauchy-Riemann equations, we have $v_y=u_x=0=u_y=-v_x.$ Thus 
we have $v$ constant as well, which means that $f$ must be a constant.  
\section{} %Problem 3 

\section{} %Problem 4 
Let us assume that the image of $f$ lies on a line passing through the origin. We have $f(t)=\gamma_1(t)+i\gamma_2(t),$ and $\exists \alpha \in \mathbb{R} 
f([0,1])\subseteq ((x,\alpha x), x \in \mathbb{R})$ or $f([0,1]) \subseteq i\mathbb{R}.$ In the second case, we have $\gamma_1(t)=0.$ Then 
$|\int_{0}^1f(t)dt| = |i\int_{0}^1\gamma_2(t)dt| = |\int_{0}^1\gamma_2(t)dt|.$ Also $\int_{0}^1 |f(t)|dt= \int_{0}^1 |\gamma_2(t)|dt.$

For the first case, we have $\gamma_1(t)=\alpha \gamma_2(t).$ Then $|\int_{0}^1 f(t)dt| = |(1+i\alpha)\int_{0}^1 \gamma_1(t)dt| = 
\sqrt{1+\alpha^2}|\int_{0}^1 \gamma_1(t)dt|.$ Also see that $\int_{0}^1 |f(t)|dt= |\int_{0}^1 |\gamma_1(t)+i\alpha \gamma_1(t)| dt = \sqrt{1+\alpha^2} 
\int_{0}^1 |\gamma_1(t)|dt.$ 


\section{} %Problem 5 
\begin{enumerate}
	\item The curve can be parametrised by $\gamma: [0,1] \to \mathbb{C}.$ We have $\gamma(t)=\omega( 1-t) +\omega^2 t.$ Expanding this, we can see that 
	$\gamma(t)= \frac{-1+ \sqrt{3}i(1-2t)}{2}.$ See that $\gamma'(t)=-\sqrt{3}i,$ then we have 
	\begin{align*}
		\int_{\gamma}|z^2|dz &= \int_{0}^{1}|\frac{-1+ \sqrt{3}i(1-2t)}{2}|^2 | (-\sqrt{3}i)|dt\\
		&= \frac{\sqrt{3}}{4}\int_{0}^1 (\frac{-1+ \sqrt{3}i(1-2t)}{2})^2dt\\
		&= \frac{\sqrt{3}}{4}\int_{0}^1 (1-3(1-2t)^2 -2\sqrt{3}(1-2t))dt\\
		&= \frac{\sqrt{3}}{4}\int_{0}^1 
	\end{align*}
	\item The curve can be represented as $\gamma:\left[\frac{2\pi}{3},\frac{4\pi}{3}\right] \to \mathbb{C}.$ We have $\gamma(t)= e^{it}.$ For this we have 
	$\gamma'(t)=ie^{it},$ so $|\gamma'(t)| =1.$ Then the integral is 
	\begin{align*}
		\int_{\gamma}|z^2|dz &= \int_{\frac{2\pi}{3}}^{\frac{4\pi}{3}}|e^{it}|^2|1|dt\\
		&= \int_{\frac{2\pi}{3}}^{\frac{4\pi}{3}}dt\\
		&= \frac{2\pi}{3}.
	\end{align*} 
	\item Here, $\gamma$ has four parts labelled $1$ to $4.$ The curves are separately parametrised by the same parameter $t,$ in a classic fashion of abuse 
	of notation. These are the four curves: $\left(\frac{1}{2},t-\frac{1}{2}\right), \left(\frac{1}{2}-t,\frac{1}{2}\right), 
	\left(-\frac{1}{2},\frac{1}{2}-t\right),$ and $\left(t-\frac{1}{2},-\frac{1}{2}\right).$ Their respective absolute values of derivatives are $1$ 
	throughout. The integral is calculated thus:
	\begin{align*}
		\int_{\gamma}|z|^2dz &= \int_{0}^1 \left(\frac{1}{2}\right)^2+\left(t-\frac{1}{2}\right)^2dt+ \int_{0}^1 
		\left(\frac{1}{2}-t\right)^2+\left(\frac{1}{2}\right)^2dt+\\
		&\int_{0}^1 \left(-\frac{1}{2}\right)^2+\left(\frac{1}{2}-t\right)^2dtz+ \int_{0}^1 \left(t-\frac{1}{2}\right)^2+\left(-\frac{1}{2}\right)^2dt\\
		&= 4(\frac{1}{4} + \frac{1}{12})=\frac{4}{3}. 
	\end{align*}
\end{enumerate}
\section{} %Problem 6 
Let $C:=\{e^{it}: t \in \left[0,\frac{pi}{2}\right]\},$ which parametrizes the curve. Then 
\begin{align*}
\int_C \overline{\text{Log}(z)} dz &= \int_{0}^{\frac{\pi}{2}}\overline{\text{Log}(e^{it})} |ie^{it}|dt\\
	&=\int_{0}^{\frac{\pi}{2}}\overline{\log(1)+it}dt\\
	&=\int_{0}^{\frac{\pi}{2}}-it dt\\
	&= -i \frac{\pi^2}{4}.
\end{align*}

\section{} %Problem 7 
\section{} %Problem 8 
\section{} %Problem 9 
We will try to describe the function $\log \log z,$ the objective is to make sure that the argument of the function lies in $(-\pi,\pi].$ 
Let $z=x+iy.$ Then we have 
\begin{align*}
	\log \log (x+iy) &= \log \left( \log |z| + i 2n\pi \arg(z)  \right)\\
	&= \log \left( \frac{1}{2} \log (x^2+y^2) + i 2n\pi \tan^{-1}\left(\frac{y}{x}\right)  \right)\\
	&= \frac{1}{2} \log \left( \frac{1}{4} \log \left(x^2+y^2\right)^2 +4n\pi^2n^2 \tan^{-1}\left(\frac{y}{x}\right)^2 \right) \\
	&+ i\left( \frac{4\pi n}{\log (x^2+y^2)}\tan^{-1}\left(\frac{y}{x}\right) \right),
\end{align*}
where $n \in \mathbb{Z}.$ 
\section{} %Problem 10
We know that $|dz| = -iR \frac{dz}{z},$  then our integral is $I=\int_{|z|=R}\frac{-iRdz}{|z-a|^2}.$ If $|a| > R,$ then our function $\frac{1}{|z-a|^2}$ is 
holomorphic in the interior of the circle $|z|=R,$ so the integral must be $0.$ In the case where $|a| < R,$ we can use Cauchy's  
\end{document}
