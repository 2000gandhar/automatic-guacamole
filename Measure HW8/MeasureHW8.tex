%%%%%%%%%%%%%%%%%%%%%%%%%%%%%%%%%%%%%%%%%
% Lachaise Assignment
% LaTeX Template
% Version 1.0 (26/6/2018)
%
% This template originates from:
% http://www.LaTeXTemplates.com
%
% Authors:
% Marion Lachaise & François Févotte
% Vel (vel@LaTeXTemplates.com)
%
% License:
% CC BY-NC-SA 3.0 (http://creativecommons.org/licenses/by-nc-sa/3.0/)
% 
%%%%%%%%%%%%%%%%%%%%%%%%%%%%%%%%%%%%%%%%%

%----------------------------------------------------------------------------------------
%	PACKAGES AND OTHER DOCUMENT CONFIGURATIONS
%----------------------------------------------------------------------------------------

\documentclass{article}

\input{structure.tex} % Include the file specifying the document structure and custom commands

%----------------------------------------------------------------------------------------
%	ASSIGNMENT INFORMATION
%----------------------------------------------------------------------------------------

\title{Algebra HW5} % Title of the assignment

\author{Gandhar Kulkarni (mmat2304)} % Author name and email address

\date{} % University, school and/or department name(s) and a date

%----------------------------------------------------------------------------------------

\begin{document}

\maketitle % Print the title

%----------------------------------------------------------------------------------------
%	INTRODUCTION
%----------------------------------------------------------------------------------------

\section{} %Problem 1 
If we have $\nu=0,$ then see that $\nu(E)=\int_{E}0 d\mu,$ thus $\mu(E)=0 \implies \nu(E)=0$ trivially, so $\nu << \mu.$ Also see that $\nu \perp \mu$, as 
$X= X \sqcup \phi,$ and see that $\mu(E)=\mu(E \cap X),$ while $\nu(E)=\nu(E \int \phi).$ 

Now let us assume that $\nu << \mu$ and $\nu \perp \mu.$ Then we have $X=A \sqcup B,$ where $\mu(E)=\mu(E \cap A),$ while $\nu(E)=\nu(E \int B).$ Let us 
pick a measurable set $E \subseteq B.$ Then we have $\mu(E)=\mu(E \cap B)=\mu(\phi)=0.$ Since $\nu << \mu,$ we have $\nu(E)=0.$ Thus $\nu$ is zero on every 
measurable subset $E$ in $B.$ For a general measurable set $E,$ we have $E=(E \cap A) \sqcup (E \cap B).$ We already know that $\nu(E\cap A)=0,$ now we see 
that $\nu(E \cap B)=0$ also. Thus $\nu(E)=\nu(E \cap A)+\nu(E \cap B)=0$ for all measurable $E \subset X.$
\section{} %Problem 2
\section{} %Problem 3 
\section{} %Problem 4 
\begin{itemize}
	\item We find the positive and negative parts of $f.$ Note that the roots of this polynomial are $3+2\sqrt{2}$ and 
	$3-\sqrt{2}.$ Let us call them $\alpha_1$ and $\alpha_2$ for sake of convenience. Then $$p^+=\begin{cases}
		x^2-6x+1 & x \in (-\infty,\alpha_2] \cup [\alpha_1, \infty)\\
		0 & \text{ else},
	\end{cases}$$ 
	and $$p^-= \begin{cases} 
		-(x^2-6x+1) & x \in (\alpha_2,\alpha_1)\\
		0 & \text{ else}.
	\end{cases}$$
	Then $\nu(E)=\int_{E}p^+d\mu - \int_{E}p^-d\mu=\nu^+-\nu^-,$ where $\nu^+:=\int_{E}p^+d\mu$ and $\nu^-:=\int_{E}p^-d\mu$ are two positive measures. Note 
	that it is not possible for both of them to attain $\infty$ together, since $\nu^-$ is a finite measure. Thus it is trivial to see that $\nu$ must be a 
	signed measure.
	\item Let $\mathbb{R}= A \sqcup B,$ where $A= (-\infty,\alpha_2] \cup [\alpha_1, \infty)$ and $B=(\alpha_2,\alpha_1).$ See that since both the positive 
	measures have their usual properties, we have that for $E \subseteq A$ measurable, we have $\nu(E)=\nu^+(E)-\nu^-(E)=\nu^+(E)-0 \geq 0,$ and likewise 
	for $E \subseteq B$ measurable, we have $\nu(E)=\nu^+(E)-\nu^-(E)=0-\nu^-(E) \leq 0.$ Thus the above construction is a Hahn decomposition.
	\item See that $\nu^+$ lives on $A,$ while $\nu^-$ lives on $B.$ That is, $\nu^+(E)=\nu(E \cap A),$ and $\nu^-(E)=-\nu(E \cap B).$ This is easy to see, 
	as $E= (E \cap A) \sqcup (C \cap B).$ Then $\nu(E)= \int_{(E\cap A) \sqcup (E \cap B)}p^+d\mu - \int_{(E\cap A) \sqcup (E \cap B)}p^-d\mu= \int_{(E\cap 
	A)}p^+ d\mu + \int_{(E\cap B)}p^+ d\mu - \int_{(E\cap A)}p^- d\mu - \int_{(E\cap B)}p^- d\mu.$ Since $p^+$ is $0$ on $B,$ and $p^-$  is $0$ on $A,$ we 
	have that $\nu^+ \perp \nu^-.$ Thus we have the Jordan decomposition.  
\end{itemize}

\section{} %Problem 5 
Let us assume that $\mu(E)=0.$ As $E$ is $\mu-$null, then $\mu(E\cap E_n)=0$ for all $n,$ by monotonicity of the measure. Then we have 
$\nu(E)=\sum_{n=1}^{N}c_n \mu(E \cap E_n)=0.$ Thus $\nu << \mu.$ See that the function $f:= \sum_{n=1}^{N}c_n \chi_{E_n}$ is a good candidate for the 
Radon-Nikodym derivative. $$\int_{E} f d\mu= \int_{E}\sum_{n=1}^{N}c_n \chi_{E_n} d\mu= \sum_{n=1}^{N}c_n \int_{X}\chi_{E} \chi_{E_n} d\mu= 
\sum_{n=1}^{N}c_n \int_{X}\chi_{E \cap E_n} d\mu=\sum_{n=1}^{N}c_n \mu(E \cap E_n),$$ which is the desired result. Thus $\frac{d\nu}{d\mu}=f.$
\section{} %Problem 6 
\section{} %Problem 7 
\section{} %Problem 8 
In $(\mathbb{N},\mathbb{P(N)}),$ $\mu$ is the counting measure. Note that the empty set is the only $\mu-$null set, since every non-empty set has 
cardinality more than zero. Then somewhat trivially we have $\mu(E)=0 \implies E=\phi \implies \nu(E)=0.$ So $\nu << \mu.$ We have $\nu$ is $\sigma-$finite, 
thus $\mathbb{N}=\sum_{n=1}^{\infty}\{n\},$ where $\nu(\{n\})<\infty.$ Thus define $f: \mathbb{N} \rightarrow \mathbb{R},$ where $f(n)=\nu(\{n\}).$ Then we 
have $\nu(E)=\int_E f d\mu=\sum_{n \in E}f(n),$ is the required function. Thus $f=\frac{d\nu}{d\mu}.$
\section{} %Problem 9 
\section{} %Problem 10 
\section{} %Problem 11
\section{} %Problem 12
\end{document}

















