%%%%%%%%%%%%%%%%%%%%%%%%%%%%%%%%%%%%%%%%%
% Lachaise Assignment
% LaTeX Template
% Version 1.0 (26/6/2018)
%
% This template originates from:
% http://www.LaTeXTemplates.com
%
% Authors:
% Marion Lachaise & François Févotte
% Vel (vel@LaTeXTemplates.com)
%
% License:
% CC BY-NC-SA 3.0 (http://creativecommons.org/licenses/by-nc-sa/3.0/)
% 
%%%%%%%%%%%%%%%%%%%%%%%%%%%%%%%%%%%%%%%%%

%----------------------------------------------------------------------------------------
%	PACKAGES AND OTHER DOCUMENT CONFIGURATIONS
%----------------------------------------------------------------------------------------

\documentclass{article}

%%%%%%%%%%%%%%%%%%%%%%%%%%%%%%%%%%%%%%%%%
% Lachaise Assignment
% Structure Specification File
% Version 1.0 (26/6/2018)
%
% This template originates from:
% http://www.LaTeXTemplates.com
%
% Authors:
% Marion Lachaise & François Févotte
% Vel (vel@LaTeXTemplates.com)
%
% License:
% CC BY-NC-SA 3.0 (http://creativecommons.org/licenses/by-nc-sa/3.0/)
% 
%%%%%%%%%%%%%%%%%%%%%%%%%%%%%%%%%%%%%%%%%

%----------------------------------------------------------------------------------------
%	PACKAGES AND OTHER DOCUMENT CONFIGURATIONS
%----------------------------------------------------------------------------------------

\usepackage{amsmath,amsfonts,amssymb, tikz-cd} % Math packages

\usepackage{enumerate} % Custom item numbers for enumerations


\usepackage[framemethod=tikz]{mdframed} % Allows defining custom boxed/framed environments

\usepackage{listings} % File listings, with syntax highlighting
\lstset{
	basicstyle=\ttfamily, % Typeset listings in monospace font
}

%----------------------------------------------------------------------------------------
%	DOCUMENT MARGINS
%----------------------------------------------------------------------------------------

\usepackage{geometry} % Required for adjusting page dimensions and margins

\geometry{
	paper=letterpaper, % Paper size, change to letterpaper for US letter size
	top=2.5cm, % Top margin
	bottom=3cm, % Bottom margin
	left=2.5cm, % Left margin
	right=2.5cm, % Right margin
	headheight=14pt, % Header height
	footskip=1.5cm, % Space from the bottom margin to the baseline of the footer
	headsep=1.2cm, % Space from the top margin to the baseline of the header
	%showframe, % Uncomment to show how the type block is set on the page
}

%----------------------------------------------------------------------------------------
%	FONTS
%----------------------------------------------------------------------------------------

\usepackage[utf8]{inputenc} % Required for inputting international characters
\usepackage[T1]{fontenc} % Output font encoding for international characters


%----------------------------------------------------------------------------------------
%	COMMAND LINE ENVIRONMENT
%----------------------------------------------------------------------------------------

% Usage:
% \begin{commandline}
	%	\begin{verbatim}
		%		$ ls
		%		
		%		Applications	Desktop	...
		%	\end{verbatim}
	% \end{commandline}

\mdfdefinestyle{commandline}{
	leftmargin=10pt,
	rightmargin=10pt,
	innerleftmargin=15pt,
	middlelinecolor=black!50!white,
	middlelinewidth=2pt,
	frametitlerule=false,
	backgroundcolor=black!5!white,
	frametitle={Command Line},
	frametitlefont={\normalfont\sffamily\color{white}\hspace{-1em}},
	frametitlebackgroundcolor=black!50!white,
	nobreak,
}

% Define a custom environment for command-line snapshots
\newenvironment{commandline}{
	\medskip
	\begin{mdframed}[style=commandline]
	}{
	\end{mdframed}
	\medskip
}

%----------------------------------------------------------------------------------------
%	FILE CONTENTS ENVIRONMENT
%----------------------------------------------------------------------------------------

% Usage:
% \begin{file}[optional filename, defaults to "File"]
	%	File contents, for example, with a listings environment
	% \end{file}

\mdfdefinestyle{file}{
	innertopmargin=1.6\baselineskip,
	innerbottommargin=0.8\baselineskip,
	topline=false, bottomline=false,
	leftline=false, rightline=false,
	leftmargin=2cm,
	rightmargin=2cm,
	singleextra={%
		\draw[fill=black!10!white](P)++(0,-1.2em)rectangle(P-|O);
		\node[anchor=north west]
		at(P-|O){\ttfamily\mdfilename};
		%
		\def\l{3em}
		\draw(O-|P)++(-\l,0)--++(\l,\l)--(P)--(P-|O)--(O)--cycle;
		\draw(O-|P)++(-\l,0)--++(0,\l)--++(\l,0);
	},
	nobreak,
}

% Define a custom environment for file contents
\newenvironment{file}[1][File]{ % Set the default filename to "File"
	\medskip
	\newcommand{\mdfilename}{#1}
	\begin{mdframed}[style=file]
	}{
	\end{mdframed}
	\medskip
}

%----------------------------------------------------------------------------------------
%	NUMBERED QUESTIONS ENVIRONMENT
%----------------------------------------------------------------------------------------

% Usage:
% \begin{question}[optional title]
	%	Question contents
	% \end{question}

\mdfdefinestyle{question}{
	innertopmargin=1.2\baselineskip,
	innerbottommargin=0.8\baselineskip,
	roundcorner=5pt,
	nobreak,
	singleextra={%
		\draw(P-|O)node[xshift=1em,anchor=west,fill=white,draw,rounded corners=5pt]{%
			Question \theQuestion\questionTitle};
	},
}

\newcounter{Question} % Stores the current question number that gets iterated with each new question

% Define a custom environment for numbered questions
\newenvironment{question}[1][\unskip]{
	\bigskip
	\stepcounter{Question}
	\newcommand{\questionTitle}{~#1}
	\begin{mdframed}[style=question]
	}{
	\end{mdframed}
	\medskip
}

%----------------------------------------------------------------------------------------
%	WARNING TEXT ENVIRONMENT
%----------------------------------------------------------------------------------------

% Usage:
% \begin{warn}[optional title, defaults to "Warning:"]
	%	Contents
	% \end{warn}

\mdfdefinestyle{warning}{
	topline=false, bottomline=false,
	leftline=false, rightline=false,
	nobreak,
	singleextra={%
		\draw(P-|O)++(-0.5em,0)node(tmp1){};
		\draw(P-|O)++(0.5em,0)node(tmp2){};
		\fill[black,rotate around={45:(P-|O)}](tmp1)rectangle(tmp2);
		\node at(P-|O){\color{white}\scriptsize\bf !};
		\draw[very thick](P-|O)++(0,-1em)--(O);%--(O-|P);
	}
}

% Define a custom environment for warning text
\newenvironment{warn}[1][Warning:]{ % Set the default warning to "Warning:"
	\medskip
	\begin{mdframed}[style=warning]
		\noindent{\textbf{#1}}
	}{
	\end{mdframed}
}

%----------------------------------------------------------------------------------------
%	INFORMATION ENVIRONMENT
%----------------------------------------------------------------------------------------

% Usage:
% \begin{info}[optional title, defaults to "Info:"]
	% 	contents
	% 	\end{info}

\mdfdefinestyle{info}{%
	topline=false, bottomline=false,
	leftline=false, rightline=false,
	nobreak,
	singleextra={%
		\fill[black](P-|O)circle[radius=0.4em];
		\node at(P-|O){\color{white}\scriptsize\bf i};
		\draw[very thick](P-|O)++(0,-0.8em)--(O);%--(O-|P);
	}
}

% Define a custom environment for information
\newenvironment{info}[1][Info:]{ % Set the default title to "Info:"
	\medskip
	\begin{mdframed}[style=info]
		\noindent{\textbf{#1}}
	}{
	\end{mdframed}
}
 % Include the file specifying the document structure and custom commands

%----------------------------------------------------------------------------------------
%	ASSIGNMENT INFORMATION
%----------------------------------------------------------------------------------------

\title{Measure Theory HW8} % Title of the assignment

\author{Gandhar Kulkarni (mmat2304)} % Author name and email address

\date{} % University, school and/or department name(s) and a date

%----------------------------------------------------------------------------------------

\begin{document}

\maketitle % Print the title

%----------------------------------------------------------------------------------------
%	INTRODUCTION
%----------------------------------------------------------------------------------------

\section{} %Problem 1 
We will use $i=1,2$ to denote the measures. Let $E_{n,i}=\{x: \frac{d\nu_i}{d\mu_i}< -\frac{1}{n}\},$ and see that 
$$\nu_i(E_{n,i})=\int_E d\nu_i= \int_{E_{n,i}}-\frac{1}{n}\int_{E_{n,i}}d\mu_i =- \frac{1}{n} \mu_i(E_{n,i}).$$ Since $\nu_i$ is a positive measure, 
$\mu_i(E_{n,i})=0=\nu_i(E_{n,i})$ for all $n$.  We have $E= \cup_{n=1}^{\infty}E_{n,i},$ that is, $E=\{x: \frac{d\nu_i}{d\mu_i}<0\}.$ 
By continuity from below, $\mu(E)=\lim_{n \to \infty}\mu(E_{n,i})=0.$ Then $\frac{d\nu_i}{d\mu_i}\geq 0$ almost everywhere. Then by Tonelli's theorem, 

\begin{align*}
	(\nu_1 \times \nu_2)(E)&= \int_{E} d(\nu_1 \times \nu_2)= \int \chi_E d(\nu_1 \times \nu_2)\\
	&= \int \int \chi_E d\nu_1 d\nu_2 = \int \left( \int \chi_E \frac{d\nu_1}{d\mu_1} d\mu_1\right)\frac{d\nu_2}{d\mu_2} d\mu_2\\
	&= \int \int \chi_E  \frac{d\nu_1}{d\mu_1} \frac{d\nu_2}{d\mu_2} d\mu_1 d\mu_2\\
	&= \int_E \frac{d\nu_1}{d\mu_1} \frac{d\nu_2}{d\mu_2} d(\mu_1 \times \mu_2).
\end{align*}
This gives us the desired result. 
\section{} %Problem 2
$f \in BV$ means that $V^b_a(f)\leq M$ for some $M \in \mathbb{N}.$ Pick a partition $\mathcal{P}$ of $[a,b],$ then we have $\sum_{\mathcal{P}} 
|f(x_{i+1})-f(x_i)| \geq \sum_{\mathcal{P}} ||(x_{i+1})| - |f(x_i)||.$ Thus we have $$V(|f|,\mathcal{P}) \leq V(f,\mathcal{P}).$$ Taking the limit over all 
partitions, we have $V(|f|)\leq V(f) < \infty.$ 
\section{} %Problem 3 
\section{} %Problem 4 
\begin{enumerate}
	\item Since $V(x)$ is an increasing function, it is differentiable almost everywhere. Choose a partition $\mathcal{P}_1$ for $[a,x],$ and a partition 
	$\mathcal{P}_2$ for $[a,x+h].$ Then we have a partition finer than both of them, say $\mathcal{P}.$ Then we can write $V(f, \mathcal{P})(a,x+h)= V(f, 
	\mathcal{P})(a,x)+V(f, \mathcal{P})(x,x+h).$ Thus $V(f, \mathcal{P})(a,x+h)-V(f, \mathcal{P})(a,x)= V(f, \mathcal{P})(x,x+h).$ Then see that $$ 
	\frac{V(f, \mathcal{P})(a,x+h)-V(f, \mathcal{P})(a,x)}{h} = \frac{V(f, \mathcal{P})(x,x+h)}{h}.$$ See that $$\frac{V(f, \mathcal{P})(x,x+h)}{h} \geq 
	\frac{|\sum_{i=1}^{N}f(x_{i+1}-x_i)|}{h}= \frac{|f(x+h)-f(x)|}{h}.$$ Since the term on the right is dependent on $h,$ and not the partition, we let it 
	go to zero, which gives us $|f'(x)|.$ The term on the left can be refined with respect to partitions, which gives us $\frac{V(x+h)-V(x)}{h}.$ Since the 
	derivative of $V$ exists almost everywhere, we have the desired inequality.
	
	\item  From monotonicity of the integral, we have $\int_{a}^{b}|f'|\leq \int_{a}^b V' \leq V(b)-V(a)=V(b),$ as required. 
	
	\item If $f$ is AC, then we have $f(x)-f(a)=\int_{a}^x g, $ for some $L^1$ function $g.$ Then see that for some partition $\mathcal{P},$ we have 
	$V(f,\mathcal{P})= \sum_{i=1}^n \left|\int_{x_i}^{x_{i+1}}g\right| \leq \int_{x_i}^{x_{i+1}}\left|g\right| = \int_{a}^b |g|.$
	Thus taking the limit over all partitions, we have $V(b) \leq \int_{a}^{b}|f'|,$ which combined with the opposite inequality shown gives us the required 
	result.
	
	For the converse, we assume $\int_{a}^{b}|f'| = V(b).$ The term on the right is also $\int_{a}^{b}V',$ so $V$ must be absolutely continuous as it is 
	monotonous and of bounded variation. See that then $ \int_{a}^b V' - |f'| =0.$ Since we know $|f'| \leq V'$ almost everywhere, we must have $V'=|f'|$ 
	almost everywhere. Choose a partition $\mathcal{P}$ such that $ \sum_{\mathcal{P}} |x_{i+1}-x_i|< \delta,$ we have $\sum_{\mathcal{P}} 
	|V(x_{i+1})-V(x_i)|< \varepsilon,$ for some $\varepsilon >0$ from the absolute continuity of $V.$ Then at any subinterval of the partition $P_k$ we have 
	$V(f,P_i) =|f(x_{i+1})-f(x_{i})| \leq V(x_{i+1})-V(x_i)= |V(x_{i+1})-V(x)|.$ Then summing over all subintervals we have our desired result. 
\end{enumerate}
\section{} %Problem 5 
Define $h_n,h$ where $h_n=\sum_{i=1}^n |f_i|, h=\sum_{i=1}^{\infty} |f_i|.$ We have an increasing sequence of positive functions that converge to $h,$ so we 
have by the monotone convergence theorem, $$\int h^p = \lim_{n \to \infty} \int h_n^p. $$ By Minkowski's inequality we have $$||h_n ||_p \leq \sum_{i=1}^n 
|| g_n||_p \leq M,$$ where $M=\sum_{i=1}^{\infty} || g_n||_p.$ Then we have $h$ is $L^p$ and it is finite a.e. We also have that $\sum_{i=1}^n g_i$ is 
convergent to some $f$ such that $|f| \leq h,$ so we have $$\left| f- \sum_{i=1}^n g_i \right|^p=\left(|f| + \sum_{i=1}^n |g_i|  \right) \leq (2h)^p,$$ so 
by the dominated convergence theorem we have $\int (f- \sum_{i=1}^n g_i)^p \to \infty$ as $n \to \infty.$ Thus we have $\sum_{i=1}^{\infty} g_i$ converging 
to $f$ in $L^p,$ as desired.   
\section{} %Problem 6 
\section{} %Problem 7 

\section{} %Problem 8 

\section{} %Problem 9 
\section{} %Problem 10 
\begin{enumerate}
	\item Let us have $f \in L^{p_2}.$ Then let $r= \frac{p_2}{p_1}>1.$ We can see that the H\"{o}lder dual of $r$ is $r'=\frac{p_2}{p_2-p_1}.$ We apply 
	H\"{o}lder's inequality on $|f|^{p_1}$ and $1,$ using $r$ and $r',$ we get 
	\begin{align*}
		\int |f|^{p_1} d\mu &\leq \left(\int |f^{p_1}|d\mu\right)^{\frac{1}{r}}\cdot \left(\int 1^{r'}d\mu\right)^{\frac{1}{r'}}\\
		&=\mu(X)^{\frac{p_2-p_1}{p_2}}\left(\int |f|^{p_2}d\mu\right)^{\frac{p_1}{p_2}}.	
	\end{align*} 
		Now raising both sides to $\frac{1}{p_1},$ we have $$ \left(|f|^{p_1} d\mu\right)^{\frac{1}{p_1}} \leq \mu(X)^{\frac{p_2-p_1}{p_2}}\left(\int 
		|f|^{p_2}d\mu\right)^{\frac{1}{p_2}}.$$ Since $\mu(X)<\infty,$ $f \in L^{p_1}.$
		
	\item Using the above inequality and putting $\mu(X)\leq 1$ gives us the required result. 
	\item In the measure space $[1,\infty)$ we have $f(x)=\frac{1}{x},$ which is a $L^2$ function, as $\int_{1}^{\infty}\frac{1}{x^2} dx= 1,$ but 
	$\int_{1}^{\infty}\frac{1}{x}dx = \infty,$ thus we have a $L^2$ function that is not $L^1.$  
\end{enumerate}
\section{} %Problem 11
We integrate $e^{-px}$ as see when it has a finite value. See that $$\int_{0}^{\infty}e^{px}dx= \lim_{n \to \infty} \left[\frac{e^{-px}}{-p}\right]_0^n= 
\lim_{n \to \infty} \frac{1-e^{np}}{p}=\frac{1}{p},$$ which implies that $e^{-x} \in L^p$ for $p \in [1,\infty].$
\section{} %Problem 12
We want to evaluate $\sum_{i=1}^{\infty}\frac{1}{(\sqrt{n}\log n)^p}.$ See that this is a decreasing sequence (let $a_n=\frac{1}{(\sqrt{n}\log n)^p}$). Then 
$\sum a_n$ converges iff $\sum 2^na_{2^n}$ converges, by the Cauchy condensation test. 
We can reduce this to $\sum 2^n \frac{1}{(\sqrt{2^n}\log (2^n))^p}= \sum 2^{k-\frac{kp}{2}} \frac{1}{(n^p (\log 2)^p}.$

\section{} %Problem 13

\end{document}

















