%%%%%%%%%%%%%%%%%%%%%%%%%%%%%%%%%%%%%%%%%
% Lachaise Assignment
% LaTeX Template
% Version 1.0 (26/6/2018)
%
% This template originates from:
% http://www.LaTeXTemplates.com
%
% Authors:
% Marion Lachaise & François Févotte
% Vel (vel@LaTeXTemplates.com)
%
% License:
% CC BY-NC-SA 3.0 (http://creativecommons.org/licenses/by-nc-sa/3.0/)
% 
%%%%%%%%%%%%%%%%%%%%%%%%%%%%%%%%%%%%%%%%%

%----------------------------------------------------------------------------------------
%	PACKAGES AND OTHER DOCUMENT CONFIGURATIONS
%----------------------------------------------------------------------------------------

\documentclass{article}

\input{structure.tex} % Include the file specifying the document structure and custom commands

%----------------------------------------------------------------------------------------
%	ASSIGNMENT INFORMATION
%----------------------------------------------------------------------------------------

\title{Algebra HW5} % Title of the assignment

\author{Gandhar Kulkarni (mmat2304)} % Author name and email address

\date{} % University, school and/or department name(s) and a date

%----------------------------------------------------------------------------------------

\begin{document}

\maketitle % Print the title

%----------------------------------------------------------------------------------------
%	INTRODUCTION
%----------------------------------------------------------------------------------------

\section{} %Problem 1 
\begin{enumerate}
	\item See that $bx-a \in \ker \pi,$ since $b \cdot \left( \frac{a}{b} \right)-a=0.$ Therefore $(bx-a) \subseteq \ker \pi.$ Now consider $f(x) \in R[x]$ 
	where $f(x) \in \ker \pi.$ We consider the polynomials $f(x)$ and $x-\frac{a}{b}$ as elements of $Q[x]$ the ring of polynomials with coefficients from 
	the fraction field of $R.$ Then this is a PID, which is why we can apply the division algorithm to see that $f(x)=q(x)\left(x-\frac{a}{b}\right)+c,$ 
	where $c \in Q, q(x) \in Q[x].$ Setting $x=\frac{a}{b}$ we get $c=0.$ Thus we have $f(x)=q(x)\left(x-\frac{a}{b}\right).$ We rewrite all polynomials are 
	primitive polynomials in $R[x];$ thus we have $f(x)=a_1 \cdot f_0(x),$ $q(x)=a_2 \cdot q_0(x),$ and $x-\frac{a}{b}=b^{-1} \cdot (bx-a).$ Then we have $ 
	a_1 \cdot f_0(x)= (a_2b^{-1}) q_0(x) (bx-a).$ We multiply on both sides by some $k \in R$ such that $ka_1b \in R $ and $ ka_2 \in R$ and the two are 
	coprime. The constant cannot divide the polynomials as they are all primitive, hence we must have $ka_1b  |  ka_2,$ and by Gauss' lemma we can say that 
	$f_0(x) | q_0(x)(bx-a),$ that is, $f \in (bx-a).$ Thus we have $\ker \pi=(bx-a).$
	
	\item Note that $(1+\sqrt{-3})\cdot (1-\sqrt{-3})=2 \cdot 2 = 4,$ which means that $R$ is not a UFD. Therefore the above result needn't apply. To show 
	that the above result strictly does not apply, we need to find $f \in \ker \pi$ such that $f \notin (2x-(1+\sqrt{-3})).$ See that $f(x)=x^2-x+1$ does 
	the trick well. It is in fact the minimal polynomial, but it is not in $(2x-(1+\sqrt{-3})).$ It is easy to prove, as the ideal of leading coefficients 
	$R \cap (2x-(1+\sqrt{-3}))=(2),$ and this clearly does not include the leading coefficient of $f(x).$ Thus $f \in \ker \pi \backslash 
	(2x-(1+\sqrt{-3})).$
	
	The underlying reason for why this fails stems from the fact that the ring of integers of the number field $\mathbb{Q}(\sqrt{-3})$ is $\mathbb{Z}\left[ 
	\frac{1+\sqrt{-3}}{2}\right] \supsetneq \mathbb{Z}[\sqrt{-3}];$ that is, the ring of integers is strictly larger than $R$ as given in this problem. This 
	has to do with the fact that $-3 \cong 1 \mod 4,$ which introduces interesting additional algebraic integers into the number field. 
\end{enumerate}
\section{} %Problem 2
\begin{enumerate}
	\item Let us assume for the sake of contradiction that $I$ has less than three generators. Then could have two generators, or even one. In case there is 
	one generator, then let $I=(f)=(x,y,z).$ See that $\frac{F[x,y,z]}{(x,y,z)}\equiv F,$ thus $I$ is maximal, and hence prime. Thus $f$ must be prime. Then 
	we have $f|x,$ which implies that $f$ divides $x,$ a prime itself, which is absurd. Hence no such $f$ exists, and $I$ cannot have just one generator.
	
	$I$ is not generated, but it could be generated by two elements. If this is the case, let $I=(f_1,f_2).$ Then consider this expression modulo $I^2.$ 
	Note that $I^2$ gives us the module of all polynomials in $R$ with degree greater than or equal to $2$. Since $I$ is maximal as an ideal in $R,$ 
	$\frac{I}{I^2}$ will be a $\frac{R}{I} \equiv F-$module, that is, a vector space. See that $x,y,z$ reduced modulo $I^2,$ are linearly independent, so 
	this means that as a vector space $\frac{I}{I^2}$ has dimension $\geq 3.$ This naturally means that two generators will not be sufficient. 
	
	\item We know that all commutative rings have a maximal ideal, thanks to Zorn's lemma. Then we have for a commutative ring $A$ the maximal ideal 
	$\mathfrak{m},$ so we have $\frac{A[x,y,z]}{\mathfrak{m}}\equiv \left(\frac{A}{\mathfrak{m}}\right)[x,y,z]=F[x,y,z],$ where $F:=\frac{A}{\mathfrak{m}}$ 
	is a field. Note that $\mathfrak{m}A[x,y,z]$ is a maximal ideal in $R=A[x,y,z],$ and $I=(x,y,z)$ is a $R-$module. Thus we can say that 
	$\frac{I}{\mathfrak{m}I}$ is a $F-$module. Now we need to see that $x$ is not affected by reduction modulo $\mathfrak{m}I.$ See that $x \in 
	\mathfrak{m}I$ means that $x=\sum_{\text{finite}} (xf_1+yf_2+zf_3),$ where $f_1,f_2,f_3 \in (\mathfrak{m})[x,y,z].$ Then by putting $y=z=0,$ we get 
	$x=\sum_{\text{finite}}x \overline{f_1(x)} \implies  \sum_{\text{finite}}\overline{f_1(x)}=1.$ Comparing the constant terms, we must have a combination 
	of 
	scalars in $\mathfrak{m}$ that add up to $1$. However, that would imply that $1 \in \mathfrak{m},$ which is absurd. Thus $x \notin \mathfrak{m}I,$ and 
	similarly for $y$ and $z.$ Now we consider the map $\pi: I \rightarrow \frac{I}{\mathfrak{m}I}$ which is the canonical map. Then see that $x,y,z$ are 
	not affected by this map as previously shown. For any $h(x,y,z)=xh_1+yh_2+zh_3 \in I,$ we have $\pi(h(x,y,z))=x\bar{h_1}+y\bar{h_2} + z\bar{h_3}.$ We 
	have $\frac{I}{\mathfrak{m}I} \cong I$ as a $\frac{R}{\mathfrak{m}}=F[x,y,z]-$module.Using the previous result, we can say that this cannot have less 
	than three generators, which gives us our answer. 
\end{enumerate}
\section{} %Problem 3 
Let $R=\frac{F[x,y]}{xy},$ and $I=(\bar{x},\bar{y}).$ Then $$\frac{R}{I} \equiv \frac{\frac{F[x,y]}{(xy)}}{(\bar{x},\bar{y})} \equiv \frac{F[x,y]}{(xy,x,y)} 
\equiv \frac{F[x]}{0 \cdot x,x} \equiv F,$$ which is a domain. Thus $I$ is prime. To see that it is not principal, we assume for the sake of contradiction 
that $I=(f_0),$ where $f_0 \in F[x,y].$ Note that once seen modulo $(xy),$ we have $\bar{f_0}=f_1(x)+f_2(y),$ where $f_1 \in F[x], f_2 \in F[y].$ If we say 
that $(f_0)=(\bar{x},\bar{y}),$ then we have $f_0 | x$ and $f_0 | y.$ $f_0(x,y)=f_1(x)+f_2(y)$ must have degree less than  or equal to $1,$ with no term of 
$y,$ hence $f_2(y)=c_2$ and $f_1(x)=c_1+dx.$ Set $c=c_1+c_2,$ then see that we must have $x=t(c+dx)$ for some $t \in F[x,y].$ Comparing degrees, we must 
have $t \in F \backslash \{0\}.$ Comparing the two sides, see that $c=0, d=1/t$ which is the only possibility. However, $x \nmid y,$ so such a $f_0$ cannot 
exist. Thus $I$ is not principal.

We know that prime ideals in $R$ correspond to prime ideals in $F[x,y]$ that contain $(xy).$ Let $\mathfrak{p}$ be the prime ideal in $F[x,y]$ containing 
$(xy)$ and $\overline{\mathfrak{p}}=\pi(\mathfrak{p}),$ where $\pi$ is the natural map from $F[x,y]$ to $R.$ Either $x \in \mathfrak{p}$ and $y \notin 
\mathfrak{p}$ or $x \notin \mathfrak{p}$ and $y \in \mathfrak{p}$ or $x \in \mathfrak{p}$ and $y \in \mathfrak{p}.$ In the first case, see that $(F[x])[y]$ 
is a polynomial over a PID. From a previous assignment, we know that a prime ideal over such a ring would either be $(0),$ $(f(y))$ for $f(y)$ irreducible 
in $(F[x])[y]$ or $(p,f(y))$ where $p$ is prime in $F[x]$ and $f(y)$ is irreducible in $\frac{(F[x])[y]}{(p)}$. The first two cases are already principal, 
we need to see that the third case is also principal. $x$ is a prime in $F[x].$ See that $f(y)=xg(x,y)+ f_1(y)$ where $f_1(y)$ is irreducible in 
$F[x,y]/(x)=F[y].$ Then we have $\mathfrak{p}=(x,f_1(y)).$ We have $\overline{\mathfrak{p}}=(x,\overline{f_1(y)}).$ By our assumption $f_1(y)$ is an 
irreducible polynomial different from $y,$ so its constant term is necessarily non-zero. Then we have in $R,$ $x(f_1(y))=cx,$ where $ c \in F$ is the 
constant term of the polynomial $f_1(y).$ Thus we have $c^{-1}cx \in (f_1(y)),$ thus $\overline{\mathfrak{p}}=(\overline{f_1(y)}),$ a principal ideal. 

The idea applies to the second case. In the third case, see that only the third type is possible. Thus we have $\mathfrak{p}=(x,f(y)).$ Since $y \in 
\mathfrak{p},$ and $f(y)$ is irreducible, we must have $f(y)=y.$ Then $\overline{f(y)}=y,$ thus $\overline{\mathfrak{p}}=(x,y),$ which as discussed is the 
non-principal ideal.

Thus every other prime ideal is principal.
\section{} %Problem 4 
Let $$A=\begin{pmatrix}
	1 & 2 & 2 &3 \\
	5 & 5 & 4 & 4\\
	6 & 7 & 7 & 8\\
	10 & 10 & 9 & 9
\end{pmatrix}.$$ Through a myriad set of row and column operations shall we reduce our matrix $A$ to form that will generate an alike cokernel. We shall use 
$R_1,R_2,$ and $R_3$ to denote the rows, while $C_1,C_2,$ and $C_3$ shall denote the columns of $A.$ First we execute $C_2 \mapsto C_2-C_1, C_4 \mapsto C_4 
- C_3.$ Then we execute $C_4 \mapsto C_4 - C_2$ to get $$A'=\begin{pmatrix}
1 & 1 & 2 & 0\\
5 & 0 & 4 & 0\\
6 & 1 & 7 & 0\\
10 & 0 & 9 & 0
\end{pmatrix}.$$
Execute $C_2 \mapsto C_2 - C_1, C_3 \mapsto C_3- 2 C_1$ to get $$A''=\begin{pmatrix}
	1 & 0 & 0 & 0\\
	5 & -5 & -6 & 0\\
	6 & -5 & -5 & 0\\
	10 & -10 & -11 &0
\end{pmatrix}. $$
Execute $R_2 \mapsto R_2 -R_1, R_3 \mapsto R_3- 6R_1,$ and $R_4 \mapsto R_4 - 10R_1.$ After this, execute $R_3 \mapsto R_3-R_2$ and $R_4 \mapsto R_4 - 2R_2$ 
to get $$A'''=\begin{pmatrix}
	1 & 0 & 0 & 0\\
	0 & -5 & -6 & 0\\
	0 & 0 & 1 & 0\\
	0 & 0 & 1 & 0
\end{pmatrix}.$$ Execute $R_2 \mapsto -R_2,$ and $R_4 \mapsto R_4 - R_3.$ After this execute $R_2 \mapsto R_2 - 6R_3$ to get $$A''''=\begin{pmatrix}
	1 & 0 & 0 & 0\\
0 & 5 & 0 & 0\\
0 & 0 & 1 & 0\\
0 & 0 & 0 & 0
\end{pmatrix}.$$

We are well aware that the cokernel is left unchanged due to our row and column operations. Thus it can be seen that the image of this matrix is $A''''(x_1 
x_2 x_3 x_4)^T=(x_1 5x_2 x_3 0).$ Therefore the image is $\mathbb{Z} \oplus 5\mathbb{Z} \oplus \mathbb{Z} \oplus 0.$ Then $\text{coker} 
A=\frac{Z^4}{\mathbb{Z} \oplus 5\mathbb{Z} \oplus \mathbb{Z} \oplus 0}= \frac{\mathbb{Z}}{5\mathbb{Z}}.$

\section{} %Problem 5 
\begin{enumerate}
	\item Since $f \circ g=0,$ we have $f(g(p))=0 \forall p \in P.$ Then we have $g(p) \in \ker f \forall p \in P \implies g(P) \subseteq \ker f.$ Thus we 
	can define $h: P \rightarrow \ker f$ as $h(p)=g(p).$ If another $h': P \rightarrow \ker f$ exists such that $g=i \circ h',$ then we have $g=i \circ h= i 
	\circ h'.$ Since $i$ is injective, for all $p \in P$ we have $i(h(p))=i(h'(p)) \implies h(p)=h'(p),$ thus we have $h=h',$ proving the uniqueness of $h.$
	
	\item Let $h(\bar{n})=g\circ \pi^{-1}(\bar{n}),$ for $\bar{n} \in \text{coker} f.$ We propose that this is the desired map. We need to see that this map 
	is well defined. $\pi^{-1}(\bar{n})= n + f(M),$ for some $n \in N.$ We need to see that the choice of representative does not matter. We can see that 
	since $g$ is $R-$linear we have $g(n + f(M))= g(n) + g \circ f(M)=g(n) \in P.$ Thus this map is well defined. To see that this map is unique, for 
	another such map $h': \text{coker} f \rightarrow P$ such that $ g=h \circ \pi,$ we have $g=h \circ \pi= h' \circ \pi,$ which implies that $h=h'$ is 
	surjective, where right cancellation is possible. Thus this map is unique. 
\end{enumerate}
\section{} %Problem 6 
We can see that $(0) \subseteq \ker f \subseteq \ker f^2 \subseteq \dots$ which is an ascending chain of submodules of $M.$ This clearly must stabilise as 
the Noetherian condition is equivalent to the ascending chain condition. That is, for some $n \in \mathbb{N}$ we have $\ker f^n=\ker f^{n+1}=\ker 
f^{n+2}=\dots.$ Now see that for some $m \in \ker f$ we have $f(m)=0.$ Since $f$ is surjective, we can find a $m' \in M$ such that $f(m')=m.$ Repeating this 
process, see that there must exist some $m_n \in M$ such that $f^n(m_n)=m.$ Applying $f$ on both sides, we have $f^{n+1}(m_n)=f(m)=0.$ Thus $m_n \in \ker 
f^{n+1}=\ker f^n,$ we must have $f^n(m_n)=m=0.$ Thus $m$ must necessarily be zero, meaning that a surjective endomorphism on a Noetherian module must 
necessarily be injective, and thus an isomorphism. 
\end{document}

















